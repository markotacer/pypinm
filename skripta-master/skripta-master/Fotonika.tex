
\documentclass[11pt,fleqn]{book} % Default font size and left-justified equations
\input{preamble}
\usepackage{amsmath}
\usepackage{amssymb}
\usepackage{amsfonts}
\pdfsuppresswarningpagegroup=1
\begin{document}

\let\cleardoublepage\clearpage
\makeatletter
\setlength{\@fptop}{0pt}
\makeatother

%--------------------------------------------------------------------------

%	TITLE PAGE
%--------------------------------------------------------------------------


\begingroup
\thispagestyle{empty}
%\AddToShipoutPicture*{\put(0,0){\includegraphics[scale=1.25]{v}}} % Image background
\centering
\vspace*{5cm}
\par\normalfont\fontsize{35}{35}\sffamily\selectfont
\textbf{FOTONIKA}
{\LARGE }\par % Book title
\vspace*{1cm}
%{\LARGE Učno gradivo za študente \\ Fakultete za fiziko
%\\ 
%Univerze v Ljubljani\\}\par % Author name
\vspace*{1cm}
%{\LARGE \textcolor{red}{Delovna verzija}\\}\par
\vspace*{8cm}
{\Large Martin ČOPIČ \\ Mojca VILFAN \\}\par % Author name
\endgroup

%--------------------------------------------------------------------
%	COPYRIGHT PAGE
%-------------------------------------------------------------------

\newpage
~\vfill
\thispagestyle{empty}

% Copyright \copyright\ 2014 Andrea Hidalgo\\ % Copyright notice
FOTONIKA \\

Martin Čopič in Mojca Vilfan\\
{\it Fakulteta za matematiko in fiziko\\
Univerza v Ljubljani}\\
in\\
{\it Institut ``Jožef Stefan'', Ljubljana}\\
 
 Recenzija: Nataša Vaupotič in Rok Petkovšek \\% Recenzent

 Risbe in diagrami: Mojca Vilfan in Andrej Petelin\\ % Risbe in diagrami
 
 Oblikovanje, postavitev in prelom: Mojca Vilfan \\ %Design

 Besedilo je bilo jezikovno pregledano.\\
 
 Naslovne slike poglavij: Mojca Vilfan, Martin Rigler (str.~121 in 201), Andrej Petelin (str.~37),
Vojna mornarica Združenih držav Amerike, U.\ S.\ Navy (str.~93), NASA (str.~175) in Irena Drevenšek (str.~229)\\ % Fotografije

 \textit{\textcopyright  
Kopiranje in razmnoževanje besedila ali njegovih delov ter slik je 
dovoljeno samo z odobritvijo avtorjev knjige.} \\% Printing/edition date

 \textsc{Ljubljana, 2020}\\

%-------------------------------------------------------------------
%	TABLE OF CONTENTS
%--------------------------------------------------------------------

\chapterimage{Lit.jpg} % heading image

\pagestyle{empty} % No headers

\renewcommand\contentsname{Kazalo}
\renewcommand{\bibname}{Bibliographie}
\setcounter{tocdepth}{1}
\tableofcontents% Print the table of contents itself

%\cleardoublepage % Forces the first chapter to start on an odd page so it's on 

\pagestyle{fancy} % Print headers again


%---------
%	PREDGOVOR
%-------------------------------------------------------------------------------

\chapterimage{slike/Nebo.jpg} % Chapter heading image

\chapter*{Predgovor}
\vskip2truecm
Fotonika je veda o svetlobi. Obsega izredno široka področja nastanka svetlobe
v svetlobnih virih, predvsem laserjih, razširjanja in modulacije svetlobe,
njenega zaznavanja in uporabe v različne namene. V zadnjih 
desetletjih je fotonika doživela izreden razcvet v raziskavah in 
tehnološki uporabi.

Pričujoča knjiga nudi obširen vpogled v najpomembnejše pojave v fotoniki.
Po kratki ponovitvi splošne optike so opisane lastnosti laserske svetlobe
in pojasnjeno delovanje laserjev. V nadaljevanju obravnavamo prenos
svetlobe po optičnih vlaknih, detekcijo svetlobe, kjer se osredotočimo 
predvsem na danes najbolj razširjene polprevodniške detektorje, in 
modulacijo laserske svetlobe, ki je nujno potrebna za prenos signalov
v modernih optičnih komunikacijah. Knjiga se zaključi s podrobno obravnavo 
nelinearnih optičnih pojavov. 

Knjiga je primerna za vse, ki jih področje fotonike zanima, 
vendar se od bralca pričakuje poznavanje osnov optike in nekaterih 
zahtevnejših matematičnih prijemov. Raziskovalci s področja optike 
-- in tudi študenti fizike, ki si želijo poglobljenega znanja fotonike -- 
bodo v knjigi našli veliko zanimivih pojavov in primerov, ki so podani tako opisno kot 
računsko. Matematično zahtevnejša poglavja so označena z zvezdico.

Za pomoč pri pripravi se najlepše zahvaljujeva prof. dr. Ireni Drevenšek Olenik, ki je 
kot dolgoletna predavateljica predmeta Fotonika na Fakulteti za matematiko Univerze v Ljubljani
znatno pripomogla k oblikovanju te knjige, dr. Andreju Petelinu za sodelovanje pri pripravi ter 
ostalim sodelavcem z Odseka za kompleksne snovi Instituta ``Jožef Stefan'' za vse vsebinske pripombe
in komentarje.

\vspace{1em}

Ljubljana, 2020

\hfill Martin Čopič in Mojca Vilfan


\cleardoublepage
\thispagestyle{empty}
\mbox{}
\cleardoublepage
%-------------------------------------------------------------------------------
%	CHAPTER 1
%-------------------------------------------------------------------------------

\chapterimage{Predgovor.jpg} % Chapter heading image

\chapter{Elektromagnetno valovanje}
Za začetek bomo osvežili osnove teorije elektromagnetnega polja in 
elektromagnetnega valovanja. Obnovili bomo zapis Maxwellovih enačb in 
valovne enačbe, opisali osnovne pojave valovanja (lom, odboj in uklon)
in si na kratko ogledali razširjanje svetlobe v anizotropnih snoveh. 

\section{Maxwellove enačbe}
Elektromagnetno polje v praznem prostoru opišemo z dvema vektorskima
poljema, električnim in magnetnim, ki sta na splošno funkciji lege $\mathbf{r}$
in časa $t$. Vsaki točki v prostoru lahko priredimo \index{Električno polje!jakost}jakost
električnega polja $\mathbf{E}(\mathbf{r},t)$ in \index{Magnetno polje!gostota}gostoto
magnetnega polja $\mathbf{B}(\mathbf{r},t)$. Za opis elektromagnetnega
polja v snovi vpeljemo dve dodatni količini. To sta \index{Električno polje!gostota}gostota
električnega polja $\mathbf{D}(\mathbf{r},t)$ in jakost magnetnega
polja\index{Magnetno polje!jakost} $\mathbf{H}(\mathbf{r},t)$.
Vse te količine povezujejo \index{Maxwellove enačbe}Maxwellove
enačbe\footnote{~Škotski fizik James Clerk Maxwell, 1831--1879.}
\boxeq{eq:Maxwell1}{
\nabla\times\mathbf{H} & =\frac{\partial\mathbf{D}}{\partial t}+\mathbf{j}_e,\\
\nabla\times\mathbf{E} & =-\frac{\partial\mathbf{B}}{\partial t}\label{eq:Maxwell2},\\
\nabla\cdot\mathbf{D} & =\mbox{\ensuremath{\rho}}_{e}\label{eq:Maxwell3} \qquad \mathrm{in}\\
\nabla\cdot\mathbf{B} & =0.\label{eq:Maxwell4}
}\\
Pri zapisu enačb smo upoštevali tudi izvore polj, to je gostoto
električnega toka $\mathbf{j}_e(\mathbf{r},t)$ in gostoto naboja $\rho_{e}(\mathbf{r},t)$, ki 
ju bomo v nadaljevanju izpuščali. Poleg Maxwellovih enačb veljata zvezi
\begin{align}
\mathbf{D} & =\epsilon_{0}\mathbf{E}+\mathbf{P} \qquad \mathrm{in}\\
\mathbf{B} & =\mu_{0}\mathbf{H}+\mu_{0}\mathbf{M},
\end{align}
pri čemer $\mathbf{P}$ označuje \index{Električna polarizacija}električno
polarizacijo, to je gostoto električnih dipolov, in  $\mathbf{M}$
\index{Magnetizacija}magnetizacijo, to je gostoto magnetnega momenta. 
Pri tem sta $\varepsilon_0 = 8,85 \times 10^{-12}~\si{As/Vm}$ influenčna in  
$\mu_0~=~4\pi\times10^{-7}~\si{Vs/Am}$ indukcijska konstanta.

Polarizacija $\mathbf{P}$ in magnetizacija $\mathbf{M}$ sta odvisni od zunanjih polj $\mathbf{E}$
in $\mathbf{B}$. Na splošno sta njuni odvisnosti zelo zapleteni.
V izotropnih in linearnih snoveh\footnote{~Med nelinearne snovi, za katere
napisani zvezi ne veljata, spadajo na primer feroelektriki in feromagneti.}
 se zvezi poenostavita v 
\begin{equation}
\mathbf{P}=\epsilon_{0}\chi_e\mathbf{E} = \epsilon_{0}(\epsilon-1)\mathbf{E} \qquad \textrm{in} 
\qquad
\mathbf{M}=\frac{(\mu-1)}{\mu\mu_0}\mathbf{B} = \chi_m \mathbf{H} = (\mu-1)\mathbf{H}
\label{eq:PM}.
\end{equation}
Vpeljali smo \index{Susceptibilnost!električna} električno $\chi_e$ in 
\index{Susceptibilnost!magnetna}magnetno $\chi_m$ susceptibilnost ter
\index{Dielektričnost}dielektričnost $\epsilon$ in
\index{Magnetna permeabilnost}magnetno permeabilnost $\mu$. Ko združimo zgornje
enačbe, zapišemo dve konstitutivni
relaciji
\begin{equation}
\mathbf{D}  =\epsilon_{0}\epsilon\mathbf{E}\qquad \textrm{in} 
\qquad
\mathbf{H}  =\frac{\mathbf{B}}{\mu_{0}\mu}.
\end{equation}
V linearnih anizotropnih snoveh moramo namesto skalarnih vrednosti $\varepsilon$
in $\mu$ zapisati tenzorje. 

\subsection*{Robni pogoji}
Navedene Maxwellove enačbe opisujejo elektromagnetno polje
v neomejeni snovi, kjer so vse komponente polj zvezne funkcije. Za
obravnavo v omejeni snovi moramo vedeti tudi, kaj se z elektromagnetnim
poljem zgodi pri prehodu iz ene snovi v drugo.\index{Maxwellove enačbe!robni pogoji}
Na meji dveh sredstev se ob odsotnosti površinskih nabojev ohranjata normalni 
komponenti gostote električnega polja in magnetnega polja 
ter tangentni komponenti jakosti
električnega in magnetnega polja (slika~\ref{fig:Robni-pogoji})
\boxeq{eq:robni-pogoji}{
D_{1n} &=  D_{2n}\hspace{1cm} B_{1n} =  B_{2n},\\
E_{1t} &=  E_{2t}\hspace{1.1cm} H_{1t} =  H_{2t}\label{eq:robni-pogoji5}.
}
Na meji z idealnim prevodnikom (kovino) je tangentna komponenta 
jakosti električnega polja enaka nič.

\begin{figure}[ht]
\centering
  \def\svgwidth{85truemm} 
  \input{slike/01_robni_pogoji.pdf_tex}
\caption{Na meji med dvema snovema se v odsotnosti površinskih tokov
in nabojev ohranjata tangentni komponenti $E_t$ in $H_t$ ter 
normalni komponenti $D_n$ in $B_n$.}
\label{fig:Robni-pogoji}
\end{figure}

\section{Valovna enačba in Poyntingov vektor}
Večinoma bomo obravnavali elektromagnetna valovanja v izotropnih, 
homogenih in linearnih snoveh brez zunanjih izvorov polja ($\mathbf{j}_e=0$ in $\rho_{e}=0$). 
Iz Maxwellovih enačb (enačbe~\ref{eq:Maxwell1}--\ref{eq:Maxwell4}) izpeljemo valovni 
enačbi\index{Valovna enačba} za jakost električnega in gostoto magnetnega polja 
\boxeq{eq:valovna-skalarna}{
\nabla^{2}\mathbf{E}-\frac{1}{c^{2}}\frac{\partial^{2}\mathbf{E}}{\partial t^{2}} = 0
}
in
\begin{equation}
\label{eq:valovna-skalarna-B}
\nabla^{2}\mathbf{B}-\frac{1}{c^{2}}\frac{\partial^{2}\mathbf{B}}{\partial t^{2}} = 0.
\end{equation}
Hitrost valovanja\index{Hitrost valovanja} v snovi je enaka 
\boxeq{eq:c}{
c=\frac{1}{\sqrt{\epsilon\epsilon_{0}\mu\mu_{0}}}=\frac{c_{0}}{n},
}
pri čemer je hitrost svetlobe v praznem prostoru $c_0 = 299~792~458~\si{m/s}$. 
Magnetne in dielektrične lastnosti snovi smo pospravili
v lomni količnik $n$\index{Lomni količnik}, ki pove, kolikokrat je hitrost 
svetlobe v snovi $c$ manjša od hitrosti svetlobe v praznem prostoru $c_0$. Velja
\begin{equation}
n=\sqrt{\epsilon\mu}.
\end{equation}
Za izotropno in nemagnetno snov ($\mu=1$) je lomni količnik $n=\sqrt{\epsilon}$.

Vpeljimo še Poyntingov 
vektor\footnote{~Angleški fizik John Henry Poynting, 1852--1914.} 
\index{Poyntingov vektor}{$\mathbf{S}$}
\boxeq{eq:Poyntingov-vektor}{
\mathbf{S} = \mathbf{E} \times \mathbf{H}.
}
Iz lastnosti vektorskega produkta sledi, da je Poyntingov vektor vedno pravokoten na 
smeri $\mathbf{E}$ in $\mathbf{H}$. Gostoto energijskega toka $\mathbf{j}$, to je količino
energije, ki v danem času preteče skozi dano ploskev, izračunamo kot časovno 
povprečje Poyntingovega vektorja \index{Gostota energijskega toka}
\begin{equation}
\mathbf{j}=\left\langle \mathbf{\mathbf{S}}\right\rangle\!.
\label{eq:jscal}
\end{equation}
Gostoto energijskega toka $\mathbf{j}$ imenujemo tudi gostota svetlobnega 
toka\index{{Gostota svetlobnega toka}|see {Gostota energijskega toka}}.
\index{Poyntingov izrek}Poyntingov izrek, ki ga izpeljemo neposredno 
iz Maxwellovih enačb in konstitutivnih relacij, predstavlja izrek o ohranitvi 
energije. Za valovanje v homogeni izotropni snovi tako velja
\begin{equation}
-\nabla\cdot\mathbf{S}=\frac{\partial w}{\partial t},
\end{equation}
pri čemer je $w$\index{Gostota energije} celotna
gostota energije elektromagnetnega polja. Zapišemo jo kot 
\boxeq{eq:gostota-energije}{
w=\frac{1}{2}\mathbf{E}\cdot\mathbf{D}+\frac{1}{2}\mathbf{B}\cdot\mathbf{H}
 = \frac{1}{2}\varepsilon\varepsilon_0\mathbf{E}^2+
 \frac{1}{2}\frac{\mathbf{B}^2}{\mu\mu_0}.
}
Valovno enačbo in ohranitvene zakone lahko zapišemo tudi za anizotropne,
nehomogene ali nelinearne snovi. Nekaj teh primerov bomo srečali v nadaljevanju.

\section{Monokromatski elektromagnetni val}
Reševanje valovne enačbe navadno poenostavimo s kompleksnim
zapisom električnega in magnetnega polja. Račun si
oglejmo na primeru monokromatskega elektromagnetnega vala. Nastavek
za monokromatski val s krožno frekvenco $\omega$ naj bo
\begin{equation}
\mathbf{E}(\mathbf{r},t)  =\mathfrak{\Re}(\mathbf{E}(\mathbf{r})e^{-i\omega t})\qquad \textrm{in} \qquad
\mathbf{B}(\mathbf{r},t)  =\mathfrak{\Re}(\mathbf{B}(\mathbf{r})e^{-i\omega t}),
\label{eq:rval}
\end{equation}
pri čemer sta $\mathbf E(\mathbf{r})$ in $\mathbf B(\mathbf{r})$ časovno
neodvisna vektorja jakosti električnega\index{Električno polje!jakost} in gostote
magnetnega polja\index{Magnetno polje!gostota} s kompleksno
amplitudo. Podobno vpeljemo kompleksne vektorje $\mathbf{P}$,
$\mathbf{M}$, $\mathbf{D}$ in $\mathbf{H},$ ki opisujejo realne količine (polarizacijo,
magnetizacijo, gostoto električnega in jakost magnetnega polja).
Čeprav bomo večinoma uporabljali kompleksni zapis polj, se moramo zavedati, da
je kompleksni zapis zgolj računski pripomoček, na koncu
je treba rezultate izraziti z realnimi količinami. 

Če vstavimo nastavka za monokromatski val (enačbi~\ref{eq:rval}) v valovni enačbi
(enačbi~\ref{eq:valovna-skalarna} in \ref{eq:valovna-skalarna-B}), 
dobimo \index{Helmholtzeva enačba}Helmholtzevi
enačbi\footnote{~Nemški fiziolog in fizik Hermann Ludwig Ferdinand von Helmholtz, 1821--1894.} 
za kompleksna vektorja jakosti električnega in gostote magnetnega polja.
V homogenem in izotropnem sredstvu ju zapišemo kot
\boxeq{eq:Helmholtz}{
\nabla^{2}\mathbf{E}(\mathbf{r})+k^{2}\mathbf{E}(\mathbf{r}) =0 \qquad \mathrm{in} \qquad 
\nabla^{2}\mathbf{B}(\mathbf{r})+k^{2}\mathbf{B}(\mathbf{r}) =0.
}
Vpeljali smo valovno število\index{Valovno število}
\boxeq{eq:k}{
k=\frac{\omega}{c} = \frac{n \omega}{c_0} = n k_0.
}
Povprečni Poyntingov vektor\index{Poyntingov vektor}, povprečen
po eni periodi, je v kompleksnem zapisu enak
\begin{equation}
\langle\mathbf{S}(\mathbf{r})\rangle=\mathbf{E}(\mathbf{r})\times\mathbf{H}^{*}(\mathbf{r})/2.
\label{eq:Poyntingov-vektor-c}
\end{equation}

\section{Ravni val}
Osnovna rešitev valovne enačbe (enačba~\ref{eq:valovna-skalarna}) je ravni 
val\index{Ravni val}. Nastavek, ki predstavlja ravni val in hkrati reši 
Helmholtzevo enačbo (enačba~\ref{eq:Helmholtz}), je oblike
\begin{equation}
\mathbf{E}(\mathbf{r},t) =
\mathbf{E}_{0}e^{i\mathbf{k}\cdot\mathbf{r}-i \omega t}
\end{equation}
in podobno za magnetno polje
\begin{equation}
 \mathbf{B}(\mathbf{r},t) =
\mathbf{B}_{0}e^{i\mathbf{k}\cdot\mathbf{r}-i \omega t},
\end{equation}
pri čemer sta vektorja $\mathbf{E}_{0}$ in $\mathbf{B}_{0}$ od kraja in časa neodvisna. 
Velikost valovnega vektorja\index{Valovni vektor} $\mathbf{k}$ je valovno število $k=nk_{0},$ 
pri čemer je $n$ lomni količnik izotropne in homogene snovi, po kateri potuje ravni val.
Zaradi enolične zveze med električnim in magnetnim poljem 
za opis ravnega vala zadošča le eno polje, navadno se odločimo za električno.

Iz Maxwellovih enačb (enačbe~\ref{eq:Maxwell1}--\ref{eq:Maxwell4}) sledijo zveze o ortogonalnosti
količin električnega in magnetnega polja v elektromagnetnem valu. Vedno sta med seboj pravokotna vektorja
jakosti električnega $\mathbf{E}$ in magnetnega polja $\mathbf{H}$, ki sta v izotropni snovi
hkrati pravokotna na valovni vektor $\mathbf{k}$ (naloga~\ref{naloga-TEM-ortogonalnost}).
Po definiciji sta $\mathbf{E}$ in $\mathbf{H}$ vedno pravokotna tudi na Poyntingov vektor $\mathbf{S}$ 
(enačba~\ref{eq:Poyntingov-vektor}), zato sta v izotropni snovi
Poyntingov vektor in smer energijskega toka vzporedna valovnemu vektorju.\footnote{~Na splošno 
velja $\mathbf{E}\perp\mathbf{S}$ (enačba~\ref{eq:Poyntingov-vektor}) in 
$\mathbf{D}\perp\mathbf{k}$ (naloga~\ref{naloga-TEM-ortogonalnost}). 
To velja vedno, tudi v anizotropnih snoveh, vendar tam $\mathbf{E} \nparallel 
\mathbf{D}$ in $\mathbf k\nparallel\mathbf S$. Za podrobnejši opis glej 
razdelek~\ref{chap:anizotropni}.} 
\vglue2truemm
\begin{definition}
\label{naloga-TEM-ortogonalnost}
Iz Maxwellovih enačb (enačbe~\ref{eq:Maxwell1}--\ref{eq:Maxwell4}) za ravni val izpelji zvezi
\begin{align}
\mathbf{k}\times\mathbf{H}_{0} & =-\omega\epsilon\epsilon_{0}\mathbf{E}_{0}\label{eq:TEM-pogoj1}\
\qquad \mathrm{in}\\
\mathbf{k}\times\mathbf{E}_{0} & =\omega\mu\mu_{0}\mathbf{H}_{0}\label{eq:TEM-pogoj2},
\end{align}
iz katerih izhaja, da je v izotropni snovi $\mathbf{E}_0\perp \mathbf{H}_0\perp 
\mathbf{k}$.
Pokaži tudi, da za ravni val vedno velja zveza
\begin{equation}
\mathbf{D}_0 \perp \mathbf{B}_0 \perp \mathbf{k}.  
\end{equation}
\end{definition}

Zapišimo še povprečno gostoto energije valovanja\index{Gostota energije} (enačba~\ref{eq:gostota-energije}). 
K energiji prispevata tako magnetno kot električno polje. Prispevka sta enaka, zato velja
\begin{equation}
\left\langle w\right\rangle =\frac{1}{4}\epsilon\epsilon_{0}\left|E_{0}\right|^{2}+
\frac{1}{4}\frac{\left|B_{0}\right|^{2}}{\mu\mu_{0}}=\frac{1}{2}\epsilon\epsilon_{0}\left|E_{0}\right|^{2}\!.
\end{equation}
Povprečna gostota energije $w$, pomnožena s hitrostjo svetlobe v
snovi $c$, je enaka velikosti gostote energijskega toka $j$\index{Gostota energijskega toka}
\boxeq{eq:jcw}{
j=cw = \frac{1}{2}c\epsilon\epsilon_{0}\left|E_{0}\right|^{2}=\frac{1}{2}c_{0}n\epsilon_{0}
\left|E_{0}\right|^{2}\!.
}
Prva enakost nazorno kaže, da je gostota svetlobnega toka pravzaprav pretok
energije. To si lahko najlepše predstavljamo, če obravnavamo valj s prečnim presekom
$S$ in dolžino $c\Delta t$. V volumnu $Sc\Delta t$ je shranjene $wSc\Delta t$
energije. Energija, ki preteče skozi presek $S$ v času $\Delta t$,
je ravno $cw$. 

Gostota svetlobnega toka je torej sorazmerna
s kvadratom amplitude jakosti električnega polja. Poglejmo dva primera.
Gostoti toka $j=1~\si{kW/m^{2}}$
(približna gostota svetlobnega toka s Sonca na Zemljinem površju) v praznem prostoru ustreza 
jakost električnega polja $E_{0}=868~\si{\volt/\meter}$, gostoti $j=1~\si{W/\micro m^{2}}$ 
(močno zbran laserski snop) pa jakost polja $E_{0}=27~\si{MV/m}$. 

Pogosto vpeljemo tudi intenziteto valovanja\index{Intenziteta} $I= |E|^2$. Uporabljamo jo 
predvsem takrat, kadar gostoto svetlobnega toka primerjamo z neko referenčno vrednostjo, 
saj je razmerje gostot energijskega toka v isti snovi kar enako razmerju intenzitet.

Gostota svetlobnega toka ravnega vala je neodvisna od kraja in časa, iz česar sledi,
da je povsod po prostoru enaka. Če bi želeli izračunati energijo,
ki jo nosi ravni val, bi opazili, da je ta energija neskončna. To
seveda ni mogoče, zato se je vedno treba zavedati, da je ravni val
le idealiziran, a nazoren in praktičen približek elektromagnetnega
vala.

\section{Polarizacija elektromagnetnega valovanja}
Jakost električnega polja elektromagnetnega valovanja v izotropnem
sredstvu leži v ravnini, ki je pravokotna na valovni vektor $\mathbf{k}$. 
Smer vektorja $\mathbf{E}_0$ v tej ravnini opiše
polarizacija\index{Polarizacija}. 

Električno polje ravnega vala v ravnini razstavimo na dve medsebojno 
pravokotni komponenti vektorja $\mathbf{E}_0$, ki
nihata sinusno z enako frekvenco, a se lahko razlikujeta v amplitudi in fazi. 
Na splošno je ravni val eliptično polariziran\index{Polarizacija!eliptična} in
vrh vektorja $\mathbf E_0$ v ravnini, ki je pravokotna 
na smer širjenja, orisuje elipso. Kadar je elipsa izrojena v daljico,
govorimo o linearno polariziranem valu\index{Polarizacija!linearna},
kadar je krog, govorimo o krožno (ali cirkularno) polariziranem valu\index{Polarizacija!krožna}. 
Poljubno polarizacijo lahko vedno zapišemo kot vsoto dveh linearno ali dveh 
krožno polariziranih valovanj. 

Priročen zapis polarizacije je s kompleksnim \index{Jonesov vektor}Jonesovim 
vektorjem\footnote{~Ameriški fizik Robert Clark Jones, 1916--2004.}
$\mathbf{J}$. Za monokromatski ravni val, ki se širi v smeri $z$ in ima 
komponenti $E_x$ in $E_y$, je Jonesov vektor
\begin{equation}
\mathbf{J}=\frac{1}{|E_{0}|}\left[\begin{array}{c}
E_{x}\\
E_{y}
\end{array}\right]\!.
\end{equation}
Dodali smo normalizacijski faktor $|E_{0}|=\sqrt{|E_{x}|^{2}+|E_{y}|^{2}}$.
Ravni val, linearno polariziran v smeri $x$, tako zapišemo kot $\mathbf{J}=\left(1,0\right)$,
val, ki je linearno polariziran pod kotom $\ang{45}$ glede na osi
$x$ in $y$, pa je $\mathbf{J}=\left(1,1\right)/\sqrt{2}$.
Za zapis krožno polariziranega valovanja ni enotnega dogovora. Tukaj pišemo
desno krožno polarizirano valovanje kot 
$\mathbf{J}=\left(1,-i\right)/\sqrt{2}$ in
levo krožno polarizirano valovanje kot $\mathbf{J}=\left(1,i\right)/\sqrt{2}$.
V našem zapisu je desno polariziran tisti val, pri katerem se jakost električnega
polja na danem mestu vrti v desno, če gledamo proti izvoru valovanja.\footnote{~Glej 
npr. G. R. Fowles, {\it Introduction to Modern Optics}, druga izdaja, Dover Publications (1975).}

Zapis z Jonesovimi vektorji je prikladen, saj omogoča preprost izračun
prehoda ravnega vala skozi optične elemente, ki spreminjajo polarizacijo,
a ohranjajo njegovo obliko. Naj bo pred prehodom skozi optični element kompleksna
amplituda $\mathbf{E}_1$ in po prehodu $\mathbf{E}_2$. Spremembo amplitude 
jakosti električnega polja lahko zapišemo v matrični obliki
\begin{equation}
\mathbf{E}_{2}=A\cdot\mathbf{E}_{1},
\end{equation}
pri čemer je $A$ Jonesova\index{Jonesova 
matrika} matrika, katere komponente so odvisne od
lastnosti elementa, skozi katerega prehaja ravni val. 
Z uporabo Jonesovih vektorjev je zapis oblike $\mathbf{J}_{2}=A\cdot\mathbf{J}_{1}$. Pri tem
$\mathbf{J}_{1}$ in $\mathbf{J}_{2}$ opisujeta polarizaciji vstopnega in izstopnega vala. 

Poglejmo nekaj primerov. Jonesova matrika za prehod skozi linearni polarizator, ki
polarizira v smeri $x$, je
\begin{equation}
A=\left[\begin{array}{cc}
1 & 0\\
0 & 0
\end{array}\right]\!.
\end{equation}
Za polarizator, zasukan pod kotom $45\si{\degree}$ v pozitivni 
smeri glede na os $x$, je Jonesova matrika
\begin{equation}
A=\frac{1}{2}\left[\begin{array}{cc}
1 & 1\\
1 & 1
\end{array}\right]\!.
\end{equation}
Jonesova matrika za optični element, ki eni komponenti doda fazni zamik $\pi$, je
\begin{equation}
A=\left[\begin{array}{cc}
1 & 0\\
0 & -1
\end{array}\right].
\end{equation}
Tak element imenujemo ploščica $\lambda/2$\index{Ploščica $\lambda/2$} in 
spremeni desno krožno polariziran val v levo krožno polariziranega in obratno, 
linearno polariziran val pa prezrcali čez koordinatno os. 

Podobno je Jonesova matrika za element, ki eni komponenti doda fazni zamik $\pi/2$ 
(imenujemo ga ploščica $\lambda/4$\index{Ploščica $\lambda/4$}), enaka
\begin{equation}
A=\left[\begin{array}{cc}
1 & 0\\
0 & i
\end{array}\right]\!.
\end{equation}
Ploščica $\lambda/4$ linearno polarizirano valovanje z Jonesovim vektorjem $(1,1)/\sqrt{2}$
spremeni v levo krožno polarizirano valovanje in krožno polarizirano
valovanje nazaj v linearno. 

\begin{definition}
Pokaži, da je Jonesova matrika za polarizator,
ki prepušča polarizacijo pod kotom $\vartheta$ glede na os $x$, podana z matriko
\begin{equation}
A=\left[\begin{array}{cc}
\cos^{2}\vartheta & \sin\vartheta\cos\vartheta\\
\sin\vartheta\cos\vartheta & \sin^{2}\vartheta
\end{array}\right]\!.
\end{equation}
Namig: matriko $A'$, ki opisuje polarizator v smeri $x$, zapiši v zasukanem
koordinatnem sistemu $A=R(\vartheta) \cdot {A'}\cdot R(\vartheta)^\textrm{T}$, 
pri čemer je $R(\vartheta)$ rotacijska matrika.
\end{definition}

\section{Lom in odboj EM valovanja}
Ko svetloba vpada na ravno mejo med dvema izotropnima dielektrikoma, se je 
del odbije po odbojnem zakonu, ki pravi, da je odbojni kot enak vpadnemu, 
preostali del svetlobe pa se lomi. Zanj velja lomni \index{Lomni zakon}zakon
\boxeq{eq:lomni_zakon}{
n_{1}\sin\vartheta_{1}=n_{2}\sin\vartheta_{2}.
}
Z $n_{1}$ in $n_{2}$ smo označili lomna količnika prve in druge snovi ter
s kotoma $\vartheta_{1}$ in $\vartheta_{2}$ vpadni in lomni
kot glede na normalo na mejno ploskev (slika~\ref{fig:Lom}).
Poglejmo, kaj se pri lomu in odboju zgodi s polarizacijo valovanja.

Dogovorimo se, da valovanje, pri katerem je jakost električnega polja pravokotna 
na vpadno ravnino, imenujemo transverzalno električno (TE) 
valovanje\index{Polarizacija!TE}. Kadar leži jakost električnega polja v
vpadni ravnini in je nanjo pravokotna jakost magnetnega polja,
govorimo o transverzalnem magnetnem valovanju (TM)\index{Polarizacija!TM}.
\begin{figure}[ht]
\centering \def\svgwidth{130truemm} 
  \input{slike/01_lom.pdf_tex}
\caption{Lom elektromagnetnega valovanja na meji dveh izotropnih dielektrikov za (a)
trans\-ver\-zal\-no električno (TE) valovanje in (b) trans\-ver\-zal\-no magnetno (TM) valovanje}
\label{fig:Lom}
\end{figure}

Z $E_1$ označimo amplitudo jakosti električnega polja vpadnega valovanja, 
z $E_2$ prepuščenega in z $E_3$ odbitega.
Nato vpeljemo amplitudno prepustnost $t$ in amplitudno odbojnost $r$, 
ki sta odvisni od vpadne polarizacije. Zapišemo
\begin{equation}
E_{2\mathrm{TE}} =t_{\mathrm{TE}}E_{1\mathrm{TE}} \qquad \mathrm{in} \qquad 
E_{3\mathrm{TE}} =r_{\mathrm{TE}}E_{1\mathrm{TE}}
\end{equation}
in podobno za TM polarizirano valovanje. Koeficiente $r$ in $t$ izračunamo iz robnih pogojev (enačbe~\ref{eq:robni-pogoji} in \ref{eq:robni-pogoji5}). 
Enačbe, ki opisujejo odvisnost amplitudne odbojnosti
in amplitudne prepustnosti od vpadnega kota za različni vpadni polarizaciji, imenujemo
\index{Fresnelove enačbe}Fresnelove enačbe\footnote{~Francoski fizik in inženir 
Augustin Jean Fresnel, 1788--1827.}. Za TE polarizacijo velja
\begin{equation}
r_{\mathrm{TE}}=\frac{n_{1}\cos\vartheta_{1}-n_{2}\cos\vartheta_{2}}{n_{1}\cos\vartheta_{1}+
n_{2}\cos\vartheta_{2}} = -\frac{\sin(\vartheta_1-\vartheta_2)}{\sin(\vartheta_1+\vartheta_2)}
\label{eq:Fresnel1}
\end{equation}
in
\begin{equation}
t_{\mathrm{TE}}=1+r_{\mathrm{TE}}=\frac{2n_{1}\cos\vartheta_{1}}{n_{1}\cos\vartheta_{1}+
n_{2}\cos\vartheta_{2}}.
\end{equation}
Amplitudna odbojnost in amplitudna prepustnost za TM polarizacijo sta
\begin{equation}
r_{\mathrm{TM}}=\frac{n_{2}\cos\vartheta_{1}-n_{1}\cos\vartheta_{2}}{n_{2}\cos\vartheta_{1}+n_{1}\cos\vartheta_{2}} = \frac{\tan(\vartheta_1-\vartheta_2)}{\tan(\vartheta_1+\vartheta_2)}
\label{eq:Fresnel2}
\end{equation}
in
\begin{equation}
t_{\mathrm{TM}}=(1-r_{\mathrm{TM}})\frac{\cos\vartheta_{1}}{\cos\vartheta_{2}}=
\frac{2n_{1}\cos\vartheta_{1}}
{n_{1}\cos\vartheta_{2}+n_{2}\cos\vartheta_{1}}.
\label{eq:Fresnel2b}
\end{equation}

Na splošno sta amplitudna odbojnost $r$ in amplitudna prepustnost $t$ kompleksni
količini, saj iz lomnega zakona sledi, da je $\cos\vartheta_{2}=
\sqrt{1-\left(n_{1}/n_{2}\right)^{2}\sin^{2}\vartheta_{1}}$
lahko kompleksen. Velikost števila $\left|r\right|$ tako predstavlja
odbojnost in argument $\arg(r)$ spremembo faze
pri odboju.

\begin{definition}
Izpelji Fresnelove enačbe (enačbe~\ref{eq:Fresnel1}--\ref{eq:Fresnel2b}).
\end{definition}

Amplitudna odbojnost $r$ in amplitudna prepustnost $t$ povesta, kako se spremeni 
kompleksna amplituda jakosti električnega polja pri odboju oziroma lomu.
Razmerje med gostoto energijskega toka odbite in vpadne svetlobe $\mathcal{R}$ oziroma 
prepuščene in vpadne svetlobe $\mathcal{T}$ izračunamo kot 
\begin{equation}
\mathcal{R}=\left|r\right|^{2} \qquad \mathrm{in} \qquad \mathcal{T}=1-\mathcal{R}.
\end{equation}
Slednja enačba sledi iz ohranitve energije. Na splošno $\mathcal{T}$
ni enak $\left|t\right|^{2}$, saj energijski tok potuje po različnih
snoveh in v različnih smereh. Velja zveza
\begin{equation}
\mathcal{T}=\frac{n_{2}\cos\vartheta_{2}}{n_{1}\cos\vartheta_{1}}\left|t\right|^{2}\!.
\end{equation}
Primer uporabe Fresnelovih enačb je pravokotni vpad svetlobe na mejo dveh sredstev. 
Zaradi simetrije sta v tem primeru odbojnost in prepustnost neodvisni od polarizacije. Sledi
\begin{equation}
r_{\mathrm{TE}} = -r_{\mathrm{TM}} = \frac{n_1-n_2}{n_1+n_2}
\qquad \mathrm{in} \qquad 
t_{\mathrm{TE}} = t_{\mathrm{TM}} = \frac{2n_1}{n_1+n_2}. 
\end{equation}
Ob pravokotnem vpadu svetlobe iz zraka na steklo ($n_1 = 1$ in $n_2 \approx 1,5$) je tako
$\mathcal{R} \approx 0,04$.

\begin{remark}
Pri prehodu skozi optične elemente se vedno nekaj
svetlobe odbije. Za zmanjšanje teh izgub optične elemente
prekrijemo z antirefleksno plastjo, to je nanosom ene ali več primerno
debelih plasti dielektrikov z ustreznimi lomnimi količniki.
Zaradi destruktivne interference valovanj, odbitih na posameznih plasteh,
se količina odbite svetlobe z izbrano valovno dolžino občutno zmanjša. Ker so laserji
izvori svetlobe s točno določeno valovno dolžino, za zmanjšanje
izgub, na primer v resonatorju laserja, uporabljamo optične
elemente (leče, kristale, modulatorje) z ustrezno antirefleksno plastjo.\footnote{~Glej npr. 
E. Hecht, {\it Optics}, peta izdaja, Pearson Education Limited (2017).}
\end{remark}

Poglejmo še odvisnost odbojnosti in prepustnosti od vpadnega kota (slika~\ref{fig:Brewster}). 
Pri tem je pomembno, ali se svetloba lomi v optično gostejše ($n_1<n_2$) ali v optično
redkejše sredstvo ($n_1>n_2$). 

Najprej obravnavajmo primer loma v optično gostejšo snov 
(sliki~\ref{fig:Brewster}\,a in c). Vidimo, da je pri neki vrednosti vpadnega kota $\vartheta_1$ odbojnost za 
TM polarizirano valovanje enaka nič. Ta kot imenujemo \index{Brewstrov kot}Brewstrov 
kot\footnote{~Škotski fizik in znanstvenik Sir David Brewster, 1781--1868.}. 
Pri Brewstrovem vpadnem kotu je vsa vpadna TM polarizirana svetloba prepuščena
in $\mathcal{T}_\mathrm{TM}(\vartheta_B)=1$. Posledično je odbito valovanje vedno TE
polarizirano. 
\begin{figure}[ht]
\centering
  \def\svgwidth{135truemm} 
  \input{slike/01_BrewsterNew.pdf_tex}
\caption{Amplitudna odbojnost $r$ za obe vpadni polarizaciji (a, b) in razmerje med 
gostoto energijskega toka odbite in vpadne svetlobe $\mathcal{R}$ za obe polarizaciji (c, d)
v odvisnosti od vpadnega kota $\vartheta_1$. Za primer na slikah (a) in (c) velja $n_1<n_2$, za primer na 
slikah (b) in (d) pa $n_1>n_2$. Z zeleno je označen Brewstrov kot $\vartheta_B$ in
z vijolično mejni kot totalnega odboja $\vartheta_T$.}
\label{fig:Brewster}
\vglue-5truemm
\end{figure}

\begin{definition}
Pokaži, da za Brewstrov kot velja 
$\vartheta_{B}=\arctan\left(\frac{n_2}{n_1}\right)\!.$
\label{eq:Brew}
\end{definition}
\vglue-3truemm
\begin{remark}
Prozorne ploščice, ki so postavljene pod Brewstrovim kotom glede na smer vpadne svetlobe, 
imenujemo Brewstrova okna\index{Brewstrovo okno}. Njihova značilnost je,
da TM polarizacijo v celoti prepustijo, TE polarizacije pa se del odbije in  
del prepusti. Brewstrova okna so zelo uporabna pri izdelavi resonatorjev 
plinskih laserjev, saj so izgube za TM polarizacijo zelo majhne, 
za TE pa razmeroma velike.  
\end{remark}

Negativni predznak amplitudne odbojnosti $r_{\mathrm{TE}}$ pomeni, da ima odbiti TE 
polarizirani val pri vpadu na optično gostejše sredstvo nasprotno 
fazo od vpadnega. Za TM polarizirani val je faza pri vpadnih kotih, manjših od Brewstrovega, 
enaka, pri večjih vpadnih kotih pa ima odbita svetloba nasprotno fazo kot vpadna. 

Pri vpadu na optično redkejše sredstvo (sliki~\ref{fig:Brewster}\,b in d) je poleg Brewstrovega kota
pomemben še en kot, to je mejni kot totalnega odboja
\boxeq{eq:totalniodbojkot}{
\vartheta_T = \arcsin\left(\frac{n_2}{n_1}\right)\!. 
}
Pri vpadnih kotih, ki so večji od $\vartheta_T$, se svetloba v celoti odbije 
$\mathcal{R}=1$. Takrat govorimo
o totalnem ali popolnem odboju\index{Totalni odboj}. 

Vendar tudi v primeru totalnega odboja jakost električnega polja v optično redkejšem sredstvu
ni enaka nič, saj se tam pojavi evanescentno polje\index{Evanescentno polje}.
To je polje, ki se širi v smeri mejne ravnine, njegova amplituda pa pojema 
eksponentno z oddaljenostjo od nje. Vdorna globina je odvisna od valovne 
dolžine valovanja, lomnega količnika snovi in tudi od vpadnega kota  (glej 
nalogo~\ref{naloga:evp}). Čeprav se v optično redkejši snovi pojavi električno polje, je Poyntingov
vektor v smeri pravokotno na mejno ploskev v povprečju enak nič, zato se energija
v drugo snov pri totalnem odboju ne prenaša.
\begin{definition}
\label{naloga:evp}
Pokaži, da je vdorna globina evanescentnega polja enaka
\begin{equation}
d = \frac{\lambda_0}{2 \pi\sqrt{n_1^2 \sin^2 \vartheta_1 - n_2^2}},
\end{equation}
pri čemer je $\lambda_0$ v praznem prostoru, 
kot $\vartheta_1$ je vpadni kot in $n_1>n_2$.
Pokaži še, da je povprečje Poyntingovega vektorja v smeri pravokotno na mejno 
ploskev enako nič. 
\end{definition}

\section{Uklon svetlobe}\index{Uklon}
Kadar svetloba vpada na oviro, za oviro nastane senca. Nastala senca ni ostra, ampak
ima zaradi uklona zabrisane robove. Obravnave 
uklona svetlobe na odprtinah ali zaslonkah se lotimo z uporabo
skalarnega približka teorije elektromagnetnega polja. To pomeni, da vpliva 
polarizacije ne upoštevamo. Ta je pomemben zgolj pri zelo majhnih odprtinah, 
kjer je velikost odprtine $a$ po velikosti podobna valovni dolžini svetlobe $a \sim \lambda$. 
Vendar so tudi v tem primeru uklonske slike za različne polarizacije podobne, 
razlikujejo se predvsem po jakosti uklonjene svetlobe.
\begin{remark}
Primer, kjer skalarni približek ne da pravih rezultatov, je uklon na mrežici, narejeni 
iz zelo tankih prevodnih žic. Takšna mrežica deluje kot polarizator za vpadno elektromagnetno valovanje.
Elektromagnetni val s polarizacijo, ki je vzporedna žicam, pri prečkanju v žicah inducira tok
in val se delno odbije in delno absorbira. Za valovanje, ki je polarizirano pravokotno 
na žice, je  inducirani tok bistveno manjši, saj je tok omejen na smer
vzdolž žice. Posledično je val, polariziran pravokotno na žice, prepuščen, 
val, polariziran vzporedno z žicami, pa ne. 
Takšni polarizatorji se uporabljajo večinoma v mikrovalovni tehniki, 
vendar se v zadnjih letih z razvojem in izboljšavo litografskih postopkov
vse pogosteje uporabljajo tudi v bližnjem infrardečem delu spektra.\index{Infrardeče valovanje}
\end{remark}

Pri velikostih odprtin $a\gg\lambda$ tako uporabimo skalarno obliko valovne enačbe 
(enačba~\ref{eq:valovna-skalarna})
\begin{equation}
\nabla^2 E - \frac{1}{c^2}\frac{\partial^2 E}{\partial t^2} = 0.
\label{eq:skalarna-valovna-enačba}
\end{equation}
Časovna odvisnost polja $E$ je harmonična funkcija in 
je sorazmerna z $e^{-i \omega t}$. Z uporabo Greeno\-ve\-ga izreka lahko
jakost polja $E_P$ v točki prostora $P$ izrazimo s poljem na 
poljubni sklenjeni ploskvi $S$, ki to točko obkroža.\footnote{~Glej npr. 
E. Hecht, {\it Optics}, peta izdaja, Pearson Education Limited (2017).} 

Zvezo opisuje Kirchhoffov integral\footnote{~Nemški fizik Gustav Robert Kirchhoff, 1824--1887.} 
\index{Kirchhoffov integral}
\begin{equation}
E_P = -\frac{1}{4\pi}\oint \left(E\,\mathbf{n}\cdot \nabla \frac{e^{ikr}}{r}-
\frac{e^{ikr}}{r}\mathbf{n}\cdot \nabla E \right) dS,
\label{eq:Kirchhoffov-integral}
\end{equation}
pri čemer je $\mathbf{n}$ normala na ploskev, po kateri teče integral, $r$ pa oddaljenost od P
do dela sklenjene ploskve $dS$ (slika \ref{fig:UklonFK}). 
\begin{figure}[ht]
\centering \def\svgwidth{80truemm} 
  \input{slike/01_uklonFK.pdf_tex}
\caption{Integracijsko ploskev $S$ v Kirchhoffovem integralu izberemo tako, da zajema odprtino 
in objema točko $P$.}
\label{fig:UklonFK}
\end{figure}

Naj svetloba iz točkastega izvora v točki $T$ vpada na zaslon
z odprtino poljubne oblike. Izračunajmo skalarno polje v točki $P$ na drugi 
strani zaslona. Vpadna svetloba naj bo
\begin{equation}
\label{eq:polje-krogelni-val}
E_T = A \frac{e^{ikr'}}{r'},
\end{equation}
pri čemer je $r'$ razdalja od izvora do točke na zaslonu, A pa zaradi ohranitve energije konstanta.

Integracijska ploskev je poljubna sklenjena ploskev, ki objema točko $P$. 
Izberemo ploskev, ki zajema odprtino na zaslonu, in naredimo še dva približka. Privzamemo, da
jakost polja $E$ in njen gradient doprineseta k integralu le na odprtini, na preostanku ploskve
pa sta njuna prispevka zanemarljivo majhna. Privzamemo tudi, da sta vrednost $E$ in njen gradient 
na odprtini takšna, kot da zaslona ne bi bilo.
Približka sta precej groba, vendar se izkaže, da se kljub temu
dobro ujemata z eksperimentalno uklonsko sliko, s čimer 
upravičimo njuno uporabo.

Kirchhoffov integral za točkast izvor svetlobe se ob omenjenih približkih 
zapiše kot integral po odprtini
\begin{equation}
E_P = -\frac{ik A e^{-i\omega t}}{4\pi}\int\frac{e^{ik(r+r')}}{rr'}\left(\cos(\mathbf{n},
\mathbf{r})-\cos(\mathbf{n},\mathbf{r'})\right) dS.
\label{eq:Fresnel-Kirchoffov-integral}
\end{equation}
Imenujemo ga Fresnel-Kirchhoffov uklonski integral\index{Fresnel-Kirchhoffov integral}.
\vglue4truemm
\begin{definition}
\label{naloga-Fresnel-Kirchhoff-uklon}
Uporabi Kirchhoffov integral (enačba~\ref{eq:Kirchhoffov-integral}) in pokaži, da 
za primer krožnega vpadnega vala (enačba~\ref{eq:polje-krogelni-val}) polje v točki 
$P$ zapišemo s Fresnel-Kirchhoffovim integralom (enačba~\ref{eq:Fresnel-Kirchoffov-integral}). 
Pri tem privzemi, da je oddaljenost točke $P$ od odprtine $r \gg \lambda$.
\end{definition}
\vglue4truemm
Oglejmo si poseben primer, ko leži točkast izvor svetlobe na osi okrogle odprtine. Polje 
v točki $P$ potem izračunamo kot 
\begin{equation}
\label{eq:Fresnelov-uklon}
E_P =  -\frac{ik}{4\pi} \int E_T\frac{ e^{ikr-i\omega t}}{r}\left(\cos(\mathbf{n},\mathbf{r})+1\right) dS.
\end{equation}
Pri tem $E_T$ predstavlja kompleksno amplitudo vpadnega polja v odprtini (enačba~\ref{eq:polje-krogelni-val}). 
Zapisana oblika Fresnel-Kirchhoffovega integrala ni pravzaprav nič drugega kot 
matematični zapis \index{Huygensovo načelo}Huygensovega 
načela\footnote{~Nizozemski znanstvenik Christiaan Huygens, 1629--1695.}. 
Spomnimo se, da Huygensovo načelo pravi, da lahko vsako točko valovne fronte obravnavamo 
kot izvor novega krogelnega vala. Točno to je zapisano tudi v enačbi~(\ref{eq:Fresnelov-uklon}). 
Vpadni val
$E_T$ v vsakem od elementov odprtine $dS$ vzbudi krogelno valovanje s
kompleksno amplitudo
\begin{equation}
E = A_0 \frac{e^{ikr}}{r},
\end{equation} 
polje v izbrani točki $P$ pa je vsota prispevkov posameznih krogelnih valovanj.
Za razliko od osnovnega Huygensovega načela v Fresnel-Kirchhoffovem integralu 
(enačba~\ref{eq:Fresnelov-uklon})
nastopa še faktor $\left(\cos(\mathbf{n},\mathbf{r})+1\right)$, ki poskrbi, da ni valovanja 
v smeri nazaj proti izvoru. Faktor $-i$, ki prav tako manjka v osnovnem Huygensovem načelu,
pomeni, da je uklonjeno valovanje fazno zakasnjeno za $\pi/2$ glede na osnovno
valovanje $E_T$.

Fresnel-Kirchhoffov uklonski integral uporabno razširimo z dodatkom prepustnostne funkcije odprtine $T$.
Z njo na splošno popišemo amplitudne in fazne spremembe, do katerih pride na raznih 
odprtinah, lečah, uklonskih mrežicah ... Razširjeni uklonski integral zapišemo kot
\boxeq{eq:Fresnelov-uklon2}{
E_P =  -\frac{ik}{4\pi} \int T(r') E_T(r') \frac{ e^{ikr-i\omega t}}{r}
\left(\cos(\mathbf{n},\mathbf{r})+1\right) dS,
}
pri čemer smo upoštevali tudi splošno obliko vpadnega vala $E_T(r')$.

\subsection*{Fraunhoferjev in Fresnelov približek}
\label{FFuklon}
Izračun Fresnel-Kirchhoffovega uklonskega integrala (enačba~\ref{eq:Fresnel-Kirchoffov-integral}) 
je na splošno zelo zapleten, zato se 
pogosto poslužujemo dveh približkov: Fraunhoferjevega\footnote{~Nemški fizik 
Joseph von Fraunhofer, 1787--1826.} in Fresnelovega\index{Fresnelov uklon}. 
\begin{figure}[ht]
\centering \def\svgwidth{100truemm} 
  \input{slike/01_uklon_koordinate.pdf_tex}
\caption{K izračunu Fraunhoferjevega in Fresnelovega uklonskega približka}
\label{fig:Uklon-koordinate}
\end{figure}

Izhajamo iz Fresnel-Kirchhoffovega integrala za točkast izvor
(enačba~\ref{eq:Fresnel-Kirchoffov-integral}). Lego točke $P$ na zaslonu zapišemo 
s koordinatami $x,y$ in $z$. Razdaljo $r$, ki je razdalja med točko v odprtini
in točko $P$, izrazimo s koordinatami točke $P$ in koordinatama v odprtini $x'$ in $y'$ 
(slika \ref{fig:Uklon-koordinate}). 
 
Privzamemo,
da je oddaljenost zaslona $z$ bistveno večja od prečnih dimenzij $x$ in $y$. 
Zapišemo razdaljo $r$ in jo razvijemo
\begin{equation}
r = \sqrt{(x-x')^2+(y-y')^2 + z^2} \approx z + \frac{(x-x')^2}{2z} +\frac{(y-y')^2}{2z}.
\label{eq:razvojuklon}
\end{equation}
Vstavimo razvoj (enačba~\ref{eq:razvojuklon}) v uklonski integral (enačba~\ref{eq:Fresnelov-uklon}), 
pri čemer $r$ v imenovalcu nadomestimo kar z $z$. Pridemo do Fresnelovega uklonskega približka 
\begin{equation}
\label{eq:FresnelApprox}
E_P(x,y,z) =  \frac{1}{i \lambda z } e^{i k z}\int \int E_T\, e^{ik ((x-x')^2+(y-y')^2)/2z} dx' dy'.
\end{equation}
Pri zapisu smo upoštevali, da sta $\mathbf{n}$ in $\mathbf{r}$ skoraj vzporedna, zato smo faktor
$\left(\cos(\mathbf{n},\mathbf{r})+1\right)$ nadomestili z 2.

Kadar je oddaljenost zaslona dovolj velika oziroma so prečne dimenzije dovolj majhne, da zadošča 
razvoj do linearnih členov, govorimo o Fraunhoferjevem uklonskem približku in uklonski integral je
\begin{equation}
\label{eq:FraunhoferApprox}
E_P(x,y,z) =  \frac{1}{i\lambda z} e^{i k (z + (x^2+y^2) /2z)}\int \int E_T\,
e^{-ik (xx'+yy')/z} dx' dy'.
\end{equation}
V njem  prepoznamo Fourierevo transformacijo polja v odprtini $E_T$.
Fraunhoferjeva uklonska slika\index{Fraunhoferjev uklon} velja za razmeroma velike oddaljenosti
zaslona od uklonske odprtine, ko lahko uklonjeni val dovolj dobro opišemo z ravnim valom
(slika~\ref{fig:UklonFF}\,a). 

Bolj zapleteno Fresnelovo uklonsko sliko moramo uporabiti, kadar obravnavamo 
primer bližnjega polja (slika~\ref{fig:UklonFF}\,b). 
Mejo med Fraunhoferjevim in Fresnelovim režimom kvalitativno določa Fresnelovo\index{Fresnelovo število}
število\footnote{~Glej npr. B. E. A. Saleh in M. C. Teich, 
{\it Fundamentals of Photonics}, druga izdaja, John Wiley \& Sons, Inc. (2007).}
\begin{equation}
F= \frac{a^2}{L\lambda}.
\label{eq:Fst}
\end{equation} 
Pri tem so $a$ karakteristična dimenzija odprtine, $L$ oddaljenost zaslona 
od odprtine in $\lambda$ valovna dolžina svetlobe. V grobem velja, da lahko 
Fraunhoferjev približek uporabimo, kadar je $F<1$ in je odstopanje faze od ravnega vala 
znotraj odprtine majhno. Sicer moramo uklon obravnavati v Fresnelovem približku ali 
celo v polni obliki. 

\begin{figure}[ht]
 \centering \def\svgwidth{110truemm} 
  \input{slike/01_uklonFF2.pdf_tex}
\caption{Značilna uklonska slika odprtine v Fraunhoferjevem (a) in Fresnelovem režimu (b)}
\label{fig:UklonFF}
\end{figure}

\begin{definition}
\label{naloga-Frauhofer-Kirchhoff-uklon_reza}
Pokaži, da je v Fraunhoferjevem približku uklonska slika reže s širino $d$:
\begin{equation}
I_P = I_0\frac{\sin(k d x / 2z)}{k d x/2z},
\end{equation}
pri čemer sta $I_P$ intenziteta valovanja na zaslonu v točki s 
koordinatama $(x,z)$ in $k$ valovno število. Izračunaj še Fraunhoferjevo uklonsko sliko
na uklonski mrežici, ki je sestavljena iz velikega števila zelo tankih rež na medsebojni 
oddaljenosti $a$.
\end{definition}
\begin{definition}
\label{naloga-Frauhofer-Kirchhoff-uklon}
Pokaži, da je v Fraunhoferjevem približku uklonska slika okrogle odprtine
\begin{equation}
I_P = I_0\frac{4 J_1(k a \rho/ z)}{k a \rho/z},
\end{equation}
pri čemer so $I$ intenziteta valovanja na zaslonu, $k$ valovno število, $a$ polmer odprtine, 
$J_1(x)$ Besslova funkcija in $\rho = \sqrt{x^2+y^2}$.
\end{definition}
\pagebreak
\section{Elektromagnetno valovanje v anizotropnih snoveh}\label{chap:anizotropni}
Do zdaj smo obravnavali elektromagnetno valovanje
v izotropnih snoveh, v katerih je di\-elek\-trič\-nost skalar. 
Na splošno so snovi anizotropne,
dielektričnost\index{Dielektričnost} je tenzor, hitrost potovanja svetlobe \index{Hitrost valovanja}
skozi snov pa je odvisna od smeri svetlobe in njene polarizacije.

Gostoto električnega polja~\index{Električno polje!gostota} v anizotropni snovi zapišemo kot 
\begin{equation}
\mathbf{D}=\epsilon_{0}\underline{\epsilon} \cdot\mathbf{E} = 
\epsilon_{0}
\left[\begin{array}{ccc}
\epsilon_{11} & \epsilon_{12}& \epsilon_{13}\\
\epsilon_{21} & \epsilon_{22}& \epsilon_{23}\\
\epsilon_{31} & \epsilon_{32}& \epsilon_{33}\\
\end{array}\right]\!\mathbf{E},
\label{eq:gostota-elektricnega-polja-tenzor}
\end{equation}
pri čemer je $\underline{\epsilon}$ tenzor drugega ranga in ima na splošno devet komponent.
V dielektričnih snoveh, v katerih ni optične aktivnosti ali absorpcije, je tenzor
realen in simetričen $\epsilon_{ij}=\epsilon_{ji}^*$. Tak tenzor lahko vedno
diagonaliziramo, torej poiščemo koordinatni sistem, v katerem je tenzor
diagonalen. V takem koordinatnem sistemu velja 
\begin{equation}
\mathbf{D} = \epsilon_{0}
\left[\begin{array}{ccc}
\epsilon_{1} & 0& 0\\
0 & \epsilon_{2}& 0\\
0 & 0& \epsilon_{3}\\
\end{array}\right]\!\mathbf{E}
\label{eq:gostota-elektricnega-polja-lastni}
\end{equation}
oziroma po komponentah
\begin{align}
D_{1}&=\epsilon_{0}\epsilon_{1}E_{1}, \nonumber \\
D_{2}&=\epsilon_{0}\epsilon_{2}E_{2} \nonumber \qquad \mathrm{in} \\
 D_{3}&=\epsilon_{0}\epsilon_{3}E_{3}.
 \end{align}
Glavne osi novega koordinatnega sistema določajo smeri, vzdolž katerih sta jakost
in gostota električnega polja vzporedni, iz lastnih vrednosti 
pa izračunamo tri lomne količnike\index{Lomni količnik} $n_{i}=\sqrt{\epsilon_{i}}$. Snovi,
za katere so vse tri vrednosti $n_i$ različne, imenujemo optično dvoosne snovi\index{Dvolomnost!dvoosne snovi}, 
medtem ko sta v optično enoosnih snoveh\index{Dvolomnost!enoosne snovi} dve lastni vrednosti enaki $n_{1}=n_{2}$. 
Če so enake vse tri lastne vrednosti, je snov izotropna.

\subsection*{Ploskev valovnega vektorja}
\index{Ploskev valovnega vektorja}
V anizotropnih snoveh je lomni količnik odvisen od smeri 
širjenja svetlobe in izkaže se, da tudi od njene polarizacije. Poglejmo 
najprej preprost primer, ko se svetloba širi vzdolž lastne osi, naj bo to os $z$.
Če je vpadno valovanje polarizirano vzdolž lastne osi $x$, se pri prehodu
skozi kristal polarizacija valovanja ohrani, lomni količnik za
tak val je $n_{1}$. Podobno velja za val, polariziran v smeri
$y$, za katerega je lomni količnik enak $n_{2}$. Če polarizacija valovanja, 
ki se širi vzdolž lastne osi $z$, ne sovpada z lastnima osema $x$ ali $y$, nastane po 
prehodu skozi kristal iz vpadnega linearno polariziranega valovanja na splošno eliptično polarizirano valovanje\index{Polarizacija!eliptična}. Lastni komponenti namreč potujeta različno
hitro, zato se med njima pojavi fazni zamik. 

Za poljubno smer širjenja valovanja in poljubno polarizacijo svetlobe je račun razmeroma zapleten. 
Formalni pristop izhaja iz valovne enačbe (enačba~\ref{eq:valovna-skalarna}), v kateri
moramo upoštevati tudi električno polarizacijo 
$\mathbf{P} = \epsilon_{0}(\underline{\epsilon}-I)\mathbf{E}$. Iz nje dobimo 
sistem enačb za komponente valovnega vektorja in jakosti električnega polja.\footnote{~Glej 
npr. G. R. Fowles, {\it Introduction to Modern Optics}, druga izdaja, Dover Publications (1975).}

Rešitev tega sistema najnazorneje predstavimo s ploskvijo valovnega vektorja, 
ki je sklenjena dvolistna ploskev (slika~\ref{kploskev}). Dvolistnost ploskve
vodi pri vsakem valovnem vektorju $\mathbf{k}$ do dveh rešitev in dveh različnih lomnih
količnikov, od katerih vsak ustreza eni od ortogonalnih polarizacij svetlobe. Točke, v katerih
se ploskev dotika sama sebe in sta lomna količnika za obe polarizaciji enaka, 
določajo smeri optičnih osi. \index{Optična os}
\vglue5truemm
\begin{figure}[ht]
\centering
\def\svgwidth{140truemm} 
\input{slike/01_kploskev.pdf_tex}
\vglue4truemm
\caption{Dvolistna ploskev valovnega vektorja, pri čemer zaradi nazornosti rišemo le presečišča
ploskve z geometrijskimi ravninami v prvem oktantu. 
V dvoosnem kristalu (a) sta dve optični osi. Druga os ni narisana, leži pa 
simetrično glede na os $z$. Privzeli smo, da velja $n_1<n_2<n_3$.
V optično enoosnem kristalu (b) je le ena optična os, \index{Polarizacija}
po dogovoru je to os $z$. Rdeče puščice označujejo smer ustrezne jakosti električnega polja $\mathbf{E}$
in oranžne smer gostote električnega polja $\mathbf{D}$, ki je pravokotna na $\mathbf{k}$. 
Kjer sta smeri jakosti in gostote električnega polja vzporedni, je narisana zgolj smer jakosti.}
\label{kploskev}
\end{figure}

\subsection*{Optično enoosni kristali}
\index{Dvolomnost!enoosne snovi}
V optično enoosnih kristalih sta dve lastni vrednosti tenzorja dielektričnosti 
enaki. Lastne vrednosti izberemo
tako, da velja $n_{1}=n_{2}\neq n_{3}$. Navadno imenujemo
$n_{1}=n_{2}=n_{o}$ redni (\textit{ordinary})
lomni količnik\index{Lomni količnik!redni}  in $n_{3}=n_{e}$ izredni 
(\textit{extraordinary}) lomni količnik\index{Lomni količnik!izredni}.\footnote{~Za $n_o$ uporabljamo tudi oznako $n_{\perp}$ in za $n_e$ oznako $n_{\parallel}$. Oznaki
nakazujeta smer pripadajoče lastne polarizacije glede na optično os.}
V eni smeri sta lomna količnika za obe polarizaciji enaka in tej smeri pravimo 
optična os. Po dogovoru je to os $z$.\index{Optična os} Hitrost valovanja, ki
se širi vzdolž optične osi, je tako neodvisna od polarizacije.
Ker je optična os samo ena, imenujemo tak kristal optično enoosen. 

Za lažjo predstavo si oglejmo ploskev valovnega vektorja (slika~\ref{kploskev}\,b). 
V tem primeru ni treba obravnavati celotne ploskve, ampak zaradi rotacijske simetrije
zadošča, da narišemo presek ploskve valovnega vektorja z vpadno ravnino, ki jo določata 
optična os in valovni vektor $\mathbf{k}$. 
Ker je pomemben le kot $\vartheta$ med valovnim vektorjem $\mathbf{k}$ 
in optično osjo $z$, lahko drugo koordinatno os poljubno izberemo. Tukaj izberemo
os $y$ (slika~\ref{fig:Elipsa}). 

V ravnini $yz$ tako dobimo krožnico s polmerom $n_o$ in elipso z glavnima polosema
$n_o$ in $n_e$. Če je $n_e>n_o$, je snov pozitivno anizotropna (slika~\ref{fig:Elipsa}\,a), 
v nasprotnem primeru je negativno anizotropna (slika~\ref{fig:Elipsa}\,b). Po 
pričakovanju se krožnica in elipsa dotikata ravno na osi $z$. 
\begin{figure}[ht]
\centering
\def\svgwidth{140truemm} 
\input{slike/01_elipsa.pdf_tex}
\caption{V optično enoosnih kristalih je lomni količnik odvisen
od smeri valovnega vektorja $\mathbf{k}$ in polarizacije. 
Poznamo pozitivno anizotropne snovi, pri katerih
je $n_e>n_o$ (a), in negativno anizotropne snovi, 
za katere velja $n_e< n_o$ (b). V obeh primerih je redni 
žarek polariziran pravokotno na vpadno ravnino. Zanj velja, 
da je $\mathbf{D} \parallel \mathbf{E}$ in $\mathbf{S} \parallel \mathbf{k}$ (c). 
Polarizacija izrednega žarka leži v vpadni ravnini. 
Smer $\mathbf{S}$ ni vzporedna z valovnim vektorjem
$\mathbf{k}$, prav tako valovne fronte niso pravokotne nanjo (d). 
Primer na (c) in (d) je narisan za pozitivno 
anizotropno snov.}
\label{fig:Elipsa}
\end{figure}

Za vsako smer valovnega vektorja, torej za vsak kot $\vartheta$, obstajata dve rešitvi, 
ki pripadata dvema lastnima polarizacijama z ustreznima lomnima količnikoma. 
Lomni količnik za žarek, ki je polariziran pravokotno na vpadno ravnino, 
je neodvisen od $\vartheta$. To je redni žarek, njegov lomni količnik 
pa je vedno $n_o$, ne glede na vpadni kot. Na skici temu žarku ustreza krožnica.

Žarek, katerega polarizacija leži v vpadni ravnini, je izredni žarek. Pripadajoči
lomni količnik\index{Lomni količnik} je odvisen od kota $\vartheta$ in 
ga izračunamo iz enačbe elipse s polosema $n_o$ in $n_e$ 
\boxeq{eq:izreden}{
\frac{1}{n^{2}(\vartheta)}=\frac{\cos^{2}\vartheta}{n_{o}^{2}}+\frac{\sin^{2}\vartheta}{n_{e}^{2}}.
}

Navadno sta pri ravnem valu vektorja $\mathbf{E}$ in $\mathbf{D}$ vzporedna, 
prav tako $\mathbf{k}$ in $\mathbf{S}$. Tak žarek se širi v smeri valovnega vektorja, pri čemer
so valovne fronte (ploskve konstantne faze) pravokotne nanj. To velja tudi za redni žarek v anizotropnih
snoveh (slika~\ref{fig:Elipsa}\,c). 

Izredni žarek ima, kot že ime nakazuje, izredne lastnosti. Vektorja
$\mathbf{E}$ in $\mathbf{D}$ nista vzporedna, zato tudi valovni vektor $\mathbf{k}$ ni vzporeden
energijskemu toku oziroma Poyntingovemu vektorju $\mathbf{S}$ (slika~\ref{fig:Elipsa}\,d). 
Smer energijskega toka za izredni žarek določimo z normalo 
na elipso pri kotu $\vartheta$. \index{Električno polje!jakost}
\index{Električno polje!gostota}\index{Valovni vektor}\index{Poyntingov vektor}

\subsection*{Dvojni lom}
Ko vpade žarek na mejo dveh sredstev, se lomi. Hitrost valovanja -- in s tem tudi 
kot, pod katerim se lomi -- je v anizotropnih snoveh odvisna od polarizacije.
Pri zapisu \index{Lomni zakon}lomnega zakona (enačba~\ref{eq:lomni_zakon}) 
v anizotropnih snoveh moramo biti zato pazljivi. Na splošno
se pojavita dva lomljena žarka z različnima polarizacijama, kar
da ime pojavu: \index{Dvolomnost}dvolomnost. 

Za redni val s TE polarizacijo (pravokotno na vpadno ravnino) velja lomni zakon, pri čemer
je lomni količnik snovi enak rednemu lomnemu količniku $n_o$
\begin{equation}
\sin\vartheta_{1}=n_{o}\sin\vartheta_{o}.
\end{equation}
Pri zapisu smo privzeli, da je lomni količnik snovi, iz katere valovanje prehaja v anizotropno snov, 
enak 1. 

Za izredni val s TM polarizacijo 
prav tako zapišemo lomni zakon
\begin{equation}
\sin\vartheta_{1}=n(\vartheta_e')\sin\vartheta_{e}.
\end{equation}
Razlika je v tem, da je lomni količnik $n(\vartheta_e')$ odvisen od smeri valovnega vektorja glede na smer optične osi
in je določen z enačbo elipse (enačba~\ref{eq:izreden}), kot $\vartheta_e$ pa je določen glede na
normalo na mejno ploskev. Na splošno je zapis precej zapleten, poenostavi se le, 
ko je optična os vzporedna mejni ploskvi ali pravokotna nanjo. 

Kadar vpada valovanje pravokotno na izotropno snov, se žarek ne lomi. V 
dvolomnih snoveh se lahko tudi pri pravokotnem vpadu svetloba razkloni (slika
\ref{fig:dvolomnost}\,b). Kadar je optična os nagnjena glede na vpadnico, sta valovna vektorja
obeh prepuščenih žarkov sicer vzporedna valovnemu vektorju vpadnega žarka ($\mathbf{k}_1 \parallel
\mathbf{k}_o \parallel \mathbf{k}_e$), vendar se razlikujeta smeri Poyntingovih vektorjev
($\mathbf{S}_1 \parallel \mathbf{S}_o \nparallel \mathbf{S}_e$). Ob prehodu skozi 
plast anizotropne snovi tako svetloba potuje v dveh smereh in nastaneta dve sliki 
z medsebojno pravokotnima polarizacijama (slika~\ref{foto:dvolom}). 

\begin{figure}[ht]
\centering
\def\svgwidth{140truemm} 
\input{slike/01_dvolom.pdf_tex}
\caption{Dvojni lom. Pri poševnem vpadu na anizotropno snov se
valovanje loči na dva različno polarizirana žarka (a). Če je optična os 
usmerjena pod poljubnim kotom glede na normalo mejne ravnine, se tudi pri pravokotnem vpadu
svetloba razkloni (b). Valovna vektorja sta v tem primeru vzporedna, 
Poyntingova vektorja pa imata različne smeri. Rdeče puščice označujejo smer
jakosti in oranžne smer gostote električnega polja. Kjer sta vektorja vzporedna, je narisana
samo smer jakosti električnega polja.}
\label{fig:dvolomnost}
\end{figure}

\begin{table}[ht]
 \centering
\begin{tabular}{|l|c|c|} \hline  
      Snov & $n_o$ & $n_e$ \\ \hline
      CaCO$_3$ (kalcit) & 1,6557 & 1,4849 \\ \hline\index{CaCO$_3$|see {Kalcit}}
      BaTiO$_3$ & 2,4042 & 2,3605 \\ \hline \index{BaTiO$_3$}
      LiNbO$_3$ & 2,2864 & 2,2022 \\ \hline \index{LiNbO$_3$}
      KH$_2$PO$_4$ (KDP) & 1,5074 & 1,4669 \\ \hline \index{KH$_2$PO$_4$|see {KDP}}\index{KDP}
      tekoči kristal 5CB ($25~\si{\degreeCelsius}$) & 1,5319 & 1,7060 \\ 
      \hline \index{Tekoči kristali} \index{Telur}\index{Tekoči kristali!5CB}
      telur ($\lambda = 10~\si{\micro\metre}$) & 4,7933 & 6,2455 \\ 
\hline 
\end{tabular}
  \caption{Redni in izredni lomni količniki za nekaj izbranih optično enoosnih kristalov. Razen v primeru telurja
   veljajo vrednosti za svetlobo z valovno dolžino 633~\si{\nano\metre}.}
\label{table:none}
\end{table}
\newpage
\begin{figure}[ht]
\centering
\def\svgwidth{140truemm} 
\input{slike/01_FotoDvolom.pdf_tex}
\caption{Dvojni lom v kristalu kalcita (islandski dvolomec). \index{Kalcit}
Po prehodu skozi kristal nastaneta dve razmaknjeni sliki in z linearnim polarizatorjem 
pokažemo, da imata sliki različni polarizaciji. Puščica označuje smer prepustnosti
linearnega polarizatorja.}\index{Polarizacija}
\label{foto:dvolom}
\end{figure}

%Final

%-------------------------------------------------------------------------------
%	CHAPTER 2
%-------------------------------------------------------------------------------

\input{Chapter_02_Koherenca}
%Final

%-------------------------------------------------------------------------------
%	CHAPTER 3
%-------------------------------------------------------------------------------

\chapterimage{slike/Laguerre.jpg} % Chapter heading image

\chapter{Koherentni snopi svetlobe}
V tem poglavju bomo zapisali obosni približek valovne enačbe in spoznali 
njegovo osnovno rešitev, to je Gaussov snop. Obravnavali bomo tudi snope višjega reda in
se naučili računati prehode Gaussovih snopov skozi optične elemente. 

\section{Omejen snop svetlobe}
Pri obravnavi elektromagnetnega valovanja pogosto uporabljamo
približek ravnih valov. Ravni val je v smeri pravokotno na smer širjenja
neomejen, čim pa ga usmerimo skozi odprtino
v zaslonu, nastane omejen snop svetlobe. V njem valovne fronte (ploskve konstantne faze) 
niso ravne in meje snopa niso vzporedne, ampak se snop zaradi uklona širi\index{Uklon} 
(slika \ref{fig:Uklon-na-rezi}).
\begin{figure}[ht]
\centering
\def\svgwidth{130truemm} 
\input{slike/03_uklon_na_rezi.pdf_tex}
\caption{Omejeni snop svetlobe nastane ob prehodu ravnega vala skozi končno odprtino. 
Polmer odprtine naj bo $a$, območje bližnjega polja označuje $b$ in $2\vartheta$ 
celotni kot širjenja snopa.}
\label{fig:Uklon-na-rezi}
\end{figure}

V veliki oddaljenosti od zaslona polje izračunamo s
\index{Fraunhoferjev uklon}Fraunhoferjevo uklonsko teorijo, za
opis polja v bližini zaslona pa je treba uporabiti Fresnelov približek
\index{Fresnelov uklon} (glej razdelek~\ref{FFuklon}). 
Vendar lahko nekaj grobih ocen naredimo tudi brez računa. Kot širjenja je približno
\begin{equation}
\vartheta\sim\frac{\lambda}{a},
\label{eq:kot_ocena}
\end{equation}
pri čemer je $a$ polmer odprtine v zaslonu.\index{Območje bližnjega polja}
S slike~\ref{fig:Uklon-na-rezi} lahko ocenimo tudi območje bližnjega polja, ki seže do $b$.
Dobimo
\begin{equation}
\frac{a}{b}\sim \vartheta \qquad \mathrm{in~tako} \qquad b\sim\frac{a^2}{\lambda}.
\label{eq:z_ocena}
\end{equation}
Bolj kvantitativen opis omejenih
snopov bi dobili s Fraunhoferjevo in Fresnelovo uklonsko teorijo,
kar ni najudobnejša pot (glej nalogo~\ref{ffuklon}). Lotimo se problema raje z 
uporabo obosnega približka valovne enačbe.

\begin{definition}
\label{ffuklon}
Pokaži, da je Fraunhoferjeva uklonska slika na odprtini, katere prepustnost se v radialni smeri
spreminja kot Gaussova funkcija $T(\xi, \eta)=\exp(-(\xi^2+\eta^2)/w_0^2)$, podana z Gaussovo funkcijo
oblike $E(x,y,z) \propto \exp(-(x^2+y^2)/w^2(z))$ in določi odvisnost $w(z)$. Izračunaj še 
uklonsko sliko v bližnjem polju po Fresnelovi uklonski teoriji.
\end{definition}

\section{Obosna valovna enačba}
Obravnavo začnemo z valovno enačbo in rešitvijo v obliki monokromatskega valovanja 
s krožno frekvenco $\omega$. Zaradi enostavnosti se omejimo le na eno polarizacijo, 
tako da $\mathbf{E}$ pišemo kot skalar. Ustrezna
Helmholtzeva enačba je \index{Helmholtzeva enačba}(enačba~\ref{eq:Helmholtz})
\begin{equation}
\nabla^{2}E+k^{2}E=0,
\label{eq:valovna-enacba-hh}
\end{equation}
pri čemer sta $k=n\omega/c_{0}$ valovno število in $n$ lomni količnik
sredstva, po katerem se valovanje širi. Rešitev iščemo v obliki omejenega snopa, 
ki se širi približno vzdolž osi $z$, z nastavkom
\begin{equation}
E=E_{0}\psi(\mathbf{r},z)e^{ikz},
\label{eq:ravni-val-nastavek}
\end{equation}
pri čemer je $\mathbf{r}$ krajevni vektor v ravnini $xy$, prečni na smer širjenja svetlobe. 
Glavni del odvisnosti od koordinate $z$ smo zapisali v faktorju $e^{ikz}$, tako da lahko
privzamemo, da se $\psi$ v smeri $z$ le počasi spreminja. Vstavimo
nastavek (enačba~\ref{eq:ravni-val-nastavek}) v Helmholtzevo enačbo (enačba~\ref{eq:valovna-enacba-hh})
in pri tem zanemarimo druge odvode $\psi$ po $z$, saj je zaradi počasnega spreminjanja
$\partial^{2}\psi/\partial z^{2}$ majhen v primerjavi s $k\partial\psi/\partial z$ in $k^{2}\psi$.
Dobimo obosno
ali paraksialno valovno enačbo\index{Obosna valovna enačba} za $\psi$
\index{Paraksialna enačba|see{Obosna valovna enačba}}
\boxeq{eq:obosna-valovna-enacba}{
\nabla_{\perp}^{2}\psi=-2ik{\frac{{\partial\psi}}{{\partial z}}}.
}
\vglue-3truemm
\begin{remark} 
Opazimo, da je obosna valovna enačba enaka Schr\"{o}dingerjevi enačbi\index{Schr\"odingerjeva enačba}
za prost delec v dveh dimenzijah, v kateri ima koordinata $z$ vlogo
časa. Omejenemu snopu v kvantni mehaniki ustreza lokaliziran delec
-- valovni paket. Ta se s časom širi, kar v optiki ustreza pojavu 
uklona.\footnote{~Glej npr. J. Strnad, {\it Fizika, 3. del}, tretja izdaja, DMFA-založništvo (2018).}
\end{remark}
Preden se lotimo reševanja obosne valovne enačbe, jo primerjajmo s Helmholtzevo enačbo
na primeru ravnega vala. Nastavek za ravni val \index{Ravni val} zapišemo v obliki
\begin{equation}
\psi=e^{ik_{1}x+ik_{2}y}\, e^{-i\beta z}.
\label{eq:ravni-val-nastavek-obosni}
\end{equation}
Da bo nastavek rešitev
obosne valovne enačbe~(enačba~\ref{eq:obosna-valovna-enacba}), mora veljati 
\begin{equation}
\beta=\frac{k_{1}^{2}+k_{2}^{2}}{2k}.
\end{equation}
Ko vstavimo nastavek za $\psi$ v izraz za polje $E$ 
(enačba~\ref{eq:ravni-val-nastavek}), dobimo ravni val, za katerega velja 
\begin{equation}
k_{3}=k-\beta=k-\frac{k_{1}^{2}+k_{2}^{2}}{2k}.
\label{eq:k3-razvoj}
\end{equation}
Pri tem označujejo $k_{3}$ vzdolžno ter $k_{1}$ in $k_{2}$ prečni komponenti valovnega 
vektorja, $k$ pa je valovno število. Po drugi strani za ravni val, ki je 
rešitev Helmholtzeve enačbe (enačba~\ref{eq:valovna-enacba-hh}), velja zveza
\begin{equation}
k_{3}=\sqrt{k^{2}-(k_{1}^{2}+k_{2}^{2})}.\label{eq:k3-tocno}
\end{equation}
Vidimo, da sledi enačba~(\ref{eq:k3-razvoj}) iz enačbe~(\ref{eq:k3-tocno})
z razvojem za majhne vrednosti $k_1$ in $k_2$. To pove, da je približek obosne 
enačbe dober, kadar sta prečni komponenti valovnega vektorja 
majhni v primerjavi z vzdolžno. 
Takrat je majhen tudi kot širjenja snopa in člene, višje od kvadratnih,
lahko zanemarimo. To je tudi območje veljavnosti Fresnelove uklonske
teorije.\index{Fresnelov uklon}

\begin{remark}\index{Fouriereva optika}
Časovno odvisnost poljubnega začetnega
stanja v kvantni mehaniki navadno izračunamo tako, da v začetnem
trenutku paket razvijemo po lastnih stanjih energije -- ravnih valovih.
Rešitev v poljubnem kasnejšem trenutku je potem dana v obliki Fourierevega
integrala. Ta pot je zelo uporabna tudi v optiki in je osnova sklopa
računskih metod, znanih pod imenom Fouriereva optika.\footnote{~Glej npr. 
E. Hecht, {\it Optics}, peta izdaja, Pearson Education Limited (2017).} V našem primeru
z njo brez težav pridemo nazaj do Fresnelovega uklonskega približka.
\end{remark}

\section{Osnovni Gaussov snop}
\label{chap:gaussovsnop}
\index{Gaussov snop}
Naša naloga je poiskati rešitve obosne valovne enačbe, ki popišejo omejene
snope. Iz kvantne mehanike vemo, da je najbolj lokaliziran in se najpočasneje
širi valovni paket Gaussove oblike. Zato poskusimo najti rešitev obosne
enačbe (enačba~\ref{eq:obosna-valovna-enacba}) z nastavkom
\begin{equation}
\psi(r,z)=e^{ikr^{2}/2q(z)}\, e^{-i\phi(z)},\label{eq:gaussov-snop-nastavek}
\end{equation}
pri čemer funkcija $q(z)$ opisuje širjenje snopa v prečni smeri,
$\phi(z)$ pa počasno spreminjanje faze snopa vzdolž osi $z$.
Vstavimo nastavek (enačba~\ref{eq:gaussov-snop-nastavek}) v obosno valovno enačbo 
(enačba~\ref{eq:obosna-valovna-enacba}). Zaenkrat se omejimo le na radialno 
simetrične rešitve in v cilindričnih koordinatah zapišemo
\begin{equation}
\nabla_{\perp}^{2}\psi=\frac{1}{r}\frac{\partial}{\partial r}\, r\,\frac{\partial\psi}{\partial r}=
\left( \frac{2ik}{q}-\frac{k^2r^2}{q^2}\right)\psi
\end{equation}
 in 
\begin{equation}
\frac{\partial\psi}{\partial z}=\left(-\frac{ikr^{2}}{2q^2}q(z)^{\prime}-i\phi^{\prime}\right)\psi.
\end{equation}
Pri tem črtica označuje odvod po koordinati $z$.
Iz obosnega približka (enačba~\ref{eq:obosna-valovna-enacba}) sledi
\begin{equation}
\frac{2ik}{q}-\frac{k^2r^2}{q^2}=ik\left(\frac{ikr^{2}}{q^2}q(z)^{\prime}+2i\phi^{\prime}\right)\!.
\end{equation}
Zapisana zveza mora veljati pri vsakem $r$, zato sta koeficienta
pri $r^{2}$ na obeh straneh enačbe enaka in člena brez odvisnosti od $r$ prav tako. Sledi
\begin{equation}
q(z)^{\prime}=1 \qquad \mathrm{in} \qquad \phi^{\prime}=-\frac{i}{q}.
\end{equation}
Z integracijo dobimo najprej 
\begin{equation}
q=z-iz_{0},
\label{eq:alpha}
\end{equation}
pri čemer smo z $-i z_{0}$ označili integracijsko konstanto. 
Integriramo še enačbo za fazo 
\begin{equation}
\phi=\int_{0}^{z}\,-\frac{i dz}{z-iz_{0}}=-i\ln(1+i\frac{z}{z_{0}}).
\label{eq:phiphi}
\end{equation}
Vstavimo izraza za $q$ in $\phi$ 
v nastavek za $\psi$ (enačba~\ref{eq:gaussov-snop-nastavek}) in zapišemo
\begin{equation}
\psi = \exp\left(i\frac{kr^{2}}{2(z-iz_0)}\right)\,\exp\left(-\ln(1+i\frac{z}{z_{0}})\right)
\end{equation}
Izraz preoblikujemo in dobimo
\begin{equation}
 \psi = \frac{1}{1+i\frac{z}{z_{0}}}\,\exp\left(-\frac{kr^{2}z_{0}}{2(z_{0}^{2}+z^{2})}+
 \frac{ikr^{2}z}{2(z_{0}^{2}+z^{2})}\right).
 \label{eq:gaussov-snop-vmesni}
\end{equation}
Najprej poglejmo realni del eksponenta, ki opisuje prečno obliko snopa. Vpeljemo novo spremenljivko $w$
in realni del prečne odvisnosti zapišemo z Gaussovo funkcijo $\exp(-r^2/w^2)$, 
ki da snopu tudi ime. Parameter $w$, ki označuje 
polmer snopa\index{Gaussov snop!polmer} pri danem $z$, je podan z
\begin{equation}
w^2 = \frac{2(z_0^2+z^2)}{kz_0}= \frac{2z_0}{k}\left(1+\left(\frac{z}{z_0}\right)^2\right)\!.
\end{equation}
Vpeljemo $w_0 = 2z_0/k$ kot polmer snopa v izhodišču (pri $z=0$) in zapišemo
\boxeq{eq:w}{
w^2 = w_{0}^{2}\left(1+\left(\frac{z}{z_{0}}\right)^{2}\right)\!.
}
Pri $z=0$ je polmer snopa najmanjši in pravimo, da je tam grlo snopa\index{Gaussov snop!grlo} 
(slika~\ref{fig:Gauss}). Parameter $z_0$ 
navadno izrazimo s polmerom snopa v grlu $w_0$, pri čemer upoštevamo $k=2 \pi/\lambda$. Dobimo
\boxeq{eq:z0}{
z_{0}=\frac{\pi w_{0}^{2}}{\lambda}.
}
Parameter $z_{0}$\index{Gaussov snop!bližnje polje} označuje oddaljenost od grla, 
pri kateri snop preide v asimptotično širjenje in tako omejuje območje,
znotraj katerega se snop ne razširi znatno. Imenujemo ga \index{Območje bližnjega polja}
\index{Rayleighova dolžina}Rayleighova dolžina\footnote{~Angleški fizik in 
nobelovec John William Strutt, 3. baron Rayleighški; lord Rayleigh, 1842--1919.}
in območje približno konstantne širine snopa Rayleighovo
območje ali območje bližnjega 
polja\index{Rayleighovo območje|see{Območje bližnjega polja}}. Zaradi simetričnosti
rešitve je celotno Rayleighovo območje dolgo $2z_0$. 
Vrednost $z_{0}$ označuje tudi oddaljenost od grla, pri kateri začne veljati 
Fraunhoferjev uklonski približek. 
\begin{figure}[ht]
\centering
\def\svgwidth{100truemm} 
\input{slike/03_Gauss.pdf_tex}
\caption{Gaussov snop. Parameter $w_0$ označuje polmer snopa v grlu pri $z=0$, z oddaljenostjo
od grla pa polmer snopa $w$ narašča. Z oddaljenostjo od grla se spreminja tudi krivinski radij front $R$.
Parameter $z_0$ je Rayleighova dolžina in kot $\vartheta$ polovični kot širjenja snopa.}
\label{fig:Gauss}
\vglue-2truemm
\end{figure}

Izračunajmo še divergenco snopa v velikih oddaljenostih od grla. Polovični kot širjenja je
\begin{equation}
\vartheta=\lim_{z \to \infty} \frac{dw}{dz} = \lim_{z \to \infty}\frac{d}{dz} \left(w_0\sqrt{1+z^2/z_0^2}\right)=
\frac{w_{0}}{z_{0}}= \frac{\lambda}{\pi w_{0}}\label{eq:divergenca-snopa}
\end{equation}
in celotna divergenca snopa\index{Gaussov snop!divergenca}
\boxeq{eq:divergenca-snopa2}{
\theta=2 \frac{w_{0}}{z_{0}}=\frac{2\lambda}{\pi w_{0}}.
}

Izraza za območje bližnjega polja (enačba~\ref{eq:z0}) in divergenco 
(enačba~\ref{eq:divergenca-snopa}) sta v skladu z grobima ocenama, ki smo ju 
napravili na začetku poglavja (enačbi~\ref{eq:kot_ocena} in \ref{eq:z_ocena}). Faktor
$\pi$ oziroma $1/\pi$ je značilen za Gaussov snop, ki ima od vseh snopov 
najmanjšo divergenco. 

\begin{remark}
Za določanje kakovosti dejanskega laserskega snopa se pogosto vpelje faktor \index{Faktor $M^2$}$M^2$,
ki opiše odstopanje oblike snopa od idealnega Gaussovega snopa 
$\theta = M^2 \, 2\lambda/\pi w_{0}$. 
Dobri laserji dosegajo vrednost $M^2 \sim 1$,
pri močnejših trdninskih ali polprevodniških laserjih je $M^2 \sim 30$ ali več. 
V grobem velja, da $M^2$ narašča z močjo laserja in oblika snopa močnih laserjev navadno znatno
odstopa od oblike idealnega Gaussovega snopa.\footnote{~A. E. Siegman, Proc. SPIE $\mathbf{1868}$, 2 (1993).}  
\end{remark}

Vrnimo se k imaginarnemu delu eksponenta v enačbi~(\ref{eq:gaussov-snop-vmesni}). Z
vpeljavo nove spremenljivke $R$ ga zapišemo v poenostavljeni obliki $\exp(ikr^2/2R)$. Parameter $R$, ki 
ga vpeljemo kot 
\boxeq{eq:R}{
R=z\left(1+\left(\frac{z_{0}}{z}\right)^{2}\right)\!,
}
meri krivinski radij valovnih front\index{Gaussov snop!krivinski radij} 
pri oddaljenosti od grla $z$. To najlažje
uvidimo, če imaginarni del primerjamo z zapisom za krogelni val, razvit 
po majhnih odmikih $r$ od osi $z$
$z$
\begin{equation}
\frac{1}{R}e^{ikR}=\frac{1}{R}e^{ik\sqrt{z^{2}+r^{2}}}\approx \frac{1}{R}e^{ikz+ikr^{2}/2z} \approx \frac{1}{R}e^{ikz+ikr^{2}/2R}.
\label{eq:krogelni-val}
\end{equation}
Krivinski radij valovnih front Gaussovega snopa je v izhodišču neskončen, kar pomeni, da so tam valovne fronte 
ravne. Pri velikih oddaljenostih krivinski radij narašča linearno z oddaljenostjo in fronte
so podobne delu krogelnega vala. 

\begin{definition}
\label{naloga-ukrivljenost-snopa}
Pokaži, da je največja ukrivljenost valovnih front snopa (in s tem najmanjši $R$) ravno pri $z=\pm z_{0}$.
Izračunaj še ukrivljenost front v grlu in v veliki oddaljenosti od grla.
\end{definition}

Na sliki~\ref{fig:ravni-Gaussov-krogelni-val} so prikazane valovne fronte (ploskve konstantne faze)
ravnega vala, Gaussovega snopa
in krogelnega vala. V bližini grla so valovne fronte v Gaussovem snopu ravne in snop je 
podoben ravnemu valu. Po drugi strani so za velike oddaljenosti fronte v Gaussovem snopu 
podobne delu krogelnega vala, le da je, kot bomo videli, faza Gaussovega snopa
zamaknjena za $\pi/2$ glede na krogelni val. 

\begin{figure}[ht]
\centering
\def\svgwidth{80truemm} 
\input{slike/03_fronte.pdf_tex}
\caption{Valovne fronte ravnega vala (a), Gaussovega snopa (b) in
krogelnega vala (c). V bližini grla je Gaussov snop podoben ravnemu valu in 
pri velikih oddaljenostih od grla krogelnemu valu, vendar z dodatnim faznim zamikom.}
\label{fig:ravni-Gaussov-krogelni-val}
\end{figure}
Ostal je  še faktor pred eksponentom v izrazu~(\ref{eq:gaussov-snop-vmesni})
\begin{equation}
\frac{1}{1+i\frac{z}{z_{0}}}=\frac{1}{\sqrt{1+(\frac{z}{z_0})^{2}}}e^{-i\eta(z)}=\frac{w_{0}}{w}e^{-i\eta(z)},
\end{equation}
 pri čemer je
\boxeq{eq:eta}{
\eta(z)=\arctan\left(\frac{z}{z_{0}}\right)\!.
}
Ta faktor poskrbi za ohranitev energijskega toka, saj opisuje zmanjševanje amplitude
jakosti električnega polja z naraščajočo oddaljenostjo od izhodišča. \index{Gaussov snop!faza} 
Dodatna faza $\eta$, imenujemo jo \index{Gouyeva faza}Gouyeva 
faza\footnote{~Francoski fizik Louis Georges Gouy, 1854--1926.},
je posledica povečane fazne hitrosti, 
kadar je valovanje omejeno v prečni smeri. Podoben pojav bomo srečali tudi pri valovanju, ki je 
omejeno v valovode (poglavje~\ref{chap:fibri}). Za velike oddaljenosti od izhodišča je Gouyjeva faya
enaka $\pi/2$,
kar predstavlja fazni zamik na sliki~\ref{fig:ravni-Gaussov-krogelni-val}.

S tem končno zapišemo izraz za jakost električnega polja osnovnega \index{Gaussov snop}Gaussovega 
snopa\footnote{~Nemški matematik, fizik in astronom Carl Friedrich Gauss, 1777--1855.}
\boxeq{eq:gaussov-snop}{
E(r,z,t)=E_{0}\,\frac{w_{0}}{w}\,e^{ikz-i\omega t}\,e^{-r^{2}/w^{2}}\,e^{ikr^{2}/2R}
e^{-i\eta},
}
pri čemer so $w(z)$, $R(z)$ in $\eta(z)$ podani z enačbami (\ref{eq:w}), (\ref{eq:R}) in (\ref{eq:eta}).
Intenziteta svetlobe\index{Gaussov snop!intenziteta} v Gaussovem snopu je 
\boxeq{eq:gaussov-snop-intenziteta}{
I(r,z)= E(r,z,t)E^*(r,z,t) = I_{0}\,\frac{w_0^2}{w^2}\,e^{-2r^{2}/w^{2}}.
}
\vglue-3truemm
\begin{figure}[ht]
\centering
\def\svgwidth{88truemm} 
\input{slike/03_Gauss_3D.pdf_tex}
\caption{Upodobitev intenzitete svetlobe v Gaussovem snopu za $z>0$ }
\label{fig:Gauss_3D}
\vglue-2truemm
\end{figure}
\begin{definition}
\vglue-3truemm
Pokaži, da je svetlobna moč v Gaussovem snopu neodvisna od $z$ in da je enaka \label{naloga-širina-snopa}
$ P = \pi w_0^2\varepsilon_0 c_0 E_0^2/4$.
\end{definition}
\vglue-3truemm
Povejmo še nekaj o parametru q(z), ki smo ga uporabili pri izračunu Gaussovega snopa v
nastavku (enačba~\ref{eq:gaussov-snop-nastavek}). Spomnimo se, da parameter $q$ narašča linearno z oddaljenostjo od grla
\vglue-10truemm
\begin{equation}
q(z) = z -iz_0.
\label{eq:q}
\end{equation}
\vglue-4truemm
Parameter $q$ imenujemo kompleksni krivinski radij\index{Kompleksni krivinski radij} in
\index{Gaussov snop!kompleksna ukrivljenost}\index{Gaussov snop!kompleksni krivinski radij}
njegov inverz kompleksna ukrivljenost\index{Kompleksna ukrivljenost}
\boxeq{eq:q-inv}{
\frac{1}{q}=\frac{1}{R}+i\frac{2}{kw^{2}}.
}
Pri tem so $q$, $R$ in $w$ seveda funkcije koordinate $z$.
Kot bomo videli v nadaljevanju, je kompleksni krivinski radij 
zelo uporaben pri obravnavi preslikav Gaussovih snopov z lečami.
\vglue-0truemm
\begin{definition}
\vglue-3truemm
Uporabi enačbi~(\ref{eq:z0}) in (\ref{eq:R}) in izpelji enačbo za $q$ (enačba~\ref{eq:q-inv}).
\end{definition}

\section{Snopi višjega reda}
Osnovna rešitev obosne valovne enačbe (enačba~\ref{eq:obosna-valovna-enacba}) 
je Gaussov snop (enačba~\ref{eq:gaussov-snop}), ki ga imenujemo tudi snop 
TEM$_{00}$\footnote{~TEM -- {\it Transverse Electromagnetic Mode}, 
transverzalno elektromagnetno valovanje.}.\index{TEM$_{00}$} 
Poleg te osnovne rešitve obstaja še veliko drugih rešitev, ki so prav tako omejene v prečni smeri. 
V kartezičnih koordinatah rešijo obosno valovno enačbo
\index{Hermite-Gaussovi snopi}Hermite-Gaussovi snopi, imenovani tudi 
TEM$_{n,m}$\index{TEM$_{n,m}$}\footnote{~Glej npr. A. Yariv in P. Yeh, {\it Photonics}, šesta izdaja, Oxford
University Press (2007).}
\begin{equation}
\psi_{n,m}(x,y)=\frac{w_{0}}{w}H_{n}\left(\frac{\sqrt{2}x}{w}\right)H_{m}\left(\frac{\sqrt{2}y}{w}\right)
\exp\left(\frac{ik(x^{2}+y^{2})}{2q}-i\eta_{n,m}\right).
\label{eq:Gauss-Hermitevi}
\end{equation}
Pri tem $H_{n}$ označuje Hermitove polinome stopnje $n$ ($H_0(x)=1$, $H_1(x)=2x$, 
$H_2(x)=4x^2-2$, $H_3(x)=8x^3-12x$ ...). Da so tudi ti snopi rešitve obosne valovne enačbe, 
se prepričamo, če izraz (enačba~\ref{eq:Gauss-Hermitevi}) vstavimo v obosno valovno 
enačbo\index{Obosna valovna enačba} (enačba~\ref{eq:obosna-valovna-enacba})
in upoštevamo zvezo med Hermitovimi polinomi
\begin{equation}
H_{n}^{\prime\prime}-2xH_{n}^{\prime}+2nH_{n}=0,
\end{equation}
pri čemer črtica pomeni odvod po $x$. 
Osnovni Gaussov snop je očitno poseben primer rešitve za $n=m=0$.
Polmer snopa $w(z)$ in kompleksni krivinski radij $q(z)$ sta za\index{Gouyeva faza}
vse $n$ in $m$ enaka kot za osnovni snop (enačbi~\ref{eq:w} in \ref{eq:q}). 
Razlika je v Gouyjevi fazi, ki je za snope višjega reda odvisna tudi od $n$ in $m$
\begin{equation}
\eta_{n,m}\left(z\right)=(n+m+1)\arctan\left(\frac{z}{z_{0}}\right)\!.
\end{equation}
Na sliki~\ref{fig:Gauss-Hermitevi-snopi} so intenzitetni profili 
Hermite-Gaussovih snopov višjih redov $|\psi_{n,m}(x, y, 0)|^2$.
Indeksa $n$ in $m$ določata število vozlov v prečnih smereh $x$ in $y$. Opazimo, da se 
širina snopa z 
naraščajočima $n$ in $m$ povečuje.
\vglue-2truemm
\begin{figure}[ht]
\centering
\def\svgwidth{90truemm} 
\input{slike/03_Hermite_Gauss.pdf_tex}
\caption{Prečni profil intenzitete Hermite-Gaussovih snopov v grlu 
za različne vrednosti $(n,m)$}
\label{fig:Gauss-Hermitevi-snopi}
\vglue-2truemm
\end{figure}

\begin{definition}
\label{naloga:HG}
\vglue-3truemm
Pokaži, da za Hermite-Gaussove snope višjih redov efektivni polmer snopa 
narašča sorazmerno s korenom iz števila prečnih vozlov $ w_{\mathrm{eff}}\propto w\sqrt{n+m}$.\\
Namig: pri zapisu prečne odvisnosti polja upoštevaj le vodilni člen Hermitovih polinomov 
in izračunaj, pri kateri oddaljenosti od središča snopa je amplituda polja $\psi$ največja.
\end{definition}

\begin{remark}
 Hermite-Gaussovi snopi tvorijo kompletni
ortogonalni sistem funkcij koordinat $x$ in $y$
\begin{equation}
\int\psi_{n,m}^{*}(x,y)\psi_{n',m'}(x,y)\, dx dy=\pi w_{0}^{2}\; 
2^{n+m-1}n!\;m!\; \delta_{n,n'}\;\delta_{m,m'}.
\end{equation}
Polje nekega valovanja, ki ga poznamo v ravnini $z=0$, pri
poljubnem $z$ izračunamo z razvojem po Hermite-Gaussovih snopih. Pri tem
je izbira polmera grla $w_{0}$ poljubna, bo pa seveda vplivala na
hitrost konvergence razvoja. Na tak način lahko obravnavamo uklon
na odprtini, pri čemer je očitno smiselno vzeti $w_{0}$ približno enak
dimenziji odprtine. Dobljeni rezultat je enako natančen kot Fresnelov
uklonski integral.\index{Fresnelov uklon}

Pri velikih $z$, kjer velja Fraunhoferjeva uklonska teorija, je
polje Fouriereva transformiranka polja pri $z=0$. Hermite-Gaussovi
snopi ohranjajo prečno obliko, ki pa se z naraščajočim $z$ širi. 
To je v skladu s tem, da je Fouriereva transformiranka Hermite-Gaussove funkcije 
$H_{n}(x)\exp(-x^{2}/2)$ kar Hermite-Gaussova funkcija.\index{Fraunhoferjev uklon}
\end{remark}

V cilindričnih koordinatah imajo snopi višjega reda obliko Laguerre-Gaussovih 
snopov\index{Laguerre-Gaussovi snopi}\footnote{~Glej npr. A. Yariv in P.
Yeh, {\it Photonics}, šesta izdaja, Oxford
University Press (2007).}
\begin{equation}
\psi_{p,l}(r,\varphi,z)=\frac{w_{0}}{w}\left(\frac{\sqrt{2}r}{w}\right)^{|l|}
L_{p}^{|l|}\left(\frac{2r^{2}}{w^{2}}\right)e^{\pm il\varphi}\exp\left(\frac{ikr^{2}}{2q}-i\eta_{p,l}\right)\!.
\label{eq:Gauss-Laguerrevi}
\end{equation}
Pri tem $L_{p}^{l}$ označujejo pridružene Laguerrove polinome oblike $L_{0}^{l}(x) = 1$, 
$L_{1}^{l}(x) = -x+l+1$, 
$L_{2}^{l}(x) = x^2/2-(l+2)x+(l+2)(l+1)/2$ ... Gouyjeva faza teh snopov je 
\begin{equation}\index{Gouyeva faza}
\eta_{p,l}\left(z\right)=(2p+l+1)\arctan\left(\frac{z}{z_{0}}\right)\!.
\label{eq:etaGL}
\end{equation}

Podobno kot v kartezičnem primeru red polinoma določa število prečnih ničel,
določa $p$ v cilindričnem primeru število vozelnih črt, kjer je gostota 
svetlobnega toka enaka nič. Na sliki~\ref{fig:Laguerrovi_presek}
je prikazanih nekaj intenzitetnih profilov Laguerre-Gaussovih snopov
višjih redov $|\psi_{p,l}(r, \varphi, 0)|^2$. Ker nastopa  odvisnost od kota
le v fazi, so intenzitetni profili snopa radialno simetrični. Opazimo, da je pri  
vseh snopih z $l \ne 0$ v središču snopa minimum. 
\begin{figure}[ht]
\centering
\def\svgwidth{100truemm} 
\input{slike/03_Laguerre_Gauss.pdf_tex}
\caption{Prečni profil intenzitete Laguerre-Gaussovih snopov v grlu 
za različne vrednosti $(p,l)$}
\label{fig:Laguerrovi_presek}
\vglue-3truemm
\end{figure}

Navadno želimo, da iz laserja izhaja snop, ki je čim bolj podoben osnovnemu
Gaussovemu snopu, vendar
pogosto opazimo tudi snope višjega reda. Da dobimo le osnovni
snop, je treba biti pri načrtovanju laserja posebej pazljiv.

\begin{remark}
Ploskve konstantne faze Laguerre-Gaussovih snopov imajo pri $l\ne0$  obliko 
vijačnic (slika~\ref{fig:Laguerrova_fronta}). 
Poyntingov vektor\index{Poyntingov vektor} 
pri njih ni vzporeden z osjo snopa, temveč ima komponento tudi v prečni smeri. Ta spreminja smer, 
zato ima snop vrtilno količino v smeri osi in snop na snov deluje z navorom. 
Pravimo, da Laguerre-Gaussovi snopi nosijo t.\ i.\ tirno vrtilno 
količino\index{Tirna vrtilna količina}.\footnote{~L. Allen, M. W. Beijersbergen, R. J. C. Spreeuw in 
J. P. Woerdman, Phys. Rev. A $\mathbf{45}$, 8185 (1992).}
V kvantni mehaniki funkcija $\psi_{p,l}$ predstavlja foton s tirno vrtilno količino $L = \hslash l$, 
medtem ko leva in desna krožna polarizacija predstavljata spin fotona. 
\begin{figure}[ht]
\centering
\includegraphics[width=10truecm]{slike/03_Laguerre_faza.png}
\caption{Ploskev konstantne faze (valovna fronta) Laguerre-Gaussovega snopa}
\label{fig:Laguerrova_fronta}
\end{figure}
\end{remark}

\section{Besslov snop}
Poglejmo še en primer omejenega snopa, to je \index{Besslov snop}Besslov
snop\footnote{~Nemški astronom, matematik in fizik Friedrich Wilhelm Bessel, 1784--1846.}. 
Nastavek za rešitev valovne enačbe (enačba~\ref{eq:valovna-skalarna}), 
pri čemer obravnavamo polje skalarno, naj bo
\begin{equation}
E=E_{0}\psi(x,y)e^{i\beta z-i\omega t}.
\end{equation}
Funkcija $\psi$ mora zadoščati Helmholtzevi enačbi\index{Helmholtzeva enačba}
(enačba~\ref{eq:Helmholtz})
\begin{equation}
\nabla_{\perp}^{2}\psi+k_{\perp}^{2}\psi=0,
\end{equation}
pri čemer je $k_{\perp}^{2}=k^{2}-\beta^{2}$. V cilindričnih
koordinatah, kjer sta $x=r\cos\varphi$ in  $y=r\sin\varphi$, se enačba prepiše v 
\begin{equation}
\frac{\partial^2 \psi}{\partial r^2}+ \frac{1}{r}\frac{\partial \psi}{\partial r}
+ \frac{1}{r^2}\frac{\partial^2 \psi}{\partial \varphi^2}+k_{\perp}^{2}\psi=0.
\end{equation}
Rešitve te enačbe so s faznim faktorjem pomnožene Besslove funkcije
\begin{equation}
\psi_m(r, \varphi)=J_{m}(k_{\perp}r)e^{im\varphi},
\end{equation}
pri čemer je $J_{m}$ zaradi zahteve po enoličnosti rešitev
celo število, $J_{m}$ pa je Besslova funkcija reda $m$. Za
$m=0$ je rešitev osnovi Besslov snop (slika~\ref{fig:Besslov_presek})
\begin{equation}
E(r,z,t)=E_{0}J_{0}(k_{\perp}r)e^{i\beta z-i \omega t}.
\label{eq:Besslov-snop}
\end{equation}
Valovne fronte osnovnega Besslovega snopa so ravne
 in njegova divergenca je enaka nič\index{Besslov snop!divergenca}, zato Besslov snop na
poljubni oddaljenosti od izhodišča ohranja svojo obliko. 

Vendar Besslov snop ni 
omejen v pravem smislu. Za velike oddaljenosti od osi snopa $r$ intenzitetni profil 
namreč pojema kot $I \propto J_{0}^{2}(k_{\perp}r)\sim (2/\pi k_{\perp}r)\cos^{2}(k_{\perp}r-\pi/4)$.
Energija takega snopa ni omejena znotraj efektivnega polmera,
kot je to pri Gaussovih snopih. Za konstrukcijo Besslovih snopov
bi (tako kot za konstrukcijo ravnega vala) potrebovali neskončno energije,
kar je seveda nemogoče. Lahko pa ustvarimo dobre približke Besslovih 
snopov, ki imajo pomembne in uporabne lastnosti. 

\begin{figure}[ht]
\centering
\def\svgwidth{60truemm} 
\input{slike/03_Bessel_profil.pdf_tex}
\caption{Prečni presek in profil intenzitete osnovnega Besslovega snopa}
\label{fig:Besslov_presek}
\end{figure}

\begin{remark}
Z uporabo stožčaste leče (aksikona) lahko Gaussov snop
preoblikujemo v približek Besslovega snopa (slika~\ref{fig:Bessel_leca}). 
Na plašču stožčaste leče se namreč Gaussov snop zlomi in valovni vektorji 
nastalega snopa opisujejo stožec, kar je sicer lastnost Besslovih snopov.
Dobljeni snop je na območju z dolžino $z_{max}$ dober približek Besslovega 
snopa.\footnote{~R. M. Herman in T. A. Wiggins, J. Opt. Soc. A $\mathbf{8}$, 932 (1991).}
Znotraj tega območja je divergenca snopa praktično enaka nič. Poleg manjše divergence
imajo ti snopi še lastnost regeneracije. To pomeni, da se snop 
za objektom, ki ga osvetljuje (na primer v optični pinceti), regenerira. 
Profil snopa na senčni strani (daleč stran od objekta) je tako enak profilu 
snopa pred objektom. 
\begin{figure}[ht]
\centering
\def\svgwidth{80truemm} 
\input{slike/03_Bessel_nastanek.pdf_tex}
\caption{Nastanek približka Besslovega snopa na stožčasti leči}
\label{fig:Bessel_leca}
\end{figure}
\end{remark}

\section{Transformacije snopov z lečami}
Vrnimo se k osnovnim Gaussovim snopom in poglejmo, kaj se zgodi z njimi pri prehodu
skozi optične elemente\index{Preslikava z lečo}. Začnemo
z enostavno tanko lečo z goriščno razdaljo $f$. 

V geometrijski optiki
je krivinski radij krogelnega vala, ki izhaja iz točke na osi, kar
enak razdalji do točke. Leča točko na optični osi preslika v točko na osi,
od koder sledi, da se krogelni val s krivinskim radijem $R_{1}$
po prehodu skozi lečo spremeni v krogelni val s krivinskim radijem $R_{2}$.
Pri tem velja zveza\footnote{~Glej npr. J. Strnad, {\it Fizika, 2. del}, osmi natis, DMFA-založništvo (2018).}
\begin{equation}
\frac{1}{R_{1}}-\frac{1}{R_{2}}=\frac{1}{f}.
\label{eq:leca}
\end{equation}
Dogovorimo se, da je krivinski radij v točki $z$ pozitiven, če je središče krožnice pri $z^{\prime}\le z$.

Kako pa je z Gaussovim snopom? Polmer snopa $w$ se pri prehodu 
skozi tanko lečo ne spremeni, zato velja po enačbah (\ref{eq:q-inv}) in 
(\ref{eq:leca}) za kompleksni krivinski radij tik pred lečo in tik za njo
\begin{equation}
\frac{1}{q_{1}}-\frac{1}{q_{2}}=\frac{1}{f}.
\label{eq:preslikava-zveza-leca}
\end{equation}
Kompleksni krivinski radij $q$ je po enačbi~(\ref{eq:q}) linearna 
funkcija koordinate $z$ in za opis Gaussovega snopa zadošča, da
v neki točki $z$ poznamo $q$. Iz realnega dela parametra $q$ določimo ukrivljenost front in iz 
imaginarnega dela polmer snopa. Enačba~(\ref{eq:preslikava-zveza-leca}) torej
zadošča za račun prehoda snopa skozi poljuben sistem leč brez aberacij, če le poznamo
njegovo goriščno razdaljo.

Kot primer poglejmo, kako s tanko zbiralno lečo zberemo Gaussov snop.
Vpadni snop naj ima grlo s polmerom $w_{01}$ in parametrom $z_{01}=\pi w_{01}^2/\lambda$. 
Grlo naj leži v točki, ki je za $x_{1}$ oddaljena od levega gorišča leče $F$ (slika
\ref{fig:Prehod-Gaussovega-snopa}). 
\begin{figure}[ht]
\centering
\def\svgwidth{145truemm} 
\input{slike/03_preslikava.pdf_tex}
\caption{Prehod Gaussovega snopa skozi
tanko lečo. Grlo s polmerom $w_{01}$ v oddaljenosti $x_{1}$ od gorišča
leče $F$ se preslika v grlo s polmerom $w_{02}$ v oddaljenosti $x_{2}$ od gorišča
leče $F$.}
\label{fig:Prehod-Gaussovega-snopa}
\end{figure}

Vpeljemo parametra
\begin{equation}
q_{1}^{F}=x_{1}-iz_{01} \qquad \mathrm{in} \qquad q_{2}^{F}=-x_{2}-iz_{02},
\label{eq:qFqF}
\end{equation}
ki predstavljata kompleksna krivinska radija v levem in desnem gorišču. Koordinatna os 
$z$ je usmerjenav desno, gledamo pa referenčno glede na lego grla vsakega posameznega snopa. 
 Za vrednosti $q$ tik pred lečo in tik za njo velja tudi
\begin{equation}
q_{1}=q_{1}^{F}+f \qquad \mathrm{in} \qquad q_{2}=q_{2}^{F}-f.
\end{equation}

Od tod z uporabo enačbe~(\ref{eq:preslikava-zveza-leca}) izpeljemo zvezo
za $q$ v goriščih v kompaktni obliki 
\boxeq{eq:qqf}{
q_{1}^{F}q_{2}^{F}=-f^{2}.
}
Uporabimo enačbi~(\ref{eq:qFqF}) in 
zapišemo posebej realni in imaginarni del
\begin{equation}
x_{1}x_{2}=f^{2}-z_{01}z_{02} \qquad \mathrm{in} \qquad 
\frac{x_{1}}{z_{01}}=\frac{x_{2}}{z_{02}}.
\end{equation}
Dobimo enačbi za preslikavo Gaussovega snopa z lečo z goriščno razdaljo $f$.
Prva enačba določa lego grla preslikanega snopa na desni strani leče
\boxeq{eq:preslikava-grlo}{
x_{2}=\frac{x_{1}f^{2}}{x_{1}^{2}+z_{01}^{2}}.
}
Druga enačba določa povečavo
\boxeq{eq:preslikava-povecava}{
\frac{w_{02}}{w_{01}}=\sqrt{\frac{x_{2}}{x_{1}}}.
}
Enačba (\ref{eq:preslikava-grlo}) se ujema z izrazom za preslikavo točke v geometrijski
optiki le, kadar je $z_{01}\ll x_{1}$. Kadar je $z_{01}\gg f$, je
val na leči skoraj raven in $x_2 \to 0$, kar pomeni, da leži
grlo na desni strani v gorišču. V praksi za Gaussove snope, ki izhajajo iz laserjev, pogosto ne
velja ne prva ne druga limita, zato je treba uporabiti zapisani izraz 
(enačba~\ref{eq:preslikava-grlo}).
Tudi velikost polmera grla na desni, podana z enačbo (\ref{eq:preslikava-povecava}),
je precej drugačna kot v geometrijski optiki.

Za primer vzemimo snop iz He-Ne laserja\index{Laser!He-Ne} z valovno dolžino 
$633~\si{\nano\metre}$. Grlo snopa naj leži na izhodu iz laserja, njegov polmer naj bo 
$w_{01}=0,5~\si{\milli\metre}$. Za tak snop je $z_{01}=124~\si{\centi\metre}$. 
Na oddaljenost $50~\si{\centi\metre}$ od grla postavimo lečo z goriščno razdaljo 
$f=25~\si{\centi\metre}$. Po enačbi (\ref{eq:preslikava-grlo})
leži grlo za lečo v oddaljenosti $1~\si{\centi\metre}$ od gorišča in torej $26~\si{\centi\metre}$ 
za lečo. Izračunani polmer je po enačbi~(\ref{eq:preslikava-povecava})
$w_{02}=100~\si{\micro\metre}$. Enačbe geometrijske optike bi
dale popolnoma napačen položaj grla $50~\si{\centi\metre}$ za lečo in polmer grla  
$0,5~\si{\milli\metre}$. Po drugi strani bi približek, da je vpadni
snop kar raven, dal grlo na desni v gorišču s približno pravim polmerom. Zakaj da približek ravnih 
valov pravilnejši rezultat, hitro uvidimo, če pogledamo
Rayleighovo dolžino snopa $z_{01}$: 
snop vpada na lečo v območju bližnjega polja ($x_1 + f < z_{01}$), znotraj katerega
ima približno obliko ravnih valov. 

Če postavimo gorišče leče v grlo snopa ($x_{1}=0$), je grlo na
desni strani tudi v gorišču ($x_{2}=0$). Razmerje polmerov grl
na eni in drugi strani leče izračunamo
\begin{equation}
\lim_{x_1 \to 0}~\frac{x_{2}}{x_{1}}=\frac{f^{2}}{z_{01}^{2}} \qquad \textrm{in} \qquad
\frac{w_{02}}{w_{01}}= \frac{f}{z_{01}}.\index{Gaussov snop!grlo}
\end{equation}
Velikost grla na desni strani je potem
\begin{equation}
w_{02}=\frac{\lambda f}{\pi w_{01}}.
\end{equation}
Če želimo doseči kar se da majhno grlo $w_{02}$ po prehodu skozi lečo, mora biti polmer
vpadnega snopa $w_{01}$ kar se da velik. Vpadni snop je tako smiselno
razširiti, vendar polmer snopa ne more biti večji od polmera leče $a$. 
Najmanjša velikost grla, ki jo še lahko dosežemo z zbiralno lečo, je tako 
\boxeq{eq:wmin}{
w_{02~\textrm{min}} = \frac{\lambda f}{\pi a}.
}
Dobri mikroskopski in fotografski
objektivi dosegajo $f/a\simeq 1$, zato je mogoče z njimi Gaussov snop
zbrati v piko velikosti $\sim\lambda$. 

Omenili smo, da je treba za dosego majhnega polmera grla za lečo snop pred lečo čim bolj razširiti.
Razširitev vpadnega snopa naredimo s teleskopom~(slika~\ref{fig:Prehod-Gaussovega-snopa-teleskop}),
pri katerem je razmik med lečama enak vsoti goriščnih razdalj leč in zato gorišči
sovpadata. Povečava teleskopa je pri taki postavitvi enaka razmerju med goriščnima razdaljama leč
(glej nalogo~\ref{teleskop}).

\begin{figure}[ht]
\def\svgwidth{145truemm} 
\input{slike/03_teleskop.pdf_tex}
\caption{Prehod Gaussovega snopa
skozi teleskop iz leč z goriščnima razdaljama $f_{1}$ in $f_{2}$}
\label{fig:Prehod-Gaussovega-snopa-teleskop}
\end{figure}

\begin{definition}
\label{teleskop}
Dve leči z goriščnima razdaljama $f_{1}$ in $f_{2}$ naj bosta na medsebojni
razdalji $d=f_{1}+f_{2}$. Pokaži, da je povečava takega teleskopa enaka  
\begin{equation}
\frac{w_{02}}{w_{01}}=\frac{f_{2}}{f_{1}}.
\label{eq:povecava-teleskop}
\end{equation}
\end{definition}

\section{Matrične (ABCD) preslikave v geometrijski optiki}
\label{chap:ABCDgeo}
Preden se lotimo splošnejšega zapisa preslikav Gaussovega snopa, 
se spomnimo, kako obravnavamo preslikave v geometrijski 
optiki.\footnote{~Glej npr. G. R. Fowles, {\it Introduction to Modern Optics}, 
druga izdaja, Dover Publications (1975).}
Slika nastane kot presečišče žarkov,
ki izhajajo iz točke predmeta pred optičnim sistemom. Žarek 
je pravokoten na valovne ploskve, pri čemer moramo vzeti še limito zelo majhne
valovne dolžine. Ukrivljenost valovne fronte je neposredno
povezana s spreminjanjem naklona žarkov, pri čemer bomo privzeli, da so 
nakloni žarkov glede na optično os (os $z$) majhni.

Žarek v izbrani ravnini $z$ opišemo z dvema parametroma: 
oddaljenostjo $y$ od optične osi in naklonom $\vartheta$ glede na optično os sistema. 
Ti dve količini sta med seboj neodvisni in ju sestavimo v vektor
\begin{equation}
\left[\begin{array}{c}
y\\
\vartheta
\end{array}\right].
\end{equation}
Preslikavo žarka potem zapišemo kot matriko, ki deluje na vpadni vektor in ga preslika
v izhodni vektor (slika~\ref{fig:K-matricni-obravnavi})
\begin{equation}
\left[\begin{array}{c}
y_2\\
\vartheta_2
\end{array}\right] = M \left[\begin{array}{c}
y_1\\
\vartheta_1
\end{array}\right]\!.
\end{equation}

\begin{figure}[ht]
\centering
\centering
\def\svgwidth{100truemm}
\input{slike/03_k_preslikavam.pdf_tex}
\caption{Preslikave žarkov lahko obravnavamo
z matrikami. Žarek zapišemo kot vektor $(y,\vartheta)$, optični element pa opišemo z matriko $M$, 
ki žarek $(y_{1},\vartheta_{1})$ preslika v $(y_{2},\vartheta_{2})$.}
\label{fig:K-matricni-obravnavi}
\end{figure}

Matrike $M$ so na splošno oblike
\begin{equation}
M = \left[\begin{array}{cc}
A & B\\
C & D
\end{array}\right]\!,
\label{eq:ABCDdef}
\end{equation}
zato jih imenujemo matrike ABCD\index{Matrike ABCD}. Poglejmo nekaj primerov. 

Pri premiku za $d$ vzdolž optične osi (osi $z$) se zaradi končnega naklona 
spremeni odmik od osi, naklon pa ostane enak
\begin{equation}
\left[\begin{array}{c}
y_{2}\\
\vartheta_{2}
\end{array}\right]=\left[\begin{array}{c}
y_{1}+d\vartheta_{1}\\
\vartheta_{1}
\end{array}\right]\!.
\end{equation}
To zapišemo v matrični obliki
\begin{equation}
\left[\begin{array}{c}
y_{2}\\
\vartheta_{2}
\end{array}\right]=\left[\begin{array}{cc}
1 & d\\
0 & 1
\end{array}\right]\cdot\left[\begin{array}{c}
y_{1}\\
\vartheta_{1}
\end{array}\right]\!.
\end{equation}
Matrika za premik $d$ vzdolž optične osi je tako
\begin{equation}
M= \left[\begin{array}{cc}
1 & d\\
0 & 1
\end{array}\right]\!.
\label{eq:MABCD1}
\end{equation}
Poglejmo še matriko za prehod skozi lečo. 
Pri prehodu skozi tanko lečo se spremeni nagib žarka. Če je žarek pred
lečo vzporeden z optično osjo, gre za lečo skozi gorišče, zato velja
\begin{equation}
\left[\begin{array}{c}
y_{2}\\
\vartheta_{2}
\end{array}\right]=\left[\begin{array}{c}
y_{1}\\
-\frac{y_{1}}{f}
\end{array}\right]=\left[\begin{array}{cc}
1 & B\\
-\frac{1}{f} & D
\end{array}\right]\cdot\left[\begin{array}{c}
y_{1}\\
0
\end{array}\right]\!.
\end{equation}
Pri tem koeficientov $B$ in $D$ še ne poznamo. Določimo jih iz drugega pogoja, 
ki pravi, da se žarek, ki gre skozi lečo na osi, ne spremeni
\begin{equation}
\left[\begin{array}{c}
y_{2}\\
\vartheta_{2}
\end{array}\right]=\left[\begin{array}{c}
0\\
\vartheta_{1}
\end{array}\right]=\left[\begin{array}{cc}
1 & B\\
-\frac{1}{f} & D
\end{array}\right]\cdot\left[\begin{array}{c}
0\\
\vartheta_{1}
\end{array}\right]\!.
\end{equation}
 Sledi $B=0$ in $D=1$. Matrika za prehod skozi tanko lečo je tako 
\begin{equation}
M= \left[\begin{array}{cc}
1 & 0\\
-\frac{1}{f} & 1
\end{array}\right]\!.
\label{eq:MABCD2}
\end{equation}
Podobno izpeljavo kot za prehod skozi lečo naredimo za odboj na krogelnem zrcalu
s krivinskim radijem $R$. Pripadajoča matrika je 
\begin{equation}
M=\left[\begin{array}{cc}
1 & 0\\
-\frac{2}{R} & 1
\end{array}\right],
\end{equation}
pri čemer je $R>0$ za konkavna zrcala. Matrika za odboj na ravnem zrcalu je identiteta.

Matriko sestavljene optične naprave zapišemo kot produkt matrik posameznih komponent, pri čemer
ne smemo pozabiti na premike med posameznimi elementi. Paziti moramo tudi na 
vrstni red množenja, saj množenje matrik ni komutativno.
Matriko preslikave z dvema optičnima elementoma, pri čemer žarek 
najprej preide element z indeksom 1 in nato element z indeksom 2, zapišemo kot 
\begin{align}
\left[\begin{array}{cc}
A & B\\
C & D
\end{array}\right] & =  \left[\begin{array}{cc}
A_{2} & B_{2}\\
C_{2} & D_{2}
\end{array}\right]\cdot\left[\begin{array}{cc}
A_{1} & B_{1}\\
C_{1} & D_{1}
\end{array}\right]\!.
\end{align}
V sistemu z več elementi zapišemo produkt matrik za vse elemente, 
vendar ne smemo pozabiti na premike med posameznimi elementi.

Poglejmo primer. Žarek naj najprej prepotuje razdaljo $d$, nato ga usmerimo
na tanko lečo z goriščno razdaljo $f$. Matrika za celoten prehod je
\begin{align}
\left[\begin{array}{cc}
A & B\\
C & D
\end{array}\right] & =  \left[\begin{array}{cc}
1 & 0\\
-\frac{1}{f} & 1
\end{array}\right]\cdot\left[\begin{array}{cc}
1 & d\\
0 & 1
\end{array}\right] =  \left[\begin{array}{cc}
1 & d\\
-\frac{1}{f} & -\frac{d}{f}+1
\end{array}\right].
\label{eq:Mdf}
\end{align}
\begin{remark}
Opisani matrični formalizem je zelo prikladen predvsem za računanje prehoda
svetlobe skozi zapletene optične sisteme, saj ga je prav lahko izvesti z računalnikom. Poleg
tega je enolično povezan z matričnim formalizmom za izračun
kompleksne ukrivljenosti Gaussovih snopov, zato omogoča preprost prenos 
rezultatov geometrijske optike v optiko Gaussovih snopov.
\end{remark}

\section{Linearne racionalne transformacije kompleksnega krivinskega radija}
\label{chap:ABCD}
Poskusimo zapisati podoben formalizem za kompleksne krivinske
radije. Za opis Gaussovega snopa zadošča, da v izbrani ravnini $z$ poznamo kompleksni
krivinski radij $q$. Vemo, da je $q$ linearna funkcija
premika po $z$ (enačba~\ref{eq:q}). Pri premiku iz ravnine $z_1$ v ravnino $z_2$, ki je od prve oddaljena
za $d$, se $q$ spremeni 
\begin{equation}
q_2=q_1+d.
\label{eq:abcdq1}
\end{equation}
Vemo tudi, kako se $q$ spremeni pri prehodu skozi tanko lečo (enačba~\ref{eq:preslikava-zveza-leca})
\begin{equation}
q_2=\frac{q_1f}{f-q_1}=\frac{q_1}{-\frac{q_1}{f}+1}.
\label{eq:abcdq2}
\end{equation}
Premik in leča dasta skupaj
\begin{equation}
q_2=\frac{q_1+d}{-\frac{q_1+d}{f}+1}=\frac{q_1+d}{-\frac{q_1}{f}-\frac{d}{f}+1}.
\label{eq:premikleca}
\end{equation}
V vseh treh primerih smo transformacijo kompleksnega krivinskega radija 
$q$ zapisali v obliki ulomljene linearne oziroma M\"obiusove 
preslikave
\boxeq{eq:ulomljena-preslikava}{
q_2=\frac{Aq_1+B}{Cq_1+D}.
}
Ko koeficiente preslikave razvrstimo v matriko 
\begin{equation}
M= \left[\begin{array}{cc}
A & B\\
C & D
\end{array}\right]
\end{equation}
in iz enačb (\ref{eq:abcdq1}, \ref{eq:abcdq2} in \ref{eq:premikleca}) razberemo 
koeficiente matrik \index{Matrike ABCD} za opisane 
preslikave, vidimo, da so povsem enaki koeficientom matrik ABCD, ki jih poznamo iz
geometrijske optike (enačbe~\ref{eq:MABCD1}, \ref{eq:MABCD2} in \ref{eq:Mdf}). Na splošno
velja, da lahko matrike, ki jih
poznamo iz geometrijske optike, uporabimo tudi za izračun preslikave snopov,
vendar v tem primeru elemente matrike razvrstimo v ulomljeno linearno preslikavo.

Omenimo še eno lastnost matrik ABCD. Kadar po prehodu skozi optične elemente snop svetlobe 
preide v snov z enakim lomnim količnikom, kot je bil na začetku, je determinanta matrike 
ABCD enaka 1. V nasprotnem
primeru je determinanta matrike enaka razmerju lomnih količnikov začetne in končne snovi
\begin{equation}
\det(M) = AD-BC = \frac{n_1}{n_2}.
\label{eq:detabcd}
\end{equation}
\newpage
\begin{table}[ht]
 \centering
  \begin{tabular}{|c|c|c|} \hline
  Opis prehoda & Skica & Matrika za prehod \\ \hline   
      Prehod skozi prostor za $d$ & \parbox[c]{3cm}{\def\svgwidth{3cm}\input{slike/03_matrika_d.pdf_tex}} & 
      $\begin{bmatrix} 1 & d\\  0 & 1 \end{bmatrix}$ \\ \hline

      Prehod skozi mejo dveh snovi & \parbox[c]{3cm}{\def\svgwidth{3cm}\input{slike/03_matrika_n.pdf_tex}} & 
      $\begin{bmatrix} 1 & 0\\ 0 & \frac{n_{1}}{n_{2}} \end{bmatrix}$ \\ \hline
      
      Prehod skozi konveksno ukrivljeno mejo $R>0$ & \parbox[c]{3cm}{\def\svgwidth{3cm}\input{slike/03_matrika_nR.pdf_tex}} & 
      $\begin{bmatrix} 1 & 0\\ \frac{(n_{1}-n_{2})}{n_{2}R} & \frac{n_{1}}{n_{2}} \end{bmatrix}$ \\ \hline
      
      Prehod skozi konveksno lečo $f>0$ & \parbox[c]{3cm}{\def\svgwidth{3cm}\input{slike/03_matrika_f.pdf_tex}} & 
      $\begin{bmatrix} 1 & 0\\ -\frac{1}{f} & 1 \end{bmatrix}$ \\ \hline
      
      Odboj na konkavnem zrcalu $R>0$ & \parbox[c]{3cm}{\def\svgwidth{3cm}\input{slike/03_matrika_R.pdf_tex}} & 
      $\begin{bmatrix} 1 & 0\\ -\frac{2}{R} & 1 \end{bmatrix}$ \\ \hline    
  \end{tabular}
  \caption{Matrike ABCD za nekaj osnovnih preslikav, ki veljajo tako v 
	  geometrijski optiki kot za izračun preslikave kompleksnega
	  krivinskega radija $q$ Gaussovih snopov}
\label{fig:Matrike-za-preslikave}
\end{table}

\begin{definition}
Pokaži, da za naslednje prehode veljajo ustrezne matrike ABCD.
\vglue2truemm
\begin{tabular}{|c|c|c|} \hline 
      Prehod skozi prostor in lečo & \parbox[c]{3cm}{\def\svgwidth{3cm}\input{slike/03_matrika_df.pdf_tex}} & 
      $\begin{bmatrix} 1 & d\\ -\frac{1}{f} & 1-\frac{d}{f} \end{bmatrix}$ \\ \hline
      \parbox[c]{12em}{Prehod skozi lečo z debelino $d$ 
      in krivinskima radijema $R_1$ in $R_2$, kjer sta $f_{i}=R_{i}/(n-1)$}& 
      \parbox[c]{3cm}{\def\svgwidth{3cm}\input{slike/03_matrika_fd.pdf_tex}} & 
      $\begin{bmatrix} 1-\frac{d}{nf_{1}} & \frac{d}{n}\\
      -\frac{1}{f_2}- \frac{1}{f_1}+\frac{d}{nf_1f_2}& 1-\frac{d}{nf_{2}} \end{bmatrix}$
      \\ \hline
      Prehod skozi zaporedje plasti & \parbox[c]{3cm}{\def\svgwidth{3cm}\input{slike/03_matrika_nN.pdf_tex}} & 
      $\begin{bmatrix} 1 & \sum_{i=1}^{N}\frac{d_{i}}{n_{i}}\\ 0 & 1 \end{bmatrix}$ \\ \hline 
\end{tabular}
\end{definition}

%Final

%-------------------------------------------------------------------------------
%	CHAPTER 4
%-------------------------------------------------------------------------------

\input{Chapter_04_Resonatorji}
%Final

%-------------------------------------------------------------------------------
%	CHAPTER 5
%-------------------------------------------------------------------------------

\chapterimage{slike/Trinivojski.png} 
\chapter{Interakcija svetlobe s snovjo}

V prejšnjih poglavjih smo obravnavali svetlobo v praznem prostoru. Oglejmo si
zdaj osnovne procese interakcije svetlobe s snovjo. To je seveda zelo
obširna tema in jo bomo obdelali le v obsegu, potrebnem za
razumevanje ojačevanja svetlobe s stimulirano emisijo, kar je osnova za
delovanje laserjev. Najprej bomo na kratko pogledali termodinamsko ravnovesje 
svetlobe v stiku s toplotnim zalogovnikom, torej sevanje črnega telesa, ki 
zahteva kvantno obravnavo elektromagnetnega polja. Nato bomo vpeljali fenomenološki
Einsteinov opis mikroskopskih procesov absorpcije, spontane in stimulirane
emisije ter pokazali, da ti procesi niso neodvisni. Izpeljali bomo
izraze za absorpcijski koeficient in koeficient ojačenja. Na koncu poglavja
bomo nakazali še kvantnomehansko izpeljavo verjetnosti za prehod
atoma iz višjega energijskega stanja v nižje s sevanjem.

\section{Kvantizacija elektromagnetnega polja}
\index{Kvantizacija polja}
Ravni valovi\index{Ravni val} so enostavne in zelo prikladne rešitve valovne 
enačbe~(enačba~\ref{eq:valovna-skalarna}), po katerih navadno razvijemo elektromagnetno polje. Razvoj
lahko naredimo po celotnem prostoru, vendar je tedaj nekoliko nerodna normalizacija. 
Če se omejimo na le del prostora, se temu problemu izognemo. Izbrani del prostora 
mora biti dovolj velik, da končni rezultat ni odvisen od izbire 
njegove velikosti in oblike.

Najpreprosteje je vzeti votlino v obliki velike kocke s stranico
$L$ in idealno prevodnimi stenami. Rešitve Maxwellovih enačb~(enačbe~\ref{eq:Maxwell1}--\ref{eq:Maxwell4}) 
znotraj take votline so ob upoštevanju robnih pogojev 
(enačbe~\ref{eq:robni-pogoji} in \ref{eq:robni-pogoji5}) 
\index{Stoječe valovanje}stoječa valovanja. Zapišemo jih v obliki
\begin{align}
E_{x} & =  E_{x0}\cos\left(\frac{\pi lx}{L}\right)\sin\left(\frac{\pi my}{L}\right)\sin\left(\frac{\pi nz}{L}\right)e^{-i\omega t},\nonumber \\
E_{y} & =  E_{y0}\sin\left(\frac{\pi lx}{L}\right)\cos\left(\frac{\pi my}{L}\right)\sin\left(\frac{\pi nz}{L}\right)e^{-i\omega t},\nonumber \\
E_{z} & =  E_{z0}\sin\left(\frac{\pi lx}{L}\right)\sin\left(\frac{\pi my}{L}\right)\cos\left(\frac{\pi nz}{L}\right)e^{-i\omega t},
\label{eq:stojece_votlina}
\end{align}
pri čemer so $l,m$ in $n$ cela števila. Vsako stoječe valovanje je določeno z valovnim 
vektorjem\index{Valovni vektor}\index{Valovno število}
\begin{equation}
\mathbf{k}=\left(\frac{\pi l}{L},\frac{\pi m}{L},\frac{\pi n}{L}\right)\!,
\end{equation} 
katerega velikost je povezana s krožno frekvenco $|\mathbf{k}|= k = \omega/c$.
Iz Maxwellove enačbe za prazen prostor $\nabla\cdot\mathbf{E}=0$ (enačba~\ref{eq:Maxwell3})
sledi $\mathbf{k}\cdot\mathbf{E}=0$. 
Za vsako trojico števil $l$, $m$ in $n$ obstajata tako dve
neodvisni polarizaciji.

\begin{definition}
\vglue-2truemm
 Pokaži, da stoječe valovanje, zapisano z enačbami~(\ref{eq:stojece_votlina}), reši 
 valovno enačbo (enačba~\ref{eq:valovna-skalarna}) v 
 kocki s stranico $L$ in zadosti robnim pogojem idealno prevodnih sten votline.
\end{definition}

Preštejmo, koliko je stoječih valovanj oziroma lastnih nihanj 
v intervalu velikosti valovnega
vektorja med $k$ in $k+dk$ -- to smo na hitro naredili že pri obravnavi
resonatorjev (enačba~\ref{eq:N-stevilo-stanj}). Mogoči valovni vektorji tvorijo tridimenzionalno
mrežo v prvem oktantu prostora vseh valovnih vektorjev. Razmik med
dvema zaporednima mrežnima točkama v smeri ene od osi je $\pi/L$.
Število točk v osmini krogelne lupine med $k$ in $k+dk$ je za dovolj
velike $l$, $m$ in $n$ enako prostornini lupine, deljeni
s prostornino, ki pripada posamezni mrežni točki, to je $(\pi/L)^{3}$.
Upoštevati moramo še, da sta pri vsakem $\mathbf{k}$ dovoljeni dve polarizaciji, zato
\begin{equation}
dN=\left(\frac{L}{\pi}\right)^{3}\pi k^{2}\, dk.
\label{4.2}
\end{equation}
Vpeljemo $V=L^3$ in zapišemo število lastnih nihanj na enoto volumna
\begin{equation}
\frac{dN}{V}=\frac{ k^{2}}{\pi^{2}} dk
\label{4.3}
\end{equation}
in ga prevedemo na frekvenčno odvisnost
\begin{equation}
\frac{dN}{V}=\frac{8 \pi \nu^{2} }{c^{3}}d\nu = \frac{\omega^2}{\pi^2c^3}d\omega.
\end{equation}
Gostota stanj $\varrho (\omega)$ je število lastnih nihanj na 
frekvenčni interval\footnote{~V tem poglavju bomo ohlapno uporabljali besedo
frekvenca tudi za krožno frekvenco. Iz zapisa bo vedno jasno, za katero frekvenco gre.}
na enoto volumna votline\index{Gostota stanj}
\boxeq{4.4}{
\rho(\omega)=\frac{dN}{V d\omega}=\frac{\omega^{2}}{\pi^{2}c^{3}}.
}

Vsote po lastnih nihanjih, to je po dovoljenih vrednostih valovnega števila $k$,
z uporabo gostote stanj spremenimo v integrale po $k$ ali po $\omega$
\begin{equation}
\sum_{k}\ldots \quad \Rightarrow \quad V\int\rho(k)\ldots dk=V\int\rho(\omega)\ldots d\omega.
\label{4.5}
\end{equation}
Označimo brezdimenzijski krajevni del rešitve~(enačbe~\ref{eq:stojece_votlina}) z 
$\mathbf{E}_{\alpha}$, pri čemer $\alpha$\index{Električno polje!jakost}
\index{Magnetno polje!gostota}
označuje trojico števil $l$, $m$ in $n$ ter še obe polarizaciji. Iz
Maxwellove enačbe (enačba~\ref{eq:Maxwell2}) izračunamo pripadajoče magnetno polje
\begin{equation}
i\omega_\alpha\mathbf{B}_{\alpha} = \nabla\times\mathbf{E}_{\alpha}.
\label{Maxalfa}
\end{equation}
Polja $\mathbf{E}_{\alpha}$ in $\mathbf{B}_{\alpha}$ tvorijo kompletni ortogonalni
sistem, zato jih lahko uporabimo za razvoj poljubnega elektromagnetnega polja v 
votlini\footnote{~Glej npr. M. Fox, {\it Quantum Optics, 
An Introduction}, Oxford University Press (2006).}
\begin{align}
\mathbf{E}(\mathbf{r},t) & =  -\frac{1}{\sqrt{V\epsilon_{0}}}
\sum_{\alpha}p_{\alpha}(t)\mathbf{E}_{\alpha}(\mathbf{r}) \qquad \mathrm{in} \label{eq:pqrazvoj1}\\
\mathbf{B}(\mathbf{r},t) & =  i\sqrt{\frac{\mu_{0}}{V}}c_0\sum_{\alpha}
\omega_{\alpha}q_{\alpha}(t)\mathbf{B}_{\alpha}(\mathbf{r}).
\label{eq:pqrazvoj}
\end{align}
Izbira predfaktorjev pri koeficientih razvoja $p_\alpha$ in $\omega_\alpha q_\alpha$ bo razvidna
v nadaljevanju. Če vstavimo splošen razvoj~(enačbi~\ref{eq:pqrazvoj1} in \ref{eq:pqrazvoj}) v 
Maxwellovi enačbi~(enačbi~\ref{eq:Maxwell1} in \ref{eq:Maxwell2}), 
upoštevamo zvezo~(enačba~\ref{Maxalfa}) in njej analogno za rotor magnetnega polja, dobimo
\begin{equation}
p_{\alpha}=\dot{q}_{\alpha} \qquad \mathrm{in} \qquad 
\omega_{\alpha}^{2}q_{\alpha}=-\dot{p}_{\alpha},
\label{4.7}
\end{equation}
pri čemer pika označuje časovni odvod. Sledi 
\begin{equation}
\ddot{p}_{\alpha}+\omega_{\alpha}^{2}p_{\alpha}=0.
\label{4.8}
\end{equation}
Ta enačba da seveda pričakovano časovno odvisnost oblike $e^{-i \omega_\alpha t}$.

\begin{definition}
\vglue-2truemm
 Uporabi razvoj polja (enačbi~\ref{eq:pqrazvoj1} in \ref{eq:pqrazvoj}) 
 in iz Maxwellovih enačb izpelji
 enačbo~(\ref{4.8}).
\end{definition}
Z upoštevanjem razvoja (enačbi~\ref{eq:pqrazvoj1} in \ref{eq:pqrazvoj}) in ob ustrezni
normalizaciji izračunamo energijo 
polja ali \index{Hamiltonova funkcija}Hamiltonovo 
funkcijo (enačba~\ref{eq:gostota-energije}) in dobimo
\begin{equation}
{\cal H}=\frac{1}{2}\int\left(\epsilon_{0}E^{2}+\frac{B^{2}}{\mu_0}\right)\, 
dV=\frac{1}{2}\sum_{\alpha}\left(p_{\alpha}^{2}+\omega_{\alpha}^{2}q_{\alpha}^{2}\right)\!.
\label{4.9}
\end{equation}
Zgornji zapis (enačbi \ref{4.8} in \ref{4.9}) kaže, 
da lahko elektromagnetno polje v votlini
obravnavamo kot sistem neodvisnih enodimenzionalnih harmonskih oscilatorjev\index{Harmonski oscilator}. 
Pri tem se koeficienti razvoja $p_{\alpha}$ in $q_{\alpha}$ obnašajo kot
gibalne količine in koordinate.

Prehod v kvantno mehaniko dosežemo tako, da klasičnim spremenljivkam gibalne količine $p_{\alpha}$
in koordinate $q_{\alpha}$ priredimo operatorje $\hat{p}_{\alpha}$ in $\hat{q}_{\alpha}$,
ki morajo zadoščati komutacijskim pravilom 
\begin{equation}
[\hat{q}_{\alpha},\hat{p}_{\beta}]=i\hslash \delta_{\alpha, \beta}.
\label{4.10}
\end{equation}

Iz kvantne mehanike vemo, da so lastne vrednosti energije posameznega harmonskega oscilatorja, 
opisanega s Hamiltonovo funkcijo (enačba~\ref{4.9}), diskretne.\footnote{~Glej npr. J. Strnad, 
{\it Fizika 3. del}, tretja izdaja, DMFA-založništvo (2018).} Njihove vrednosti so 
enake
\boxeq{4.11}{
W_{n}=\hslash\omega\left(n+\frac{1}{2}\right)\!\!, \quad \mathrm{pri~\check{c}emer~je} \quad n= 0, 1, 2 \ldots
}
{\bf Razliki energije harmonskega oscilatorja, če se \textit{\textbf{n}}
spremeni za 1, pravimo foton.}\index{Foton} Energija
fotona je enaka $\hslash \omega$, $n$ pa predstavlja število fotonov. Celotna
energija kvantiziranega elektro\-magnetnega polja v votlini je vsota prispevkov
posameznih oscilatorjev, pri čemer ničelno energijo zanemarimo
\begin{equation}
W = \sum_\alpha \hslash \omega_\alpha n_\alpha.
\end{equation}
Tudi v nadaljevanju bomo ničelno energijo izpuščali, 
saj je to energija osnovnega stanja, ki se ne more sprostiti. 
\vglue-2truemm
\begin{remark}
Vidna svetloba z valovno dolžino $500~\si{\nano\metre}$ ima frekvenco
$\nu = 6 \times 10^{14}~\si{\hertz}$. Ustrezna energija fotona je
$W = 4 \times 10^{-19}~\si{\joule}$ oziroma $W = 2,5~\mathrm{e}\si{\volt}$.
To vrednost si velja zapomniti.
\end{remark}

\section{Sevanje črnega telesa}
Obravnavajmo sevanje v votlini, ki je v toplotnem ravnovesju s stenami s temperaturo
$T$. Izberemo lastno nihanje votline s krožno frekvenco $\omega$. Iz statistične fizike 
vemo, da verjetnost $P$, da je v izbranem  nihanju število fotonov enako $n$, 
zapišemo z Boltzmannovo porazdelitvijo\index{Boltzmannova porazdelitev}\footnote{~Glej
npr. I. Kuščer in S. Žumer, {\it Toplota}, tretji natis, DMFA-založništvo (2017).}
\begin{equation}
P(n)=\frac{e^{-W_{n}/k_BT}}{\sum_{n}e^{-W_{n}/k_BT}} = 
\frac{e^{-\beta\hslash\omega n}}
{\sum_{n}e^{-\beta\hslash\omega n}}=
e^{-\beta\hslash\omega n}\left(1-e^{-\beta\hslash\omega}\right)\!,
\label{4.12}
\end{equation}
pri čemer sta $k_B$ Boltzmannova konstanta in $\beta = 1/k_BT$. 
Povprečno število fotonov v izbranem nihanju je potem \index{Sevanje črnega telesa}
\begin{equation}
\langle n\rangle =\sum_{n}n P(n)=\frac{1}{e^{\beta\hslash\omega}-1}.
\label{4.13}
\end{equation}

Povprečno energijo izbranega lastnega nihanja zapišemo kot produkt energije fotona in 
povprečnega števila fotonov
\begin{equation}
\langle W\rangle = \hslash \omega \langle n \rangle
= \frac{\hslash \omega}{e^{\beta\hslash\omega}-1}.
\end{equation}

Ravnovesno gostoto energije elektromagnetnega polja\index{Gostota energije} v votlini na
frekvenčni interval izračunamo tako, da povprečno energijo posameznega
nihanja $\langle W \rangle$ pomnožimo z gostoto stanj $\varrho (\omega)$ 
(enačba~\ref{4.4}). Dobimo znan izraz za energijo na enoto volumna na enoto frekvence ($u$), 
to je \index{Planckov zakon}Planckov zakon\footnote{~Nemški fizik in nobelovec 
Max Karl Ernst Ludwig Planck, 1858--1947.} (slika~\ref{fig:Planck}).
Planckov zakon opiše spektralno gostoto energije svetlobe
\index{Spektralna gostota energije}, izsevane iz \index{Spekter!Planckov}
\index{Sevanje črnega telesa}črnega telesa, ki je v toplotnem ravnovesju z 
okolico s temperaturo $T$
\boxeq{eq:Planck}{
u(\omega)=\hslash\omega\langle n\rangle \rho(\omega)
=\frac{\hslash}{\pi^{2}c^{3}}\frac{\omega^{3}}{e^{\beta\hslash\omega}-1}.
}

\begin{figure}[ht]
\centering
\def\svgwidth{110truemm} 
\input{slike/05_PlanckOmega.pdf_tex}
\caption{Planckov spekter za sevanje črnega telesa pri različnih temperaturah.
Z naraščajočo temperaturo se vrh spektra pomika k višjim frekvencam in 
v območje vidne svetlobe.}
\label{fig:Planck}
\end{figure}

\section{Absorpcija, spontano in stimulirano sevanje}
\label{chap:ASSS}
Oglejmo si osnovne procese interakcije svetlobe s snovjo. Naj
bo v votlini poleg elektro\-mag\-net\-nega polja še $N$ atomov, ki se med
seboj ne motijo. Za začetek naj bodo atomi prav enostavni\index{Dvonivojski sistem}:
imajo naj le dva energijska nivoja z energijama $E_{1}$ in $E_{2}$ (slika~\ref{sl4.1}\,a),
pri čemer naj bo $E_2$ višja energija od $E_1$. Razliko med energijama nivojev zapišemo kot 
\begin{equation}
 E_2 - E_1 = \hslash \omega_0.
\end{equation}
Zaradi interakcije s poljem
atomi prehajajo iz nižjega nivoja v višjega, in obratno. Prehajanje 
med nivojema opisujejo trije procesi: 
absorpcija, spontano sevanje in stimulirano sevanje.

\begin{figure}[ht]
\centering
\def\svgwidth{145truemm} 
\input{slike/05_Dvonivojski.pdf_tex}
\caption{Shema energijskih nivojev dvonivojskega atoma 
(a) in treh vrst prehodov med njima:
absorpcija (b), spontano sevanje (c) in stimulirano sevanje (d). Črna črta označuje atomski
prehod, rdeča pa foton.}
\label{sl4.1}
\end{figure}

\subsection*{Absorpcija fotona}
Absorpcija fotona\index{Absorpcija fotona} je prehod, pri katerem se foton 
z ustrezno energijo absorbira, atom pa preide iz nižjega energijskega nivoja 
v višjega (slika~\ref{sl4.1}\,b). 
Verjetnost za prehod na časovno enoto\index{Verjetnost za prehod}, ki jo označimo z $r_{12}$, 
je sorazmerna spektralni gostoti energije polja \index{Spektralna gostota energije}$u(\omega)$, 
to je energiji na enoto volumna in frekvenčni interval, pri frekvenci prehoda $\omega_{0}$.
Sorazmernostni koeficient označimo z $B_{12}$ in 
zapišemo
\begin{equation}
r_{12}=B_{12}u(\omega_{0}).
\label{4.16}
\end{equation}
To je enostavno razumeti. Več kot je fotonov v votlini pri frekvenci prehoda, 
več fotonov se lahko absorbira in večja je verjetnost za prehod. Pri absorpciji se
seveda število fotonov v enem od lastnih nihanj polja pri frekvenci
$\omega_{0}$ zmanjša za ena.
\vglue2truemm

\subsection*{Spontano sevanje}
Ko se atom nahaja v vzbujenem stanju, ni stabilen. Prej ali slej se zato spontano vrne 
v osnovni nivo, pri čemer izseva foton. Temu pojavu pravimo spontano sevanje\index{Spontano sevanje} 
ali spontana emisija (slika~\ref{sl4.1}\,c).

Pri spontanem sevanju je foton izsevan v katerokoli stanje polja v bližini 
frekvence prehoda. Smer izsevane svetlobe je poljubna, v odsotnosti zunanjega polja
je poljubna tudi polarizacija izsevane svetlobe.
Verjetnost za prehod na časovno enoto\index{Verjetnost za prehod} označimo z $A_{21}$.
Za dovoljene prehode je vrednost $A_{21} \sim 10^6$--$10^8/\si{\second}$ in 
za prepovedane okoli $\sim 10^4/\si{\second}$. Karakteristični (naravni) 
razpadni čas\index{Razpadni čas} vzbujenega stanja vpeljemo kot
$\tau = 1/A_{21}$. 

Zaradi končnega življenjskega časa ima vzbujeno stanje končno spektralno 
širino. Če ni Dop\-pler\-je\-ve razširitve, je atomska spektralna 
črta $g$ \index{Spektralna črta} najpogosteje kar 
Lorentzove oblike\index{Spekter!Lorentzov} z vrhom pri $\omega_0$
(enačba~\ref{eq:spekter-primer})
\boxeq{4.21}{
g(\omega-\omega_0)=\frac{1}{\pi}\frac{\gamma}{(\omega-\omega_{0})^{2}+\gamma^{2}}.
}

Funkcija $g(\omega)$ je normirana, tako da velja
\begin{equation}
\int_{-\infty}^\infty g(\omega)\, d\omega=1.
\label{4.20}
\end{equation}
Za grobe ocene pogosto naredimo približek in funkcijo $g$ nadomestimo 
s pravokotnikom širine
$2\gamma$ in višine $1/2\gamma$.

\subsection*{Stimulirano sevanje}
Tretji pojav je prehod atoma iz višjega nivoja v nižjega zaradi interakcije
s poljem. Ko na vzbujen atom vpade foton, se atom vrne v osnovni nivo in pri 
tem izseva foton, ki je povsem enak vpadnemu (slika~\ref{sl4.1}\,d). 
Temu procesu pravimo stimulirano sevanje\index{Stimulirano sevanje} ali 
stimulirana emisija. Tudi verjetnost za stimuliran prehod na časovno enoto $r_{21}$ 
je sorazmerna s spektralno gostoto energije polja pri frekvenci prehoda $\omega_{0}$
\begin{equation}
r_{21}=B_{21}u(\omega_{0}).
\label{4.17}
\end{equation}
V tem primeru smo sorazmernostni koeficient označili z $B_{21}$. Kadar pride do
stimuliranega sevanja, se število atomov v vzbujenem stanju zmanjša, 
število fotonov v stanju, ki je prehod povzročilo, pa se poveča. Pri tem je 
ključnega pomena, da je foton, ki nastane pri stimuliranjem sevanju, enak vpadnemu fotonu.
Izsevana svetloba ima tako enako fazo, frekvenco, polarizacijo in smer potovanja kot 
vpadna. Tipične vrednosti parametra so $B_{21} \sim 10^{16}$--$10^{20}~\si{\metre^3/\joule\second^2}$.

Poglejmo natančneje izraza za absorpcijo
(enačba~\ref{4.16}) in stimulirano emisijo (enačba~\ref{4.17}).
Zapisani enačbi veljata le v primeru, kadar je spektralna gostota \index{Spektralna gostota energije}
elektromagnetnega polja $u(\omega)$
znotraj celotne spektralne širine prehoda $g(\omega - \omega_0)$ približno konstantna 
(slika~\ref{fig:spektri}\,a). To je gotovo res, če
obravnavamo sevanje v votlini, ki je v termičnem ravnovesju z okolico (črno telo).

\begin{figure}[ht]
\centering
\def\svgwidth{140truemm} 
\input{slike/05_Spektri.pdf_tex}
\caption{Pri izračunu verjetnosti za absorpcijo in stimulirano emisijo je 
pomembna oblika spektralne gostote vpadnega elektromagnetnega polja $u(\omega)$. V prvem primeru (a) je 
bistveno širša, v drugem (b) pa bistveno ožja od širine atomske spektralne črte $g(\omega-\omega_0)$.}
\label{fig:spektri}
\end{figure}

V splošnem primeru, ko se spekter vpadne svetlobe spreminja znotraj 
atomske spektralne črte, moramo sešteti prispevke po ozkih frekvenčnih 
intervalih. To naredimo z integralom in zapišemo
\begin{equation}
r_{12}=B_{12}\int g(\omega-\omega_0)\, u(\omega)\, d\omega.
\label{4.19}
\end{equation}
Zapis preverimo na primeru spektra črnega telesa, ki se ne spreminja 
dosti v območju prehoda. Takrat $u(\omega_0)$ postavimo pred integral in po pričakovanju
dobimo znano zvezo~(enačba~\ref{4.16}). 

Če na atome svetimo s svetlobo s spektrom, ki je ozek v primerjavi s spektralno 
širino prehoda (na primer iz laserskega resonatorja), je verjetnost za prehod 
odvisna od tega, kako blizu osrednje frekvence prehoda je frekvenca vpadne 
svetlobe (slika~\ref{fig:spektri}\,b). Naj bo  $w_{R}$ gostota energije skoraj
monokromatske vpadne svetlobe s frekvenco $\omega_R$. Verjetnost za absorpcijo na časovno 
enoto je\index{Gostota energije}
\boxeq{4.18}{
r_{12}=B_{12}g(\omega_R-\omega_0)\, w_{R}.
}

Koeficiente $A_{21}$, $B_{12}$ in $B_{21}$, s katerimi smo opisali spontano sevanje,
absorpcijo in stimulirano emisijo, je prvi vpeljal Einstein\footnote{~Nemški fizik
in nobelovec Albert Einstein, 1879--1955.}, zato jih imenujemo 
Einsteinovi koeficienti\index{Einsteinovi koeficienti}.\footnote{~A. Einstein,
Phys. Z. $\mathbf{XVIII}$, 121 (1917).} 

\subsection*{Einsteinovi koeficienti}
\label{AB}\index{Einsteinovi koeficienti}
Število atomov v določenem atomskem nivoju imenujemo 
zasedenost.\index{Zasedenost stanj} 
Ker zaenkrat obravnavamo preproste modele atomov z zgolj 
dvema nivojema, zapišemo samo dve zasedenosti\index{Dvonivojski sistem}. Naj 
bodo $N_1$ zasedenost 
nižjega nivoja, $N_{2}$ zasedenost višjega nivoja in skupno število atomov 
$N_1+N_2=N$. V prisotnosti svetlobe se število atomov v spodnjem in zgornjem 
nivoju lahko spreminja, skupno število pa se ohranja.

Obravnavajmo termično ravnovesje, ko je spekter svetlobe bistveno širši
od širine atomskega prehoda (slika~\ref{fig:spektri}\,a). Verjetnosti za prehoda potem
zapišemo z enačbama~(\ref{4.16}) in (\ref{4.17}). Zasedenost višjega nivoja $N_2$
se zmanjšuje zaradi spontanih in stimuliranih prehodov v nižji nivo ter
povečuje zaradi absorpcije. To zapišemo z enačbo
\begin{align}
\frac{dN_{2}}{dt} &= -A_{21}N_2 - r_{21}N_2 + r_{12}N_1 \nonumber \\ 
&= -A_{21}N_{2}-B_{21}u(\omega_{0})N_{2}+B_{12}u(\omega_{0})N_{1}.
\label{4.22}
\end{align}
Zaradi ohranitve skupnega števila atomov velja 
\begin{equation}
\frac{dN_{1}}{dt}=-\frac{dN_{2}}{dt}.
\end{equation}
V termičnem ravnovesju sta zasedenosti konstantni, tako da lahko zapišemo 
\begin{equation}
\frac{dN_{1}}{dt}=A_{21}N_{2}+B_{21}u(\omega_{0})N_{2}-B_{12}u(\omega_{0})N_{1}=0.
\label{4.23}
\end{equation}
Za zasedenosti $N_{1}$ in $N_{2}$ v termičnem ravnovesju velja
Boltzmannova porazdelitev\index{Boltzmannova porazdelitev}
\begin{equation}
\frac{N_{2}}{N_{1}}=e^{-\beta(E_{2}-E_{1})} = e^{-\beta \hslash \omega_0},
\label{4.25}
\end{equation}
pri čemer je $\beta=1/k_BT$. 

Izrazimo spektralno gostoto\index{Spektralna gostota energije} $u(\omega_0)$ 
iz enačbe~(\ref{4.23})
\begin{equation}
u(\omega_{0})=\frac{A_{21}}{B_{12}\frac{N_{1}}{N_{2}}-B_{21}}.
\label{4.24}
\end{equation}
Z uporabo enačbe~(\ref{4.25}) dobimo
\begin{equation}
u(\omega_{0})=\frac{A_{21}/B_{12}}{e^{\beta\hslash\omega_{0}}-B_{21}/B_{12}}.
\label{4.26}
\end{equation}
Po drugi strani vemo, da v termičnem ravnovesju spektralno gostoto energije sevanja
$u(\omega_0)$ opišemo s Planckovim zakonom~(enačba~\ref{eq:Planck}).
Iz primerjave obeh zapisov določimo zvezo med koeficientoma $A_{21}$ in $B_{12}$
\boxeq{4.27}{
A_{21}=\frac{\hslash\omega^{3}}{\pi^{2}c^{3}}\, B_{12}.
}
Poleg tega ugotovimo, da morata biti koeficienta $B_{21}$ in $B_{12}$ enaka
\boxeq{4.27a}{
B_{12}=B_{21}.
}
Koeficient pred $B_{12}$ v enačbi~(\ref{4.27}) je ravno enak gostoti stanj elektromagnetnega polja
$\rho(\omega)$ (enačba~\ref{4.4}), pomnoženi z energijo fotona $\hslash\omega$. 
Videli bomo, da to ni slučaj, saj to izhaja iz verjetnosti za prehod v kvantni 
elektrodinamiki (razdelek~\ref{chap:verjetnost}).
Pozoren bralec je lahko tudi opazil, da je z enačbo~(\ref{4.26}),
ki smo jo dobili le z uporabo Boltzmannove porazdelitve za atome, že
določena oblika Planckove formule, ne da bi kar koli rekli o fotonih.

\begin{remark}
 Zveza $B_{12}=B_{21}$ velja le v primeru nedegeneriranih stanj. V realnih sistemih
 so stanja pogosto degenerirana in je treba zvezo $B_{12}=B_{21}$ ustrezno popraviti v
\begin{equation}
\frac{B_{21}}{B_{12}} = \frac{g_1}{g_2},
\label{eq:ABdeg}
\end{equation}
pri čemer $g_{1}$ in $g_2$ označujeta degeneriranost 
stanj.\footnote{~W. T. Silfvast, {\it Laser Fundamentals}, druga izdaja, Cambridge University Press (2004).}
\end{remark}

\section{Absorpcijski koeficient}
\index{Absorpcija}
V plinu dvonivojskih atomov naj bo zasedenost osnovnega nivoja $N_1$ in zasedenost vzbujenega
$N_2$. Izbran volumen takega plina osvetlimo s snopom svetlobe s frekvenco
$\omega$, ki je blizu frekvence atomskega prehoda $\omega_{0}$. Gostota
vpadnega energijskega toka je $j=w_{\omega}c$ (enačba~\ref{eq:jcw}), 
pri čemer je $w_{\omega}$ gostota energije. Obravnavajmo primer, ko je 
spekter vpadnega snopa ozek v primerjavi s širino atomskega prehoda
(slika~\ref{fig:spektri}\,b). V tej obliki je zapis enačb sicer bolj zapleten,
a bo bolj priročen pri obravnavi laserja. Privzemimo še, da
je stanje stacionarno. 

Ko svetlobni snop vpade na tanko plast plina debeline $dz$, se gostota
energijskega toka zmanjša zaradi absorpcije in hkrati poveča zaradi 
stimulirane emisije (slika~\ref{fig:abs}). 
Spontano sevanje, ki je tudi prisotno, lahko zanemarimo, saj
je svetloba izsevana na vse strani enakomerno in le majhen del je izsevan v smeri snopa.
Sprememba energije snopa v časovni enoti je enaka razliki med 
številom absorpcij in stimuliranih prehodov v tem času, pomnoženih z 
energijo fotona\footnote{~Zaradi preglednosti tukaj pišemo obliko atomske spektralne črte kot $g$, 
pri čemer je to vrednost Lorentzove krivulje z osrednjo frekvenco $\omega_0$ pri $\omega$, 
torej $g(\omega-\omega_0)$.}
\begin{equation}
dP=r_{12}\,\frac{(N_{2}-N_{1})}{V}\,\,S dz\, \, \hslash\omega = 
\frac{(N_{2}-N_{1})}{V}\,B_{21}g w_{\omega} \, \hslash\omega \,S dz.
\label{4.28}
\end{equation}
Verjetnost za prehod smo izrazili iz enačbe~(\ref{4.18}),
$S$ označuje presek snopa in $V$ volumen plina. 
\begin{figure}[ht]
\centering
\def\svgwidth{70truemm} 
\input{slike/05_Absorpcija.pdf_tex}
\caption{Ob prehodu skozi plin se vpadna svetloba absorbira in $dj < 0$. Gostota vpadnega
toka je $j$, debelina plasti plina $dz$ in $S$ prečni presek snopa.}
\label{fig:abs}
\end{figure}

Sledi
\begin{equation}
dj=\frac{(N_{2}-N_{1})}{V}\, B_{21}g\, \hslash\omega\,\frac{j}{c}\, dz.
\label{4.29}
\end{equation}
Priročno je vpeljati presek za absorpcijo\index{Presek za absorpcijo} 
\boxeq{sigmaabs}{
\sigma(\omega)=\frac{B_{21}\, g\, \hslash\omega}{c}.
}
Z njim se izraz (\ref{4.29}) poenostavi v 
\boxeq{4.30}{
\frac{dj}{dz}=\frac{\Delta N}{V}\sigma(\omega)j,
}
pri čemer je $\Delta N = N_{2}-N_{1}$.
Navadno obravnavamo pline, ki so blizu termičnega ravnovesja. V tem primeru 
je $N_{2}<N_{1}$ in svetloba se absorbira.
Zapišemo 
\begin{equation}
\frac{dj}{j} = -\mu dz.
\label{eq:jabs}
\end{equation}
Vpeljali smo absorpcijski koeficient $\mu$ \index{Absorpcijski koeficient}
\begin{equation}
\mu=\frac{\Delta N}{V}\sigma(\omega)=
\frac{\Delta N}{V}\, B_{21}\, g\frac{\hslash\omega}{c}.
\label{eq:muabs1}
\end{equation}
Makroskopski koeficient absorpcije svetlobe v plinu atomov $\mu$ 
smo tako povezali
z Einsteinovim koefici\-entom $B_{21}$. Povejmo še, da so 
velikosti presekov za absorpcijo $\sigma \sim 10^{-24}$--$10^{-16}~\si{\metre^2}$.
\vglue-4truemm
\begin{remark}
Energija se pri absorpciji v plinu dvonivojskih atomov 
ne izgublja. Atom, ki je prešel v vzbujeno stanje, se s spontano 
emisijo vrne v osnovno. Pri tem se svetloba izseva na vse strani -- se siplje. 
\end{remark}

\section{Nasičenje absorpcije}
\label{chap:NasAbs}
Čeprav je videti izraz za zmanjševanje 
gostote svetlobnega toka pri prehodu skozi absorbirajoči plin (enačba~\ref{eq:jabs}) 
preprost, ga ni mogoče enostavno integrirati, saj je  absorpcijski koeficient 
$\mu$ odvisen od 
gostote energijskega toka $j$. Pri dovolj velikem svetlobnem toku namreč z 
absorpcijo znaten delež atomov preide v višji nivo, zato se zmanjša razlika $\Delta N$
in posledično tudi absorpcijski koeficient $\mu$. Takrat se absorpcija
v plinu nasiti in pojavu pravimo nasičenje absorpcije\index{Nasičena absorpcija}.

Naj na dvonivojski plin vpada snop monokromatske svetlobe. 
Atomi v plinu prehajajo med nivojema zaradi absorpcije, spontane in stimulirane emisije. 
Podobno kot smo zapisali termično ravnovesje v primeru
širokega spektra (enačba~\ref{4.23}), zapišemo stacionarno enačbo 
\vglue-8truemm
\begin{equation}
\frac{dN_{1}}{dt}=A_{21}N_{2}+B_{21}\,g\,\Delta N\,\frac{j}{c}=0,
\label{4.32}
\end{equation}
pri čemer smo za verjetnost za prehod uporabili
enačbo~(\ref{4.18}) in upoštevali $w=j/c$. Zasedenost nivoja $N_{2}$ izrazimo s celotnim številom atomov $N$ in razliko zasedenosti $\Delta N$
\vglue-8truemm
\begin{equation}
N_{2}=\frac{1}{2}(N_1+N_2) + \frac{1}{2}(N_2-N_1) = \frac{1}{2}N+\frac{1}{2}\Delta N.
\label{4.321}
\end{equation}
Izračunamo razliko zasedenosti 
\vglue-8truemm
\begin{equation}
\Delta N=-\frac{N}{1+2\frac{B_{21}g}{cA_{21}}j}.
\label{4.33}
\end{equation}
Pri majhni gostoti toka $j$ so praktično vsi atomi v osnovnem stanju in prispevajo
k absorpciji. Pri velikih gostotah toka  imenovalec v izrazu za $\Delta N$
močno naraste, razlika zasedenosti gre proti nič in absorpcija se zmanjšuje.
Ko drugi člen v imenovalcu (enačba~\ref{4.33}) doseže vrednost 1, pravimo, da
gostota energijskega toka doseže vrednost saturacijske gostote\index{Saturacijska gostota toka}.
Zapišemo jo kot 
\begin{equation}
j_{s}(\omega)=\frac{cA_{21}}{2B_{21}g}=
\frac{\hslash\omega^{3}}{2\pi^{2}c^{2}g},
\label{4.34}
\end{equation}
pri čemer smo upoštevali zvezo med koeficientoma $A_{21}$ in $B_{21}$
(enačba~\ref{4.27}).

Saturacijska gostota $j_s$ je torej odvisna le od krožne frekvence vpadnega valovanja 
in vrednosti $g(\omega-\omega_0)$, ki je približno obratna vrednost širine atomskega 
prehoda. Za črto z valovno dolžino $600~\si{\nano\metre}$ 
in širino $10^{8}~\si{\second}^{-1}$ znaša saturacijska gostota svetlobnega toka okoli 
$20~\si{\milli\watt/\centi\metre^2}$. Tako veliko gostoto svetlobnega toka je v tako ozkem
frekvenčnem intervalu z navadnimi svetili praktično nemogoče doseči, 
medtem ko jo z laserji z lahkoto.

Izraz za razliko zasedenosti stanj zapišemo v preglednejši obliki
\begin{equation}
\Delta N=-\frac{N}{1+j/j_{s}}.
\label{4.35}
\end{equation}
Vstavimo ga v enačbo za zmanjševanje gostote toka (enačba~\ref{4.30}) in dobimo
\boxeq{4.36}{
dj=-\frac{\mu_{0}}{1+j/j_{s}}\, j\, dz,
}
pri čemer je
\boxeq{eq:mu0abs}{
\mu_{0}=\frac{N}{V}\sigma = \frac{N\,B_{21}\,g\,\hslash\omega}{Vc}
}
absorpcijski koeficient\index{Absorpcijski koeficient} pri majhnih gostotah vpadnega toka.
Enačbo (\ref{4.36}) integriramo
\begin{equation}
\ln\frac{j}{j_{0}}+\frac{j-j_{0}}{j_{s}}=-\mu_{0}\, z.
\label{4.37}
\end{equation}
\begin{figure}[ht]
\centering
\def\svgwidth{90truemm} 
\input{slike/05_jabs.pdf_tex}
\caption{Pojemanje gostote svetlobnega toka $j$ v absorbirajočem plinu (enačba~\ref{4.37}). 
Gostota vpadnega svetlobnega toka je $j_0$, primer je narisan za $j_s/j_0=0,3$. 
Pri $j>j_s$ je absorbcija nasičena in pojemanje linearno,
pri $j<j_s$ je pojemanje eksponentno.}
\label{fig:abs2}
\vglue-3truemm
\end{figure}

Z $j_{0}$ smo označili začetno gostoto svetlobnega toka. Kadar je ta dosti
manjša od $j_{s}$, lahko drugi člen v enačbi~(\ref{4.37}) zanemarimo in gostota svetlobnega
toka eksponentno pojema (slika~\ref{fig:abs2})
\begin{equation}
j = j_0 e^{-\mu_0 z}.
\end{equation}
Pri zelo velikih vpadnih gostotah, ko je absorpcija nasičena, lahko prvi člen v izrazu 
zanemarimo in gostota svetlobnega toka pojema linearno
\begin{equation}
j=j_{0}-\mu_{0}j_{s}z.
\label{4.38}
\end{equation}
V primeru močnega vpadnega toka sta zasedenosti osnovnega in vzbujenega nivoja skoraj
enaki in absorpcijo omejuje hitrost vračanja atomov
v osnovno stanje s spontanim sevanjem. 

\section{Optično ojačevanje}
\index{Optično ojačevanje}
Do zdaj smo obravnavali prehod svetlobe skozi dvonivojski plin. V 
termičnem ravnovesju je zasedenost zgornjega nivoja manjša od zasedenosti spodnjega in 
svetloba, ki vpada na plin, se v njem absorbira. 
Če dosežemo  obrnjeno zasedenost\index{Obrnjena zasedenost},
za katero velja $N_{2}>N_{1}$, se bo snop svetlobe pri prehodu skozi plin ojačeval. Ta pojav
je osnova za delovanje laserjev. 

Stanje obrnjene zasedenosti seveda ni v termičnem ravnovesju in ga je treba vzdrževati z dovajanjem 
energije plinu -- črpanjem\index{Črpanje}.
Načinov črpanja za dosego obrnjene zasedenosti je veliko. Zaenkrat poglejmo
nekaj osnovnih mehanizmov, podrobneje jih bomo spoznali na konkretnih primerih 
laserjev (poglavje~\ref{chap:Primeri}).

V plinih je najpogostejši način črpanja vzbujanje z električnim tokom. 
Elektroni se zaletavajo v atome ali ione plina in jih vzbujajo
v višje nivoje. To povzroči obrnjeno zasedenost med
nekim parom nivojev. Tako črpanje uporabljamo na primer v argonskem laserju\index{Laser!argonski}. 

Pogost proces v plinih je tudi prenos energije med atomi s trki. V
mešanici dveh plinov, pri katerih se nek nivo enih atomov ujema po energiji z
nekim nivojem drugih atomov, lahko vzbujen atom prve vrste pri trku preda 
energijo brez sevanja atomu druge vrste, ta pa iz osnovnega stanja preide v 
ustrezen višji nivo. Če je pod tem nivojem še drugo vzbujeno stanje, katerega
življenjski čas je krajši od življenjskega časa zgornjega nivoja, pride
do obrnjene zasedenosti. Primer takega črpanja je He-Ne laser\index{Laser!He-Ne}.

V trdnih neprevodnih kristalih sta v optičnem področju absorpcija
in sevanje navadno posledica primesi.
Obrnjeno zasedenost para nivojev primesi navadno dosežemo tako, da
kristal obsevamo s svetlobo s frekvenco, ki ustreza prehodu na nek
nivo, ki leži nad izbranim parom nivojev. Tako črpanje uporabljamo na primer v Nd:YAG in Ti:safir 
laserjih.\index{Laser!Nd:YAG} \index{Laser!Ti:safir}

V polprevodniških ali diodnih laserjih \index{Laser!polprevodniški}dosežemo 
obrnjeno zasedenost med 
prevodnim in valenčnim pasom z vbrizgavanjem elektronov in vrzeli v območje spoja $p$-$n$ 
z električnim tokom v prevodni smeri. 

\section{Optično črpanje trinivojskega sistema}
Optično ojačevanje si oglejmo na najpreprostejšem modelu optičnega črpanja.
Obravnavajmo plin atomov s tremi nivoji, tako imenovani trinivojski sistem (slika~\ref{fig:3nivojski}\,a). Osnovno stanje, ki 
ga označimo z $|0\rangle$,  naj ima energijo $E_0$. Poleg tega naj imajo atomi še 
dve vzbujeni stanji z energijo $E_1$ (stanje $|1\rangle$) in energijo $E_2>E_1$
(stanje $|2\rangle$)\index{Trinivojski sistem}, tako da je energijska razlika med vzbujenima nivojema $E_2-E_1 = \hslash \omega_0$.

Na tak trinivojski plin svetimo s črpalno svetlobo, ki vzbuja atome iz osnovnega stanja 
$|0\rangle$ v stanje $|2\rangle$, pri čemer je lahko spektralna gostota $u_{p}$ črpalne 
svetlobe široka. Po plinu naj se širi še monokromatska svetloba z gostoto 
energije $w$ in frekvenco $\omega$, ki je blizu frekvence prehoda $\omega_{0}$. 
Ugotoviti želimo, pri katerih pogojih  dosežemo obrnjeno zasedenost med 
stanjema $|1\rangle$ in $|2\rangle$ ter s tem ojačevanje svetlobe s frekvenco blizu
$\omega_{0}$ (slika~\ref{fig:3nivojski}\,b).

\begin{remark}
Trinivojski laserski sistem na sliki~\ref{fig:3nivojski}\,b je pravzaprav 
poseben primer bolj realističnega štiri\-nivojskega sistema\index{Štirinivojski sistem}, 
pri katerem zgornji črpalni nivo sovpada z zgornjim laserskim nivojem. Sicer se tretji vzbujeni nivo, 
v katerega črpamo, praviloma zelo hitro prazni v drugega vzbujenega in od tam počasi v prvega vzbujenega, 
kot kaže slika~\ref{fig:3nivojski}\,c.
Obravnava štirinivojskih sistemov je bolj zapletena od obravnave trinivojskih sistemov, 
ki za opis delovanja laserjev povsem zadošča. Podrobneje bomo večnivojske sisteme 
obravnavali na konkretnih laserskih primerih (poglavje~\ref{chap:Primeri}).
\end{remark}

\begin{figure}[ht]
\centering
\def\svgwidth{130truemm} 
\input{slike/05_Trinivojski.pdf_tex}
\caption{Shema energijskih nivojev trinivojskega sistema in oznake koeficientov za prehode
med njimi (a). V plinskih laserjih je stanje obrnjene zasedenosti navadno med vzbujenima 
stanjema (b). Pogosto so laserji štiri- ali večnivojski (c).}
\label{fig:3nivojski}
\end{figure}

Zapišimo enačbe za spreminjanje zasedenosti posameznih stanj. Osnovno stanje
$|0\rangle$ se prazni zaradi absorpcije črpalne svetlobe in polni zaradi
spontanih prehodov\index{Spontano sevanje} iz stanj $|1\rangle$ in $|2\rangle$. Stimulirane
prehode iz stanja $|2\rangle$ v osnovno stanje bomo zanemarili. Zasedenost stanja $|2\rangle$ se
povečuje zaradi absorpcije s spodnjih nivojev in zmanjšuje
zaradi spontanega in stimuliranega sevanja\index{Stimulirano sevanje}. Srednje stanje se polni
s stimuliranimi in spontanimi prehodi iz stanja $|2\rangle$ in prazni
zaradi absorpcije in spontanih prehodov. Zasedbene enačbe so tako\index{Zasedenost stanj}
\begin{align}
\frac{dN_{0}}{dt} & =  -rN_0+A_{20}N_{2}+A_{10}N_{1}, \label{4.39.1}\\
\frac{dN_{1}}{dt} & =  -A_{10}N_{1}+B_{21}\,g\,w\, (N_{2}-N_{1})+A_{21}N_{2} \label{4.39.2} \qquad \mathrm{in}\\
\frac{dN_{2}}{dt} & =  rN_0-A_{20}N_{2}-A_{21}N_{2}-B_{21}\,g\,w\, (N_2-N_1).
\label{4.39}
\end{align}
Poleg tega je vsota zasedenosti enaka številu 
vseh atomov in $N_{0}+N_{1}+N_{2}=N$. Pri zapisu zasedbenih enačb smo še 
predpostavili, da je $N_0 \approx N \gg N_1, N_2$. Črpanje $B_{20}\, 
u_{p} (N_0-N_2)$, ki je praktično konstantno, smo zapisali s koeficientom $r$. 
Mehanizem črpanja 
smo tako skrili v $r$ in prav nič ni pomembno, na kakšen način 
poteka.\index{Optično črpanje}
S tem smo obravnavo posplošili z optičnega na druge sisteme črpanja. 

Zanima nas stacionarno stanje, ko so vsi trije časovni odvodi enaki nič. 
Tako iz druge enačbe sistema~(enačba~\ref{4.39.2}) sledi
\begin{eqnarray}
B_{21}\,g\,w\, N_{2}+A_{21}N_{2} = B_{21}\,g\,w\, N_{1} + A_{10}N_{1} 
\end{eqnarray}
in
\begin{eqnarray}
N_2 = \frac{B_{21}\,g\,w + A_{10}}{B_{21}\,g\,w+A_{21}}N_1.  
\end{eqnarray}
Brez škode lahko zanemarimo spontano sevanje iz stanja
$|2\rangle$ v osnovno stanje. Tako iz prve enačbe sistema~(enačba~\ref{4.39.1}) dobimo
\begin{equation}
N_1= \frac{rN}{A_{10}}
\end{equation}
in razliko zasedenosti zapišemo kot 
\begin{equation}
N_{2}-N_{1}=\left(\frac{N_2}{N_1}-1\right)N_1=\left(\frac{A_{10}-A_{21}}{A_{21}+
B_{21}g\,w}\right)\,\frac{rN}{A_{10}}.
\label{4.42}
\end{equation}
Sledi, da je zasedenost obrnjena\index{Obrnjena zasedenost}, 
kadar je $A_{10}>A_{21}$, torej kadar je
razpadni čas stanja $|1\rangle$ krajši od razpadnega časa stanja $|2\rangle$.
Tak rezultat smo seveda pričakovali.

V praktičnih primerih navadno velja $A_{10}\gg A_{21}$. Ob upoštevanju zveze $j=wc$ povežemo
razliko zasedenosti z gostoto vpadnega svetlobnega toka
\begin{equation}
N_{2}-N_{1}=\frac{rN}{A_{21}} \, \frac{1}{1+\frac{B_{21}gj}{c A_{21}}} = 
\frac{rN}{A_{21}} \, \frac{1}{1+j/j_s}.
\label{eq:3n_N}
\end{equation}
Konstante smo pospravili v saturacijsko gostoto svetlobnega toka\index{Saturacijska gostota toka} 
\begin{equation}
j_s = \frac{c A_{21}}{B_{21}g}.
\label{eq:jsatg}
\end{equation}
\vglue-3truemm
\begin{remark}
 Vidimo, da je  izraz za saturacijsko gostoto toka v trinivojskem sistemu 
 (enačba~\ref{eq:jsatg}) zelo podoben izrazu za saturacijsko gostoto v dvonivojskem
 sistemu (enačba~\ref{4.34}), razlikujeta se le
v faktorju 2. Ta razlika je posledica različnega števila nivojev, saj pogoj $N_{1}+N_{2}=N$
v trinivojskem sistemu ne velja. 
\end{remark}
Poglejmo, kaj se zgodi s svetlobo ob vpadu na plast trinivojskega plina. Naj ima vpadna
svetloba krožno frekvenco $\omega$ in gostoto svetlobnega toka $j=wc$. Račun je zelo podoben 
računu za absorpcijo (enačba~\ref{4.29}). Spremembo gostote toka na debelini $dz$ zapišemo kot
\begin{equation}
dj=\frac{(N_{2}-N_{1})}{V}\, B_{21}g\, \frac{\hslash\omega}{c}j\, dz.
\label{eq:dj}
\end{equation}
Pri tem gostota toka $j$ nastopa tudi v izrazu za razliko
zasedenosti $N_2-N_1$. Če upoštevamo enačbo~(\ref{eq:3n_N}), 
dobimo diferencialno enačbo za gostoto toka $j$
\begin{equation}
\frac{1}{j}\left(1+\frac{j}{j_{s}}\right)\, dj=G\, dz
\label{4.43}
\end{equation}
oziroma
\boxeq{eq:djG}{
dj=\frac{G}{1+j/j_{s}}\, j\, dz,
}
ki je spet zelo podobna enačbi za absorpcijo (enačba~\ref{4.36}).
Pri tem $G$ označuje koeficient ojačenja pri majhni gostoti vpadnega
toka\index{Koeficient ojačenja}. Podan je z 
\begin{equation}
G=\frac{N}{V}\frac{r}{A_{21}}\sigma=\frac{rNB_{21}\hslash\omega g}{VcA_{21}}.
\label{4.44}
\end{equation}
Rešitev diferencialne enačbe (enačba~\ref{eq:djG}) je prikazana na sliki~\ref{fig:ojacanje}. 
\begin{figure}[ht]
\centering
\def\svgwidth{80truemm} 
\input{slike/05_joja.pdf_tex}
\caption{Naraščanje gostote svetlobnega toka $j$ pri optičnem ojačevanju. 
Gostota vpadnega svetlobnega toka je $j_0$, primer je narisan za $j_s/j_0 = 20$.  Pri $j<j_s$ je naraščanje eksponentno, 
pri $j>j_s$ je zaradi nasičenja linearno.
}
\label{fig:ojacanje}
\end{figure}
Obnašanje gostote svetlobnega toka ima, tako kot pri absorpciji, dva režima. 
Pri majhnih gostotah toka $j\ll j_{s}$ je naraščanje eksponentno 
\begin{equation}
j(z)=j_{0}e^{Gz}.
\label{4.45}
\end{equation}
Pri velikih gostotah toka zaradi nasičenja gostota svetlobnega
toka narašča linearno
\begin{equation}
j(z)=j_{0}+j_{s}Gz.
\label{4.46}
\end{equation}
V tem primeru je gostota toka tako velika, da vsi atomi, ki jih
s črpanjem spravimo v najvišje stanje, preidejo v stanje $|1\rangle$
s stimuliranim sevanjem. Pri konstantnem črpanju je tedaj 
linearno naraščanje gostote toka razumljivo. 

Vrnimo se k preseku za stimulirano sevanje $\sigma$
 (enačba~\ref{4.44})\index{Presek
za stimulirano sevanje}. Opazimo, da je enak preseku za
absorpcijo (enačba~\ref{sigmaabs}) dvonivojskega sistema\index{Presek za absorpcijo}.
Odvisen je od frekvence svetlobe in sorazmeren vrednosti atomske spektralne 
črte pri frekvenci prehoda. Za He-Ne laser\index{Laser!He-Ne} 
($\lambda\,=\,633~\si{nm}$ in $\Delta\nu\,\sim\,1,5~\si{\giga\hertz}$) znaša   
$\sigma\,\sim\,10^{-16}~\si{\metre}^2$, 
za Nd:YAG laser \index{Laser!Nd:YAG} ($1064~\si{nm}$ in 
$\Delta\nu\,\sim\,150~\si{\giga\hertz}$) pa 
$\sigma\,\sim\,10^{-22}~\si{\metre}^2$.
Zaradi različnih presekov, različnih gostot atomov in različnih načinov črpanja se 
koeficienti ojačenja v večnivojskih sistemih med seboj precej razlikujejo. Tipično
ojačenje v He-Ne laserju z dolžino $L\,=\,0,5~\si{\metre}$ je 
$GL\,\sim\,1,015$ in v Nd:YAG laserju z dolžino ojačevalnega sredstva 
$L\,=\,10~\si{\centi\metre}$ $GL \sim 50$. Pri prvem laserju je 
velik presek za stimulirano sevanje, a majhna gostota atomov v obrnjeni zasedenosti. V drugem primeru močno črpanje prevlada nad majhnim presekom 
in ojačenje je veliko.\index{Optično ojačevanje}

\section{Homogena in nehomogena razširitev spektralne črte}
\label{Razsiritev}
Doslej smo privzeli, da svetijo vsi atomi v snovi
pri isti krožni frekvenci $\omega_{0}$ in z isto spektralno širino, ki smo
jo popisali s funkcijo $g(\omega-\omega_0)$ z vrhom pri $\omega_0$. Če to velja, 
je razširitev spektralne črte homogena\index{Spektralna črta!homogena razširitev}. 
Funkcija $g(\omega-\omega_0)$ je v tem primeru Lorentzove oblike\index{Spekter!Lorentzov} 
\boxeq{eq:homogenasirina}{
g_L(\omega-\omega_0)=\frac{1}{\pi}\frac{\gamma}{(\omega-\omega_{0})^{2}+\gamma^{2}},
}
s širino črte $\Delta \omega_L = 2\gamma$ (glej sliko~\ref{fig:SpekterAc}). 
Primera homogene razširitve sta naravna širina in razširitev zaradi trkov med atomi.
Homogena razširitev je pogosto večja od obratne vrednosti razpadnega časa nivoja. 
V plinu namreč prihaja do trkov, ki lahko zmotijo le fazo sevanja, ne da bi povzročili 
prehod, vendar razširijo spektralno črto. V trdni snovi homogeno razširitev 
brez prehoda povzročajo termična nihanja lokalnega polja. 

Spektralna črta je lahko razširjena tudi zato, ker svetloba, izhajajoča iz različnih
atomov, nima povsem iste frekvence. Tedaj govorimo o nehomogeni 
razširitvi\index{Spektralna črta!nehomogena razširitev}.
Najpomembnejši primer nehomogene razširitve je Dopplerjeva 
\index{Dopplerjeva razširitev} razširitev v plinu. 
Atomi plina vedno sevajo pri praktično isti frekvenci $\omega_0$, vendar jih zaradi gibanja
opazovalec v mirujočem (laboratorijskem) sistemu v skladu z Dopplerjevim pojavom 
zazna pri različnih frekvencah. 

Naj se atom giblje s hitrostjo $v$ glede na smer opazovanja. Potem opazovane krožne 
frekvence posameznih atomov zapišemo kot  
\begin{equation}
\omega=\omega_{0}-\frac{v}{c}\omega_{0}=\omega_{0}-k_{0}v.
\label{4.81}
\end{equation}
Označimo z ${\cal N}(v)$ porazdelitev gostote atomov po hitrostih, pri čemer se omejimo 
na premikanje v smeri opazovanja. V termičnem ravnovesju je ${\cal N}(v)$
Maxwellova porazdelitev\index{Maxwellova porazdelitev}\footnote{~Glej npr. J. Strnad, 
{\it Fizika 1. del}, druga izdaja, DMFA-založništvo (2016).}
\begin{equation}
{\cal N}(v)=\frac{N}{V}\left(\frac{m}{2\pi k_{B}T}\right)^{1/2}\exp\left(-\frac{mv^{2}}{2k_{B}T}\right)\!,
\label{4.82}
\end{equation}
pri čemer je $m$ masa posameznega atoma.
Porazdelitev atomov po frekvencah izračunamo tako, da hitrost izrazimo
iz enačbe~(\ref{4.81}), poleg tega funkcijo $g_{D}(\omega-\omega_0)$
normiramo. Dobimo
\boxeq{4.821}{
g_{D}(\omega-\omega_0)=\frac{c}{\omega_{0}}\left(\frac{m}{2\pi 
k_{B}T}\right)^{1/2}\exp \left(-\frac{mc^{2}}{2k_{B}T}\frac{(\omega-\omega_{0})^{2}}{\omega_{0}^2}\right)\!.
}
Dopplerjeva razširitev v plinu je torej Gaussove oblike\index{Spekter!Gaussov}.
Njena širina pri polovični 
višini\footnote{~Celotno širino na polovični višini imenujemo FWHM -- \it{Full Width at Half Maximum}.} je
\begin{equation} 
\Delta\omega_{D}=2 \sqrt{\frac{2k_{B}T \ln 2}{mc^{2}}}\omega_{0}.
\label{4.83}
\end{equation}
\vglue-1truemm
\begin{definition}
\vglue-2truemm
Izpelji obliko nehomogeno razširjene črte za primer Dopplerjeve razširitve (enačba~\ref{4.821})
in pokaži, da je širina črte podana z enačbo~(\ref{4.83}).
\end{definition}

Izračunajmo Dopplerjevo razširitev za He-Ne laser. Za prehod
atoma neona pri valovni dolžini $633~\si{nm}$ in temperaturi $300~\si{K}$ je izračunana vrednost
$\Delta\omega_{D}\,=\,8\times 10^{9}~\si{\second}^{-1}$ oziroma 
$\Delta\nu\,=\,1,4~\si{\giga\hertz}$. 
Dejanske izmerjene vrednosti širine črte za He-Ne \index{Laser!He-Ne}laser 
znašajo okoli $1,5~\si{\giga\hertz}$, kar je znatno več od naravne širine
črte ($1,2~\si{\mega\hertz}$). Še izrazitejše so razširitve zaradi nehomogenosti
v trdninskih laserjih, na primer v Nd:YAG laserju, v katerem \index{Laser!Nd:YAG}
je širina črte $\Delta \nu\,=\,150~\si{\giga\hertz}$. Nehomogena razširitev zaradi Dopplerjevega pojava v 
redkem plinu ali zaradi nehomogenosti v trdnih snoveh je tako kar nekaj redov velikosti 
večja od homogene naravne širine in razširitve zaradi trkov.\footnote{~W. T. Silfvast, {\it Laser Fundamentals}, druga izdaja, Cambridge University Press (2004).}

\begin{remark}
Pri nehomogenih razširitvah bi za natančnejši izračun morali upoštevati 
tudi naravno širino posameznega atoma. To bi zapisali s konvolucijo Lorentzove
in Gaussove funkcije ter dobili tako imenovani Voigtov 
profil, ki ga ne moremo preprosto analitično zapisati\index{Spekter!Voigtov}.
\end{remark}

\section{*Nasičenje nehomogeno razširjene absorpcijske črte}
\label{NasabsNehom}
\index{Nasičena absorpcija!nehomogeno razširjene črte}
V razdelku~\ref{chap:NasAbs} smo obravnavali nasičenje absorpcije pri homogeno 
razširjenem prehodu. Pri nasičenju absorpcije, kadar prevladuje nehomogena razširitev,
nastopijo pomembni novi pojavi.

Začnimo z dvonivojskim plinom\index{Dvonivojski sistem}, na katerega 
vpada močan snop monokromatske svetlobe s frekvenco $\omega_S$,
ki je blizu osrednje frekvence $\omega_{0}$ Dopplerjevo razširjene 
črte\index{Dopplerjeva razširitev}, in gostoto toka $j$. S svetlobo
lahko sodeluje le skupina atomov, pri kateri se Dopplerjevo premaknjena
frekvenca od $\omega_S$ ne razlikuje več kot za homogeno širino, ki
jo opisuje funkcija $g(\omega-\omega_S)$. Zato ne moremo zapisati zasedbenih
enačb za vse atome hkrati, ampak le za tiste, ki imajo hitrost med
$v$ in $v+dv$ in ki absorbirajo svetlobo pri frekvenci $\omega_{0}-kv$.

Naj bosta ${\cal N}_{1}(v)$ in ${\cal N}_{2}(v)$ hitrostni porazdelitvi
atomov v osnovnem in vzbujenem stanju. Gostota
${\cal N}_{2}(v)$ se spreminja podobno kot celotna
zasedenost v homogenem primeru (enačba~\ref{4.22})
\begin{equation}
\frac{d{\cal N}_{2}(v)}{dt}=-A{\cal N}_{2}(v) -B\, g(\omega_S-\omega_{0}+kv)
\frac{j}{c}\,
\left({\cal N}_{2}(v)-{\cal N}_{1}(v)\right)\!.
\label{4.85}
\end{equation}
Upoštevali smo, da je zaradi Dopplerjevega pojava prehod premaknjen k frekvenci
$\omega_{0}-kv$. Velja
\begin{equation}
 \frac{d{\cal N}_{2}(v)}{dt}=-\frac{d{\cal N}_{1}(v)}{dt}.
\label{4.86}
\end{equation}
Vpeljemo ${\cal N}(v)={\cal N}_{1}(v)+{\cal N}_{2}(v)$ in 
${\cal Z}(v)={\cal N}_{1}(v)-{\cal N}_{2}(v)$. Podobno kot 
v enačbi~(\ref{4.321}) zapišemo
\begin{equation}
{\cal N}_{2}(v)=\frac{1}{2}{\cal N}(v)-\frac{1}{2}{\cal Z}(v)
\end{equation}
in dobimo 
\begin{equation}
\frac{d{\cal Z}(v)}{dt}=-A{\cal Z}(v)+A{\cal N}(v)
-2B\,g(\omega_S-\omega_{0}+kv)\frac{j}{c}
{\cal Z}(v).
\label{4.87}
\end{equation}
V stacionarnem stanju je odvod enak nič in dobimo
\begin{equation}
{\cal Z}(v)=\frac{{\cal N}(v)}{1+\frac{2B}{Ac}g(\omega_S-\omega_{0}+kv)j}.
\label{4.88}
\end{equation}
 Če je gostota vpadnega svetlobnega toka majhna, lahko imenovalec razvijemo
\begin{equation}
{\cal Z}(v)\approx{\cal N}(v)\left(1-\frac{2B}{Ac}g(\omega_S-\omega_{0}+kv)j\right)\!.
\label{4.89}
\end{equation}
Porazdelitev ${\cal Z}(v)$ je podobna nemoteni porazdelitvi atomov
po hitrosti ${\cal N}(v)$, le da je pri hitrosti $v=(\omega_{0}-\omega_S)/k$
zmanjšana zaradi vpliva vpadne svetlobe (slika \ref{fig:Bennet}). 
Atomi s to hitrostjo namreč svetlobo
absorbirajo in s tem prehajajo v zgornje stanje. V porazdelitvi
atomov tako nastane vdolbina\index{Bennettova vdolbina}, ki jo imenujemo
Bennettova vdolbina\footnote{~Ameriški fizik William Ralph Bennett Jr., 1930--2008.}.
Širina vdolbine je določena
s homogeno širino prehoda, to je s funkcijo $g(\omega_S-\omega_{0}+kv)$,
globina pa z gostoto vpadnega toka~$j$.\footnote{~W. R. Bennett Jr., Phys. Rev. $\mathbf{126}$, 580 (1962).}\index{Spektralna črta!homogena razširitev}
\begin{figure}[ht]
\centering
\def\svgwidth{90truemm} 
\input{slike/05_Hole.pdf_tex}
\caption{Porazdelitev atomov v osnovnem stanju ${\cal N}_1$ po hitrosti $v$, v kateri zaradi
absorbcije svetlobe nastane Bennettova vdolbina. Podobno obliko ima 
tudi absorpcijski koeficient za testno svetlobo.}
\label{fig:Bennet}
\end{figure}

Naj na snov poleg močne vpadne svetlobe pri $\omega_S$ vpada še šibko testno valovanje pri 
frekvenci $\omega^{\prime}$. Izračunajmo absorpcijski koeficient za valovanje pri $\omega^\prime$. 
Upoštevati moramo, da k absorpciji testne svetlobe prispevajo vsi atomi, katerih
hitrost je taka, da je prehod dovolj blizu $\omega^{\prime}$. Absorpcijski 
koeficient\index{Absorpcijski koeficient} potem izračunamo s seštevanjem 
po porazdelitvi ${\cal Z}(v)$
(enačba~\ref{eq:muabs1})
\begin{equation}
\mu(\omega^{\prime})=\frac{\hslash\omega^{\prime}}{c}\int{\cal Z}(v)Bg(\omega^{\prime}-\omega_{0}+k'v)\, dv.
\label{4.90}
\end{equation}
Homogena razširitev je dosti manjša od Dopplerjeve širine, zato
v prvem približku Lorentzovo funkcijo $g$ v enačbi (\ref{4.90}) nadomestimo kar z
$\delta(\omega)$. V izrazu za $\cal Z$ (enačba~\ref{4.88}) jo pustimo. 
Tako je absorpcijski koeficient za šibko testno svetlobo 
\begin{align}
\label{eq:mumumu}
\mu(\omega^{\prime}) & =  \frac{\hslash\omega^{\prime}}{k'c}B\frac{{\cal N}
(\frac{\omega_0-\omega'}{k'})}{1+\frac{2Bj}{Ac}g(\omega_S-\omega')} \nonumber \\ 
 & \approx  \hslash B{\cal N}\left(\frac{\omega_0-\omega'}{k'}\right)\left(1-\frac{2Bj}{Ac}g(\omega_S-\omega')\right)\!.
\end{align}
V drugi vrstici smo uporabili približek (enačba~\ref{4.89}). Vidimo, da je 
odvisnost $\mu(\omega^{\prime})$ Gaussove oblike z vdolbino pri $\omega_S$ in je tako
podobna porazdelitvi, kot jo kaže slika~\ref{fig:Bennet}. Odvisnost 
$\mu(\omega^{\prime})$ lahko tudi izmerimo, tako da spreminjamo 
frekvenco testnega snopa $\omega^{\prime}$.

\begin{remark}
 Merjenje absorpcije s testnim
žarkom omogoča opazovanje oblike homogene črte kljub mnogo večji
nehomogeni Dopplerjevi razširitvi. V moderni spektroskopiji ima zato ta metoda
velik pomen.
\end{remark}

Izračunajmo še absorpcijski koeficient za prvi, močan vpadni snop, tako da v
enačbo~(\ref{eq:mumumu}) vstavimo $\omega^{\prime}=\omega_S$. Vodilni člen ${\cal N}((\omega_0-
\omega_S)/k)$ opisuje običajno Gaussovo obliko Dopplerjevo
razširjene črte, pri čemer izraz v oklepaju da zmanjšanje absorpcije
zaradi nasičenja, ki je odvisno le od vrednosti $g(0)$ in zato enako za vse $\omega_S$. 
Z enim samim vpadnim snopom svetlobe torej vdolbine v absorpciji ne moremo zaznati, saj 
je izmerjena črta kljub nasičenju Gaussove oblike. 

Namesto z dvema snopoma, od katerih šibkemu testnemu snopu spreminjamo
frekvenco, lahko vdolbino v porazdelitvi zaznamo tudi z enim samim snopom
spremenljive frekvence, ki se po prvem prehodu skozi plin odbije od
zrcala in vrne v nasprotni smeri. V porazdelitvi atomov se
v spodnjem stanju simetrično pri hitrostih $\pm(\omega_{0}-\omega_S)/k$
pojavita dve Bennettovi vdolbini (slika \ref{fig:Lamb}\,a).
Kadar je $\omega_S$ blizu $\omega_{0}$, se vdolbini vsaj delno prekrivata, 
nasičenje se poveča in v krivulji za absorpcijo svetlobe se pojavi 
vdolbina (slika \ref{fig:Lamb}\,b).\footnote{~R. A. McFarlane,
W. R. Bennet Jr. in W. E. Lamb Jr, Appl. Phys. Lett. $\mathbf{2}$, 189 (1963).}
Imenujemo jo Lambova vdolbina\footnote{~Ameriški fizik in nobelovec 
Willis Eugene Lamb Jr., 1913--2008.}\index{Lambova vdolbina}.
\begin{figure}[ht]
\centering
\def\svgwidth{140truemm} 
\input{slike/05_Lamb.pdf_tex}
\caption{Porazdelitev atomov v osnovnem stanju ${\cal N}_1$ po hitrosti $v$, kadar svetloba prehaja 
skozi plin v dveh smereh (a). Če frekvenca vpadne svetlobe $\omega_S$ približno 
sovpada z osrednjo frekvenco prehoda $\omega_0$, se vdolbini prekrivata in absorpcija $\mu$ 
se zmanjša (b).}
\label{fig:Lamb}
\end{figure}

Zapišimo enačbe še za ta primer. Vpadni snop svetlobe povzroči spremembo zasedenosti
pri prehodu skozi plin v obeh smereh, zato je zdaj 
\begin{equation}
{\cal Z}(v)\approx{\cal N}(v)\left(1-\frac{2Bj}{Ac}\left(g(\omega_S-\omega_{0}+kv)+
g(\omega_S-\omega_{0}-kv)\right)\right)\!.
\label{4.92}
\end{equation}
Podobno kot prej izračunamo absorpcijski koeficient za širjenje svetlobe v
pozitivni smeri 
\begin{align}
\mu_{+}(\omega_S) & =  \frac{\hslash\omega}{c}B\int{\cal Z}(v)g(\omega_S-\omega_{0}+kv)\, dv\nonumber \\
 & \approx  \hslash B{\cal N}\left(\frac{\omega_S-\omega_0}{k}\right)\left(1-\frac{2Bj}
 {Ac}\left(g(0)+g(2(\omega_S-\omega_{0}))\right)\right)\!.
 \label{eq:lamb}
\end{align}
Izmerjeni absorpcijski profil je odvisen od frekvence vpadne svetlobe $\omega_S$ in ima pri $\omega_0$ vdolbino,
ki je podobna homogeno razširjeni črti. Faktor 2 v argumentu
funkcije $g(2(\omega_S-\omega_{0}))$ je posledica  grobega približka,
ko smo v integraciji $g(\omega_S-\omega_{0}+kv)$ nadomestili kar z
$\delta$ funkcijo. Natančnejši račun pokaže, da je vrh pri $\omega_{0}$
oblike $g(\omega_S-\omega_{0})$.

\begin{definition}
\vglue-3truemm
Pokaži, da je rezultat natančnejše izpeljave absorpcijskega koeficienta  
\begin{equation}
 \mu_{+}(\omega_S) = \hslash B{\cal N}\left(\frac{\omega_S-\omega_0}{k}\right)\left(1-\frac{Bj}
 {Ac}\left(g(0)+g(\omega_S-\omega_{0})\right)\right)\!.
\end{equation}
Pri računu privzemi, da je širina Dopplerjeve porazdelitve bistveno večja od širine homogene
razširitve (enačba~\ref{eq:homogenasirina}), in Maxwellovo porazdelitev postavi pred integral. 
\end{definition}

\section{*Izpeljava verjetnosti za prehod}
\label{chap:verjetnost}
\index{Verjetnost za prehod}
Verjetnosti za prehod atoma iz enega stanja v drugo s sevanjem, ki
smo jih opisali s fenomenološkimi Einsteinovimi koeficienti $A_{21}$
in $B_{21}$ (razdelek~\ref{AB}), \index{Einsteinovi koeficienti}
je mogoče izpeljati tudi drugače.
Pri tem se poslužimo kvantne\index{Kvantizacija polja} elektrodinamike, 
kar pomeni kvantno obravnavo 
tako atoma kot elektromagnetnega polja. Povsem strog račun je zahteven in presega
okvir te knjige, zato si na kratko oglejmo le, kako pridemo do rezultata z uporabo
Fermijevega zlatega pravila.\index{Fermijevo zlato pravilo}\footnote{~Glej npr. M. 
Fox, {\it Quantum Optics, An Introduction}, Oxford University Press (2006).}

Postavimo dvonivojski atom v votlino z elektromagnetnim poljem.\index{Dvonivojski sistem}
Izračunajmo verjetnost, da zaradi interakcije s poljem atom
preide iz stanja $|2\rangle$ v stanje $|1\rangle$, pri čemer se
število fotonov v izbranem stanju elektromagnetnega polja $\alpha$
poveča z $n_{\alpha}$ na $n_{\alpha}+1$. V vseh ostalih stanjih
polja naj bo število fotonov enako nič.

Med atomom in poljem privzamemo električno dipolno interakcijo 
\begin{equation}
\hat{H}_{i}=-e\hat{E}(\mathbf{r},t)\hat{x},
\label{4.47}
\end{equation}
pri čemer je $\hat{x}$ operator koordinate elektrona v atomu. 
Privzeli
smo, da je nihajoče polje polarizirano v smeri osi $x$. Stanja celotnega sistema, 
to je atoma in polja, zapišemo v obliki produkta atomskih stanj in
stanja elektromagnetnega polja, pri čemer navedemo število fotonov
v posameznih lastnih nihanjih votline $\alpha$. Zapišemo okrajšano
\begin{equation}
|i,n_{\alpha}\rangle\equiv|i\rangle|\{n_{\alpha}\}\rangle.
\label{4.48}
\end{equation}
Začetno stanje celotnega sistema je torej $|2,n_{\alpha}\rangle$, kar pomeni, da je
atom v vzbujenem stanju (stanju 2), polje pa ima $n_{\alpha}$ fotonov v stanju $\alpha$.
Ustrezno končno stanje po prehodu je $|1,n_{\alpha}+1\rangle$.

V prvem redu teorije motenj je verjetnost za prehod na časovno enoto enaka
\begin{equation}
w_{21}=\frac{2\pi}{\hslash}|\langle1,n_{\alpha}+
1|\,\hat{H}_{i}\,|2,n_{\alpha}\rangle|^{2}\,
\delta(E_{2}-E_{1}-\hslash\omega_{\alpha}).
\label{4.49}
\end{equation}
S funkcijo $\delta$ izberemo prehod, pri katerem se energija ohranja.

Operator elektromagnetnega polja razvijemo po lastnih nihanjih votline (enačba~\ref{eq:pqrazvoj}) 
\begin{equation}
\hat{E}(\mathbf{r},t)=-\frac{1}{\sqrt{V\epsilon_{0}}}\sum_{\alpha}
\hat{p}_{\alpha}(t)E_{\alpha}(\mathbf{r}),
\label{4.50}
\end{equation}
pri čemer sta $\hat{p}_{\alpha}$ operator gibalne količine stanja $\alpha$ in $E_{\alpha}$ funkcija, ki opisuje krajevno odvisnost polja. Vemo, da se vsako lastno
elektromagnetno nihanje votline obnaša kot harmonski oscilator (enačba~\ref{4.9}).\index{Harmonski oscilator}
Vpeljemo kreacijske in anihilacijske operatorje
\begin{align}
\hat{a}_{\alpha}^{\dagger} & =  \frac{1}{\sqrt{2\hslash\omega_{\alpha}}}\,
(\omega_{\alpha}\hat{q}_{\alpha}-i\hat{p}_{\alpha}) \qquad \mathrm{in} \\
\hat{a}_{\alpha} & =  \frac{1}{\sqrt{2\hslash\omega_{\alpha}}}\,(\omega_{\alpha}\hat{q}_{\alpha}+i\hat{p}_{\alpha}).
\end{align}
Kreacijski operatorji povečujejo, anihilacijski pa zmanjšujejo število
fotonov v danem stanju. Tako velja
\begin{align}
\hat{a}_{\alpha}^{\dagger}|n_{\alpha}\rangle & =  \sqrt{n_{\alpha}+1}
|n_{\alpha}+1\rangle\qquad \mathrm{in} \\
\hat{a}_{\alpha}|n_{\alpha}\rangle & =  \sqrt{n_{\alpha}}|n_{\alpha}-1\rangle.
\end{align}
Edini od nič različni matrični elementi so tako oblike
\begin{align}
\langle n_\alpha +1|\, \hat{a}_{\alpha}^{\dagger}\,|n_{\alpha}\rangle & = 
\sqrt{n_{\alpha}+1} \qquad \mathrm{in} \label{eq:ankr.1}\\
\langle n_\alpha-1|\,\hat{a}_{\alpha}\,|n_{\alpha}\rangle & =  \sqrt{n_{\alpha}}.
\label{eq:ankr}
\end{align}
Operatorje $\hat{p}_{\alpha}$ izrazimo s kreacijskimi in anihilacijskimi
operatorji ter jih vstavimo v razvoj električnega polja (enačba~\ref{4.50})
\begin{equation}
\hat{E}(\mathbf{r},t)=-i\sum_{\alpha}\sqrt{\frac{\hslash\omega_{\alpha}}{2V\epsilon_{0}}}\,
\left(\hat{a}_{\alpha}^{\dagger}-\hat{a}_{\alpha}\right)E_{\alpha}(\mathbf{r}).
\label{4.53}
\end{equation}
Nadaljujemo z izračunom matričnega elementa, pri čemer upoševamo, da operator koordinate
$\hat{x}$ deluje na atomski del stanja in $\hat{E}$ na elektromagnetno
polje. Dobimo
\begin{align}
\langle1,n_{\alpha}+1|\,\hat{H}_{i}\,|2,n_{\alpha}\rangle & =  -e\,
\langle1,n_{\alpha}+1|\,\hat{E}\,\hat{x}\,|2,n_{\alpha}\rangle \nonumber \\
 & =  -e\,\langle1|\,\hat{x}\,|2\rangle\langle n_{\alpha}+1|\,\hat{E}\,|n_{\alpha}\rangle.
\end{align}
Vstavimo polje, ki smo ga izrazili s kreacijskimi in anihilacijskimi operatorji (enačba~\ref{4.53}),
upoštevamo zvezi~(\ref{eq:ankr.1}) in (\ref{eq:ankr}) in zapišemo
\begin{align}
\langle n_{\alpha}+1|\, \hat{E}\,|n_{\alpha}\rangle & = 
 -i\sum_{\beta}\sqrt{\frac{\hslash\omega_{\beta}}{2V\epsilon_{0}}}
\langle n_{\alpha}+1|\,\hat{a}_{\beta}^{\dagger}-\hat{a}_{\beta}\,|n_{\alpha}\rangle\, 
E_{\beta}(\mathbf{r})\nonumber \\
 & =  -i\sqrt{\frac{\hslash\omega_{\alpha}}{2V\epsilon_{0}}}
 \sqrt{n_{\alpha}+1}\, E_{\alpha}(\mathbf{r}).
\end{align}
Od vseh operatorjev v razvoju polja je namreč od nič različen matrični
element le za kreacijski operator za stanje $\alpha$.

Vpeljemo $\mathcal{V} = -e \langle1|\hat{x}|2\rangle$.
 Iskana verjetnost za prehod iz 
začetnega stanja, v katerem so v votlini vzbujen en atom in $n_{\alpha}$ fotonov, v končno
stanje, v katerem so atom v osnovnem stanju in $n_{\alpha}+1$ fotonov v stanju $\alpha$, je tako
\begin{equation}
w_{21}=\frac{\pi \omega_{\alpha}\mathcal{V}^{2}}{V\epsilon_{0}}
(n_{\alpha}+1)\,E_{\alpha}^{2}(\mathbf{r})\,\delta(E_{2}-E_{1}-\hslash\omega_{\alpha}).
\label{4.56}
\end{equation}
Verjetnost za prehod je sorazmerna z $n_{\alpha}+1$ in je od nič
različna, tudi če je število kvantov polja enako nič. To opisuje 
spontano sevanje\index{Spontano sevanje}. Prispevek, sorazmeren 
s številom že prisotnih fotonov, predstavlja stimulirano 
sevanje\index{Stimulirano sevanje}. Verjetnost za prehod vsebuje
še kvadrat prostorske odvisnosti polja $E_{\alpha}^{2}(\mathbf{r})$.
Če ne poznamo natančne lege atoma ali če je plin atomov enakomerno
porazdeljen po votlini, ta člen nadomestimo s povprečno vrednostjo.
Za stoječe valovanje je to 1/2.

Kolikšna pa je verjetnost za spontano emisijo?
Spontana emisija je mogoča v vsa elektromagnetna nihanja votline s
pravo frekvenco. Celotno verjetnost za prehod atoma iz vzbujenega stanja
v osnovno izračunamo tako, da seštejemo verjetnosti za prehod z izsevanim fotonom 
v določenem stanju. Vemo, da je ta verjetnost ravno enaka 
Einsteinovemu koeficientu $A_{21}$\index{Einsteinovi koeficienti} (enačba~\ref{4.27})
\begin{equation}
A_{21}=\sum_{\alpha}w_{21}=\sum_{\alpha}\frac{\pi \omega_{\alpha}\mathcal{V}^{2}}
{2V\epsilon_{0}}\,\delta(E_{2}-E_{1}-\hslash\omega_{\alpha}).
\label{4.57}
\end{equation}
Za prostorsko odvisnost polja $E^{2}(\mathbf{r})$ smo vzeli povprečje
1/2. Vsoto po nihanjih z uporabo enačbe~(\ref{4.5}) spremenimo v integral
in upoštevamo enačbo~(\ref{4.4}). Dobimo
\begin{equation}
A_{21}=\frac{\pi \mathcal{V}^{2}}{2\hslash\epsilon_{0}}\int\rho(\omega_{\alpha})\omega_\alpha\, 
\delta(\omega_{0}-\omega_{\alpha})\, d\omega_{\alpha}=\frac{\omega_{0}^{3}\mathcal{V}^{2}}{\epsilon_{0} h c^{3}},
\label{4.58}
\end{equation}
 pri čemer smo z $\omega_{0}=(E_{2}-E_{1})/\hslash$ označili frekvenco prehoda. Tako smo 
 izpeljali vrednost Einsteinovega koeficienta $A_{21}$. 
\begin{remark}
Pri izračunu Einsteinovega koeficienta $A_{21}$ smo privzeli, da so vsi dipoli urejeni  
 v smeri jakosti električnega polja svetlobe. Če želimo rezultat izenačiti s koeficientom, ki smo ga vpeljali
 za izotropno sevanje črnega telesa, ga moramo pomnožiti s faktorjem $\langle \cos^2\vartheta
 \rangle = 1/3$.
\end{remark}
Zaradi spontanega sevanja vzbujeno atomsko stanje nikoli ni popolnoma
stacionarno. Poleg tega energija stanja s končnim razpadnim časom ni natančno
določena, zato moramo verjetnost za stimulirano sevanje (enačba~\ref{4.56}) malo 
popraviti. Delta funkcijo energije nadomestimo s končno široko  
funkcijo $g(\omega-\omega_0)$, ki ima vrh pri $\omega_{0}$. Zaradi 
spremembe integracijske spremenljivke dobimo še dodaten faktor $1/\hslash$ in zapišemo
\begin{equation}
w_{21}=\frac{\pi \omega_{\alpha}\mathcal{V}^{2}}{2V\epsilon_{0}\hslash}
(n_{\alpha}+1)g(\omega_{\alpha}-\omega_0).
\label{4.59}
\end{equation}

Poglejmo še Einsteinov koeficient za stimulirano sevanje $B_{21}$. Lahko ga 
izrazimo iz enačbe~(\ref{4.18}), če upoštevamo, da je gostota energije 
polja $n_{\alpha}\hslash\omega_{\alpha}/V$
\begin{equation}
B_{21}=\frac{V\,w_{21}}{n_{\alpha}\,\hslash\omega_{\alpha}\, g(\omega_{\alpha}-\omega_0)}
=\frac{\pi \mathcal{V}^{2}}{2\epsilon_{0}\hslash^{2}}.
\label{4.60}
\end{equation}
Razmerje Einsteinovih koeficientov izračunamo iz enačb~(\ref{4.58}) in 
(\ref{4.60}). Dobimo
\begin{equation}
 \frac{A_{21}}{B_{21}}=\frac{\hslash \omega_0^3}{\pi^2 c^3},
\end{equation}
ki se ujema z razmerjem, ki smo ga izpeljali z uporabo
Planckove formule (enačba~\ref{4.27}). Prehojena pot jasno kaže zvezo med spontanim in
stimuliranim sevanjem ter gostoto stanj elektromagnetnega polja. 

\section{*Rabijeve oscilacije}
Če je svetloba, ki vpada na dvonivojski sistem, zelo močna, se lahko v primeru, da je frekvenca 
vpadne svetlobe $\omega$ blizu frekvence prehoda $\omega_0$, energija med 
svetlobnim poljem in dvonivojskim sistemom periodično izmenjuje.\index{Dvonivojski sistem} 
Oscilacije števila fotonov oziroma pričakovane 
vrednosti zasedenosti nivojev\index{Rabijeve oscilacije} imenujemo Rabijeve 
oscilacije\footnote{~Ameriški fizik in nobelovec Isidor Isaac Rabi, 1898--1988.}. 

Obravnavajmo sklopitev dvonivojskega sistema z elektromagnetnim valovanjem 
v semiklasičnem modelu.\index{Semiklasični model}\footnote{~Glej npr. M. Fox, {\it Quantum Optics, 
An Introduction}, Oxford University Press (2006).} 
To pomeni, da dvonivojski sistem obravnavamo kvantno in 
svetlobo, ki vpada nanj, kot klasično skalarno polje. 
V odsotnosti električnega polja zapišemo Hamiltonian\index{Hamiltonova
funkcija} za elektron kot
\begin{equation}
H_0 = \hslash \omega_1 |1\rangle \langle1| + \hslash \omega_2 |2\rangle \langle2|,
\end{equation}
pri čemer je $\omega_2- \omega_1 = \omega_0$ frekvenca prehoda. V prisotnosti 
svetlobnega polja moramo dodati še člen, ki opisuje dipolno interakcijo. Celoten
Hamiltonian postane časovno odvisen in ga zapišemo kot
\begin{equation}
H = \hslash \omega_1 |1\rangle \langle1| + \hslash \omega_2 |2\rangle \langle2|
-e\hat{x}E_0 \cos (\omega t).
\label{eq:sk-H}
\end{equation}
Schr\"odingerjevo enačbo\index{Schr\"odingerjeva enačba}
\begin{equation}
i \hslash \frac{\partial}{\partial t}|\psi\rangle = H|\psi\rangle
\label{eq:sk-S}
\end{equation}
rešujemo z nastavkom
\begin{equation}
|\psi\rangle = c_1(t)e^{-i \omega_1t}|1\rangle + c_2(t)e^{-i \omega_2t}|2\rangle,
\label{eq:sk-n}
\end{equation}
saj je valovna funkcija, ki popisuje stanje sistema, na splošno
kombinacija obeh stanj. Nastavek (enačba~\ref{eq:sk-n}) in Hamiltonian 
(enačba~\ref{eq:sk-H}) vstavimo v enačbo (\ref{eq:sk-S}), ki jo enkrat pomnožimo 
z $\langle1|$ in drugič z $\langle2|$. Izpeljemo sistem dveh sklopljenih enačb
\begin{equation}
\frac{d c_1}{dt}=-\frac{i}{\hslash} \mathcal{V} E_0\cos (\omega t) e^{-i\omega_0 t}\, c_2 
\quad \mathrm{in} \quad
\frac{d c_2}{dt}=-\frac{i}{\hslash} \mathcal{V} E_0\cos (\omega t) e^{i\omega_0 t}\, c_1,
\label{eq:c1c2}
\end{equation}
pri čemer je $\mathcal{V} = -e\langle1|\hat{x}|2\rangle$. Zapišemo $\cos(\omega t)$ kot
kompleksno število in zanemarimo hitro spreminjajočo se komponento pri $\omega_0 + \omega$,
tako da enačbi prepišemo v 
\begin{equation}
\frac{d c_1}{dt}=-\frac{i}{2\hslash} \mathcal{V} E_0 e^{-it\Delta}\, c_2 
\quad \mathrm{in} \quad
\frac{d c_2}{dt}=-\frac{i}{2\hslash} \mathcal{V} E_0 e^{it\Delta}\, c_1,
\label{eq:rabi2}
\end{equation}
pri čemer je $\Delta = \omega_0-\omega$. Dodamo začetni pogoj, ki pravi, da je na začetku
sistem v osnovnem stanju in torej $c_1(0)=1$ in $c_2(0)=0$. Rešitvi enačb
(\ref{eq:rabi2}) sta tako
\begin{align}
c_1(t)&=e^{-it\Delta/2}\, \left(\cos\left(\frac{\Omega t}{2}\right) + 
i \frac{\Delta}{\Omega} \sin\left(\frac{\Omega t}{2}\right) \right)\qquad \mathrm{in} 
\label{eq:rabi3} \\
c_2(t)&=\frac{\mathcal{V}E_0}{i\hslash \Omega} e^{it\Delta/2}\, \sin\left(\frac{\Omega t}{2}\right)\!.
\label{eq:rabi4}
\end{align}
Pri tem smo vpeljali krožno frekvenco
\begin{equation}
\Omega = \sqrt{\Delta^2 + \left(\frac{\mathcal{V}E_0}{\hslash}\right)^2} = \sqrt{(\omega_0-\omega)^2 
+ \left(\frac{\langle1|\hat{x}|2\rangle\, eE_0}{\hslash}\right)^2} = 
\sqrt{(\omega_0-\omega)^2 + \Omega_R^2}.
\end{equation}
Krožno frekvenco $\Omega_R = \mathcal{V} E_0/\hslash$ imenujemo Rabijeva krožna frekvenca.\index{Rabijeva frekvenca}

\begin{definition}
Pokaži, da enačbi~(\ref{eq:rabi3}) in (\ref{eq:rabi4}) rešita
sistem enačb (\ref{eq:rabi2}) ob izbranih začetnih pogojih.
\end{definition}
Poglejmo rezultat podrobneje. Verjetnost, da najdemo atom v stanju $|2\rangle$, je enaka
\begin{equation}
P_2(t) = |c_2(t)|^2 = \frac{\mathcal{V}^2E_0^2}{\hslash^2 \Omega^2}\sin^2(\Omega t/2).
\end{equation}
Če je frekvenca vpadne svetlobe točno enaka frekvenci prehoda, je $\Delta = 0$ in 
$\Omega = \Omega_R = \mathcal{V}E_0/\hslash$. Takrat je amplituda nihanja zasedenosti vzbujenega stanja kar enaka 1
in sistem v celoti periodično prehaja iz osnovnega stanja v vzbujeno in nazaj (slika~\ref{fig:Rabi}). To pomeni,
da prihaja izmenično do popolne absorpcije svetlobe in do popolne stimulirane emisije.
Pri odstopajoči vpadni frekvenci se amplituda nihanja zmanjša, hkrati se poveča
frekvenca oscilacij. Frekvenca oscilacij ni odvisna zgolj od frekvence vpadnega valovanja, 
ampak tudi od jakosti električnega polja. Groba ocena
Rabijeve frekvence je $\Omega_R \sim~\si{MHz}$.
\begin{figure}[ht]
\centering
\def\svgwidth{90truemm} 
\input{slike/05_rabi.pdf_tex}
\caption{Rabijeve oscilacije za tri različne vrednosti odstopanja frekvence vpadne
svetlobe od frekvence prehoda $\Delta=\omega_0-\omega$. 
Z naraščajočim odstopanjem se amplituda oscilacij
zmanjšuje, njihova frekvenca pa povečuje.}
\label{fig:Rabi}
\end{figure}

Omenjeno velja, kadar je vpadna svetloba zelo močna, povsem koherentna in v sistemu ni 
motenj. V realnih sistemih so prisotni relaksacijski pojavi, kot na primer trki med atomi
ali spontana emisija, zato so Rabijeve oscilacije dušene. Zaznamo jih lahko le v času, ki 
je krajši od obratne vrednosti širine spektralne črte ($\sim 10^{-10}~\si{s}$).

\begin{remark}
Rabijeve oscilacije niso omejene samo na optične prehode, ampak se pojavijo pri 
vrsti dvonivojskih sistemov, ki interagirajo z močnim spreminjajočim se zunanjim poljem. 
Poznamo jih na primer pri jedrski magnetni resonanci (NMR) ali kvantnih logičnih vezjih.
\end{remark}

%Final

%-------------------------------------------------------------------------------
%	CHAPTER 6
%-------------------------------------------------------------------------------

\chapterimage{slike/Navy.jpg} 
\chapter{Laser}

V prejšnjih poglavjih smo spoznali resonatorje, pojasnili proces ojačevanja svetlobe in 
opisali črpanje, ki je potrebno za vzdrževanje obrnjene zasedenosti v ojačevalnem sredstvu. 
V tem poglavju bomo vsa ta
spoznanja združili in komponente sestavili v eno samo napravo -- laser. Zapisali
bomo zasedbene enačbe, pojasnili delovanje laserjev in spoznali prednosti 
laserske svetlobe pred svetlobo iz navadnih svetil. Opisali bomo način delovanja 
sunkovnih laserjev in na koncu predstavili semiklasični model laserja. 

\section{Laser}
\index{Laser}\index{Spontano sevanje}
\index{Optično ojačevanje}
V sredstvu, v katerem med dvema nivojema dosežemo obrnjeno 
zasedenost, se svetloba z ustrezno valovno dolžino ojačuje. 
Postavimo tako sredstvo v optični resonator. \index{Resonator} 
Na začetku nastaja predvsem spontano izsevana svetloba in 
v resonatorju se vzbujajo tista lastna nihanja, katerih frekvenca je blizu frekvence
atomskega prehoda. Resonator poskrbi, da se svetloba odbija nazaj v ojačevalno 
sredstvo. Če v njem vzdržujemo obrnjeno zasedenost, se svetloba ob prehodu skozi 
sredstvo ojačuje.
Na začetku je ojačenje za
izbrana nihanja veliko, z naraščajočo intenziteto pa se ojačenje zmanjšuje.
\index{Izgube v resonatorju} 
Ko se ojačenje na prelet izenači z izgubami, sistem preide v stacionarno stanje in 
seva močno koherentno svetlobo. Tak
izvor svetlobe imenujemo laser. Beseda laser je nastala iz kratice za {\it Light
Amplification by Stimulated Emission of Radiation} --  ojačenje svetlobe s
stimuliranim sevanjem.\footnote{~R. G. Gould, {\it The Ann Arbor Conference on Optical Pumping: 
the University of Michigan}, ur. P. A. Franken in R. H. Sands (1959).}

\begin{figure}[ht]
\centering
\def\svgwidth{90truemm} 
\input{slike/06_shema.pdf_tex}
\caption{Shema laserja s ključnimi deli: 
ojačevalno sredstvo, črpalni mehanizem in resonator.
Odbojnost izhodnega zrcala $\mathcal{R}_2<1$, da svetloba lahko zapusti resonator.}
\label{fig:shemalaserja}
\end{figure}
\vglue-6truemm
\begin{remark}
Kot klasično analogijo za laser vzamemo klarinet, ki je sestavljen iz 
cevi in ustnika. Cev deluje kot resonator, v katerem nastane 
stoječi zvočni val, pri čemer je frekvenca stoječega vala določena z 
dolžino cevi in s številom vozlov. Naloga ustnika je dovajanje energije 
in s tem vzdrževanje konstantne amplitude nihanja. To glasbenik doseže s 
pihanjem v ustnik in tresenjem prožnega jezička, ki s tresljaji proizvaja 
zvok. Tresenje jezička je približno periodično in vsebuje mnogo različnih 
frekvenc, tudi take, ki ustrezajo frekvenci stoječih valov v cevi. 
Ko amplituda tlaka v cevi naraste nad neko mejo, nastopi zanimiv
pojav. Nihanje tlaka v zgornjem koncu cevi povratno deluje na ustnik
in ga sili, da niha s frekvenco najmočneje vzbujenega stoječega vala v cevi,
druge frekvence pa zamrejo. Moč pihanja gre le še v
nihanje jezička s pravo frekvenco in ojačuje nihanje zračnega stolpca. 
Tako se s povratno zvezo med nihanjem jezička in stoječim valovanjem v cevi
vzdržuje stoječe valovanje s konstantno amplitudo. 
\end{remark}

V grobem ima laser tri ključne sestavne dele: ojačevalno sredstvo, 
črpalni mehanizem in resonator, ki je v najpreprostejšem primeru sestavljen iz dveh 
ukrivljenih zrcal (slika~\ref{fig:shemalaserja}). Črpalni \index{Črpanje}
mehanizem vzdržuje obrnjeno zasedenost v ojačevalnem sredstvu, medtem ko resonator 
omogoča, da se svetloba med številnimi prehodi skozi ojačevalno sredstvo dovolj ojači.
Odbojnost vsaj enega od zrcal 
mora biti manjša od 1, da skozenj lahko izhaja svetloba.

Zaenkrat se omejimo na najpreprostejši model laserja in 
privzamemo, da frekvenca le enega resonatorskega lastnega nihanja sovpada s 
frekvenco prehoda aktivne snovi. Ta predpostavka v večini laserjev ni
avtomatično izpolnjena, vendar jo je pogosto mogoče doseči z dodatnimi elementi 
v resonatorju (glej razdelek~\ref{section:vecfrek}). 
Aktivno snov oziroma ojačevalno sredstvo v laserju stalno 
črpamo in s tem vzdržujemo obrnjeno zasedenost. 

Naj bo $W$ energija svetlobnega valovanja v resonatorju. Zaradi izgub skozi
zrcali, absorpcije in sipanja v resonatorju se energija na en obhod 
resonatorja zmanjša za (enačba~\ref{eq:Lambda})
\begin{equation}
\Delta W_{\rm izgube}=-\Lambda W=-\left(1-{\cal {R}}_{1}+1-{\cal {R}}_{2}
+2\alpha L\right)W,
\label{5.1}
\end{equation}
pri čemer so $\Lambda $ celotne izgube, $\alpha$ izgube na enoto poti zaradi
absorpcije in sipanja, $L$ dolžina resonatorja ter \index{Izgube v resonatorju}
${\cal {R}}_{1}$ in ${\cal {R}}_{2}$ odbojnosti
zrcal. V ojačevalnem sredstvu, v katerem vzdržujemo obrnjeno zasedenost,
pride do ojačevanja s stimuliranim sevanjem. Energija nihanja resonatorja 
se tako na en obhod po enačbi (\ref{eq:djG}) poveča za 
\index{Stimulirano sevanje}
\begin{equation}  
\Delta W_{\rm oja\check{c}enje}=\frac{G}{1+W/W_s}\,W\, 2L'.
\label{5.2}
\end{equation}
Vpeljali smo saturacijsko energijo $W_s=Vj_s/c$, $G$ pa je koeficient ojačenja.
Dolžino ojačevalnega sredstva,
ki se v splošnem razlikuje od dolžine resonatorja $L$, smo označili z $L'$.
\index{Saturacijska energija}
Zapis sicer pogosto poenostavimo in vzamemo $L'=L$, vendar tukaj zaradi
jasnosti obdržimo ločen zapis. Privzeli smo tudi, da je ojačenje na en
obhod dovolj majhno, da enačbe (\ref{eq:djG}) ni treba integrirati.

V stacionarnem stanju se zmanjšanje energije zaradi izgub ravno izenači 
s povečanjem energije zaradi ojačenja. Zapišemo
\beq
|\Delta W_{\rm izgube}|=|\Delta W_{\rm oja\check{c}enje}|,
\eeq
od koder sledi
\begin{equation}  
\Lambda\, W=\frac{G\,2L'}{1+W/W_s}\,W.
\label{5.3}
\end{equation}
\begin{figure}[ht]
\centering
\def\svgwidth{140truemm} 
\input{slike/06_stacio.pdf_tex}
\caption{Za majhne vrednosti ojačenja $G$ ima enačba~(\ref{5.3}) eno samo 
rešitev, to je $W=0$ (a). Pri večjih ojačenjih ima tudi neničelno rešitev (b).}
\label{fig:stacio}
\end{figure}

Enačba~(\ref{5.3}) ima pri majhnem ojačenju $G$ eno samo rešitev, to je 
$W=0$ (slika~\ref{fig:stacio}). Pri večjih vrednostih ojačenja $G$ obstaja še neničelna rešitev
za energijo svetlobnega nihanja 
\boxeq{5.4}{ 
W=W_s \left(\frac{G}{G_\mathrm{pr}}-1\right)\!,
}
pri čemer smo vpeljali ojačenje na pragu delovanja\index{Prag delovanja laserja}
\boxeq{5.5}{
G_\mathrm{pr} = \frac{\Lambda}{2L'}.
}
Združimo rešitvi: energija svetlobe v laserju je pri ojačenju, ki je manjše
od ojačenja na pragu delovanja $G_\mathrm{pr}$, enaka
nič. Nad pragom energija svetlobe linearno narašča z ojačenjem $G$ (slika~\ref{fig:energija}).
Ojačenje pa je odvisno od stopnje obrnjene zasedenosti, ki je povezana
z močjo črpanja.
\begin{figure}[ht]
\centering
\def\svgwidth{60truemm} 
\input{slike/06_energija.pdf_tex}
\caption{Odvisnost energije svetlobe v laserju $W$ od ojačenja $G$. 
Pod pragom $G_\mathrm{pr}$ je energija enaka nič, 
nad pragom pa linearno narašča z ojačenjem.}
\label{fig:energija}
\end{figure}

Izhodna moč laserja je enaka energiji, ki zapusti\index{Izhodna moč laserja}
resonator skozi izhodno zrcalo, deljeni z obhodnim časom resonatorja $2L/c$ 
\boxeq{5.6}{
P=(1-{\cal {R}}_{2})\frac{c}{2L}\,W.
}
Predfaktorji v enačbi so konstantni, zato je izhodna moč sorazmerna
energiji svetlobe v resonatorju. Odvisnost izhodne moči laserja od črpanja je 
tako do konstante enaka energiji, prikazani na sliki~\ref{fig:energija}. 

Gostoto izhodnega svetlobnega toka za ojačenje nad pragom ($G>G_\mathrm{pr}$) 
zapišemo kot
\begin{equation}
 j = \frac{P}{S} = \frac{1}{2} (1-{\cal {R}}_{1}) j_s \left(\frac{G}{G_\mathrm{pr}}-1\right)\!.
\end{equation}
\begin{definition}
Izračunaj izhodno moč laserja pri dani dolžini resonatorja ($L=L'$), 
odbojnosti enega zrcala ${\cal {R}}_{1}=1$, 
notranjih izgubah na enoto dolžine $\alpha$ in ojačenju $G$. Pokaži, da
je izhodna moč največja pri odbojnosti izhodnega zrcala 
\beq
{\cal {R}}_2 = 1-2\alpha L \left(\sqrt{\frac{G}{\alpha}}-1\right)\!.
\eeq
\end{definition}

\section{Zasedbene enačbe}
\index{Zasedbene enačbe}
Za podrobnejši opis delovanja laserja zapišemo zasedbene enačbe. 
Enačbam za zasedenost atomskih nivojev \index{Trinivojski sistem}
v trinivojskem sistemu (enačbe~\ref{4.39.1}--\ref{4.39}) dodamo še enačbo za 
energijo lastnega nihanja v resonatorju. Vzbujeno naj bo 
le eno lastno resonatorsko nihanje, opazujemo pa prehode med prvim 
in drugim vzbujenim atomskim stanjem (slika~\ref{fig:3nivojski}\,b). 

Preden zapišemo enačbe, napravimo še nekaj poenostavitev. Najprej privzamemo, 
da je razpadni čas spodnjega atomskega stanja $|1\rangle$, 
ki ga določa koeficient $A_{10}$, dosti krajši od razpadnega
časa zgornjega stanja $|2\rangle$. Tedaj vsi atomi iz spodnjega stanja zelo 
hitro preidejo v osnovno stanje in $N_1 \approx 0$, če le ni preveč
stimuliranega sevanja. Zanemarimo tudi spontano sevanje iz drugega vzbujenega
nivoja $A_{20} \approx 0$. 

Celoten sistem potem opišemo z dvema 
\index{Obrnjena zasedenost}
spremenljivkama: prva je $N_2$, ki označuje zasedenost drugega vzbujenega
stanja in hkrati približno obrnjeno zasedenost, druga pa je $n$, ki pove število
fotonov v izbranem lastnem nihanju resonatorja. Število fotonov določa energijo 
polja v resonatorju, ki je enaka $W = \hslash\omega n$ (enačba~\ref{4.11}). Ustrezna
gostota energije je $w = \hslash\omega n/V$, pri čemer je $V$ volumen resonatorja.

Zasedbeni enačbi sta
\begin{equation}
\frac{dN_2}{dt}=rN-A_{21}N_2-B_{21}gN_2\frac{\hslash \omega}{V}\,n
=rN-A_{21}N_2-\frac{\sigma c}{V}\, N_2\,n
\label{5.7}
\end{equation}
in \index{Izgube v resonatorju}\index{Življenjski čas nihanj}
\begin{equation}
\frac{dn}{dt}=\frac{\sigma c}{V}\, N_2\,(n+1)-\frac{2}{\tau}\,n.
\label{5.8}
\end{equation}

Prva enačba sledi neposredno iz enačbe~(\ref{4.39}) ob upoštevanju zgoraj navedenih
poenostavitev. Drugo  dobimo z naslednjim razmislekom: energija svetlobe 
oziroma število fotonov v resonatorju se povečuje predvsem 
zaradi stimuliranega sevanja, opisanega z zadnjim členom v enačbi~(\ref{5.7}).
Po drugi strani vemo, da je verjetnost za prehod atoma iz višjega v nižje stanje z 
izsevanjem fotona v izbrano stanje elektromagnetnega polja sorazmerna z 
$n+1$ (enačba~\ref{4.59}), pri čemer je $n$ število fotonov v izbranem stanju. 
Če torej namesto $n$ v zadnjem členu enačbe~(\ref{5.7}) pišemo $n+1$, 
opišemo poleg stimuliranega sevanja tudi prispevek spontanega sevanja. Dodamo še 
člen, s katerim popišemo zmanjševanje energije svetlobe v resonatorju zaradi 
izgub, kar opišemo z razpadnim časom $\tau/2$ (enačba~\ref{eq:dW}). 
\index{Ojačenje s stimuliranim sevanjem|see{Stimulirano sevanje}}
\index{Stimulirano sevanje}
\index{Koeficient ojačenja}

Enačbi~(\ref{5.7} in \ref{5.8}) predstavljata sistem dveh sklopljenih diferencialnih enačb za 
časovni razvoj števila fotonov v resonatorskem stanju in za zasedenost 
zgornjega atomskega stanja. Enačbi sta nelinearni in nimata  
analitične rešitve. Vseeno lahko nekaj povemo o takem sistemu.

Zapišimo najprej stacionarne rešitve. Za njih velja $dN_2/dt=0$ in 
$dn/dt=0$. Iz enačbe~(\ref{5.7}) izrazimo $N_{2}$ in ga vstavimo v enačbo~(\ref{5.8}).
Dobimo 
\begin{equation}
\frac{2}{\tau }n\left(A_{21}V+\sigma c\,n\right)=
\sigma c\, r\,N\,(n+1).
\label{5.9}
\end{equation}
Enačbo zapišemo bolj pregledno, če vpeljemo koeficient ojačenja $G$ (enačba~\ref{4.44})
in ojačenje na pragu $G_\mathrm{pr}$ (enačbi~\ref{taulambda} in \ref{5.5})
\beq
G_\mathrm{pr}\, n\, \left(1+\frac{\sigma c}{VA_{21}}n \right)= G(n+1).
\label{5.9.a}
\eeq

Vpeljemo brezdimenzijsko konstanto $p$, pri čemer upoštevamo zvezo
med Einsteinovimi koeficienti (enačba~\ref{4.27})
\begin{equation}
p=\frac{VA_{21}}{\sigma c} = 
\frac{VA_{21}}{B_{21}\hslash \omega g}=\frac{V\omega ^{2}}{\pi
^{2}c^{3}g}\approx
\frac{V\omega ^{2}}{\pi ^{2}c^{3}}\Delta \omega.  
\label{5.10}
\end{equation}
V zadnjem koraku smo privzeli, da je $g\approx 1/\Delta \omega $. 
Parameter $p$ je približno enak produktu 
gostote stanj elektromagnetnega polja v resonatorju (enačba~\ref{4.4}),
širine atomskega prehoda in volumna, torej kar številu vseh stanj 
v frekvenčnem intervalu atomskega prehoda. To število je navadno precej 
veliko $p \sim 10^{8}$--$10^{10}$. 

\begin{definition}
Primerjaj izraz za $p$ (enačba~\ref{5.10}) z izrazom za saturacijsko 
gostoto toka $j_s$ (enačba~\ref{eq:jsatg}). Pokaži, da velja
\begin{equation}
p = \frac{W_s}{\hslash \omega}.
\end{equation}
Parameter $p$ je torej enak številu fotonov v resonatorju, pri katerem pride 
do nasičenja ojačenja, če je frekvenca nihanja resonatorja blizu 
centra atomske črte. 
\end{definition}

Enačbo~(\ref{5.9.a}) prepišemo in dobimo
\begin{equation}
\frac{n^2}{p}-\left(\frac{G}{G_\mathrm{pr}}-1\right)\,n-\frac{G}{G_\mathrm{pr}}=0
\label{5.11}
\end{equation}
s pozitivno rešitvijo 
\begin{equation}
n=\frac{p}{2}\left( \left(\frac{G}{G_\mathrm{pr}}-1\right)+\sqrt{\left(\frac{G}{G_\mathrm{pr}}
-1\right)^{2}+ \frac{4G}{p\,G_\mathrm{pr}}}\right)\!.
\label{5.12}
\end{equation}
Ker je $p$ zelo veliko število, lahko koren razvijemo, če le ni ojačenje
preveč blizu praga, ko je $G/G_\mathrm{pr}\sim 1$. 
\index{Prag delovanja laserja} 

Pod pragom je $G<G_\mathrm{pr}$ in velja
\begin{equation}
n\approx \frac{p}{2}\left( \left(\frac{G}{G_\mathrm{pr}}-1\right)+\left(1
-\frac{G}{G_\mathrm{pr}}\right)+\frac{2G}{p(G_\mathrm{pr}-G)}\right) =\frac{G}{G_\mathrm{pr}-G}.
\label{5.13}
\end{equation}

Nad pragom je
število fotonov enako
\boxeq{5.14}{
n\approx p\left(\frac{G}{G_\mathrm{pr}}-1\right) = \frac{W_s}{\hslash \omega} \left(\frac{G}{G_\mathrm{pr}}-1\right)\!.
}
Rezultat, ki je po pričakovanju skladen z enačbo~(\ref{5.4}), si oglejmo podrobneje. 
Pod pragom so izgube večje od črpanja in gre praktično vsa moč, ki jo dovedemo v sistem, 
s spontanim sevanjem v veliko število stanj elektromagnetnega polja. 
Število fotonov v izbranem resonatorskem nihanju je tako okoli 
ena vse do neposredne bližine praga (slika~\ref{fig:p}). 
Nad pragom povsem prevlada stimulirano sevanje 
v eno samo izbrano nihanje resonatorja in število fotonov je reda velikosti $p$ (enačba~\ref{5.14}).
Prehod čez prag je zaradi velikega $p$ tako hiter, da ga ni mogoče izmeriti.
Izjema so polprevodniški laserji, \index{Laser!polprevodniški}katerih 
volumen -- in posledično tudi $p$ -- 
je tako majhen, da je mogoče opaziti zvezen prehod čez prag.
\begin{figure}[ht]
\centering
\def\svgwidth{70truemm} 
\input{slike/06_p.pdf_tex}
\caption{Odvisnost števila fotonov v resonatorju od ojačenja za $p=10^5$. 
Pod pragom je število fotonov majhno, ob pragu skokovito naraste in tudi 
pri močnem črpanju ostaja reda velikosti $p$.}
\label{fig:p}
\end{figure}

Izračunajmo še stacionarno zasedenost zgornjega atomskega nivoja. Iz 
enačbe~(\ref{5.8}) sledi
\boxeq{5.15}{
N_2=\frac{2V}{\tau \sigma c}\frac{n}{n+1} \approx \frac{2V}{\tau \sigma c}.
}
Na pragu je po enačbi (\ref{5.12}) $n=\sqrt{p}$. 
Sledi 
\begin{equation}  
N_{\rm 2pr}=\frac{2V}{\tau \sigma c}\frac{\sqrt{p}}{\sqrt{p}+1}.
\label{5.16}
\end{equation}
Ker je $p$ zelo veliko število, lahko zasedenost zgornjega
nivoja (oziroma obrnjena zasedenost) narašča le do bližine praga. Nad pragom ostaja praktično
konstantna in skoraj natanko enaka kot na pragu. Tega ni težko razumeti. Nad
pragom je število fotonov v resonatorju veliko in linearno narašča 
z močjo črpanja. S tem se povečuje hitrost praznjenja zgornjega atomskega 
stanja s stimuliranim sevanjem, kar ravno izniči učinek povečanja črpanja. 
V stacionarno delujočem laserju obrnjene zasedenosti ($\approx N_2$)
torej ni mogoče povečati 
nad vrednost na pragu $N_{\rm 2pr}$. To ima pomembne praktične posledice, kot bomo
videli v nadaljevanju.

\begin{remark}
Obravnava laserja z zasedbenimi enačbami je seveda zelo groba. Nismo
upoštevali, da je prostorska odvisnost polja v delujočem laserju 
drugačna od lastnega nihanja praznega resonatorja. Poleg tega smo
privzeli, da so atomi lahko le v lastnih energijskih stanjih, kar je res le
v primeru stacionarnih stanj brez zunanjega, časovno odvisnega polja
svetlobe. Podrobnejši pristop je semiklasični model, pri katerem 
za opis svetlobe uporabimo klasično valovno enačbo, za atome
pa kvantno mehaniko (glej razdelek~\ref{chap:semiklasicni}). Ta model
zadošča za opis skoraj vseh pojavov v laserjih razen vpliva spontanega sevanja. 
Za dosledno obravnavo tega je treba svetlobo opisati s pomočjo 
kvantne elektrodinamike.
\end{remark}

Povzemimo na kratko, kaj smo spoznali o delovanju 
enofrekvenčnega laserja. Pri ojačenju, ki ravno pokriva izgube 
resonatorja, sta v stacionarnem stanju energija in s tem amplituda 
izbranega lastnega nihanja resonatorja različni od nič. Frekvenca svetlobe je
določena z izbranim lastnim nihanjem resonatorja, ki določa tudi prostorsko
odvisnost valovanja v resonatorju in obliko izhodnega snopa. V navadnem stabilnem 
resonatorju je polje po obliki zelo blizu Gaussovemu snopu, zato je tak tudi izhodni snop.
Gaussova prostorska odvisnost izhodnega snopa je morda najpomembnejša lastnost
laserjev. Gaussov snop se, kot vemo, najmanj širi zaradi uklona in ga je mogoče
zbrati v piko reda velikosti valovne dolžine. Laser se tako najbolj 
približa idealno točkastemu izvoru svetlobe.

\section{Spektralna širina enega laserskega nihanja}
\index{Spektralna črta}
Povejmo še nekaj o spektralni širini svetlobe enofrekvenčnega laserja. 
Če bi se lastno stanje 
elektromagnetnega polja v resonatorju obnašalo kot klasično 
harmonsko nihalo, bi bil spekter laserja neskončno ozek. Vendar 
imajo laserji končno spektralno širino -- v idealnem primeru zaradi
kvantizacije elektromagnetnega polja, v praksi pa zaradi zunanjih motenj.
Poskusimo oceniti razširitev zaradi vpliva kvantizacije. Zaradi nje
je poleg stimuliranega sevanja vedno prisotno tudi spontano sevanje. To 
predstavlja kvantni šum, ki povzroči razširitev spektra. 

Predstavimo amplitudo nihanja $E(t)$ na izbranem mestu v resonatorju kot kompleksno 
število, ki ga v kompleksni ravnini določata dolžina $|E(t)|$ in faza
$\varphi$ (slika~\ref{fig:fazor}). 
Pri tem fazo določimo glede na neko začetno izbrano fazo. 
Ker je energija svetlobe sorazmerna s številom fotonov, je dolžina $|E(t)|$
sorazmerna s korenom iz števila fotonov v izbranem lastnem nihanju. 
Stimulirano sevanje, ki ravno pokriva izgube resonatorja, vzdržuje
dolžino $|E(t)|$ praktično konstantno, nespremenjena ostaja 
tudi faza. Spontano sevanje velikosti amplitude nihanja ne spreminja dosti, 
vendar stohastična narava spontano izsevanih fotonov vpliva na njeno fazo.
Majhen prispevek spontanega sevanja zaradi spreminjajoče se faze
skrajša koherenčni čas in določa spodnjo mejo za širino spektralne črte. 

\begin{figure}[ht]
\centering
\def\svgwidth{75truemm} 
\input{slike/06_fazor.pdf_tex}
\caption{Amplituda polja v resonatorju in njena sprememba zaradi 
spontanega sevanja}
\label{fig:fazor}
\end{figure}

Pri spontani emisiji se izseva en foton s poljubno fazo. Prispevek h kompleksni
amplitudi ima torej dolžino 1 in poljubno smer (slika~\ref{fig:fazor}). Zanima
nas povprečje kvadrata spremembe faznega kota pri enem spontano izsevanem fotonu
\begin{equation}
\overline{(\Delta \varphi_{1})^{2}}=\overline{\left(\frac{\cos\psi}{\sqrt{n} }\right)^2}
=\frac{1}{2\overline{n}},
\label{5.17}
\end{equation}
pri čemer je kot $\psi$ označen na sliki~\ref{fig:fazor}. 

Zaporedne spontane emisije so med seboj neodvisne, zato izračunamo
povprečni kvadrat spremembe faze pri $m$ emisijah tako, da seštejemo
povprečne kvadrate za posamezne fotone 
\begin{equation}
\overline{(\Delta \varphi_{m})^{2}}=m\overline{(\Delta \varphi_{1})^{2}}=
\frac{m}{2\overline{n}}.
\label{5.18}
\end{equation}
Ocenimo še število spontano izsevanih fotonov na časovno enoto.
Vemo, da stimulirano sevanje ravno pokrije izgube resonatorja, zato je
stimulirano izsevanih fotonov na časovno enoto $2\overline{n}/\tau $. Vemo tudi, 
da je razmerje med verjetnostjo za stimulirano in spontano sevanje enako \index{Stimulirano sevanje}
številu fotonov v danem stanju polja (enačba~\ref{4.56}), zato je število 
spontanih sevanj na časovno enoto kar $2/\tau $.\index{Spontano sevanje}

Tako je število spontano izsevanih fotonov v času $t$ enako $m=2t/\tau $ in 
\begin{equation}
\overline{\Delta \varphi^{2}(t)}=\frac{t}{\overline{n}\tau }.
\label{5.19}
\end{equation}
Čas $t_{p}$, v katerem se faza znatno spremeni, je torej
velikostnega reda 
\begin{equation}
t_{p}\sim \overline{n}\tau =\frac{W}{\hslash \omega }\,\tau =\frac{P}{\hslash
\omega }\tau ^{2}.
\label{5.20}
\end{equation}
Ker je število fotonov v izbranem nihanju nad pragom zelo veliko ($\sim 10^9$ v majhnem 
He-Ne laserju) in $\tau~\sim 10^{-7}$, je karakteristični\index{Laser!He-Ne}
čas za fazno razširitev idealnega laserja $t_p \sim 100~\si{s}$. 

Iz enačbe (\ref{5.20}) vidimo tudi, da je spektralna širina, ki je
podana z $1/t_{p}$, obratno sorazmerna z izhodno močjo laserja. Spodnjo
mejo za spektralno širino pri dani izhodni moči laserja 
podaja Schawlow-Townseva limita\footnote{~A. 
L. Schawlow in C. H. Townes, Phys. Rev. {\bf 112}, 1940 (1958).}
\beq
\Delta \nu_\mathrm{min} = \frac{ \pi h \nu}{P} \Delta \nu_R^2,
\eeq
pri čemer $\Delta \nu_R$ predstavlja širino nihanja praznega 
resonatorja.\footnote{~Natančnejši izračun odstopa od preprosto izpeljanega za 
faktor 2. V zapisanem izrazu je že upoštevan pravilen predfaktor.}
V neposredni bližini praga, kjer je $\overline{n}\sim 1$, je 
spektralna širina približno enaka širini nihanja praznega resonatorja.

Dejanski laserji seveda nimajo niti približno tako ozkega spektra, kot smo ga
pravkar ocenili. Vemo, da je frekvenca laserja določena z dolžino resonatorja 
($\nu=N c/2L$), pri čemer je $N$ zelo veliko celo število. Že majhna
sprememba dolžine resonatorja povzroči spremembo frekvence laserja, pri 
znatnejši spremembi dolžine lahko preskoči tudi vzbujeno
nihanje v resonatorju, torej se spremeni število $N$. Dolžina resonatorja se 
spreminja predvsem zaradi zunanjih mehanskih motenj in zaradi spreminjanja
temperature. Če se posebej ne potrudimo s konstrukcijo resonatorja, so 
fluktuacije frekvence kar reda velikosti razmika med sosednimi stanji 
resonatorja, to je reda velikosti $\sim 100~\si{MHz}$. 
Fluktuacije dolžine je mogoče zmanjšati s skrbno konstrukcijo, 
temperaturno stabilizacijo in uporabo materialov z majhnim toplotnim raztezkom. 
Na tak način je mogoče narediti laser z efektivno spektralno širino pod $\sim 1~\si{MHz}$.

\begin{remark}
Zapisali smo, da lahko s posebno konstrukcijo laserjev dosežemo
spektralno širino pod $\sim 1~\si{MHz}$. Vendar najmanjša dosežena 
spektralna širina znaša $\sim 10~\si{mHz}$, kar je še osem
velikostnih redov manj! Za to je potreben prav poseben 
laser iz monokristalov silicija, hlajen na $-150~\si{\celsius}$. Fluktuacije
dolžine resonatorja so pogojene s termičnimi fluktuacijami v odbojnih plasteh, 
ki znašajo okoli $10^{-17}\si{\metre}$. Koherenčna dolžina takega laserja je več
milijonov kilometrov.\footnote{~D. G. Matei et al., Phys. Rev. Lett. {\bf 118}, 263202 (2017).} 
\end{remark}

Velja opozoriti, da je narava spektralne razširitve v laserju 
drugačna kot v navadnih svetilih. V drugem poglavju smo videli, da 
intenziteta svetlobe navadnega svetila fluktuira na časovni skali 
koherenčnega časa, ki je obraten spektralni širini (razdelek~\ref{chap:kns}). 
Šum navadnih svetil je torej amplitudno moduliran šum. Pri \index{Šum}
enofrekvenčnem laserju je drugače. Amplituda in intenziteta 
izhodne svetlobe sta konstantni, fluktuira le frekvenca oziroma
faza. Šum laserja je torej v obliki frekvenčne modulacije.

\section{Primerjava laserjev in navadnih svetil}
Primerjajmo enofrekvenčni laser, v katerem je vzbujeno eno lastno nihanje
resonatorja Gaussove oblike, z navadnim nekoherentnim izvorom svetlobe.

Svetlobni snop iz laserja ima dve takoj očitni odliki: je zelo\index{Laser!enofrekvenčni}
usmerjen in zelo enobarven. Prva lastnost je posledica tega, da je
lastno nihanje stabilnega resonatorja Gaussove oblike in je zato tak tudi
izhodni snop. Divergenca Gaussovega snopa je posledica uklona in je 
najmanjša možna. Valovne fronte izhodne svetlobe so gladke in na dani oddaljenosti ves 
čas enake, zato je laserski snop prostorsko idealno koherenten. 
Koherentni Gaussov snop lahko z ustrezno optiko zberemo v piko velikosti
valovne dolžine, s čimer dosežemo že pri razmeroma majhni izhodni moči zelo veliko
gostoto svetlobnega toka. To je zelo uporabno v tehnologiji, na primer za natančno in
čisto obdelavo materialov, ter v medicini za
zahtevne kirurške posege.

Kako pa je z navadnimi svetili? V njih atomi sevajo neodvisno, zato
izsevana svetloba ni prostorsko koherentna. Valovna fronta na danem 
mestu je nepravilna in se znotraj koherenčnega časa znatno spremeni. 
Vendar tudi iz svetlobe navadnega nekoherentnega svetila lahko pridobimo
koherentni snop, če na dano razdaljo od svetila postavimo zaslonko, ki
je manjša od koherenčne ploskve na tistem mestu (glej 
razdelek~\ref{Prostorska-koherenca}). Ocenimo moč tako dobljenega
koherentnega snopa za zaslonko.\index{Koherenčna ploskev}

Svetilo naj ima svetlost\footnote{~Svetilnost je moč, izsevana v dan 
prostorski kot; svetlost je svetilnost na enoto ploskve
$B= dP/Sd\Omega$.} $B$.  
Moč koherentnega snopa za zaslonko, ki prepušča svetlobo skozi prostorski kot 
$\Delta\Omega$, je (slika~\ref{fig:svetlost})\index{Svetilnost}\index{Svetlost}
\begin{equation}
P=BS_{0}\Delta \Omega =\frac{BS_{0}S_{c}}{z^{2}}\sim \frac{BS_{0}}{z^{2}}\,
\frac{\lambda ^{2}z^{2}}{S_{0}}=B\lambda ^{2}.
\label{5.21}
\end{equation}
Pri tem so $S_{0}$ površina svetila, $z$ oddaljenost zaslonke od svetila in
$S_{c}$ velikost koherenčne ploskve, za katero smo uporabili oceno 
(enačba~\ref{eq:koherencna-ploskev}). Da iz $S_0=1~\si{mm}^2$ velikega svetila 
dosežemo koherentni snop svetlobe z valovno dolžino $550~\si{nm}$, 
mora biti premer zaslonke, ki jo postavimo $1~\si{m}$ od izvora, 
okoli $0,6~\si{mm}$. V tem primeru znaša 
moč, ki prehaja skozi zaslonko, pri svetlosti $100~\si{W/cm^{2}}$ 
le približno $3\times 10^{-7}~\si{W}$.
Pri podobni divergenci snopa je močno navadno svetilo torej štiri rede
velikosti šibkejše od šibkih laserjev z močjo $1~\si{mW}$. 
\begin{figure}[ht]
\centering
\def\svgwidth{100truemm} 
\input{slike/06_svetlost.pdf_tex}
\caption{K izračunu moči koherentnega snopa svetlobe iz nekoherentnega svetila. $S_0$
je površina svetila, $z$ oddaljenost do zaslona, $\Delta \Omega$ prostorski kot odprtine
 in $S_c$ velikost koherenčne ploskve na zaslonu.}
\label{fig:svetlost}
\end{figure}

Druga odlična lastnost svetlobe iz enofrekvenčnega laserja je njena zelo majhna
spektralna širina, ki je lahko z nekaj truda pod $1~\si{kHz}$. Po drugi strani so emisijske 
črte v plinu zaradi Dopplerjeve razširitve široke vsaj nekaj $\si{GHz}$, 
pa še to le v razmeroma redkem in hladnem plinu, kjer je svetlost majhna.

Primerjajmo spektralno gostoto moči laserja in navadnih svetil. Majhen He-Ne
laser seva $1~\si{mW}$ v približno $10^{7}~\si{Hz}$ in spektralna gostota
moči je $dP/d\nu \sim 10^{-10}~\si{W/Hz}$. Po drugi strani zelo svetla 
živosrebrna svetilka seva v močno razširjene spektralne črte s širino okoli 
$10~\si{nm}$, kar ustreza $\sim 10^{13}~\si{Hz}$. \index{Spektralna gostota moči}
Spektralna gostota v koherentnem snopu, ki smo ga pripravili iz
navadne svetilke, je tako le okoli $3\times 10^{-20}~\si{W/Hz}$. Šolski
He-Ne laser torej prekaša najmočnejše nekoherentno svetilo za deset \index{Laser!He-Ne}
velikostnih redov. Z laserji je seveda mogoče doseči znatno večje
moči, v sunkih tipično do okoli $10^{12}$ W, tako da po spektralni gostoti moči v
koherentnem snopu laserji prekašajo običajna svetila za $20$--$25$
velikostnih redov. Verjetno v zgodovini težko najdemo kak drug izum, 
ki je prinesel tolikšno izboljšavo v neki bistveni količini, in tako ni 
nenavadno, da je iznajdba laserjev na začetku šestdesetih let dvajsetega stoletja povzročila preporod optike.

\section{*Stabilizacija frekvence laserja na nasičeno absorpcijo}
\label{chap:stabilizacija}
\index{Nasičena absorpcija!{stabilizacija frekvence}}\index{Spektralna črta}
Svetloba, izsevana iz laserjev, ima končno spektralno širino, ki 
je v realnih sistemih pogosto pogojena s fluktuacijami dolžine resonatorja ali 
temperature ojačevalnega sredstva. Eden od načinov, kako lahko zožamo spektralno črto,
je zato aktivna stabilizacija
dolžine resonatorja. Ideja je naslednja: svetlobo iz laserja primerjamo
z nekim standardom in iz razlike določimo spremembo dolžine. S
piezoelektričnim nosilcem nato premaknemo eno od zrcal in popravimo 
dolžino resonatorja. 

Poglavitna težava je najti dovolj stabilen primerjalni standard za frekvenco. 
Ena možnost je, da izhodno svetlobo usmerimo skozi pomožni 
interferometer, \index{Fabry-Perotov interferometer}
ki je skoraj v resonanci z laserjem in ima dovolj ozek vrh 
prepustnosti.\footnote{~R. W. P. Drever et al., Appl. Phys. B $\mathbf{31}$, 
97 (1983).}
Že majhen 
premik frekvence laserja povzroči spremembo prepuščenega svetlobnega toka. 
Na prvi pogled je videti, da z uporabo interferometra nismo pridobili, 
saj je tudi resonančna frekvenca interferometra stabilna le toliko, kot je 
stabilna njegova dolžina. Vendar je z izolacijo in temperaturno stabilizacijo 
mogoče ohranjati dolžino praznega resonatorja -- interferometra -- znatno
natančneje kot dolžino resonatorja, v katerem se nahaja aktivno sredstvo, ki mu 
dovajamo energijo.

Druga možnost je stabilizacija laserja na primerno molekularno absorpcijsko
črto.\footnote{~A. Brillet in P. Cerez, J. Phys. Col. $\mathbf{42}$, C8-73 (1981).}
Te so v razredčenem plinu lahko zelo ozke, zato je tudi spekter 
laserja lahko izredno ozek, lahko tudi pod $1~\si{kHz}$. 
Pri tem moti Dopplerjeva razširitev absorpcijske črte, vendar se ji
izognemo s pojavom nasičenja 
absorpcije. \index{Nasičena absorpcija}V razdelku
(\ref{NasabsNehom}) smo spoznali, da se pri dvakratnem prehodu
monokromatskega snopa svetlobe skozi plin v nasprotnih smereh pojavi v
sredini Dopplerjevo razširjene črte Lambova vdolbina,\index{Lambova vdolbina} ki 
ima obliko homogeno razširjene črte (slika~\ref{fig:Lamb}). 
Homogena širina je navadno mnogo manjša od Dopplerjeve in
je zato vdolbina uporabna kot frekvenčni standard. 

S spreminjanjem dolžine resonatorja spreminjamo frekvenco laserja, in 
ko se ta znotraj homogene širine približa sredini absorpcijske črte (pri $\omega_0$), 
se absorpcija zmanjša in moč laserja poveča. To naredimo tako, da 
eno od zrcal pritrdimo na piezoelektrični nosilec, na katerega je priključena
izmenična napetost s krožno frekvenco $\Omega$. Zaradi periodičnega spreminjanja dolžine
laserja se spreminja izhodna moč. Kadar je krožna frekvenca laserja 
enaka $\omega_0$, se izhodna 
moč spreminja z dvojno krožno frekvenco modulacije $2\Omega$. 
Kadar pa srednja krožna frekvenca laserja odstopa od $\omega_0$, se izhodna moč 
pri odmiku zrcala v eno stran spremeni drugače kot pri premiku v drugo stran, 
kar pomeni, 
da je v izhodnem signalu tudi komponenta s krožno frekvenco $\Omega$. Da držimo 
srednjo krožno frekvenco laserja enako $\omega_0$, moramo torej meriti komponento 
izhodne moči pri modulacijski frekvenci in jo s povratno zanko ohranjati na nič.

Napravimo še kvantitativno oceno opisane stabilizacijske sheme. Odvisnost
izhodne moči od krožne frekvence laserja $\omega$ lahko približno zapišemo v
obliki (enačba~\ref{eq:spekter-primer})
\begin{equation}  
\label{5.40}
P(\omega)=P_0 + \frac{P_1\gamma^2}{(\omega- \omega_0)^2+\gamma^2}.
\end{equation}
Pri tem so $P_0$ moč laserja brez saturacijskega vrha pri $\omega_0$, $P_1$ 
povečanje moči pri $\omega_0$ in $\gamma$ homogena širina črte. Privzeli  
smo, da se ojačenje laserja in nehomogeno razširjeni del absorpcije ne 
spreminjata znatno znotraj homogene širine absorberja in je zato moč $P_0$ 
približno konstantna. Krožno frekvenco laserja moduliramo, tako da je 
\begin{equation}  
\label{5.41}
\omega=\omega_0+\Delta\omega+b \sin \Omega t.
\end{equation}
Z $\Delta\omega$ smo označili odstopanje srednje krožne frekvence laserja od
sredine absorpcijske črte $\omega_0$ in z $b$ amplitudo modulacije. 

Če sta $b$ in $\Delta\omega$ majhna v
primerjavi z $\gamma$, lahko imenovalec v enačbi~(\ref{5.40})
razvijemo
\begin{equation}  
\label{5.42}
P=P_0+P_1 \left(1-\frac{\Delta\omega^2}{\gamma^2} +\frac{2}{\gamma^2} b
\Delta\omega \sin\Omega t - \frac{b^2}{\gamma^2}\sin^2\Omega t\right)\!.
\end{equation}
Amplituda signala pri krožni frekvenci modulacije $\Omega$ je 
\begin{equation}
P(\Omega) = \frac{2P_1 b \Delta\omega}{\gamma^2}. 
\end{equation}
Najmanjša moč, ki jo ločeno zaznamo na detektorju, je določena s šumom 
meritve. Kot bomo videli v razdelku~\ref{chap:sum}, je najmanjša 
sprememba svetlobne moči, ki jo lahko izmerimo, enaka 
\begin{equation}  
\label{5.43}
P_N\approx \sqrt{\hslash\omega P \frac{1}{\tau}},
\end{equation}
pri čemer je $P$ celotna vpadna svetlobna moč, $\tau$ pa čas
meritve. Vzemimo za primer He-Ne laser\index{Laser!He-Ne}. Stabiliziramo ga tako,
da vanj dodamo jodovo komoro, v kateri pride do nasičene absorpcije.
Povprečna moč laserja naj bo $P_0 = 10~\si{mW}$ in $P_1= 0,1~\si{mW}$. 
Širina absorpcijske črte je $\gamma= 10^6~\si{s}^{-1}$. Izberimo amplitudo 
modulacije $b=10^5~\si{s}^{-1}$ in $\tau=10^{-4}~\si{s}$. Časovna konstanta 
$\tau$ ne sme biti prevelika, določa namreč, kako hitro prilagajamo dolžino 
laserja. Za najmanjšo zaznavno moč pri $\Omega$ dobimo 
$P_N(\Omega) = 5\times10^{-9}~\si{W}$.
Najmanjše merljivo odstopanje krožne frekvence laserja je tedaj 
\begin{equation}  
\label{5.44}
\Delta\omega_N=\frac{P_N \gamma^2}{2P_1 b}=250~\si{s}^{-1}.
\end{equation}
Takšno in še boljšo stabilnost frekvence se danes lahko doseže. 

\begin{remark}
Na absorpcijsko črto stabiliziranega laserja navadno ne uporabljamo
neposredno, temveč z njim kontroliramo drug laser. Del izhodne svetlobe iz
obeh laserjev zmešamo na detekcijski fotodiodi. V signalu dobimo utripanje,
ki je enako razliki frekvenc obeh laserjev. S spreminjanjem dolžine drugega
laserja skrbimo, da je frekvenca utripanja konstantna. Na ta način lahko v
ozkem frekvenčnem intervalu še spreminjamo frekvenco drugega laserja.
Z opazovanjem utripanja med dvema stabiliziranima laserjema določamo tudi
njuno stabilnost.
\end{remark}

\section{Večfrekvenčni laser}
\label{section:vecfrek}
Do zdaj smo obravnavali laserje, v katerih je bilo vzbujeno eno samo stoječe
valovanje. Vendar je ojačevalna širina večine aktivnih (ojačevalnih) sredstev 
navadno večja od razlike med frekvencami posameznih \index{Laser!večfrekvenčni}
nihanj resonatorja. V plinih, na primer, je ojačevalna širina zaradi 
Dopplerjevega pojava vsaj nekaj $\si{GHz}$, medtem ko so lastne frekvence 
resonatorja 
pri $30~\si{cm}$ dolgem resonatorju razmaknjene za $500~\si{MHz}$. Ojačenje 
v laserju tako za več lastnih nihanj hkrati 
presega ojačenje na pragu. Ker je v resonatorju vzbujenih več lastnih nihanj, 
svetloba iz takega večfrekvenčnega laserja ni monokromatska,
temveč je sestavljena iz množice ozkih črt znotraj ojačevalnega pasu.
Izsevana svetloba ni bistveno bolj monokromatska od ustrezne spektralne 
komponente svetlečega plina. Ostaja seveda prostorsko koherentna.

Za holografijo, interferometrijo in nekatere spektroskopske metode
potrebujemo ozko spektralno črto. Zato moramo poskrbeti, da je v laserju vzbujeno le
eno lastno nihanje, najbolje tisto, ki je najbližje vrhu ojačenja
aktivnega sredstva. To dosežemo tako, da za vsa ostala nihanja povečamo izgube,
na primer s Fabry-Perotovim etalonom, \index{Fabry-Perotov interferometer}
ki ga vstavimo v laserski resonator
(slika~\ref{fig:FPres}). 

\begin{figure}[ht]
\centering
\def\svgwidth{70truemm} 
\input{slike/06_FPres.pdf_tex}
\caption{Shema laserja z vstavljenim Fabry-Perotovim etalonom}
\label{fig:FPres}
\vglue-2truemm
\end{figure}

Prepustnost etalona je odvisna od krožne frekvence 
svetlobe $\omega$, odbojnosti na stenah 
${\cal R}$, debeline ploščice $L$, lomnega količnika ploščice $n$ in kota $\vartheta$, pod katerim potuje
svetloba znotraj ploščice glede na normalo 
 (enačba~\ref{eq:Fabry-Perot-prepustnost}):
\begin{equation}
T=\frac{1}{1+\frac{4{\cal {R}}}{(1-{\cal {R}})^{2}}\sin^{2}
(\omega L n \cos \vartheta/c)}.
\label{5.22}
\end{equation}
\vglue-3truemm
\begin{figure}[ht]
\centering
\def\svgwidth{80truemm} 
\input{slike/06_FPmodes.pdf_tex}
\caption{Lastne frekvence resonatorja (a) in frekvenčna odvisnost 
ojačenja z označenim pragom delovanja (b). Z rdečo črto je označeno ojačenje, s črtkano
rdečo črto prag delovanja, z zeleno pa tiste 
lastne frekvence, ki so nad pragom in se v laserju ojačujejo. 
Ko dodamo Fabry-Perotov etalon z dano prepustnostjo (c, modra črta), povečamo
izgube za vse načine, ki bi bili ojačeni, razen za enega. 
Tako dosežemo delovanje laserja pri eni sami frekvenci (d).}
\label{fig:FPmodes}
\end{figure}

Debelino etalona in njegov nagib 
izberemo tako, da vrh prepustnosti sovpada z izbranim lastnim nihanjem resonatorja. 
Izgube za ostala nihanja, ki bi sicer bila ojačena, so znatno večje in laser
sveti pri eni sami izbrani frekvenci. Ta proces je shematsko prikazan na 
sliki~\ref{fig:FPmodes}: izmed vseh možnih stanj v resonatorju~(a) svetijo le tista, za katere
je ojačenje nad pragom~(b). Ko v resonator dodamo Fabry-Perotov etalon z razmeroma velikim
razmikom med sosednjimi vrhovi prepustnosti~(c), je ojačeno eno samo nihanje~(d). 
Ker zadošča že zmerno povečanje izgub, je odbojnost sten etalona navadno dokaj nizka, 
pod $0,5$. 

Opisali smo, kako v laserju dosežemo delovanje pri enem samem longitudinalnem nihanju.
Poleg tega je treba omejiti tudi ojačenje višjih prečnih nihanj. To navadno 
dosežemo z zaslonkami, saj imajo snopi višjih redov večji efektivni polmer od osnovnega. 
\vglue-3truemm
\begin{remark}
Nagib etalona omogoča natančno spreminjanje izbrane frekvence, poleg tega
 je nujen, da se neprepuščena svetloba odbije ven iz smeri osi resonatorja. Če bi 
bila os etalona vzporedna z osjo resonatorja, bi se pojavile dodatne resonance, 
kar bi močno motilo delovanje laserja. Poleg Fabry-Perotovega etalona se uporabljajo
tudi prizme in uklonske mrežice.
\end{remark}

\section{Relaksacijske oscilacije}
\index{Relaksacijske oscilacije}
Stacionarno delovanje laserjev smo že dodobra spoznali. Za obravnavo
nestacionarnega delovanja  se moramo vrniti k obravnavi zasedbenih enačb 
(\ref{5.7}) in (\ref{5.8}), ki jih je na splošno treba reševati numerično. 
Vseeno lahko nekaj povemo o obnašanju laserja
v bližini stacionarnega delovanja. 

Spet se omejimo na enofrekvenčni laser, v katerem zasedenost vzbujenega stanja $N_2$
in število fotonov $n$ opišemo z enačbama (\ref{5.7})
in (\ref{5.8}). Najprej zaradi preglednosti vpeljemo
brezdimenzijski čas $t^{\prime}=t A$ in $\tau^{\prime}=\tau A$, kar pomeni, da merimo 
čas v enotah življenjskega časa laserskega nivoja. Ponovno uporabimo parameter
$p=VA/(B\hslash\omega g)$ (enačba~\ref{5.10}), ki pomeni število lastnih stanj 
elektromagnetnega polja v volumnu $V$ in znotraj spektralne širine laserskega prehoda. 
Enačbi zapišemo kot
\begin{align}  
\frac{d N_2}{d t^{\prime}}&=-\frac{nN_2}{p}-N_2+N_{20} \label{5.23a} \qquad \mathrm{in}\\
\frac{d n}{d t^{\prime}}& =  \frac{nN_2}{p}-\frac{2}{\tau^{\prime}}n.
\label{5.23}
\end{align}
Pri tem smo vpeljali konstanto $N_{20}= rN/A$, ki ima tudi nazoren pomen.
Predstavlja zasedenost, ki bi jo dobili pri danem stacionarnem črpanju, če v
izbranem stanju ne bi bilo fotonov in s tem ne stimuliranega sevanja. Meri torej 
moč črpanja. V enačbi~(\ref{5.23})
smo zanemarili prispevek spontanega sevanja, za katerega smo že ugotovili,
da se pozna le do praga.

Poiščimo približne rešitve sistema sklopljenih diferencialnih enačb z 
linearizacijo. Naj laser najprej deluje stacionarno, nato se v nekem trenutku  
nekoliko izmakne iz stacionarnega stanja. To se lahko zgodi, na primer, če v nekem 
trenutku spremenimo moč črpanja. Trenutno zasedenost $N_2$ in število fotonov $n$
zapišemo v obliki 
\begin{equation}  
N_2= N_{2s}+x \qquad \mathrm{in} \qquad n=n_s+y,
\label{5.24}
\end{equation}
pri čemer sta $N_{2s}$ in $n_s$ vrednosti v stacionarnem stanju. Zanju velja 
\begin{equation}  
N_{2s}=\frac{2p}{\tau^{\prime}}\qquad \mathrm{in}\qquad  
n_s=p\frac{N_{20}-N_{2s}}{N_{2s}}=p(a-1).
\label{5.26}
\end{equation}
Prva enačba je v skladu s tem, da je stacionarna zasedenost 
enaka zasedenosti na pragu, ta pa je odvisna od izgub resonatorja. 
Razmerje $a=N_{20}/N_{2s}$ je mera za moč črpanja in
mora biti v delujočem laserju večje od 1. V večini praktičnih primerov
doseže vrednosti $a \sim 5$.

Vstavimo nastavka (enačbi~\ref{5.24}) v enačbi (\ref{5.23a} in \ref{5.23}). 
Dobimo sistem enačb
\begin{align}  
\frac{d x}{d t^{\prime}} &=-\frac{n_sN_{2s}}{p}-N_{2s}+N_{20}- \frac{1}{p}
(n_sx+N_{2s}y+xy)-x \qquad \mathrm{in}\\
\frac{d y}{d t^{\prime}} &= \frac{n_sN_{2s}}{p}-\frac{2}{\tau^{\prime}}n_s
+ \frac{1}{p}(n_s x+N_{2s} y+xy)-\frac{2}{\tau^{\prime}}y.
\label{5.27}
\end{align}

Ker sta $x$ in $y$ majhna v primerjavi s stacionarnimi vrednostmi, lahko
mešani produkt $xy$ zanemarimo. Vsi členi, ki vsebujejo le stacionarne vrednosti,
dajo ravno 0, saj smo jih tako določili. Če upoštevamo še izraza 
za stacionarni vrednosti (enačbi~\ref{5.26}), 
sta linearizirani diferencialni enačbi za odmika od stacionarnih vrednosti 
\begin{equation}
\frac{dx}{dt^{\prime }} =-a\,x-\frac{2}{\tau ^{\prime }}\,y   \qquad \mathrm{in} \qquad
\frac{dy}{dt^{\prime }} =(a-1)\,x.
\label{5.28}
\end{equation}
Rešitev sistema diferencialnih enačb s konstantnimi
koeficienti poiščemo z nastavkom
\begin{equation}
x=x_{0}e^{\lambda t^{\prime}} \qquad \mathrm{in} \qquad 
y=y_{0}e^{\lambda t^{\prime}}.
\label{5.29}
\end{equation}
Dobimo homogen sistem linearnih enačb 
\begin{align}
(a+\lambda )x_{0}+\frac{2}{\tau ^{\prime }}y_{0} &=0  \label{5.30} \qquad \mathrm{in}\\
-(a-1)x_{0}+\lambda y_{0} &=0,
\end{align}
ki ima netrivialno rešitev le, če je njegova determinanta enaka nič
\begin{equation}
\lambda ^{2}+a\lambda +\frac{2}{\tau ^{\prime }}(a-1)=0.  
\label{5.301}
\end{equation}
Rešitvi sta 
\begin{equation}
\lambda =-\frac{a}{2}\pm \sqrt{\frac{a^{2}}{4}-\frac{2}{\tau ^{\prime }}(a-1)}.
\label{5.31}
\end{equation}
Obnašanje rešitve je odvisno od velikosti brezdimenzijskega relaksacijskega
časa nihanja resonatorja $\tau ^{\prime }=A\tau $. 

Kratek račun pokaže, da je 
za $\tau ^{\prime }>2$ izraz pod korenom za vse vrednosti $a$ pozitiven in laser 
se eksponentno vrača v stacionarno stanje. Za $\tau' <2$ je koren v določenem območju
parametra $a$ imaginaren in laser se v stacionarno stanje vrača z
dušenim nihanjem, ki mu pravimo relaksacijske 
oscilacije.\index{Relaksacijske oscilacije}
Primer takega nihanja 
pri vključitvi laserja kaže slika~\ref{fig:relax}. Opazimo, da sta tako 
frekvenca relaksacijskih oscilacij kot karakteristični čas dušenja oscilacij
odvisna od parametra $a$, torej od moči črpanja.\footnote{~O. Svelto in D. C. Hanna, 
{\it Principles of Lasers}, peta izdaja, 
Springer (2010).}
\begin{figure}[ht]
\centering
\def\svgwidth{80truemm} 
\input{slike/06_relax.pdf_tex}
\caption{Relaksacijske oscilacije števila fotonov $n$ po vklopu laserja. Po začetnem
nihanju se število fotonov ustali pri stacionarni vrednosti $n_s$.}
\label{fig:relax}
\end{figure}

Poglejmo primer. Vzemimo vrednost $\tau \sim 10^{-7}~\si{s}$ in razpadno
konstanto laserskega nivoja $A \sim 10^5/\si{s}$. Tedaj je $\tau^{\prime}\sim 10^{-2}$ 
in relaksacijske oscilacije se pojavijo pri vseh dosegljivih vrednostih črpanja 
nad pragom, to je za $a>1$. Ker $a$ v praksi ni nikoli dosti večji od $3$--$5$, 
je krožna frekvenca oscilacij v brezdimenzijskih enotah približno
enaka $\omega^{\prime}_r\sim 1/\sqrt{\tau^{\prime}}$. Ko preidemo
nazaj na enote časa, dobimo $\omega_r\sim \sqrt{A/\tau}$. Krožna frekvenca 
relaksacijskih oscilacij je v tem primeru velikostnega reda geometrijske 
sredine med razpadnima konstantama nihanja resonatorja in atomskega stanja. 
Tipične vrednosti frekvence relaksacijskih oscilacij so $\sim 10^5~\si{Hz}$, 
karakteristični čas dušenja pa $\sim 10^{-5}~\si{s}$.
\vglue-3truemm
\begin{remark}
Relaksacijske oscilacije so praktično pomembne, saj določajo zgornjo mejo
hitrosti, s katero lahko izhodna moč laserja sledi modulaciji črpanja.
Poleg tega se pri tej frekvenci pojavi resonanca, pri kateri se šum črpanja
ojačeno prenaša v šum izhodne moči. 
\end{remark}

\section{Sunkovni laserji}
Kadar potrebujemo veliko izhodno moč laserske svetlobe pri zmerni povprečni porabi 
energije, zvezno delujoči laserji niso primerni. Uporabiti moramo sunkovne laserje, 
ki v kratkem časovnem intervalu delujejo z zelo veliko izhodno močjo. 
\index{Sunkovni laser}

Poglejmo primer. 
Sunkovni laser naj oddaja svetlobo v $10~\si{ns}$ dolgih sunkih 
s povprečno energijo $1~\si{J}$ in s ponovitvijo $1000/\si{s}$.
Povprečna moč, s katero deluje tak laser, je $1~\si{kW}$ in moč, ki jo 
dosega v sunkih, $100~\si{MW}$. Če to primerjamo z zvezno delujočim laserjem  
z izhodno močjo $1~\si{kW}$, vidimo, da pri isti povprečni moči
dosežemo moči, ki so za $5$--$10$ velikostnih redov večje od moči 
zvezno delujočih laserjev. 

V grobem ločimo dva načina sunkovnega delovanja laserjev. V prvem primeru
periodično spreminjamo črpalno moč, v drugem pa periodično 
spreminjamo izgube v resonatorju. V to skupino uvrščamo laserje na 
preklop dobrote in laserje, ki uklepajo faze valovanj.   

Obravnavajmo najprej način, pri katerem periodično moduliramo črpanje.
To dosežemo na primer z bliskavico ali drugim sunkovnim laserjem. Ko je 
črpanje nad pragom, svetloba izhaja iz laserja.\index{Sunkovni 
laser!{preklop črpanja}}\index{Črpanje}
Ko je črpanje pod pragom, so izgube prevelike in laser ne 
sveti (slika~\ref{fig:Gswitch}\,a). Za dosego velike
obrnjene zasedenosti v ojačevalnem sredstvu mora biti sprememba črpanja zelo hitra, 
da v času preklopa števila fotonov še ne poveča znatno. 
Najpogosteje se modulacija črpanja uporablja
v polprevodniških laserjih, saj v njih črpanje poteka z električnim tokom, ki 
ga je zelo preprosto modulirati z razmeroma velikimi frekvencami. 
\begin{figure}[ht]
\centering
\def\svgwidth{140truemm} 
\input{slike/06_pulseG.pdf_tex}
\caption{Delovanje sunkovnega laserja s periodično moduliranim črpanjem (a). Laser sveti,
kadar je črpanje dovolj veliko in ojačenje $G$ nad pragom $G_{\mathrm{pr}}$. 
Pri hitrem preklapljanju črpanja se lahko pojavijo neželene oscilacije izhodne moči $P$ in 
izhodni sunek se popači (b).}
\label{fig:Gswitch}
\end{figure}

Pri modulaciji črpanja se lahko pojavi težava. Ko ob močnem črpanju 
obrnjena zasedenost znatno preseže zasedenost praga (v nestacionarnem stanju 
je to mogoče), laser posveti in v kratkem času zasedenost pade nazaj pod prag. 
Če tedaj črpanje še traja, zasedenost naraste in laser ponovno posveti. 
To se lahko večkrat ponovi. Razmiki med sunki\index{Relaksacijske oscilacije}
so reda velikosti periode relaksacijskih oscilacij in so lahko precej
nepravilni  (slika~\ref{fig:Gswitch}\,b). Pri takem delovanju
razpoložljiva energija črpanja ne izstopa iz laserja v obliki enega samega lepo oblikovanega sunka, 
ampak v zaporedju sunkov. Omenjene oscilacije močno omejujejo uporabnost tega pristopa.

\section{Delovanje v sunkih s preklopom dobrote}
\index{Preklop dobrote|see{Sunkovni laser}}
\label{qswitch}
Namesto modulacije črpanja lahko v sunkovnih laserjih periodično spreminjamo 
izgube.\index{Sunkovni laser!{preklop dobrote}} Čim večje so namreč izgube, višji 
je prag za delovanje laserja. \index{Obrnjena zasedenost}\index{Izgube v resonatorju}
Posledično lahko dosežemo večjo stopnjo obrnjene zasedenosti in v sistemu atomov shranimo več 
energije kot pri majhnih izgubah.\index{Sunkovni laser}
Ko je enkrat ustvarjena velika obrnjena zasedenost, izgube zelo hitro zmanjšamo. 
Optično ojačenje je veliko in energija svetlobe v kratkem času močno naraste. 
S tem se tudi obrnjena zasedenost hitro zniža na vrednost močno pod pragom.
Podobno kot prvi nihaj relaksacijskih oscilacij, le da
je začetno stanje daleč od stacionarnega in zato linearni približek ne drži več.
Iz laserja dobimo kratek in zelo močan sunek svetlobe.\footnote{~F. J.
McClung in R. W. Hellwarth, J. Appl. Phys. $\mathbf{33}$, 828 (1962).} Energija
sunka je skoraj tolikšna, kot je bila energija obrnjene zasedenosti. 
Tipične dolžine sunkov, ki jih dosežemo na ta način, so $t \sim 10~\si{ns}$, sunki
pa se ponavljajo s frekvenco $\nu \sim 1$--$100~\si{kHz}$.
Dogajanje v laserju kaže slika~\ref{fig:pulseQ}.
\vglue-3truemm
\begin{remark}
V elektrotehniki se izgube resonatorjev podajajo z dobroto $Q$, to je razmerjem
med frekvenco lastnega nihanja in njegovo širino (enačba~\ref{eq:qdef}). 
Ker s povečanjem izgub spremenimo 
širino črte in z njo dobroto, opisano tehniko imenujemo preklop 
dobrote.\index{Dobrota resonatorja}
\end{remark}
\vglue-5truemm
\begin{figure}[ht]
\centering
\def\svgwidth{90truemm} 
\input{slike/06_pulseQ.pdf_tex}
\caption{Delovanje laserja v režimu preklopa dobrote: izgube $\Lambda$ (a), relativno 
ojačenje $G/G_\mathrm{pr}$ (b), 
zasedenost višjega nivoja $N_2$ (c) in moč izsevane svetlobe $P$ (d) v odvisnosti od časa $t$.}
\label{fig:pulseQ}
\end{figure}

Izgube resonatorja je mogoče spreminjati na več načinov. Najpreprosteje
je mehansko, na primer z vrtečo prizmo ali zrcalom. Tedaj je resonator uglašen le v kratkem
intervalu, ko žarek vpada pravokotno na zrcalo resonatorja. Bolj razširjen način 
je z vgradnjo elektro-optičnega ali akusto-optičnega modulatorja, o katerih
bomo pisali v nadaljevanju (poglavje~\ref{chap:modulacija}). Na kratko povejmo, 
da z njimi izgube preklapljamo zelo hitro, saj to naredimo s spreminjajočo se 
električno napetostjo.

Kot smo že zapisali, sistema nelinearnih enačb za zasedenost vzbujenega stanja $N_2$ 
in število fotonov $n$
(enačbi~\ref{5.7} in \ref{5.8}) ne moremo analitično rešiti. Vendar lahko na hitro
naredimo nekaj ocen. Dolžina izhodnega sunka je odvisna od hitrosti, 
s katero se izprazni zgornji laserski nivo. To se ne more zgoditi
hitreje kot v nekaj preletih sunka skozi resonator. Trajanje sunka je torej
vsaj nekajkrat $2L/c$, to je za $15~\si{cm}$ dolg resonator vsaj nekaj $\si{ns}$.

Ocenimo še hitrost naraščanja števila fotonov na začetku in 
njegovega upadanja na koncu sunka. Še enkrat zapišemo enačbi za zasedenost in za število
fotonov, pri čemer nas zanima le dogajanje v času sunka,
ki je zelo kratek v primerjavi z atomskim razpadnim časom, zato 
ustrezni člen v enačbi (\ref{5.7}) zanemarimo. Navadno je tudi črpanje prešibko, da
bi med sunkom samim znatno vplivalo na zasedenost, zato tudi člen $rN$
izpustimo. Seveda je črpanje upoštevano v tem, da je v sistemu velika začetna 
zasedenost $N_{20}$. Tako ostane 
\begin{equation}
 \frac{d N_2}{d t}=-\frac{\sigma c}{V}\,n\,N_2 
 \label{5.32a}
\end{equation}
in 
\begin{equation}
 \frac{d n}{d t}=\frac{\sigma c}{V}\,n\,N_2 - \frac{2}{\tau}\,n.
 \label{5.32}
\end{equation}
Na začetku sunka je $N_2$ velik in se ne razlikuje dosti 
od začetne vrednosti $N_{20}$. Takrat 
je vrednost $n$ majhna, zato drugi člen v enačbi~(\ref{5.32})  
zanemarimo. Izračunamo, da število fotonov
na začetku sunka narašča približno eksponentno
\begin{equation}  
n(t)=n_0e^{t\,N_{20}\,\sigma c/V}= n_0e^{t/\tau_r},
\label{5.33}
\end{equation}
pri čemer je $t_r = V/N_{20} \sigma c$.
Začetnega števila fotonov ne poznamo, vendar vemo, da je velikost\-ne\-ga reda 1,
saj predstavlja spontano emisijo. Da $n$ znatno naraste, recimo nad 
$10^{10}$ fotonov, je potreben čas blizu $25~\tau_r$.
Proti koncu sunka $N_2$ pojema zaradi sevanja svetlobe in v enačbi~(\ref{5.32}) lahko 
zanemarimo člen z $N_2$. Ostane samo še drugi člen, ki da preprosto rešitev
\begin{equation}  
n(t)=\tilde{n}_0e^{-2t/\tau}.
\label{5.33a}
\end{equation}
Eksponentno pojemanje števila fotonov na koncu sunka je torej določeno z izgubami
v resonatorju (enačba~\ref{taulambda}). 

Dogajanja v vmesnih časih ne moremo enostavno popisati, vendar lahko najdemo
medsebojno zvezo med $n$ in $N_2$. 
Izrazimo $dt$ iz enačbe (\ref{5.32a}) in ga vstavimo v enačbo~(\ref{5.32}).
Dobimo
\begin{equation}
dn=-dN_{2}+\frac{\tilde{N}_2}{N_{2}}dN_{2},\label{5.341}
\end{equation}
pri čemer smo zapisali, da je $\tilde{N}_{2}=2V/(\sigma c\,\tau)$.

Enačbo brez težav integriramo
\boxeq{5.351}{
n=N_{20}-N_{2}+\tilde{N}_{2}\ln \frac{N_{2}}{N_{20}}.
}
Pri tem smo privzeli, da je na začetku sunka $n=0$ in $N_{2}=N_{20}$. 
Najprej izračunamo, kolikšna je zasedenost na koncu sunka 
$N_{2k}$. Izhajamo iz pogoja, da je število fotonov na koncu sunka $n=0$, 
kar da transcendentno enačbo za $N_{2k}$
\beq
\ln \frac{N_{2k}}{N_{20}} = \frac{N_{2k}}{\tilde{N}_2}- \frac{N_{20}}{\tilde{N}_2}.
\label{xaenacba}
\eeq
Enačbo preprosto numerično rešimo (slika~\ref{fig:Qeq}).

\begin{figure}[ht]
\centering
\def\svgwidth{90truemm} 
\input{slike/06_Qeq.pdf_tex}
\caption{Rešitev enačbe~(\ref{xaenacba}), ki pove, kolikšna je
zasedenost vzbujenega stanja na koncu sunka $N_{2k}$ pri dani začetni 
zasedenosti $N_{20}$. Večja kot je začetna zasedenost, bolj se 
izprazni vzbujeni nivo.
}
\label{fig:Qeq}
\end{figure}

Če vpeljemo parametra $x=N_{2k}/\tilde{N}_2$ in $a=N_{20}/\tilde{N}_{2}$, se
enačba prepiše v 
\beq
\ln \frac{x}{a}= x-a.
\eeq
Za vrednosti $a<1$ enačba nima rešitev in 
v $\tilde{N}_{2}$ prepoznamo zasedenost vzbujenega stanja na pragu delovanja. Kadar je začetna 
zasedenost $N_{20}$ le malo nad pragom, končna zasedenost $N_{2k}$ ne pade 
dosti pod prag, zato je izraba energije slabša. Pri večjih začetnih vrednostih 
$N_{20}$ pade končna zasedenost praktično na nič. Za $a=2$, na primer,
je $x=0,41$, medtem ko je že pri $a=4$ vrednost $x$ le še 0,08. 

Ko enkrat poznamo začetno in končno zasedenost, lahko izračunamo 
celotno energijo v sunku $W=\hslash \omega (N_{20}-N_{2k})$. Kadar je začetna vrednost
dovolj nad pragom ($N_{20}\gg\tilde{N}_{2}$), končno zasedenost zanemarimo ($N_{2k}\ll\tilde{N}_{2}$) in 
celotna energija svetlobe, izsevane v sunku, je 
\boxeq{QW}{
W\approx\hslash \omega\,N_{20}.
}
\begin{definition}
\label{nalpmax}
Trenutna moč svetlobe, ki izhaja iz laserja, je enaka 
\begin{equation}
P=n \hslash \omega \frac{2}{\tau}.
\end{equation}
Pokaži, da je trenutna izsevana moč največja natanko tedaj, ko pade zasedenost na vrednost pri 
pragu ($N_{2}=\tilde{N}_{2}$). 
\end{definition}

Največja izsevana moč iz laserja je \index{Izhodna moč laserja}
\beq
P_{\rm max}=\frac {n_{\rm max} \hslash \omega}{2L/c}\left(1-\cal{R}\right) = 
\frac {2 n_{\rm max} \hslash \omega}{\tau},
\eeq
pri čemer smo z $n_\textrm{max}$ označili število fotonov v vrhu sunka. 
Ko vstavimo še vrednost za $n_{\rm max}$ pri $N_{2}=\tilde{N}_{2}$ (glej 
nalogo~\ref{nalpmax}), dobimo
\beq
P_{\rm max}=\frac {2\hslash \omega}{\tau} \left(N_{20}-\tilde{N}_{2}-\tilde{N}_{2}
\ln (N_{20}/\tilde{N}_{2})\right)\!.
\eeq
Ker je navadno $N_{20}\gg \tilde{N}_2$, je $n_{\rm max} \approx N_{20}$
in 
\boxeq{QP}{
P_{\rm max} \approx \frac{2}{\tau} \hslash \omega N_{20}.
}

Poglejmo primer. Naj bosta presek za stimulirano sevanje $\sigma=B\hslash \omega g/c$ 
okoli $10^{-19}~\si{cm}^{2}$ in začetna gostota zasedenosti $N_{20}/V=10^{19}~\si{cm}^{-3}$.
Tedaj sta $\tau_{r}=30~\si{ps}$ in čas naraščanja
sunka $\sim 1~\si{ns}$. Število fotonov se nato zmanjšuje s
karakterističnim razpadnim časom resonatorja $\tau /2\sim 2L/(c(1-{\cal R}))$. 
Za dosego kratkih sunkov svetlobe je zato v laserjih s preklopom dobrote odbojnost 
izhodnega zrcala navadno dokaj nizka, recimo $0,5$. Pri dolžini resonatorja 
$L=15~\si{cm}$ je $\tau=4~\si{ns}$.
Celotno trajanje sunka je v izbranem primeru $~\sim~10~\si{ns}$, pri
čemer traja okoli $100~\si{ns}$ od preklopa dobrote do trenutka, ko sunek zraste iz šuma
spontanega sevanja. Energija v sunku je blizu $N_{20}\hslash \omega $, kar pri
aktivnem volumnu $0,5~\si{cm}^3$ znaša $\sim~1~\si{\joule}$. Grobo ocenimo, da je
izhodna moč v vrhu sunka $\sim~100~\si{MW}$.

\section{Uklepanje faz}
\index{Uklepanje faz}\index{Sunkovni laser}
\label{chap:Uklepanje}
Krajše sunke kot s preklopom dobrote je mogoče dobiti z uklepanjem 
faz.\footnote{~L. E. Hargrove, R. L. Fork in M. A. Pollack, Appl. Phys. Lett. $\mathbf{5}$, 4 (1964).}
Pri tem gre za povsem drugačen način, ki je prav presenetljiva manifestacija 
koherentnosti laserske svetlobe. Spoznali smo že, da je v laserju navadno 
vzbujenih več lastnih nihanj hkrati, pri čemer so njihove krožne frekvence 
enakomerno razmaknjene za $\Delta \omega =\pi c/L$ 
(enačba~\ref{eq:delta-omega-resonator}). Celotno električno
polje v neki točki v laserju zapišemo kot vsoto 
\begin{equation}
E(t)=\sum_{m=-N/2}^{N/2}A_{m}e^{-i(\omega _{0}+m\Delta \omega )t+i\varphi
_{m}(t)},
\label{5.342}
\end{equation}
pri čemer je $N$ število vseh vzbujenih nihanj. Upoštevali smo, da ima vsako
vzbujeno nihanje lahko poljubno fazo $\varphi _{m}(t)$, ki je na splošno predvsem
zaradi zunanjih motenj slučajna funkcija časa. Zaradi tega se
celotno polje v resonatorju slučajno spreminja, kar močno zmanjšuje uporabnost
takega laserja.

Denimo, da nekako dosežemo enake faze vseh nihanj, ki naj bodo vse enake nič. 
Poleg tega zaradi enostavnosti
računa privzamemo, da so enake tudi vse amplitude vzbujenih nihanj $A_{m}= A_0$. Tedaj
postane vsota (\ref{5.342}) geometrijska in jo brez težav seštejemo 
\begin{equation}
E(t)=A_{0}\,e^{-i\omega _{0}t}\frac{\sin (N\Delta \omega t/2)}{\sin(\Delta
\omega t/2)}.
\label{5.352}
\end{equation}
Moč izhodne svetlobe ima časovno odvisnost (slika~\ref{s5.10})\index{Izhodna moč laserja}
\begin{equation}
P(t)=P_{0}\,\frac{\sin ^{2}(N\Delta \omega t/2)}{\sin ^{2}(\Delta \omega t/2)}.
\label{5.36}
\end{equation}

\begin{definition}
\label{naloga:mlock}
Pokaži, da izhodno moč iz laserja v primeru enakih faz vzbujenih lastnih nihanj
zapišemo z enačbo~(\ref{5.36}). Pokaži še, da je razmik med posameznimi sunki (vrhovi
moči) enak $T=2L/c$, dolžina posameznega sunka $\tau = T/N$, vrednost moči v vrhu 
sunka $N^{2}P_{0}$ in povprečna moč $NP_{0}$.
\end{definition}

Izhodna svetloba predstavlja zaporedje sunkov, 
ki si sledijo s periodo $T=2\pi /\Delta \omega =2L/c$, kar je enako času obhoda
svetlobe v resonatorju (glej nalogo~\ref{naloga:mlock}).  Izračunamo lahko tudi 
dolžino sunkov
\begin{equation}
\tau=\frac{T}{N}=\frac{2\pi }{N\Delta \omega }=\frac{2\pi }{\Delta\omega_{G}} = 
\frac{1}{\Delta\nu_{G}}.
\label{5.37}
\end{equation}
\begin{figure}[ht]
\centering
\def\svgwidth{135truemm} 
\input{slike/06_pulseML.pdf_tex}
\caption{Valovanja z različno lastno frekvenco, a enako fazo, se seštejejo v 
posamezne vrhove (a). 
Časovna odvisnost izhodne moči večfrekvenčnega laserja z uklenjenimi fazami $P$ je 
v obliki izrazitih sunkov (b).}
\label{s5.10}
\vglue-3truemm
\end{figure}

Ker je $N$ število vzbujenih lastnih nihanj, ki se med seboj po krožni 
frekvenci razlikujejo za $\Delta \omega$, je produkt $N\Delta \omega$ ravno enak
širini ojačenja $\Delta \omega_{G}$. Dolžina sunka je torej obratno sorazmerna s širino
ojačevanja aktivnega sredstva. Za dosego zelo kratkih sunkov z metodo uklepanja
faz je treba uporabiti laser s kar se da veliko širino ojačevanja. 

Poglejmo primer. V Ti:safir laserju z dolžino resonatorja $L=1,5~\si{m}$ \index{Laser!Ti:safir}je 
širina ojačenja enaka $\Delta \nu_G = 100~\si{THz}$. Iz takega laserja izhajajo
sunki,ki so  dolgi $\tau = 1/\Delta \nu_G = 10~\si{fs}$, med posameznima sunkoma 
pa poteče $T = 2L/c = 10~\si{ns}$. Število fazno uklenjenih lastnih načinov je 
$N = T/\tau = 10^6$.

Premislimo, kakšna je prostorska odvisnost električnega polja v
resonatorju. Polje na danem mestu opisuje enačba~(\ref{5.352}). Krajevno 
odvisnost dobimo, če v enačbi~(\ref{5.352}) parameter $t$ zamenjamo s $(t-z/c)$. To
predstavlja svetlobni paket, ki potuje sem in tja med zrcaloma
resonatorja. Na izhodnem zrcalu se vsakič del sunka odbije in del zapusti
resonator (slika~\ref{fig.5.11}). Razmik med sunki, ki izhajajo iz
resonatorja, je $2L$, prostorska dolžina posameznega sunka pa $\tau c=2L/N$.
\begin{figure}[ht]
\centering
\def\svgwidth{130truemm} 
\input{slike/06_MLvsota.pdf_tex}
\caption{Prostorska odvisnost fazno uklenjenih sunkov. Pri vsakem odboju del
sunka zapusti resonator. Razmik med zaporednima sunkoma je dvakratnik dolžine resonatorja.}
\label{fig.5.11}
\end{figure}
\vglue-5truemm
\begin{remark}
V našem računu predpostavka, da so vse amplitude $A_{m}$ enake, ni ključnega 
pomena. Če vzamemo, da so amplitude oblike 
$A_{m}=A_{0}\exp (-(m\Delta \omega /\Delta \omega_{G})^{2})$, 
kar je bolj realistično (slika~\ref{fig:FPmodes}), vsote (enačba~\ref{5.342}) 
ne znamo točno sešteti. Lahko jo
približno pretvorimo v integral, ki je Fouriereva transformiranka Gaussove
funkcije (pri prehodu z diskretne vsote na integral seveda izgubimo
periodičnost zaporedja sunkov). Ta je zopet Gaussova funkcija, katere
širina je obratna vrednost širine prvotne funkcije, podobno, kot
smo dobili zgoraj. Odvisnost amplitud vzbujenih lastnih nihanj od $m$ 
vpliva torej le na točno obliko sunkov.
\end{remark}

Pri uklepanju faz je bistvena predpostavka, da so vse faze $\varphi_m$ enake. 
V večfrekvenčnih laserjih so resonatorska stanja na splošno med seboj
neodvisna, zato so njihove faze poljubne in se zaradi motenj lahko še spreminjajo.
Za dosego istih faz posameznih nihanj moramo poskrbeti posebej. Tako ujemanje
oziroma uklepanje faz lahko dosežemo na več načinov. V grobem ločimo dva načina:
aktivno in pasivno uklepanje faz.

Pri aktivnem uklepanju faz moduliramo izgube v resonatorju
s frekvenco,\index{Uklepanje faz!aktivno} ki je ravno enaka razliki 
frekvenc med lastnimi resonatorskimi nihanji (slika~\ref{fig:aktpas}\,a). 
To dosežemo tako, da v resonator dodamo
modulator, na primer akusto-optičnega (glej poglavje~\ref{chap:modulacija}).
Ko v akusto-optičnem modulatorju vzbudimo stoječe zvočno valovanje, se 
svetloba na njem uklanja in izgube so velike. Stoječe valovanje periodično izginja
in takrat se uklonske izgube zmanjšajo. Če se to dogaja v časovnih\index{Akusto-optični pojav}
razmikih, ki so enaki $T=2L/c$, se v resonatorju ojačuje le kratek sunek svetlobe. 
Navadno zadošča razmeroma majhna sinusna modulacija izgub, pri kateri je relativna 
prepustnost v minimumu za nekaj desetink manjša od največje. Ta način se uporablja
v šibkejših sunkovnih laserjih, na primer v Nd:YAG laserjih.\index{Laser!Nd:YAG}
\begin{figure}[ht]
\centering
\def\svgwidth{140truemm} 
\input{slike/06_aktpas.pdf_tex}
\caption{Izgube v resonatorju in izhodna moč pri aktivnem (a) in pasivnem (b) uklepanju faz}
\label{fig:aktpas}
\end{figure}

Kako se pri modulaciji izgub faze uklenejo, lahko uvidimo še drugače.
Modulacija amplitude $m$-tega lastnega nihanja povzroči, da se v spektru nihanja
pojavita še stranska pasova pri krožnih frekvencah $(\omega_0 + m\Delta \omega) \pm \Delta\omega$. Ta
se ravno ujemata z obema sosednjima nihanjema in se konstruktivno
prištejeta, če imata enako fazo. Podobno tudi naprej za ostala nihanja. 
S tem se zmanjšajo izgube in delovanje laserja z uklenjenimi fazami ima najnižji prag. 

Pri pasivnem uklepanju faz dodanega elementa ne krmilimo od zunaj, 
ampak\index{Uklepanje faz!pasivno} je njegova prepustnost odvisna od intenzitete
izbranega nihajnega načina (slika~\ref{fig:aktpas}\,b). 
Ena vrsta pasivnih elementov je plast raztopljenega
barvila, ki močno absorbira svetlobo pri majhni gostoti svetlobnega toka, pri veliki 
gostoti pa pride do nasičenja absorpcije (glej\index{Nasičena absorpcija}
razdelek~\ref{chap:NasAbs}) in praktično vsa vpadna svetloba je 
prepuščena.\footnote{~E. P. Ippen, C. V. Shank in A. Dienes, Appl. Phys. Lett. $\mathbf{21}$, 348 (1972).}
V laserju je po vklopu prisotno predvsem 
spontano sevanje, ki se pri enem prehodu skozi aktivno snov deloma ojači. 
Barvilo najmanj absorbira fluktuacijo z največjo intenziteto. Pri dovolj 
velikem ojačenju se ta izbrana fluktuacija ojačuje, ostale pa ne, zato
se pojavi fazno uklenjeni sunek. Vzbujeni atomi se morajo čim hitreje
vrniti v osnovno stanje, da se svetloba lahko spet absorbira. To pomeni, da
 mora biti relaksacijski čas barvila  krajši od časa
obhoda $T$, tipično je $\sim \si{ps}$. 

Drugi način pasivnega uklepanja faz je z uporabo nelinearne optike 
(glej \index{Kerrov pojav!optični}
razdelek~\ref{OKP}). Z optičnim Kerrovim pojavom se snop svetlobe, ki vpade na 
optično nelinearno sredstvo, zoža, pri čemer je njegova širina odvisna od 
intenzitete svetlobe. Z dodatkom zaslonke poskrbimo, 
da je prepuščen le zelo močen svetlobni sunek, šibkejši, ki imajo večji polmer,
pa so zadušeni. Ta način pogosto
uporabimo v Ti:safir laserjih,\index{Laser!Ti:safir} s čimer dosežemo zelo
kratke izhodne sunke.

Z uklepanjem faz je mogoče dobiti sunke, krajše od $100~\si{fs}$. 
Tak sunek traja le še nekaj deset optičnih period. S posebnimi prijemi 
jih lahko še skrajšamo na okoli $10~\si{fs}$ (glej razdelek~\ref{kompdisp}). 
Zelo kratki svetlobni sunki so uporabni za študij hitre molekularne dinamike 
in kratkoživih vzbujenih elektronskih stanj v trdnih snoveh. Z uporabo fazno
uklenjenih sunkov svetlobe se je časovna ločljivost merilnih metod povečala za 
več redov velikosti.

\section{*Frekvenčni glavnik in absolutno merjenje frekvence laserja}
\index{Frekvenčni glavnik}\index{Uklepanje faz}\index{Sunkovni laser}
Iz fazno uklenjenega laserja izhajajo zelo kratki sunki svetlobe, ki 
so sestavljeni iz velikega števila vzbujenih lastnih nihanj. Spekter 
izhodne svetlobe tako vsebuje veliko število spektralnih črt, ki\index{Spekter}
so med seboj enakomerno razmaknjene (slika~\ref{fig:comb}). Zaradi podobnosti spektra
z glavnikom imenujemo tak izvor svetlobe frekvenčni glavnik.\footnote{~Za odkritje 
sta John L. Hall in Theodor W. H\"ansch leta 2005 prejela 
Nobelovo nagrado.} Značilna širina ojačenega
območja je $300~\si{THz}$, razmik med posameznimi črtami pa $100~\si{MHz}$, tako 
da tipični glavnik vsebuje okoli več milijonov spektralnih črt.\footnote{~
J. L. Hall, Rev. Mod. Phys. $\mathbf{78}$, 1279 (2006) in 
T. W. H\"ansch, Rev. Mod. Phys. $\mathbf{78}$, 1297 (2006).}
\begin{figure}[ht]
\centering
\def\svgwidth{110truemm} 
\input{slike/06_comb.pdf_tex}
\caption{Spekter svetlobe iz frekvenčnega glavnika}
\label{fig:comb}
\end{figure}

Frekvence izhodne svetlobe na splošno zapišemo kot 
\begin{equation}
\nu = \nu_0 + m\Delta \nu,
\label{eq:comb}
\end{equation}
pri čemer so $m$ naravno število, $\Delta \nu = c/2L$ razlika med dvema zaporednima 
frekvencama in $\nu_0$ frekvenčni zamik. S stabilizacijo $\Delta \nu$ in 
frekvenčnega zamika $\nu_0$ so frekvence izhodne svetlobe  
natančno določene, zato frekvenčni glavnik uporabljamo kot referenco 
za izredno natančno določanje frekvenc. Ko svetloba z neznano frekvenco
interferira s svetlobo iz frekvenčnega glavnika, nastopi utripanje -- pojavi
se signal pri razliki obeh frekvenc. Zaradi velikega števila lastnih nihanj 
in posledično majhnih razlik v frekvencah je frekvenca utripanja navadno v 
radijskem območju. To pa znamo zelo natančno izmeriti in tako določiti neznano 
vpadno frekvenco. 

V primeru, da so sunki svetlobe iz laserja povsem periodični in se ujemajo tako
v amplitudi kot v fazi, so frekvence izhodne svetlobe kar 
večkratniki $\Delta \nu$. V praksi se zaradi Gouyeve faze,\index{Gouyeva faza}
disperzije \index{Disperzija}in nelinearnosti pojavi zamik
v fazi $\Delta \varphi$ in vrh ovojnice sunka na splošno ne sovpada z vrhom amplitude
nihanja (slika~\ref{fig:comb2}). V izrazu za
frekvenco izhodne svetlobe se zato pojavi dodatni zamik $\nu_0 \neq 0$ (enačba~\ref{eq:comb}). 
Za absolutno
določitev frekvence moramo ta zamik seveda natančno poznati. 
\begin{figure}[ht]
\centering
\def\svgwidth{110truemm} 
\input{slike/06_comb2.pdf_tex}
\caption{Časovna odvisnost amplitude izhodne svetlobe iz frekvenčnega glavnika. V splošnem
vrh ovojnice sunka ne sovpada z vrhom amplitude nihanja, vendar zamik $\nu_0$ znamo izmeriti.}
\label{fig:comb2}
\end{figure}

Zamik $\nu_0$ lahko izmerimo
z interferometrom, v katerem med seboj primerjamo valovanje pri 
osnovni in pri podvojeni frekvenci.\footnote{~Tak interferometer imenujemo $f$--$2f$
interferometer, kar nakazuje osnovno ($f$) in podvojeno ($2f$) frekvenco.}
Valovanje pri osnovni frekvenci najprej frekvenčno podvojimo. To dosežemo 
z nelinearnimi optičnimi pojavi, ki omogočajo generacijo valovanja pri frekvenci, 
ki je enaka dvakratni frekvenci vpadnega valovanja 
(glej razdelek~\ref{chap:SHG}). Podvojeno frekvenco $2(\nu_0 + 
m\Delta \nu)= 2\nu_0 + 2m\Delta \nu$ nato primerjamo s frekvenco pri 
točno dvakrat večjem $m$, to je pri $\nu_0 + 2m\Delta \nu$. Pri razliki med tema
dvema frekvencama, ki je ravno $\nu_0$, se na detektorju pojavi utripanje.
Zamik $\nu_0$ na ta način izmerimo, hkrati ga s povratno zanko
vzdržujemo konstantnega. Ko določimo oba parametra $\nu_0$ in $\Delta \nu$, 
poznamo absolutne vrednosti frekvenc posameznih spektralnih črt v frekvenčnem 
glavniku z relativno natančnostjo okoli $10^{-18}$.

\subsection*{Definicija metra}
Danes je meter definiran kot pot, ki jo svetloba 
v vakuumu prepotuje v \index{Definicija metra}
$1/299\,\,792\,\,458~\si{s}$.  Vendar ni bilo vedno tako. Do šestdesetih let dvajsetega
stoletja je bil meter definiran z dolžino prametra, to je palice iz platine in iridija, 
pri atmosferskem tlaku in temperaturi taljenja ledu. 

Leta 1960 so definicijo 
izboljšali in jo vpeljali glede na svetlobo kriptonove svetilke. Meter je bil 
definiran kot $1\,\,650\,\,763,73$ valovnih dolžin svetlobe, ki jo izseva kriptonov
izotop $^{86}$Kr med nivojema $2p_{10}$ in $5p_5$. Vendar je širina črte
kriptonove svetilke razmeroma velika, zato je bil meter 
definiran le z relativno negotovostjo $4 \times 10^{-9}$ oziroma absolutno 
negotovostjo $4~\si{nm}$. 

Z odkritjem laserjev so hitro spoznali, da definicija s kriptonovo svetlobo ni 
najprimernejša, saj se je izkazalo, da je njen spekter razmeroma širok in asimetričen. 
S heterodinsko tehniko \index{Heterodinska detekcija}so primerjali valovno dolžino 
svetlobe \index{Laser!He-Ne}
iz He-Ne laserja, stabiliziranega z metanom ($3,39~\si{\micro\meter}$), z osnovno
cezijevo uro in določili absorpcijsko črto z natančnostjo cezijeve ure.
Frekvenca izhodne svetlobe je bila določena na $88,376181627(50)~\si{THz}$.
Po drugi strani so lahko izmerili valovno dolžino izsevane svetlobe s primerjavo 
z valovno dolžino kriptonove svetlobe in tako določili hitrost svetlobe na
$299\,\,792\,\,458~\si{m/s}$.

Leta 1983 so definicijo metra vnovič spremenili in jo prek hitrosti svetlobe v
vakuumu vezali na enoto sekunde. Ta definicija velja še danes. Sekunda je določena
z nihaji cezijevih atomov in to omogoča določitev metra z relativno 
negotovostjo $10^{-13}$. 

Z novo definicijo izbira snovi, ki izseva svetlobo, ni več ključnega
pomena. Priporočena realizacija definicije metra je s He-Ne laserjem, 
stabiliziranim z jodovo komoro, ki z absorpcijo še zmanjša spektralno širino črte.
V tem primeru je meter $1\,\,579\,\,800,762042(33)$ valovnih dolžin izsevane svetlobe. 
Laser je tako postal sekundarni standard za dolžino -- vendar je laser pri tem le pomožna 
naprava, standard je ustrezen molekularni prehod. 

\section{*Semiklasični model laserja}
\label{chap:semiklasicni}
\index{Semiklasični model}
Doslej smo laserje obravnavali le z modelom zasedbenih enačb. Ta je zelo
grob, saj smo pri tem zanemarili nekaj pomembnih pojavov. Svetlobo v 
resonatorju smo opisali s celotno energijo oziroma številom fotonov in se za
njeno valovno naravo nismo menili. Privzeli smo, da sta frekvenca
delujočega laserja in oblika polja lastnega nihanja v njem enaki kot v
praznem resonatorju. Aktivno snov smo opisali z zasedenostjo zgornjega
in spodnjega atomskega nivoja in s tem izpustili možnost, da se zaradi
interakcij z elektromagnetnim poljem atomi nahajajo v nestacionarnem,
mešanem stanju.\index{Zasedbene enačbe}

Navedene pomanjkljivosti delno odpravimo s tem, da elektromagnetno polje v
resonatorju obravnavamo klasično z valovno enačbo, atome aktivne snovi pa
kvantno, in upoštevamo, da se pokoravajo Schr\"odingerjevi enačbi. S tem dobimo 
semiklasični model laserja. Za še natančnejši opis bi morali obravnavati
kvantno tudi svetlobo.

Aktivna snov naj bo množica enakih dvonivojskih 
atomov s stanjema $|1\rangle$ in $|2\rangle$, ki imata energiji $E_1$ in $E_2$.
Atomi s svetlobo interagirajo z dipolno interakcijo oblike $H = -eE(t)\hat{x}$, 
pri čemer je $E(t)$ polje v resonatorju, ki naj bo zaradi preprostosti 
polarizirano v smeri osi $x$. Časovno odvisno stanje atomov
zapišemo v obliki 
\begin{equation}  \label{5.45}
|\psi\rangle=c_1(t)|1\rangle\exp(-iE_1t/\hslash)+
c_2(t)|2\rangle\exp(-iE_2t/\hslash).
\end{equation}
Iz Schr\"odingerjeve enačbe (enačba~\ref{eq:sk-S}) dobimo za časovna odvoda
koeficientov $c_1(t)$ in $c_2(t)$ zvezi
(glej enačbi~\ref{eq:c1c2})
\begin{equation}
\frac{d c_1}{dt}=-\frac{i}{\hslash} \mathcal{V} E(t) e^{-i\omega_0 t}\, c_2 
\qquad \mathrm{in} \qquad
\frac{d c_2}{dt}=-\frac{i}{\hslash} \mathcal{V} E(t) e^{i\omega_0 t}\, c_1,
\label{5.46}
\end{equation}
pri čemer je $\mathcal{V} = -e\langle1|\hat{x}|2\rangle$ realen in $\omega_0=(E_2-E_1)/\hslash$.

Električni dipolni moment atoma v stanju ${\psi}$ je 
\begin{equation}  
\label{5.47}
p=e\langle\psi|\hat{x}|\psi\rangle=-
(c_1^{\ast}c_2e^{-i \omega_0t}+c_1c_2^{\ast}e^{i \omega_0 t}) \mathcal{V}.
\end{equation}
Razdelimo $p$ na dva dela in zapišemo
\begin{equation}  
\label{5.48}
p=p^+ + p^-=-\mathcal{V}\left(\eta(t)+\eta^{\ast}(t)\right)\!,
\end{equation}
pri čemer smo vpeljali $\eta(t)=c_1^{\ast}c_2e^{-i \omega_0 t}$.

Zanima nas, kako se dipolni moment spreminja s časom. Zadošča, da vemo, kako se 
s časom spreminja parameter $\eta(t)$. Njegov časovni odvod izrazimo
iz enačb (\ref{5.46}) in dobimo
\begin{equation}  
\label{5.49}
\frac{d\eta}{dt}=- i \omega_0\eta+\frac{i}{\hslash}\,\mathcal{V}E(t) \left(|c_2|^2-|c_1|^2\right)\!.
\end{equation}
Spomnimo, da je $|c_i|^2$ verjetnost za zasedenost stanja $|i\rangle$. Izraz v oklepaju
na desni strani enačbe~(\ref{5.49}) torej meri razliko zasedenosti obeh stanj, ki jo označimo
s $\zeta$. Podobno kot zgoraj izrazimo spreminjanje razlike zasedenosti s časovnim odvodom 
\begin{equation}  
\label{5.50}
\frac{d\zeta}{dt}=\frac{2i}{\hslash}\, \mathcal{V} E(t)\left(\eta- \eta^{\ast}\right)\!.
\end{equation}
S tem smo iz Schr\"odingerjeve enačbe dobili enačbi za časovni razvoj
dipolnega momenta in obrnjene zasedenosti, vendar ju moramo še dopolniti.

Naj bo atom na začetku v vzbujenem stanju $|2\rangle$ in naj bo $E(t)=0$. Začetna
vrednost obrnjene zasedenosti je tako $\zeta(0)=1$. Po enačbi (\ref{5.50}) naj bi 
bil odvod obrnjene zasedenosti enak nič in $\zeta$ konstantna. 
Vendar vemo, da se atom, ki je v vzbujenem stanju, sčasoma vrne v
osnovno stanje. Verjetnost za tak spontan prehod na časovno enoto smo označili z $A$ (glej 
razdelek~\ref{chap:ASSS}).

Poleg tega moramo upoštevati še črpanje, s katerim\index{Črpanje}
vzdržujemo obrnjeno zasedenost in s tem lasersko delovanje. Za podroben
opis črpanja bi morali v Hamiltonov operator dodati ustrezne člene in
morda upoštevati še druga stanja atomov, vendar nas take podrobnosti 
ne zanimajo. Če ne bi bilo črpanja, bi bila stacionarna vrednost
v odsotnosti laserskega polja
\begin{equation}
 \zeta_{\mathrm{stac}}= |c_2|^2-|c_1|^2 = -1.
\end{equation}
Zaradi črpanja zavzame obrnjena zasedenost neko vrednost $-1<\zeta_0<1$, 
odvisno od moči črpanja. Enačbi (\ref{5.50}) dodamo ustrezen člen
\begin{equation}  
\label{5.51}
\frac{d\zeta}{dt}=A\left(\zeta_0-\zeta\right)+\frac{2i}{\hslash}\,\mathcal{V}E(t)\left(\eta-\eta^{\ast}\right)\!,
\end{equation}
ki opisuje vpliv črpanja in spontane prehode v nižje stanje. 

Dopolnimo še enačbo za časovno spreminjanje električnega dipola 
(enačba~\ref{5.49}). Pri $E(t)=0$ da zapisana enačba časovno odvisnost 
$\eta \propto e^{-i \omega_0 t}$, to je brez dušenja. Vendar vemo, da električna
polarizacija v mešanem stanju razpada vsaj zaradi spontanega sevanja, lahko
pa še zaradi drugih vplivov, na primer trkov. Označimo
koeficient dušenja polarizacije snovi, ki meri tudi spektralno širino svetlobe, 
izsevane pri prehodu $2\rightarrow 1$, z $\gamma$. Zapišemo
dopolnjeno enačbo
\begin{equation}  
\label{5.52}
\frac{d\eta}{dt}=- \left(i \omega_0+\gamma\right)\eta+\frac{i}{\hslash}\,\mathcal{V} E(t) \zeta
\end{equation}
in še kompleksno konjugirano enačbo
\begin{equation}  
\label{5.52c}
\frac{d\eta^*}{dt}=-\left(-i \omega_0+\gamma\right)\eta^*-\frac{i}{\hslash}\,\mathcal{V}E(t) \zeta.
\end{equation}
Dobili smo sistem sklopljenih diferencialnih enačb, ki opisuje časovno
spreminjanje obrnjene zasedenosti in dipolnega momenta atoma. 
\vglue-2truemm
\begin{remark}
 Enačbe~(\ref{5.51}), (\ref{5.52}) in (\ref{5.52c}) pogosto imenujemo \index{Maxwell-Blochove enačbe}
 Maxwell-Blochove ali optične Blochove enačbe\footnote{~Švicarsko-ameriški fizik 
 in nobelovec Felix Bloch, 1905--1983.}. Osnovne Blochove enačbe opisujejo gibanje 
 jedrskega magnetnega  momenta v elektromagnetnem polju, zato so jih najprej uporabili za opis 
 jedrske magnetne in elektronske spinske resonance. 
\end{remark}
\vglue-2truemm
Za opis potrebujemo še enačbo za polje $E(t)$. To naredimo klasično, tako da
jakost električnega polja zadošča valovni enačbi. Upoštevati moramo, 
da je v snovi tudi električna polarizacija različna od nič. 
Valovna enačba v skalarni obliki je tedaj\index{Valovna enačba}
\begin{equation}  
\label{5.54}
\nabla^2 E-\frac{1}{c^2}\frac{\partial^2 E}{\partial t^2}=\mu_0 \frac{\partial^2 P}{\partial t^2}. 
\end{equation}
V primeru, da so vsi atomi enakovredni, je električna polarizacija enaka
\begin{equation}  
\label{5.53}
P=\frac{N}{V}p = -\frac{N}{V}\,\mathcal{V}\left(\eta+\eta^{\ast}\right)=P^+ + P^-,
\end{equation}
pri čemer smo z $V$ označili volumen in z $N$ število atomov.

Namesto mikroskopske količine $\zeta$ uvedemo gostoto obrnjene
zasedenosti $Z=(N/V)\zeta$ in enačbe (\ref{5.51}), (\ref{5.52}) in (\ref{5.52c})
prepišemo v obliko 
\begin{align}
\frac{dZ}{dt} &= A\left(Z_0-Z\right)+\frac{2i}{\hslash}E\left(P^- - P^+\right)\!\!, \label{5.56} \\
\frac{dP^+}{dt}&=\left(-i \omega_0-\gamma\right)P^{+}-\frac{i}{\hslash} \mathcal{V}^2 E  Z \qquad 
\mathrm{in} \label{5.56b}\\
\frac{dP^-}{dt}&=\left(i \omega_0-\gamma\right)P^{-}+\frac{i}{\hslash} \mathcal{V}^2 E  Z.\label{5.56c} 
\end{align}
Opozorimo še, da je prehod z enačb (\ref{5.51}--\ref{5.52c}) na (\ref{5.56}--\ref{5.56c}) 
mogoč le, kadar so vsi atomi enakovredni, to je, kadar ni nehomogene razširitve 
spektra.\footnote{~Primer nehomogene razširitve je opisan v npr.  
H. Haken, {\it Laser Theory}, Springer (1984).}

Enačbe (\ref{5.56}--\ref{5.56c}), skupaj z valovno enačbo (enačba~\ref{5.54}) podajajo
semiklasični opis interakcije svetlobe s snovjo. Iz izpeljave je razvidno, da
je v semiklasičnem opisu spontano sevanje obravnavano pomanjkljivo, le s fenomenološkim
nastavkom. To je moč popraviti s kvantizacijo elektromagnetno polja. 
Kljub tej pomanjkljivosti je s semiklasičnim modelom mogoče zelo podrobno opisati 
večino pojavov v laserjih in tudi druge pojave širjenja svetlobe po snovi. Je pa 
reševanje zapisanega sistema nelinearnih parcialnih diferencialnih enačb  
na splošno zelo težavno.

\subsection*{Primer enofrekvenčnega laserja}\index{Laser!enofrekvenčni}
Semiklasične enačbe pobliže spoznajmo na najenostavnejšem primeru. To naj bo laser, 
v katerem je vzbujeno le eno resonatorsko nihanje, označimo ga z $n$.
Polje ima tedaj obliko
\begin{equation}  
\label{5.57}
E(\mathbf{r},t)= E_n(t)u_n(\mathbf{r}),
\end{equation}
pri čemer je $u_n(\mathbf{r})$ krajevni del lastnega nihanja resonatorja, ki
zadošča enačbi 
\begin{equation}  
\label{5.58}
\nabla^2 u_n+\frac{\omega_n^2}{c^2}u_n=0.
\end{equation}
Funkcija $E_n(t)$ opisuje časovno odvisnost. Za laser v stacionarnem
delovanju je periodična, vendar njena krožna frekvenca $\Omega$ ni nujno enaka lastni
krožni frekvenci praznega resonatorja $\omega_n$. Krožno frekvenco 
$\Omega$ moramo še izračunati.

Razvijmo še električno polarizacijo po lastnih funkcijah $u_n(\mathbf{r})$. 
Ker so lastne funkcije med seboj ortogonalne, preide valovna enačba (\ref{5.54}) 
ob upoštevanju enačbe~(\ref{5.58}) v 
\begin{equation}  
\label{5.59}
\omega_n^2 E_n+\frac{d^2 E_n}{dt^2}= 
-\frac{1}{\epsilon_0}\frac{d^2P_n}{dt^2}.
\end{equation}
Razstavimo $E_n(t)$ na dva dela: 
\begin{equation}  \label{5.60}
E_n(t)=E_n^+(t)+E_n^-(t)=A^+(t)e^{-i
\omega_nt}+A^-(t)e^{i \omega_nt}.
\end{equation}
Dejanska krožna frekvenca laserja je blizu $\omega_n$, zato pričakujemo,
da se bosta amplitudi $A^{\pm}(t)$ v primerjavi z $e^{-i \omega_nt}$ le
počasi spreminjali. Izračunajmo 
\begin{equation}  
\label{5.61}
\frac{d^2 E_n^+}{dt^2}=-\omega_n^2 E_n^+ - 2i \omega_n 
\frac{dA^+}{dt} e^{-i \omega_nt} \approx 
-\omega_n^2 E_n^+-2i \omega_n\left(\frac{dE_n^+}{dt}+
i \omega_nE_n^+\right)\!.
\end{equation}
Napravili smo približek počasi spreminjajoče se amplitude, tako da smo izpustili drugi odvod $A$ po času. 

Električna polarizacija snovi je približno periodična s krožno frekvenco $\omega_0 \approx \omega_n$.
Tudi amplituda polarizacije se le počasi spreminja, zato lahko privzamemo, da je 
drugi odvod $P_n^+$ po času približno enak $-\omega_0^2 P_n^+$. Z uporabo tega
približka in enačbe~(\ref{5.61}) preide valovna enačba (enačba~\ref{5.54}) za izbrano nihanje v 
\begin{equation}  
\label{5.62}
\frac{dE_n^+}{dt}=-i \omega_n E_n^++
\frac{i \omega_0}{2\epsilon_0}P_n^+.
\end{equation}

Upoštevajmo še, da je polje v praznem resonatorju dušeno, in ustrezno popravimo enačbo
\begin{equation}  
\label{5.63}
\frac{dE_n^+}{dt}=\left(-i \omega_n-\frac{1}{\tau}\right) E_n^+ 
+\frac{i \omega_0}{2\epsilon_0}P_n^+.
\end{equation}
Kadar v resonatorju ni snovi, je dobljena enačba enaka enačbi (\ref{3.36}).

Enačbe (\ref{5.56}--\ref{5.56c}) so nelinearne, zato jih ni moč kar tako
prepisati za razvoj po lastnih nihanjih resonatorja. Pri enačbah za
razvoj polarizacije (enačbi~\ref{5.56b} in \ref{5.56c}) v zadnjem členu na desni 
nastopa produkt komponente polja $E_n$ in obrnjene zasedenosti $Z$. Bistveni
prispevek je od krajevnega povprečja $\overline{Z}$, ki se s časom le počasi spreminja.
Seveda vsebuje $Z$ tudi krajevno odvisne komponente, ki so pomembne 
predvsem zato, ker sklapljajo različna lastna nihanja resonatorja, vendar to
presega našo obravnavo. Iz enačbe~(\ref{5.56b}) dobimo
\begin{equation}  
\label{5.64}
\frac{d P_n^+}{dt}=\left(-i \omega_0-\gamma\right)P_n^{+}-\frac{i}{\hslash}
\mathcal{V}^2 E_n^+ \overline{Z}.
\end{equation}
Povprečje $\overline{Z}$ izrazimo iz enačbe (\ref{5.56}). V zadnjem členu nastopajo produkti 
$E^{\pm}P^{\pm}=E_n^{\pm}P_n^{\pm} u_n^2\left(\mathbf{r}\right)$,
ki jih moramo prostorsko povprečiti. Funkcije $u_n\left(\mathbf{r}\right)$ so
normirane, tako da je $\int u_n^2\left(\mathbf{r}\right) \,dV=V$ in $\overline{u_n^2\left(\mathbf{r}\right)}=1$.
Sledi
\begin{equation}  
\label{5.65}
\frac{d\overline{Z}}{dt}= A\left(\overline{Z}_0-\overline{Z}\right)+ \frac{2i}{\hslash}\left(E_n^+
+E_n^-\right)\left(P_n^- - P_n^+\right)\!,
\end{equation}
pri čemer je $\overline{Z}_0$ povprečje nenasičene zasedenosti $Z_0$. V zadnjem
členu nastopajo produkti, ki nihajo s krožnimi frekvencami $\omega_n-
\omega_0$ in $\omega_n+ \omega_0$. Frekvenci sta si zelo blizu,
zato je njuna vsota znatno večja od razlike. Hitro spreminjajoče se člene 
$E_n^+P_n^+$ in $E_n^- P_n^-$ izpustimo, 
saj skoraj nič ne vplivajo na valovanje blizu $\omega_n$. Časovno odvisnost 
$\overline{Z}$ potem zapišemo 
\begin{equation}  
\label{5.66}
\frac{d\overline{Z}}{dt}= A\left(\overline{Z}_0-\overline{Z}\right)+\frac{2i}{\hslash}\left(E_n^+
P_n^- - E_n^- P_n^+\right)\!.
\end{equation}
Enačbe (\ref{5.63}), (\ref{5.64}) in (\ref{5.66}) predstavljajo skupaj s konjugirano
kompleksnimi enačbami za $E_n^-$ in $P_n^-$ zaključen
sistem, ki opisuje delovanje enofrekvenčnega laserja. Uporabimo jih za
izračun frekvence izhodne svetlobe.

Naj bo stanje stacionarno. Tedaj polje zapišemo v obliki $E_{n
}^{+}=E_{0}e^{-i\Omega t}$, pri čemer je $E_{0}$ realna konstanta, krožna frekvenca
svetlobe $\Omega$ pa je blizu $\omega _{0}$ in $\omega _{n }$. V
stacionarnem stanju ima električna polarizacija enako časovno odvisnost in 
$P_{n }^{+}=P_{0}e^{-i\Omega t}$. Tedaj je v enačbi~(\ref{5.66}) drugi
oklepaj konstanten in povprečna gostota obrnjene zasedenosti $\overline{Z}$ 
od časa neodvisna. 

Sistem enačb (\ref{5.63}), (\ref{5.64}) in (\ref{5.66}) 
tako da 
\begin{align}
\left(i\left(\Omega - \omega_n\right)-\frac{1}{\tau}\right) E_{0}+\frac{i\omega _{0}}
{2\epsilon _{0}}\,P_{0} &=0,  \label{5.67} \\
\left(i\left(\Omega-\omega_{0}\right)-\gamma\right)P_{0}-\frac{i}{\hslash}\mathcal{V}^{2}
E_{0}\overline{Z} &=0 \qquad \mathrm{in}\label{5.67b}\\
A\left(\overline{Z}_{0}-\overline{Z}\right)+\frac{2i}{\hslash }\,E_{0}\left(P_{0}^{*}-P_{0}\right) &=0.
\end{align}

Najprej iz druge enačbe izrazimo $P_0$, ga vstavimo v tretjo in izračunamo $\overline{Z}$. Dobimo
\begin{equation}  
\label{5.68}
\overline{Z}=\overline{Z}_0\left(1+\frac{4\mathcal{V}^2}{\hslash^2 A}\,E_0^2\, \frac{\gamma}
{\left(\omega_0-\Omega\right)^2+\gamma^2}\right)^{-1}\!.
\end{equation}
Ta izraz že poznamo. V njem prepoznamo Einsteinov
koeficient $B$ (enačba~\ref{4.60}),
$E_0^2$ je sorazmeren gostoti energije polja v resonatorju, medtem ko 
zadnji ulomek v oklepaju podaja obliko homogeno razširjene atomske
črte (\ref{eq:homogenasirina}). Izraz prepišemo v
\begin{equation}  
\label{5.69}
\overline{Z}=\overline{Z}_0\left(1+\frac{2B}{Ac}\,g\left(\omega_0- \Omega\right)j\right)^{-1}\!.
\end{equation}
To je enako izrazu za nasičenje zasedenosti stanj, ki smo ga
izpeljali iz zasedbenih enačb v četrtem poglavju (enačba~\ref{4.33}).

Vstavimo $P_0$ iz prve enačbe sistema (\ref{5.67}) v drugo (\ref{5.67b})
\begin{equation}  
\label{5.70}
E_0\left(-i\Omega+i\omega_n+\frac{1}{\tau}\right) \left(i\Omega- i\omega_0
-\gamma\right)=-\frac{\mathcal{V}^2 \omega_0}{2\hslash\epsilon_0}\,E_0\,\overline{Z}.
\end{equation}
V delujočem laserju je $E_0\ne 0$, zato ga lahko krajšamo. Vrednost $\overline{Z}$ je
realna, tako da mora biti imaginarni del leve strani enak nič. Dobimo
\begin{equation}  
\label{5.71}
\left(\Omega- \omega_n\right)\gamma+\left(\Omega- \omega_0\right)\frac{1}{\tau} = 0.
\end{equation}
Od tod lahko izračunamo krožno frekvenco delujočega laserja 
\begin{equation}  \label{5.72}
\Omega=\frac{\omega_n\gamma+ \omega_0\frac{1}{\tau}}{\gamma + \frac{1}{\tau}}.
\end{equation}
Krožna frekvenca torej ni enaka krožni frekvenci praznega resonatorja $\omega_n$,
temveč je premaknjena proti centru atomske črte $\omega_0$. Premik je
odvisen od razmerja širine atomske črte in izgub v resonatorju.

Opisani primer uporabe semiklasičnih enačb je zelo preprost. Prava moč
modela se pokaže pri obravnavi večfrekvenčnega laserja, na primer pri
računu uklepanja faz laserskih nihanj, kar presega okvir te 
knjige.\footnote{~Glej npr. Y. Khanin, {\it Fundamentals of Laser Dynamics}, Cambridge International Science
Publishing (2006).} 

%v delu

%-------------------------------------------------------------------------------
%	CHAPTER 7
%-------------------------------------------------------------------------------


\chapterimage{slike/Primeri.jpg} 
\chapter{Primeri laserjev}
\label{chap:Primeri}
V tem poglavju bomo spoznali nekaj najpomembnejših vrst laserjev.
V grobem laserje razlikujemo po aktivnem sredstvu, ki je lahko plin, trdna snov, 
organsko barvilo ali polprevodnik. Tudi pri izbranem
sredstvu obstaja veliko različnih izvedb in načinov delovanja laserja. Za vsak 
obravnavani primer bomo navedli osnovne karakteristike, v podrobnosti 
izvedbe pa ne bomo spuščali.\footnote{~Za podrobnejši opis glej npr. W. T. Silfvast,
{\it Laser Fundamentals}, druga izdaja, Cambridge University Press (2004) ali
O. Svelto in D. C. Hanna, {\it Principles of Lasers}, peta izdaja, Springer (2010).}

\section{Laserski sistemi}
\index{Laserski sistemi}
Laser \index{Laser} je lahko dokaj preprosta naprava z malo sestavnimi deli,
lahko pa je zelo velik in zapleten sistem. Večina laserskih sistemov
je sestavljena iz osnovnega laserja, ki ni posebno močan, a daje kakovosten
snop svetlobe, in iz enega ali več ojačevalnikov. V njih se svetloba 
ojačuje v sredstvu, ki je enako kot v osnovnem laserju in ki je v 
visokem stanju obrnjene zasedenosti. V več ojačevalnih korakih 
se tako doseže zelo velika svetlobna moč. 

Pri velikih laserskih močeh nastopi vrsta novih težav. Da gostota 
svetlobnega toka ne poškoduje optičnih komponent, mora biti
premer ojačevanega snopa (in vseh vmesnih ojačevalnih stopenj) razmeroma velik. 
Na zadnjih stopnjah največjih laserskih sistemov je 
premer snopa večji od pol metra, kar seveda pomeni, da morajo imeti tolikšno odprtino 
vse optične komponente v sistemu. Poleg tega je
treba skrbno paziti, da se odbita svetloba ne vrača v prejšnji
ojačevalnik ali v osnovni laser in s tem moti delovanje. Med
posamezne ojačevalne stopnje zato damo optične izolatorje, ki temeljijo na Faradayevem
pojavu vrtenja polarizacije v snovi z magnetnim poljem.
\begin{figure}[ht]
\centering
\includegraphics[width=90truemm]{slike/07_NIF_Laser_Bay.jpg}
\caption{Eden izmed najmočnejših laserskih sistemov na svetu, ki doseže 
$500~\si{\tera\watt}$ moči v sunku. Vir: National Ignition Facility, Livermore, Kalifornija.}
\label{fig:NIF}
\end{figure}

Laserski sistemi lahko oddajajo svetlobo z zelo veliko izhodno močjo. 
Najmočnejši zvezno delujoči laserji dosegajo moči 
$\sim 100~\si{\kilo\watt}$. Še bistveno večje moči dosegajo sunkovni laserji, 
saj lahko v sunku dosežejo moč tudi $\sim 500~\si{\tera\watt}$ (slika~\ref{fig:NIF}). 
Vendar so sunki s tako veliko svetlobno močjo izredno kratki, tipično reda pikosekunde, tako da
znaša celotna energija v sunku do nekaj $\si{\mega\joule}$. Pomemben
parameter pri sunkovnih laserjih je tudi čas, ki poteče med dvema zaporednima
sunkoma (repeticija). Najmočnejši laserski sistemi lahko izsevajo največ nekaj sunkov 
dnevno.\footnote{~S. N. Dixit et al. CLEO: 2013, San Jose, CA (2013).}

\section{Helij-neonski laser}
\index{Laser!He-Ne}\index{Energijski nivoji!He-Ne}
\index{Trinivojski sistem}
Najprej si oglejmo helij-neonski (He-Ne) laser, ki je bil prvi zvezno 
delujoči laser in je še danes zelo razširjen.\footnote{~A. Javan, W. R. Bennet Jr. in 
D. R. Herriott, Phys. Rev. Lett. $\mathbf{6}$, 106 (1961).} Najpogosteje deluje 
pri valovni dolžini $632,8~\si{\nano\metre}$ v rdečem delu spektra, lahko 
pa tudi pri infrardečih $1,15~\si{\micro\metre}$ in 
$3,39~\si{\micro\metre}$ ter nekaterih drugih\index{Infrardeče valovanje}
valovnih dolžinah v oranžnem in zelenem delu spektra. Laser deluje v zveznem 
načinu delovanja s tipičnimi močmi $0,5$--$100~\si{\milli\watt}$.

Ojačevalno sredstvo je plin, mešanica helija in neona, katerih relevantni
energijski nivoji so prikazani na sliki~\ref{fig:HeNeE}. 
Atome helija
s trki z elektroni vzbudimo v eno izmed dveh dolgoživih metastabilnih stanj $2^3$S ali
$2^1$S z razpadnima časoma $0,1~\si{\milli\second}$ in $5~\si{\micro\second}$.
Ti dve stanji slučajno praktično sovpadata z dvema stanjema neona ($4$s in $5$s).
Ko heliju dodamo neon, se energija s trki 
prenese z vzbujenih helijevih atomov na atome neona, ki s tem preidejo v 
že omenjeni vzbujeni stanji. Helijevi atomi se po trku vrnejo v osnovno stanje, od koder
jih ponovno vzbudimo. Prenos energije z atomov helija na atome neona s trki je 
zelo učinkovit, zato zasedenost vzbujenih neonovih stanj hitro naraste. Ko preseže 
zasedenost nižjih vzbujenih stanj, dosežemo obrnjeno zasedenost. 

\begin{figure}[ht]
\centering
\def\svgwidth{100truemm} 
\input{slike/07_HeNeE.pdf_tex}
\caption{Shema energijskih nivojev v He-Ne laserju. Nivoji helija so označeni
z modro in nivoji neona z zeleno, laserski prehodi pa z rdečimi barvami in pripisano
ustrezno valovno dolžino.}
\label{fig:HeNeE}
\end{figure}

Znana rdeča svetloba He-Ne laserja z valovno dolžino $632,8~\si{\nano\metre}$ nastane 
pri prehodu iz stanja $5$s v eno od stanj $3$p. Pri tem je življenjski čas 
stanja $5$s okoli $100~\si{\nano\second}$, stanja $3$p pa okoli $10~\si{\nano\second}$, zato
se spodnji nivo s spontano emisijo hitro prazni v metastabilno stanje $3$s. 
V tem stanju se atomi kopičijo, saj so dipolni sevalni prehodi v osnovno stanje prepovedani.
Atomi prehajajo v osnovno stanje le s trki ob steno cevi. Da pospešimo
praznjenje nivoja $3$s in omogočimo večjo obrnjeno zasedenost, zmanjšamo 
premer razelektritvene cevi. Zaradi gibanja atomov je spektralna 
črta Dopplerjevo razširjena\index{Dopplerjeva razširitev} ($\Delta \nu = 1,5~\si{GHz}$). 

Laser deluje tudi pri prehodu iz $5$s v stanje $4$p, pri katerem 
ima izsevana svetloba valovno dolžino $3,39~\si{\micro\metre}$. 
Ojačenje je za ta prehod celo precej večje kot za
prehod pri $632,8~\si{\nano\metre}$, deloma zaradi nižje frekvence 
(glej zvezo med Einsteinovima koeficientoma $A$ in $B$, enačba~\ref{4.27}), 
deloma zaradi kratke življenjske dobe spodnjega laserskega nivoja $4$p. 
Zato bi pričakovali, da bo He-Ne laser svetil v infrardečem delu in ne vidnem. 
To delno prepreči absorpcija v steklu, delno pa izgube namerno povečamo s selektivno odbojnostjo
resonatorskih zrcal, ki dvigne prag delovanja za $3,39~\si{\micro\metre}$ 
nad prag za $632,8~\si{\nano\metre}$. V laser lahko dodamo tudi
komoro metana, ki infrardeči del svetlobe močno absorbira, vidnega pa ne.
Omenimo še prehode iz stanja $4$s, ki ga dosežejo neonovi atomi s trki
z vzbujenimi helijevimi atomi iz nivoja $2^3$S. Prehod $4$s v $3$p, ki da svetlobo
pri $1,15~\si{\micro\metre}$, je bil prvi opažen prehod v He-Ne laserjih.
Zaradi razcepov posameznih nivojev je možnih prehodov še veliko več.

Tipičen He-Ne laser je razmeroma preprosto zgrajen (sliki~\ref{fig:HeNeShema}
in \ref{fig:Iskra}).\index{Laser!zgradba}
V razelektritveni cevi (napetost  $\sim 1~\si{\kilo\volt}$), skozi
katero teče električni tok ($\sim 10~\si{\milli\ampere}$), 
se nahaja mešanica helija in neona v razmerju od
$5:1$ do $10:1$. Skupni tlak v cevi je nizek, le okoli $3~\si{\milli\bar}$, 
cev pa je tipično dolga okoli $0,5~\si{\metre}$ s premerom $1$--$2~\si{\milli\metre}$.  
Cev s plinom na obeh straneh zapirata okni, ki sta nagnjeni za Brewstrov kot, 
tako da so izgube pri odboju za eno polarizacijo kar se da majhne.\index{Brewstrovo okno}
Izhodna svetloba iz laserja je zato polarizirana. V manjših laserjih
so namesto Brewstrovih oken na razelektritveno cev privarjena kar
resonatorska zrcala, zaradi česar so taki laserji nepolarizirani. 
Navadno je razelektritvena cev obdana z dvema ukrivljenima zrcaloma, 
ki imata zelo veliko odbojnost za izbrano valovno dolžino.
Nekaj tipičnih podatkov za He-Ne laser je zbranih v tabeli~\ref{tab:Ar}.
\begin{figure}[ht]
\centering
\def\svgwidth{100truemm} 
\input{slike/07_HeNeShema.pdf_tex}
\caption{Shema He-Ne laserja: R -- razelektritvena cev, IZ -- izhodno zrcalo, Z -- zrcalo
z veliko odbojnostjo, B -- Brewstrovi okni}
\label{fig:HeNeShema}
\end{figure}

\begin{figure}[ht]
\centering
\includegraphics[width=100truemm]{slike/07_HeNe.jpg}
\caption{Primer starejšega He-Ne laserja, izdelanega v Sloveniji}
\label{fig:Iskra}
\end{figure}

He-Ne laserji so preprosti, stabilni, zanesljivi, poceni, imajo visoko kakovost
snopa in dolgo služijo ($\sim~50\,\,000$ ur).
Danes jih sicer izrivajo polprevodniški laserji, vendar so še vedno v uporabi
v merilnih napravah, v optičnih bralnih sistemih, v šolah, v raziskovalnih 
laboratorijih za interferometrijo, holografijo itd. Na njem je osnovan tudi 
standard za meter.

\section{Argonski laser}
\index{Laser!argonski}
Kot drugi primer plinskega laserja obravnavajmo argonski laser ali natančneje 
laser na argonove ione (Ar$^+$). Zanj je značilno zvezno 
delovanje v modrem in zelenem delu spektra pri 
valovnih dolžinah $488,0~\si{\nano\metre}$ in $514,5~\si{\nano\metre}$, deluje 
pa tudi v bližnjem ultravijoličnem delu spektra.\footnote{~W. B. Bridges,
Appl. Phys. Lett. $\mathbf{4}$, 128 (1964).} Tipične moči delovanja argonskega laserja
so od $100~\si{\milli\watt}$ do $50~\si{\watt}$.\index{Ultravijolično valovanje}

Kot večino plinskih laserjev tudi tega črpamo z električnim tokom.
Atome argona vzbudimo s trki z elektroni v ione argona, ti pa z nadaljnjimi
trki preidejo v vzbujena stanja. Obrnjeno zasedenost
dosežemo med nivojema $4$p in $4$s (slika~\ref{fig:ArE}). 
Ta dva nivoja vsebujeta veliko podnivojev, zato je tudi prehodov med
njima zelo veliko. Argonski laser tako seva pri več kot tridesetih različnih
valovnih dolžinah, najznačilnejši sta že omenjeni 488~nm in 514,5~nm. 
Življenjski čas zgornjega nivoja je okoli $10~\si{ns}$, kar je približno 
desetkrat več od življenjskega časa spodnjega nivoja, od koder se ioni
z rekombinacijo z elektroni vrnejo v osnovno stanje. Tudi pri tem laserju
je poglavitni vzrok za razširitev črte \index{Dopplerjeva razširitev}Dopplerjev 
pojav ($\Delta \nu = 3,5~\si{GHz}$).\index{Energijski nivoji!argon}

\begin{figure}[ht]
\centering
\def\svgwidth{80truemm} 
\input{slike/07_ArE.pdf_tex}
\caption{Shema energijskih nivojev v argonskem laserju}
\label{fig:ArE}
\end{figure}

Argonski laser je v osnovi zgrajen podobno kot He-Ne laser. \index{Laser!zgradba}
V razelektritveni cevi
(tipična dolžina $1~\si{\metre}$ in premer $1$--$2~\si{\milli\metre}$)
se nahaja argon pri pritisku okoli $10~\si{\milli\bar}$. 
Ker gre pri vzbujanju atomov argona za dvostopenjski proces, mora biti električni tok, 
s katerim dosežemo obrnjeno zasedenost, precej velik, lahko tudi nekaj deset amperov. 
Pri tipični napetosti nekaj kV to pomeni, 
da so za delovanje potrebne velike električne moči, 
pogosto več deset $\si{\kilo\watt}$. Močnejši argonski laserji morajo biti
zaradi velike količine odvečne toplote hlajeni, najpogosteje vodno.

\begin{figure}[ht]
\centering
\def\svgwidth{100truemm} 
\input{slike/07_ArShema.pdf_tex}
\caption{Poenostavljena shema argonskega laserja s prizmo: R -- razelektritvena cev, 
IZ -- izhodno zrcalo, Z -- zrcalo z veliko odbojnostjo, B -- Brewstrovi okni,
P -- prizma\index{Brewstrovo okno}
}
\label{fig:ArS}
\end{figure}

\begin{remark}
V argonskih laserjih pogosto ustvarimo vzdolžno magnetno polje, ki preprečuje 
elektronom, da bi predčasno zapustili ojačevalno območje in trčili ob steno. S
tem se poveča izhodno moč laserja, hkrati pa preprečuje poškodbe na stenah, ki bi jih 
povzročili visokoenergijski elektroni. Iz istega razloga so pri močnejših
laserjih zrcala izven plinske cevi. 
\end{remark}
\vglue-4truemm
V resonator argonskega laserja moramo vgraditi še element, ki omogoči
izbiro ene same spektralne črte. Najpogosteje za ta frekvenčno selektiven element
uporabimo kar majhno prizmo pred enim od obeh zrcal (slika~\ref{fig:ArS}). Zaradi disperzije
v prizmi se snopi različnih valovnih dolžin lomijo pod različnimi koti in le tisti 
snop, ki vpada pravokotno na zrcalo, se ojačuje. Tako z vrtenjem prizme ali zrcala 
izbiramo valovno dolžino izhodne svetlobe. Nekaj tipičnih podatkov za argonski
laser je zbranih v tabeli~\ref{tab:Ar}.

Argonski laserji so zanesljivi in dajejo zelo dober osnovni Gaussov snop pri eni
sami frekvenci. Uporabljajo se v optični spektroskopiji,
interferometriji, holografiji in merilni tehniki. Delujejo v zveznem načinu,
zaradi razmeroma široke črte ojačenja jih uporabljamo tudi za fazno uklenjene
sunkovne laserje z dolžino sunkov okoli $150~\si{\pico\second}$. 
V kombinaciji s kriptonskimi laserji, ki so zelo podobni argonskim, le da delujejo
v rdečem in oranžnem delu spektra, se uporabljajo tudi v zabavni industriji.
V zadnjem času jih vse bolj izrivajo polprevodniški laserji in frekvenčno
podvojeni Nd:YAG laserji. 

\section{Laser na ogljikov dioksid}
\index{Laser!CO$_2$}
Osnova za delovanje do zdaj opisanih plinskih laserjev so elektronski prehodi
v atomih ali ionih. Osnova za delovanje laserja na ogljikov dioksid pa so 
prehodi med vibracijskimi stanji molekul 
CO$_2$, pri čemer elektroni ostanejo v osnovnem stanju.\footnote{~C. K. N. Patel,
Phys. Rev. $\mathbf{136}$, A1187 (1964).}
Zaradi majhnih energijskih razlik med vibracijskimi stanji deluje
ta laser v infrardečem delu spektra, najpogosteje pri \index{Infrardeče valovanje}
$9,6~\si{\micro\metre}$ in $10,6~\si{\micro\metre}$. Laser deluje v zveznem
in sunkovnem načinu. Odlikujejo ga zelo velik izkoristek ($~\sim 30~\%$) in 
posledično zelo velike moči, tipično od $1~\si{\watt}$ do $10~\si{\kilo\watt}$. 

Preden opišemo delovanje laserja, si na kratko oglejmo lastna nihanja molekule 
ogljikovega dioksida. Molekula CO$_2$ je v osnovnem stanju linearna molekula 
(slika~\ref{fig:CO2}\,a). 
Za molekule take oblike obstajajo trije osnovni načini nihanja atomov glede na težišče:
upogib, pri katerem atomi nihajo v smeri pravokotno na os (slika~\ref{fig:CO2}\,b),
simetrični razteg, pri katerem atoma kisika nihata simetrično vzdolž osi molekule, ogljik pa pri tem miruje (slika~\ref{fig:CO2}\,c), in asimetrični razteg, pri 
katerem atoma kisika nihata v isti smeri vzdolž osi, ogljik pa v nasprotni 
(slika~\ref{fig:CO2}\,d). Najvišjo frekvenco nihanja ima asimetrični razteg,
najnižjo pa upogib.
Vsako vibracijsko stanje molekule lahko razstavimo na osnovne nihajne načine in 
ga opišemo s številom energijskih kvantov v posameznem osnovnem nihanju, 
torej s trojico celih števil $(n_1,n_2,n_3)$. Po dogovoru stanje 100 opisuje
osnovni simetrični razteg, stanje 010 osnovni upogib in stanje 001 
osnovni asimetrični razteg.

\begin{figure}[ht]
\centering
\def\svgwidth{100truemm} 
\input{slike/07_CO2.pdf_tex}
\caption{Molekula CO$_2$ (a) in trije osnovni načini nihanja molekule:
upogib~(b), simetrični razteg~(c) in asimetrični razteg~(d). Atomi ogljika so označeni s črno
in kisika z rdečo barvo.}
\label{fig:CO2}
\end{figure}

Vibracijska stanja molekule vzbudimo z električnim tokom skozi plin. 
\index{Energijski nivoji!CO$_2$}
Zato v razelektritveno cev dodamo dušik (N$_2$) in podobno kot pri He-Ne laserju
se tudi CO$_2$ črpa predvsem preko trkov z dušikovimi molekulami. 
Dušikova molekula je dvoatomna in ima zato zgolj eno vibracijsko stanje, ki po energiji
praktično sovpada z energijo stanja 001 (slika~\ref{fig:CO2E}). Iz tega zgornjega
nivoja molekule prehajajo v stanje 100 ($10,6~\si{\micro\metre}$) ali stanje
020 ($9,6~\si{\micro\metre}$). Da pospešimo prehod nazaj v osnovno stanje, 
plinski mešanici dodamo helij, s katerim trkajo molekule.
Razmerje parcialnih tlakov je navadno 1\,:\,1\,:\,8 za CO$_2$\,:\,N$_2$\,:\,He pri tlaku $1~\si{\milli\bar}$. 
Pri tako nizkih tlakih je poglavitna razširitev spektralne črte Dopplerjeva, 
\index{Dopplerjeva razširitev}ki 
je v primerjavi z ostalimi plinskimi laserji zaradi nizkih frekvenc zelo majhna,
le okoli $70~\si{\mega\hertz}$. V laserjih z višjim tlakom 
prevlada razširitev zaradi medmolekulskih trkov. Pri tlakih okoli $20~\si{\bar}$
znaša razširitev okoli $500~\si{\giga\hertz}$, kar omogoča izdelavo fazno uklenjenih 
sunkovnih laserjev s sunki dolžine $\sim 1~\si{\pico\second}$. Nekaj tipičnih podatkov 
za laser na ogljikov dioksid je zbranih v tabeli~\ref{tab:Ar}.

\begin{figure}[ht]
\centering
\def\svgwidth{95truemm} 
\input{slike/07_CO2E.pdf_tex}
\caption{Shema vibracijskih nivojev v laserju na ogljikov dioksid. 
Nivoji dušika so označeni z modro in nivoji CO$_2$ z zeleno, laserski prehodi 
pa z rdečimi barvami.}
\label{fig:CO2E}
\end{figure}

Najpreprostejši laser na ogljikov dioksid \index{Laser!zgradba} 
je po svoji zgradbi podoben drugim plinskim laserjem. 
Razelektritvena cev (polmer $\sim 1~\si{\centi\metre}$ in dolžina $0,5$--$2~\si{\metre}$) 
je na obeh koncih zaključena z Brewstrovima oknoma in zrcaloma. Vsi optični elementi
v laserju morajo biti seveda prepustni oziroma odbojni za infrardeči del spektra. Ker lahko 
laser deluje pri različnih valovnih dolžinah, dodamo v resonator frekvenčno selektiven
člen, na primer uklonsko mrežico (slika~\ref{fig:CO2S}).\index{Uklonska mrežica}

\begin{figure}[ht]
\centering
\def\svgwidth{100truemm} 
\input{slike/07_CO2Shema.pdf_tex}
\caption{Poenostavljena shema laserja na ogljikov dioksid: R -- razelektritvena cev, 
IZ -- izhodno zrcalo, Z -- zrcalo z veliko odbojnostjo, B -- Brewstrovi okni, 
U -- uklonska mrežica\index{Brewstrovo okno}
}
\label{fig:CO2S}
\end{figure}

Laserji na ogljikov dioksid se zaradi svoje velike izhodne moči uporabljajo v 
industriji za zahtevne obdelave materialov, na primer za rezanje 
kovin, vrtanje, ablacijo, varjenje in tudi za vojaške namene. Zaradi velike
absorpcije izsevane svetlobe v vodi se uporabljajo tudi v medicinske namene, predvsem
za rezanje tkiv in dermatologijo. 
Obdelava z laserji omogoča veliko natančnost, čistočo in je zelo fleksibilna.

\begin{table}
\small
\begin{center}
\begin{tabular}{|l|c|c|c|c|}\hline
Laser & He-Ne & Ar$^+$ & CO$_2$ & ekscimer\\ \hline
Valovna dolžina  $\lambda$ & $632,8~\si{\nano\metre}$& $488$ in
$514,5~\si{\nano\metre}$ & $9,6$ in $10,6~\si{\micro\metre}$ & UV
\\ \hline
Verjetnost za spontani prehod $A$ & $3,4 \times 10^6/\si{\second}$ & 
$7,8 \times 10^7/\si{\second}$ & $0,25/\si{\second}$ & $\sim 10^8/\si{\second}$ \\ \hline
Presek za stimulirano emisijo $\sigma$ & $3 \times 10^{-17}~\si{\metre}^2$&  $2,6 \times 10^{-16}~\si{\metre}^2$ & $3 \times 10^{-22}~\si{\metre}^2$ & $ 10^{-20}~\si{\metre}^2$ \\ \hline
Spektralna širina črte $\Delta \nu$ & $1,5 \times 10^{9}~\si{\hertz}$ & 
$3,5 \times 10^{9}~\si{\hertz}$ &$7 \times 10^{7}~\si{\hertz}$ & $10^{13}~\si{\hertz}$ \\ \hline
Obrnjena zasedenost $\Delta N/V$ & $5 \times 10^{15}/\si{\metre}^3$ & $2 \times 10^{15}/\si{\metre}^3$ & $3 \times 10^{21}/\si{\metre}^3$ & $10^{20}/\si{\metre}^3$\\ \hline
\end{tabular}
\caption{Izbrani podatki za helij-neonski in argonski laser ter laser na ogljikov dioksid in tipičen ekscimerni laser}
\index{Laser!He-Ne}
\index{Laser!argonski}
\index{Laser!CO$_2$}
\index{Laser!ekscimerni}
\label{tab:Ar}
\end{center}\vglue-3truemm
\end{table}

\section{Ekscimerni laser}
\index{Laser!ekscimerni}
Beseda ekscimer ({\it excited dimer, excimer}) označuje vzbujena vezana 
stanja dveh atomov, 
ki se v osnovnem stanju ne bi vezala.\footnote{~N. G. Basov et al., JETP Lett.
$\mathbf{12}$, 329 (1970).}
Za laserje so zanimivi predvsem ekscimeri
težkih žlahtnih plinov in halogenov, na primer Ar$_2^*$ ($126~\si{\nano\metre}$), 
Kr$_2^*$ ($146~\si{\nano\metre}$), Xe$_2^*$ ($172~\si{\nano\metre}$),
ArF ($193~\si{\nano\metre}$), KrF ($248~\si{\nano\metre}$), 
XeCl ($308~\si{\nano\metre}$), ArBr ($161~\si{\nano\metre}$) in 
NeF ($108~\si{\nano\metre}$). Te molekule obstajajo samo v vzbujenem stanju,
v osnovnem stanju  je zaradi prevelike odbojne sile med atomoma molekula neobstojna.
Vsi našteti primeri oddajajo lasersko svetlobo v\index{Ultravijolično valovanje}
ultravijoličnem delu, ki ga drugi laserski sistemi le težko pokrivajo. 
Ekscimerni laserji delujejo v sunkih, pri čemer je tipična oddana energija v sunku 
$\sim 1~\si{\joule}$, dolžina sunka pa $10$--$100~\si{\nano\second}$ pri repeticiji
$\sim 100~\si{\hertz}$.

Dva atoma se vežeta, kadar je ionizacijska energija prvega
atoma manjša od vsote elektronske afinitete drugega atoma in
elektrostatične energije vezave obeh ionov. Vzemimo za primer klor in
kripton. Ionizacijska energija kriptona v osnovnem stanju je 14~eV, v
vzbujenem pa 5~eV. Elektronska afiniteta klora je 3,75~eV in
elektrostatična vezavna energija KrCl okoli 7~eV. Tako je treba za nastanek
molekule KrCl v osnovnem stanju dodati okoli 4~eV, pri tvorbi
molekule v vzbujenem stanju pa se sprosti okoli 6~eV. Odvisnost
potencialne energije molekule KrCl v osnovnem in vzbujenem stanju
kaže slika~\ref{fig:exE}. Molekula, ki je vezana v vzbujenem stanju, po
sevalnem prehodu v osnovno stanje takoj razpade, zato je zelo lahko doseči
obrnjeno zasedenost. Razpadni čas vezanega stanja je $\sim~10~\si{\nano\second}$ in
spodnjega nevezanega okoli $0,1~\si{\pico\second}$.
Da nastanejo ekscimeri, vzbujamo mešanico 
plinov v heliju. Pritisk
je razmeroma velik ($\sim 3~\si{\bar}$), zato plin v cevi vzbujamo prečno.
Velika je tudi spektralna širina prehoda ($\Delta\nu = 10^{13}~\si{Hz}$). 

Ekscimerni laserji delujejo v sunkih s precej veliko energijo in se zaradi kratke valovne
dolžine in velike natančnosti uporabljajo v fotolitografiji in izdelavi mikroprocesorjev 
ter v medicini, predvsem oftalmologiji in kirurgiji.
\vskip-2truemm
\begin{figure}[ht]
\centering
\def\svgwidth{50truemm} 
\input{slike/07_exE.pdf_tex}
\caption{Shema energije $E$ v odvisnosti od razdalje med jedroma atomov $d$. V vzbujenem stanju
se atoma povežeta v molekulo, po prehodu v nižji nivo  molekula razpade.}
\label{fig:exE}
\end{figure}

\section{Neodimski laser}
Druga skupina laserjev, ki jo bomo obravnavali, so trdninski laserji. Taki laserji
temeljijo na elektronskih prehodih v ionih primesi, ki jih dodamo v kristal ali steklo,
črpamo pa jih optično. Primesi so navadno redke zemlje ali prehodne kovine, 
kristali pa so oksidi ali fluoridi. Izdelava ojačevalnih sredstev na osnovi stekla
je bistveno preprostejša in cenejša, vendar ima steklo precej nižjo toplotno prevodnost
od kristalov in se zato bolj greje. 
Začeli bomo z opisom dveh primerov neodimskega laserja, Nd:YAG
in Nd:steklo.\footnote{~
Podobne laserje dobimo, če v kristalu YAG itrijeve ione deloma
nadomestimo z iterbijem ($1030~\si{\nano\metre}$) ali 
erbijem ($2940~\si{\nano\metre}$).\index{Iterbij}\index{Erbij}} 

\subsection{Nd:YAG}
\index{Laser!Nd:YAG}
V Nd:YAG laserju\footnote{~J. E. Geusic,
H. M. Marcos in L. G. Van Uitert, Appl. Phys. Lett. $\mathbf{4}$, 182 (1964).} je ojačevalno sredstvo
itrij-aluminijev granat (Y$_3$Al$_5$O$_{12}$, YAG) s primesmi neodimovih ionov Nd$^{3+}$. 
Laser deluje pri valovni dolžini $1,064~\si{\micro\meter}$ ali frekvenčno podvojeni
$532~\si{\nano\metre}$. Deluje v zveznem \index{Infrardeče valovanje}
načinu pri močeh do $5~\si{\kilo\watt}$ ali sunkovnem z dolžino sunkov okoli 
$100~\si{\nano\second}$ in energijo sunka $\sim 1~\si{\joule}$.

Neodimski laser je primer štirinivojskega laserja, 
\index{Štirinivojski sistem}pri čemer je 
laserski prehod med stanjema $^4$F$_{3/2}$ in $^4$I$_{11/2}$ neodimovih ionov 
(slika~\ref{fig:NdE}). S svetlobo z valovno dolžino tipično okoli $800~\si{\nano\metre}$
črpamo elektrone v višje nivoje, ki hitro 
preidejo v zgornji laserski nivo. Življenjski čas zgornjega nivoja je 
okoli $230~\si{\micro\second}$, spodnjega pa precej krajši, zato je 
lahko doseči veliko obrnjeno zasedenost. Spodnje stanje je dovolj visoko nad 
osnovnim, da pri sobni temperaturi v ravnovesju ni znatno zasedeno. 
Razširitev črte je homogena in je posledica predvsem \index{Spektralna črta!homogena razširitev}
termičnega nihanja kristalne mreže ($\Delta \nu = 130~\si{GHz}$). 
Prag neodimskega laserja za zvezno delovanje je nizek in ga je lahko doseči, 
prav tako dobro neodimski laser deluje v sunkih, predvsem s preklopom dobrote.
\index{Energijski nivoji!Nd:YAG}
\begin{figure}[ht]
\centering
\def\svgwidth{85truemm} 
\input{slike/07_NdE.pdf_tex}
\caption{Shema energijskih nivojev neodimovih ionov v Nd:YAG laserju}
\label{fig:NdE}
\vglue-3truemm
\end{figure}

Laser optično črpamo z diodnimi laserji ali močnimi ksenonskimi svetilkami za zvezno delovanje 
in podobnimi bliskavicami za sunkovno delovanje (slika~\ref{fig:Nd}\,a). 
Aktivna snov v laserju je v obliki paličice dolžine od nekaj cm do dobrih 
$10~\si{\centi\metre}$ in premera $\sim 1~\si{\centi\metre}$. 
V kristalu YAG neodimovi ioni nadomestijo približno $1~\%$ itrijevih, zato je ojačevalno
sredstvo na videz rahlo rožnato (slika~\ref{fig:Nd}\,b). 
Aktivna paličica in svetilka sta vgrajeni v cilindrično ali eliptično votlino z 
zrcalnimi ali belimi stenami, tako da se čim večji del črpalne svetlobe absorbira v 
laserski paličici (slika~\ref{fig:Nd}\,c).

\begin{figure}[ht]
\centering
\def\svgwidth{120truemm} 
\input{slike/07_Nd.pdf_tex}
\caption{Ksenonska bliskavica (a), ojačevalno sredstvo v Nd:YAG laserju (b) 
in shema prečnega preseka eliptične črpalne votline (c)}
\label{fig:Nd}
\vglue-4truemm
\end{figure}

Pri črpanju s ksenonsko svetilko je le manjši del črpalne svetlobe v
absorpcijskih pasovih, zato je izkoristek razmeroma slab, tipično 
pod $1~\%$. Za izhodno moč zvezno delujočega Nd:YAG laserja $\sim 10~\si{\watt}$ je tako
potrebna električna moč $\sim 1~\si{kW}$. Velika večina porabljene moči 
gre v gretje, zato je v laserjih z nekoliko večjo povprečno
močjo potrebno vodno hlajenje. Gretje povzroča tudi toplotne deformacije
laserske paličice, kar lahko močno spremeni lastnosti resonatorja. Toplotni
učinki so poglavitna praktična težava pri izdelavi neodimskih
laserjev s klasičnimi svetilkami. Danes zato zvezno delujoče neodimske laserje
črpamo z diodnimi laserji, ki svetijo v območju največje
absorpcije Nd$^{3+}$. Črpanje je lahko prečno ali vzdolžno (slika~\ref{fig:NdS}). 
Pri diodnem črpanju je izkoristek dosti večji in je manj gretja, kar omogoča 
bolj kompaktno konstrukcijo in boljšo stabilnost izhodne moči.
\index{Laser!zgradba} 
\begin{figure}[ht]
\centering
\def\svgwidth{110truemm} 
\input{slike/07_NdS.pdf_tex}\index{Črpanje!vzdolžno}
\caption{Shema vzdolžno črpanega Nd:YAG laserja: O -- ojačevalno sredstvo, 
IZ -- izhodno zrcalo, D -- dikroično zrcalo, 
prepustno za črpalno svetlobo in odbojno za lasersko, DL -- diodni 
laser za črpanje, L -- leča
}
\label{fig:NdS}
\vglue-4truemm
\end{figure}

Neodimski laserji so zelo razširjeni, tako v osnovni kot v frekvenčno 
podvojeni različici. Uporabni so za obdelavo materialov (vrtanje, varjenje, 
litografija). Ker snop preprosto sklopimo v optično vlakno, so 
zelo uporabni tudi v medicini (endoskopska kirurgija in dermatologija). 
Pomemben proizvajalec sunkovnih Nd:YAG laserjev 
za medicinske namene je podjetje Fotona d.o.o. iz Ljubljane.\index{Fotona d.o.o.}

\begin{table}[ht]
\small
\begin{center}
\begin{tabular}{|l|c|c|c|}\hline
Laser & Nd:YAG & Nd:steklo & Ti:safir \\ \hline
Valovna dolžina  & $1064~\si{\nano\metre}$ & $1050~\si{\nano\metre}$ & 
 $600-1100~\si{\nano\metre}$\\ \hline
Verjetnost za spontani prehod $A$ & $4 \times 10^3/\si{\second}$ & $3 \times 10^3/\si{\second}$
& $3 \times 10^5/\si{\second}$\\ \hline
Presek za stimulirano emisijo $\sigma$ & $3 \times 10^{-23}~\si{\metre}^2$ &
$3 \times 10^{-24}~\si{\metre}^2$ & $3 \times 10^{-23}~\si{\metre}^2$\\ \hline
Spektralna širina črte $\Delta \nu$ & $1,3 \times 10^{11}~\si{\hertz}$ &
$7 \times 10^{12}~\si{\hertz}$ & $1 \times 10^{14}~\si{\hertz}$\\ \hline
Gostota obrnjene zasedenosti $\Delta N/V$ & $1,6 \times 10^{23}/\si{\metre}^3$ &
$8 \times 10^{23}/\si{\metre}^3$ & $6 \times 10^{23}/\si{\metre}^3$\\ \hline
\end{tabular}
\caption{Tipični podatki za Nd:YAG, Nd:steklo in Ti:safir laser}
\index{Laser!Nd:YAG}
\index{Laser!Nd:steklo}
\index{Laser!Ti:safir}
\label{tab:nd}
\end{center}
\end{table}
\newpage
\subsection{Nd:steklo}
\index{Laser!Nd:steklo}
Namesto v kristal lahko neodimove ione Nd$^{3+}$ vgradimo tudi v steklo. 
Laser z ojačevalnim sredstvom Nd:steklo  deluje 
pri valovni dolžini $1,050~\si{\micro\meter}$ v sunkovnem načinu 
s preklopom dobrote ali z uklepanjem faz z energijami sunkov $~\sim 1~\si{\joule}$.
Zaradi amorfne strukture stekla in posledično 
nehomogenega lokalnega polja je laserska črta nehomogeno razširjena in razmeroma široka
($\Delta \nu = 7~\si{THz}$).
\index{Spektralna črta!nehomogena razširitev}
Ojačenje je manjše kot v Nd:YAG in za prag laserskega delovanja je
potrebna precej večja črpalna moč. Laserji z ojačevalnim sredstvom Nd:steklo se zato uporabljajo le v 
sunkovnem načinu in za tako delovanje so celo primernejši od Nd:YAG laserjev.
Zaradi manjšega ojačenja pri dani obrnjeni zasedenosti 
je v laserju s preklopom dobrote mogoče doseči večjo načrpanost, preden pride do praznjenja
zaradi ojačevanja spontanega sevanja v enem preletu paličice. Problem teh laserjev
predstavlja nizka toplotna prevodnost stekla, ki omejuje repeticijo sunkov.
Velika širina spektralne črte je zelo primerna za delovanje v načinu uklepanja faz, s 
katerim dosegamo ultrakratke sunke ($\sim 100~\si{\femto\second}$). 
\vglue-3truemm
\begin{remark}
Energije izsevanih sunkov je mogoče še povečati z ojačevalniki. Med največjimi je
laserski sistem Nd:steklo, ki ga uporabljajo za raziskave fuzije (NIF -- National Ignition Facility, 
Livermore, Kalifornija).
Okoli $1~\si{\nano\second}$ dolg sunek iz osnovnega laserja razdelijo na 192
ojačevalnih vej, ga postopoma ojačujejo in nato na koncu spet združijo.
Končna energija sunka je tako nad $\sim 1~\si{\mega\joule}$. Z njim z vseh strani posvetijo na
kroglico iz devterija in tritija, ki se dovolj segreje in stisne, da pride do 
njunega zlivanja. Vršna moč laserskega sunka je okoli $10^{15}$~W. 
Če laserski snop zberemo na površino 1~mm$^2$, je v gorišču jakost električnega polja
okoli $5 \times 10^{11}$~V/m, kar je približno enako električnemu polju v vodikovem atomu.
\end{remark}

\section{Titan-safirni laser}
\index{Laser!Ti:safir}
Titan-safirni (Ti:safir) laser je trdninski laser\footnote{~P. F. Moulton, J. Opt. Soc. Am. B $\mathbf{3}$
125 (1986).}, pri katerem so v kristal safirja
Al$_2$O$_3$ primešani ioni titana Ti$^{3+}$. Njegova najpomembnejša značilnost je
zvezna nastavljivost valovne dolžine v zelo širokem frekvenčnem pasu 
($600$--$1100~\si{\nano\metre}$) z največjo učinkovitostjo pri okoli $800~\si{\nano\metre}$. Deluje
v zveznem načinu z močmi do $50~\si{\watt}$ in sunkovno  v fazno uklenjenem načinu 
z dolžino sunkov $\sim~10~\si{\femto\second}$ z vršnimi močmi nad $10^{12}~\si{\watt}$. 
\index{Energijski nivoji!Ti:safir}
\begin{figure}[ht]
\centering
\def\svgwidth{75truemm} 
\input{slike/07_TiE.pdf_tex}
\caption{Energijski nivoji v Ti:safir laserju. Nivoja sta zaradi vibracij
razcepljena na veliko število podnivojev, ki se med seboj deloma prekrivajo.
Zelo podobna je tudi shema energijskih nivojev organskih barvil. 
}
\label{fig:TiE}
\vglue-3truemm
\end{figure} 

Ojačevalno sredstvo v Ti:safir laserju je aluminijev oksid, v katerem 
približno $0,2~\%$ aluminijevih ionov nadomestimo s titanovimi. Titanovi ioni imajo 
v taki konfiguraciji zgolj eno vzbujeno stanje, vendar se zaradi sklopitve s fononi
vibracijski nivoji posameznega stanja med seboj prekrivajo in prehod je močno razširjen. 
Z optičnim črpanjem vzbudimo titanove ione iz osnovnega stanja v eno izmed vibracijskih 
stanj vzbujenega stanja. Ioni nato hitro preidejo v najnižje vzbujeno stanje. 
Laserski prehod poteka med najnižjim vzbujenim stanjem in enim od vibracijskih 
nivojev osnovnega stanja (slika~\ref{fig:TiE}). Življenjski čas
vzbujenega stanja je kratek ($3,2~\si{\micro\second}$) in širina črte največja med
vsemi trdninskimi laserji ($\Delta \nu =  100~\si{THz}$). 
Ker je vrh absorpcijskega pasu blizu $500~\si{\nano\metre}$,
laser črpamo z zeleno svetlobo (argonski laser oziroma
frekvenčno podvojen neodimski laser za sunkovno delovanje). 

Najpomembnejša uporaba Ti:safir laserjev je v raziskovalnih laboratorijih za ustvarjanje zelo 
kratkih sunkov svetlobe z dolžino $\sim 10~\si{\femto\second}$. Pot, ki jo 
v tem času prepotuje svetloba, je le nekaj valovnih dolžin svetlobe. 

\section{Laserji na organska barvila}
\index{Laser!organska barvila}
V laserjih na organska barvila je barvilo raztopljeno v tekočini, praviloma vodi ali alkoholu. 
To so bili prvi laserji z veliko spektralno širino in nastavljivo valovno dolžino
izhodne svetlobe. Delujejo lahko kot zvezni laserji in z izbiro barvila dosežemo
delovanje v območju $300$--$1500~\si{\nano\metre}$ pri močeh do $\sim 2~\si{\watt}$.
Široka spektralna širina omogoča sunkovno delovanje z uklepanjem faz. Dolžina sunkov je nekaj 
femtosekund sunki, energija v sunkih pa nekaj $100~\si{\joule}$.

Shema energijskih nivojev molekule tipičnega organskega barvila
je zelo podobna shemi energijskih nivojev Ti:safir laserja (slika~\ref{fig:TiE}).
Vsi elektronski nivoji so razcepljeni v vibracijske in rotacijske podnivoje. 
V toplotnem ravnovesju je molekula na dnu osnovnega elektronskega stanja S$_0$. 
Z absorpcijo vidne svetlobe primerne frekvence preide v neko vzbujeno
singletno stanje S$_1$. Preko trkov z molekulami topila vzbujena barvilna molekula
zelo hitro, v času okoli pikosekunde, preide na dno vzbujenega stanja, od
koder s sevanjem preide nekam v osnovno stanje S$_0$, od tam pa s trki
hitro nazaj na dno osnovnega stanja. Ker
sta obe elektronski stanji zaradi vibracij in rotacij razširjeni, sta 
absorpcijska in emisijska fluorescenčna črta široki ($\Delta \nu = 30~\si{THz}$).
Energija izsevane svetlobe je zmanjšana za energijo
prehodov s trki, zato je emisijska črta premaknjena k nižjim
frekvencam glede na absorpcijsko. Absorpcijski in fluorescenčni spekter prehoda S$_0-$S$_1$
za barvilo rodamin 6G kaže slika~\ref{fig:RhG}.
\vglue5truemm
\begin{figure}[ht]
\centering
\def\svgwidth{80truemm} 
\input{slike/07_RhG.pdf_tex}
\caption{Absorpcijski in emisijski spekter barvila rodamin 6G, ki se uporablja v laserjih}
\label{fig:RhG}
\end{figure} 

\begin{table}[ht]
\begin{center}
\begin{tabular}{|l|c|}\hline
Valovna dolžina  & $300$--$1500~\si{\nano\meter}$\\ \hline
Verjetnost za spontani prehod $A$ & $ \sim 10^8/\si{\second}$ \\ \hline
Presek za stimulirano emisijo $\sigma$ & $3 \times 10^{-20}~\si{\metre}^2$ \\ \hline
Spektralna širina črte $\Delta \nu$ & $3 \times 10^{13}~\si{\hertz}$  \\ \hline
Gostota obrnjene zasedenosti $\Delta N/V$ & $ \sim 10^{22}/\si{\metre}^3$ \\ \hline
\end{tabular}
\caption{Tipični podatki za laserje na organska barvila}
\label{tab:orgb}
\end{center}
\end{table}

Laser na organska barvila lahko deluje pri vseh frekvencah znotraj široke
fluorescenčne črte. Zato moramo v resonator vgraditi frekvenčno
selektivni element, s katerim nastavljamo frekvenco izhodne svetlobe. Za to lahko 
uporabimo prizmo ali eno od zrcal nadomestimo 
z uklonsko mrežico, ki je zasukana pod takim kotom, da se po osi resonatorja odbije le svetloba
izbrane valovne dolžine.\index{Uklonska mrežica}
Barvilne laserje črpamo ali z bliskavico ali z drugim laserjem primerne 
valovne dolžine, na primer argonskim ali ekscimernim laserjem. 

Slabost laserjev na organska barvila je njihova degradacija. Barvila v
laserjih je treba pogosto menjati (tipično na 100 ur delovanja), poleg tega je ravnanje
z njimi zahtevno, saj je veliko barvil in topil strupenih ali korozivnih.
Laserji na organska barvila so zaradi svoje nastavljive valovne dolžine
uporabni v spektroskopiji, za ločevanje izotopov, v 
medicini (dermatologija, odstranjevanje ledvičnih kamnov) ...
 
\section{Vlakenski laserji}
\index{Laser!vlakenski}
Posebna vrsta laserjev so vlakenski laserji, v katerih za aktivno 
sredstvo uporabimo optično vlakno, dopirano z ioni \index{Optično vlakno}redkih zemelj.\footnote{~Za
podroben opis optičnih vlaken glej poglavje~\ref{chap:fibri}.}
Valovna dolžina, pri kateri oddajajo svetlobo, je odvisna od snovi, s katerimi
je vlakno dopirano. Najpogosteje je to erbij ($1550~\si{nm}$),\index{Erbij}
iterbij ($\sim 1100~\si{nm}$)\index{Iterbij} ali neodim ($1064~\si{nm}$).
Vlakenske laserje odlikujeta\index{Infrardeče valovanje}
velik izkoristek (tipično okoli $70$--$80~\%$, lahko tudi več) 
in posledično zelo velika moč (do $20~\si{kW}$). Zanje sta značilni tudi
izredno velika kakovost snopa (faktor $M^2<1,1$, glej razdelek~\ref{chap:gaussovsnop}) in 
razmeroma majhna občutljivost na zunanje motnje. Delujejo lahko v zveznem
ali sunkovnem načinu.\index{Faktor $M^2$}

\begin{figure}[ht]
\centering
\def\svgwidth{127truemm} 
\input{slike/07_FibEr.pdf_tex}
\caption{Energijski nivoji v erbijevem vlakenskem laserju (a) ter 
absorpcijski in emisijski spekter za erbij (b). Dodaten absorpcijski 
vrh pri $980~\si{nm}$ ni prikazan.}
\label{fig:ErFib}
\end{figure} 

Oglejmo si vlakenski laser, katerega vlakno je dopirano z ioni erbija 
(masni delež $\sim 1~\%$). Vlakna so pogosto dodatno dopirana z iterbijem, kar
poveča absorpcijo črpalne svetlobe in s tem izkoristek laserja. Laser črpamo
optično z lasersko diodo pri $980~\si{nm}$ ali $1480~\si{nm}$, laserski prehodi 
pa se zgodijo ob povratku v osnovno stanje. Osnovno stanje je razcepljeno v več podnivojev
(slika~\ref{fig:ErFib}), zato je valovna dolžina oddane svetlobe v razmeroma 
širokem intervalu ($1520$--$1560~\si{nm}$). Velika spektralna širina
($\Delta \nu = 3~\si{THz}$)
omogoča delovanje z uklepanjem faz. 

Zgradba vlakenskih laserjev se razlikuje od do zdaj opisanih. Glavna razlika je
seveda v resonatorju, ki je v tem primeru kar optično vlakno. Tipičen premer je 
$\sim 5~\si{\micro\meter}$ in dolžina več metrov. Na koncih vlakna lahko
postavimo dikroični zrcali, ki omogočata longitudinalno sklopitev črpalne svetlobe 
v vlakno. Namesto zrcal se pogosto uporablja periodične strukture 
na koncih vlakna, na katerih se valovanje izbrane valovne dolžine Braggovo odbija
(slika~\ref{fig:Fibshema}). \index{Laser!zgradba}\index{Braggov odboj}
S selektivnim odbojem se širina spektra izhodnega valovanja bistveno zmanjša. 

Navadno uporabljamo vlakna, ki so sestavljena iz sredice in dveh plaščev. Laserska
svetloba ostaja ujeta v sredici vlakna, medtem ko črpalno svetlobo vodimo po notranjem plašču. To
omogoča bistveno lažjo sklopitev črpalne svetlobe v vlakno. Poleg tega so zaradi povečanja
efektivnega polmera snopa vršne intenzitete manjše in posledično manjše verjetnosti
pojava neželenih nelinearnih pojavov (glej razdelek~\ref{NLOFIB}).

\begin{figure}[ht]
\centering
\def\svgwidth{100truemm} 
\input{slike/07_Fibshema.pdf_tex}
\caption{Shema vlakenskega laserja: LD -- črpalna laserska dioda, 
BPS -- Braggova periodična struktura, V -- optično vlakno
}
\label{fig:Fibshema}
\end{figure}

Vlakenski laserji se uporabljajo v telekomunikacijah, saj oddajajo svetlobo 
valovnih dolžin, pri katerih je v vlaknih najmanjša disperzija (glej razdelek~\ref{chap:Disperzija}). 
Velika intenziteta izhodne svetlobe omogoča obdelavo materialov, varjenje, vrtanje in rezanje kovin. 
Zaradi svojih mehanskih lastnosti so primerni tudi za premično lasersko obdelavo snovi.

\begin{table}[ht]
\begin{center}
\begin{tabular}{|l|c|}\hline
Valovna dolžina  & $1550~\si{\nano\meter}$\\ \hline
Verjetnost za spontani prehod $A$ & $ \sim 90/\si{\second}$ \\ \hline
Presek za stimulirano emisijo $\sigma$ & $7 \times 10^{-25}~\si{\metre}^2$ \\ \hline
Spektralna širina črte $\Delta \nu$ & $3 \times 10^{12}~\si{\hertz}$  \\ \hline
Gostota obrnjene zasedenosti $\Delta N/V$ & $ \sim 10^{24}/\si{\metre}^3$ \\ \hline
\end{tabular}
\caption{Tipični podatki za erbijev vlakenski laser}
\label{tab:fib}
\end{center}
\end{table}

\begin{remark}
 Namesto vlaken, dopiranih z ioni redkih zemelj, lahko za izdelavo vlakenskih laserjev
 izkoristimo pojav stimuliranega Ramanovega sipanja (glej razdelek~\ref{chap:SRS}). 
 Pri tem pojavu se črpalna svetloba neelastično siplje, ojači pa se valovanje 
 pri nižji frekvenci. Razlika frekvenc ustreza vibracijskim prehodom 
 molekul, ki prevzamejo preostanek energije. Signal, ki se pri prehodu ojačuje, 
 ostaja pretežno ujet v vlakno z Braggovimi periodičnimi strukturami na koncih.
 Zavedati se moramo razlike med navadnim vlakenskim laserjem, ki deluje zaradi vzpostavljene
 obrnjene zasedenosti, in Ramanskim laserjem, v katerem se ojači sipana
 svetloba.\index{Laser!Ramanski}\index{Ramanovo sipanje!stimulirano}
\end{remark}

\section{Polprevodniški laserji}
\label{chap:SCL}
\index{Laser!polprevodniški}\index{Laser!diodni|see {Laser, polprevodniški}}
Danes so nedvomno najpomembnejši polprevodniški oziroma diodni 
laserji.
Njihove glavne značilnosti so veliko ojačenje in zato majhna 
dimenzija ($\sim 10$--$100~\si{\micro\metre}$), nizka cena, 
velik izkoristek ($\sim 50~\%$) in neposredno črpanje z električnim tokom. 
Za črpanje zadoščajo majhni tokovi \index{Črpanje}\index{Ultravijolično valovanje}
(tipično $\sim 100~\si{\milli\ampere}$), kar omogoča zelo hitro modulacijo (več $10~\si{GHz}$)
svetlobne moči s spreminjajočim se črpanjem. Slabosti polprevodniških laserjev sta razmeroma širok 
spekter in posledično majhna koherenca. Polprevodniški laserji delujejo v območju 
valovnih dolžin od $\sim 375~\si{nm}$ do več $\si{\micro\meter}$. Izhodne moči
so zelo odvisne od valovne dolžine: v ultravijoličnem (UV) območju so razmeroma nizke ($\sim 100~\si{mW}$),
sicer pa dosegajo vrednosti $\sim 3~\si{\watt}$.

Na hitro lahko rečemo, da delovanje diodnih laserjev temelji na rekombinaciji  
elektronov iz prevodnega pasu z vrzelmi v valenčnem pasu, pri čemer se izseva foton. Ta proces je lahko 
spontan, kot v svetlečih diodah, ali stimuliran, zaradi česar se svetloba ojača. 
Za podrobnejšo razlago ojačenja v polprevodniških 
laserjih moramo poznati osnove polprevodniške fizike, zato jo na kratko 
ponovimo.\footnote{~Glej npr. N. W. Ashcroft in N. D. Mermin, {\it Solid State Physics}, Harcourt College
Publishers (1976).}

\subsection*{Energijski pasovi v polprevodnikih}
V trdnih snoveh elektroni niso lokalizirani in zaradi interakcij med sosednjimi atomi
se sicer ostra energijska stanja razširijo v energijske pasove. Zgornji pas je lahko 
le delno zaseden (kovine), lahko pa se med najvišjim polno zasedenim 
(valenčnim) pasom in najnižjim nezasedenim (prevodnim) pasom pojavi energijska reža. 
Če je velikost energijske reže $E_g \sim 1~\si{eV}$ , je snov polprevodnik 
(tabela~\ref{table:gap}), sicer je izolator.\index{Polprevodnik}
\begin{table}[ht]
 \centering
\begin{tabular}{|l|c|c|c|c|c|c|} \hline  
      Snov & InSb & InAs & Ge & Si & GaAs & GaP \\ \hline
      $E_g~[\si{eV}]$ & 0,17 & 0,36 & 0,67 & 1,124 & 1,43 & 2,26  \\ \hline  
\end{tabular}
  \caption{Širina energijske reže $E_g$ v nekaterih polprevodnikih}
\label{table:gap}
\end{table}
\vglue-5truemm
Ko na polprevodnik vpade foton z energijo $\hslash\omega > E_g$, se foton absorbira, elektron
iz valenčnega pasu preide v prevodni pas in v valenčnem pasu ostane vrzel. Pričakovali bi, 
da se elektron, ki hitro preide na dno prevodnega pasu, spontano vrne v valenčni 
pas, pri čemer se svetloba izseva. Vendar je pri \index{Energijska reža}
prehodu treba upoštevati tudi ohranitev gibalne količine. Pri najobičajnejših polprevodnikih, 
siliciju in germaniju, leži vrh valenčnega pasu pri valovnem vektorju $\mathbf{k}=0$, 
dno prevodnega pasu pa pri $\mathbf{k} \neq 0$ (slika~\ref{fig:Ek}\,a).
Prehod elektrona prek take indirektne reže je malo verjeten, saj mora zaradi ohranitve 
gibalne količine priti še do interakcije s fononom. Prehod je veliko bolj verjeten v 
snoveh z direktno režo, pri katerih ležita tako dno prevodnega pasu kot vrh valenčnega pasu 
pri $\mathbf{k}=0$ (slika~\ref{fig:Ek}\,b). Snovi z direktno režo so na primer GaAs in 
druge spojine elementov III. skupine (Al, Ga, In) in V. skupine (P, As, Sb), ki so najuporabnejše 
za izdelavo diodnih laserjev.\index{Silicij}\index{Germanij}
\index{GaAs}\index{GaP}\index{InAs}\index{InSb}
\begin{figure}[ht]
\centering
\def\svgwidth{145truemm} 
\input{slike/07_Ek.pdf_tex}
\caption{Energijski pasovi v polprevodniku, kjer modra označuje valenčni pas, 
zelena pa prevodnega. Reža je lahko indirektna (a) ali direktna (b). V vzbujenem stanju (c) so 
najnižja mesta v prevodnem pasu zasedena in najvišja mesta valenčnega pasu
izpraznjena. 
}
\label{fig:Ek}
\end{figure}

V najpreprostejši sliki obliko prevodnega pasu v bližini minimuma opišemo 
s parabolično odvisnostjo od velikosti valovnega vektorja $\mathbf{k}$
\beq
E_p = E_g + \frac{\hslash^2 k^2}{2m_p}.
\label{pp:Ec}
\eeq
Pri tem $m_p$ označuje efektivno maso elektrona v prevodnem pasu, ki upošteva
interakcije z mrežo in se zato razlikuje od mase prostega elektrona $m_0$.

Podobno z efektivno maso zapišemo energijo vrzeli v valenčnem pasu
\beq
E_v = - \frac{\hslash^2 k^2}{2m_v}.
\label{pp:Ev}
\eeq
Zapišemo še gostoti stanj na energijski interval za prevodni in valenčni pas. Izhajamo iz zveze
$\varrho(k) dk = k^2dk /\pi^2$ (enačba~\ref{4.3}) in z upoštevanjem 
enačb~(\ref{pp:Ec} in \ref{pp:Ev}) dobimo
\beq
\varrho_p(E) = \frac{1}{2\pi^2}\left(\frac{2m_p}{\hslash^2} \right)^{3/2} \sqrt{E-E_g}
\qquad \mathrm{in}\qquad
\varrho_v(E) = \frac{1}{2\pi^2}\left(\frac{2m_v}{\hslash^2} \right)^{3/2} \sqrt{-E}.
\label{eq:rho_p}
\eeq
Pri tem sta ključna parametra efektivni masi 
elektronov in vrzeli. Ti dve masi sta značilni za posamezen polprevodnik
in znašata v GaAs $m_p = 0,067~m_0$ in $m_v = 0,5~m_0$. 
Gostota stanj za vrzeli je v GaAs zato približno 
dvajsetkrat večja od gostote stanj za elektrone.\index{Gostota stanj}

Verjetnost za zasedenost stanj je podana s Fermi-Diracovo funkcijo 
\begin{equation}  
f_p(E)=\frac{1}{e^{(E-E_F)/k_B T}+1},
\label{eq:7FD}
\end{equation}
pri čemer $E_F$ označuje Fermijevo energijo, $k_B$ pa Boltzmannovo konstanto. 
Pri $T=0$ so vsa stanja pod \index{Fermijeva energija}
Fermijevo energijo zasedena in nad njo prazna. Fermijeva energija leži
v energijski reži (slika~\ref{fig:Fermi}\,a). Pri končni temperaturi se na dnu prevodnega pasu nahajajo 
termično vzbujeni elektroni in na vrhu valenčnega pasu vrzeli (slika~\ref{fig:Fermi}\,b). Verjetnost 
za pojav vrzeli v valenčnem pasu je $f_v = 1 - f_p$.\index{Fermi-Diracova porazdelitev}
\begin{figure}[ht]
\centering
\def\svgwidth{145truemm} 
\input{slike/07_Fermi.pdf_tex}
\caption{Verjetnost za zasedenost stanj. Pri $T=0$ je valenčni pas poln in prevodni prazen (a).
Pri končni temperaturi termično vzbujeni elektroni preidejo v prevodni pas (b). Za opis neravnovesnega
stanja uporabimo dve Fermijevi energiji ($F_v$ in $F_p$), za vsak pas svojo (c).
}
\label{fig:Fermi}
\end{figure}
\vglue-0.5truecm
\begin{remark}
Fermijeva energija $E_F$ se določi iz pogoja, da je število elektronov v prevodnem pasu enako 
številu vrzeli v valenčnem pasu in $N_{p0} = N_{v0}$. Fermijeva energija
leži na sredini energijske reže le v primeru, da sta efektivni masi 
za elektrone in vrzeli enaki. Sicer se Fermijeva energija premakne
proti pasu z manjšo efektivno maso. 
\end{remark}

Število elektronov v prevodnem pasu na prostorninsko enoto izračunamo kot 
produkt gostote stanj in verjetnost, da je stanje zasedeno, integrirano po 
celotnem energijskem pasu
\beq
N_{p0}=\int_{E_g}^{\infty}\,\rho_p(E)f_p(E)\,dE.
\label{6.3a}
\eeq
Število vrzeli v valenčnem pasu je 
\beq
N_{v0}=\int_{-\infty}^{0}\,\rho_v(E)f_v(E)\,dE.
\label{6.3b}
\eeq
Število elektronov v prevodnem pasu (in vrzeli v valenčnem) je pri
$T=0$ enako nič in tudi pri končnih temperaturah ostaja razmeroma nizko. 
Znatno ga lahko povečamo, če polprevodnik dopiramo.

Dopiranje polprevodnika pomeni nadzorovano dodajanje ustreznih nečistoč. 
Če dodamo donorske primesi, ki povečajo število elektronov v snovi, 
govorimo o polprevodniku tipa $n$. Če dodajamo akceptorske snovi, ki 
elektrone sprejemajo, govorimo o polprevodniku tipa $p$. \index{Polprevodnik!tip $n$}
Primeri donorjev za GaAs so žveplo, selen ali telur,\index{Polprevodnik!tip $p$}
primera akceptorjev pa cink in kadmij. 

Zaradi primesi se v energijski 
reži pojavi dodaten energijski nivo, pri čemer je donorski nivo navadno 
tik pod prevodnim pasom in akceptorski tik nad valenčnim. V polprevodniku tipa $n$ se
Fermijeva energija premakne navzgor, pri močnem dopiranju lahko tudi v 
prevodni pas. V tem primeru so zasedena vsa stanja v valenčnem pasu in vsa
stanja do Fermijeve energije v prevodnem. Podobno je v močno dopiranem polprevodniku tipa 
$p$, v katerem se Fermijeva energija pomakne navzdol in število vrzeli v valenčnem 
pasu močno naraste. Prosta so tako vsa stanja v prevodnem pasu in stanja do 
Fermijeve energije v valenčnem pasu.

\begin{remark}
 Zgolj z dopiranjem se verjetnost za prehod in izsevanje fotona ne spremeni. V
 polprevodniku tipa $n$, na primer, se število elektronov v prevodnem pasu sicer znatno poveča, 
 vendar v valenčnem pasu ni ustreznih vrzeli, s katerimi bi se elektroni rekombinirali.
\end{remark}

Ko elektrone vzbudimo iz valenčnega v prevodni pas, se v valenčnem pasu pojavijo
vrzeli. Dokler se ne rekombinirajo (tipično nekaj $\si{ns}$),
vlada v prevodnem pasu kvazi-termično ravnovesje, saj je relaksacija 
elektronov znotraj pasu bistveno hitrejša (tipično $\si{ps}$). 
Pri veliki populaciji elektronov v prevodnem in vrzeli v 
valenčnem pasu Fermijeva funkcija ni več dobra za opis zasedenosti stanj 
(sliki~\ref{fig:Ek}\,c in \ref{fig:Fermi}\,c). Uporabimo \index{Kvazi-Fermijeva energija}
koncept kvazi-Fermijevih nivojev $F_p$ in $F_v$, s katerima opišemo porazdelitvi
v vsakem pasu posebej, za prevodni in valenčni pas
\beq
f_p(E)=\frac{1}{e^{(E-F_p)/k_B T}+1}
\label{eq:fDppas}
\eeq
in 
\beq
f_v(E)=\frac{1}{e^{(E-F_v)/k_B T}+1}.
\label{eq:fDvpas}
\eeq
To lahko naredimo, ker je hitrost rekombinacije elektronov in vrzeli znatno počasnejša
od vzpostavljanja kvazi-ravnovesja znotraj pasu. V termičnem ravnovesju je razlika med 
kvazi-Fermijevima energijama $F_{p}-F_v$ enaka nič, z naraščajočim vzbujanjem pa se 
razlika povečuje.

\subsection*{Ojačenje svetlobe v polprevodnikih}
Posvetimo na polprevodnik v vzbujenem stanju s svetlobo s krožno frekvenco $\omega$.
Vpadna svetloba povzroča prehode med stanji z energijo $E_a$ v valenčnem 
in stanji z energijo $E_b$ v prevodnem pasu (slika~\ref{fig:Ek}\,b).
Če je prehodov iz prevodnega pasu v valenčni pas več kot prehodov v obratni
smeri, se svetloba ojačuje.\index{Ojačenje v polprevodnikih}\footnote{~Glej npr. 
B. E. A. Saleh in M. C. Teich, 
{\it Fundamentals of Photonics}, druga izdaja, John Wiley \& Sons, Inc. (2007).}

Za zapis verjetnosti za prehod med dvema stanjema na časovno
enoto uporabimo Fermijevo zlato pravilo. Verjetnost, da je zgornje stanje zasedeno, je 
$f_p(E_b)$, verjetnost za zasedenost spodnjega stanja pa je $1-f_v(E_a)$. Verjetnost 
za prehod pri določenem valovnem vektorju je\index{Fermijevo zlato pravilo}
\begin{equation}  
w_s(k)=\frac{2\pi}{\hslash}|H_{pv}|^2\delta(E_b-E_a- \hslash\omega)
f_p(E_b)\left(1-f_v(E_a)\right)\!,
\label{6.5}
\end{equation}
pri čemer je $H_{pv}= \langle p| \hat{x}|v\rangle E $ matrični element za dipolni
prehod  med prevodnim in valenčnim pasom v svetlobnem polju $E$. Podobna je
verjetnost za absorpcijo 
\begin{equation}  
w_a(k)=\frac{2\pi}{\hslash}|H_{pv}|^2\delta(E_b-E_a- \hslash\omega)
f_v(E_a)\left(1-f_p(E_b)\right)\!.
\label{6.6}
\end{equation}
Upoštevamo enačbi~(\ref{pp:Ec}) in (\ref{pp:Ev}) in zapišemo razliko energij
\begin{equation}  
E_b-E_a= E_g + \frac{\hslash^2 k^2}{2}(\frac{1}{m_p}+ \frac{1}{m_v})= E_g + \frac{\hslash^2 k^2}{2m_r},
\label{6.8}
\end{equation}
pri čemer smo z $m_r=m_v m_p/(m_v+m_p)$ označili reducirano maso elektrona in vrzeli.

Število emisij oziroma absorpcij na enoto volumna v danem času izračunamo tako,
da verjetnosti za prehod integriramo po vseh $\mathbf{k}$. Razlika med številom 
emisij $N_{pv}$ in absorpcij $N_{vp}$ na enoto volumna je
\begin{align}  
N_{pv}-N_{vp}&=\int\left(w_s-w_a\right)\rho(k)\,dk  \nonumber \\
&=\frac{2}{\pi\hslash} \int |H_{pv}|^2\left(f_p(E_b)-f_v(E_a)\right)
\delta \left(\frac{\hslash^2 k^2}{2m_r}+E_g -\hslash\omega\right) k^2\,dk.
\label{6.7}
\end{align}
Upoštevali smo, da je gostota stanj $\rho(k)=k^2\, dk/\pi^2$. Vpeljemo
novo spremenljivko 
\beq
X = \frac{\hslash^2 k^2}{2m_r}+E_g -\hslash\omega
\eeq
in zapišemo integral
\beq
N_{pv}-N_{vp}=\frac{1}{\pi\hslash}\left(\frac{2m_r}{\hslash^2}\right)^{3/2}\!
\int |H_{pv}|^2 \left(f_p(E_b)-f_v(E_a)\right)
\sqrt{\left(X-E_g+\hslash \omega\right)}\,
\delta (X) dX.
\label{6.7a}
\eeq
Z upoštevanjem lastnosti funkcije $\delta$ dobimo
\begin{equation}  
N_{pv}-N_{vp}=\frac{1}{\pi\hslash}\left(\frac{2m_r}{\hslash^2}\right)^{3/2}
|H_{pv}|^2 \sqrt{\hslash \omega-E_g}\left(f_p(E_b)-f_v(E_a)\right)\!.
\label{6.11}
\end{equation}
Da se vpadna svetloba ojačuje namesto absorbira, mora biti število prehodov
iz prevodnega v valenčni pas torej večje od števila prehodov v obratno smer. 

Pogoj
$N_{pv}-N_{vp} >0$ je izpolnjen, kadar je razlika v oklepaju pozitivna. Vstavimo izraza
za verjetnost (enačbi~\ref{eq:fDppas}~in~\ref{eq:fDvpas}) in zapišemo
\begin{equation}  
\frac{1}{e^{(E_b-F_{p})/k_B T}+1}>\frac{1}{e^{(E_a-F_v)/k_B T}+1}.
\label{6.12}
\end{equation}
Upoštevamo zvezo $E_b-E_a = \hslash \omega$ in dobimo pogoj za ojačevanje
\boxeq{6.13}{  
E_g \leq \hslash\omega<F_{p}-F_{v}.
}
Energija fotonov, ki se v snovi ojačujejo, mora biti po pričakovanjih večja od
energije reže, sicer se niti ne absorbirajo niti ne ojačujejo. Pogoj za ojačevanje pove tudi, 
da za ojačenje svetlobe ne zadošča le nekaj vzbujenih elektronov in nekaj ustreznih vrzeli. 
V prevodnem pasu mora biti toliko elektronov, da pri neki energiji zasedejo vsaj polovico stanj, 
hkrati  mora biti v valenčnem pasu toliko vrzeli, da je vsaj polovica stanj nezasedena.
Ojača se torej le svetloba z energijo fotonov, ki je manjša od razlike med kvazi-Fermijevima
nivojema. 

Koeficient ojačenja vpadne svetlobe pri dani krožni
frekvenci $\omega$ je
\beq
\gamma(\omega) = K \sqrt{\hslash \omega - E_g}\left(f_p(E_b)-f_v(E_a)\right)\!,
\label{eq:gainSC}
\eeq
pri čemer smo konstante pospravili v sorazmernostni faktor $K$. Za ojačenje velja
\beq
dj = \gamma(\omega) j dz.
\eeq
Z naraščajočo stopnjo vzbujenosti 
koeficient ojačenja razumljivo narašča, manjša pa se z naraščajočo 
temperaturo. Njegovo odvisnost od energije vpadnih fotonov kaže slika~\ref{s6.11}. Vidimo,
da se z naraščajočo temperaturo zmanjšuje tudi širina pasu, v katerem se svetloba ojačuje.
\begin{figure}[ht]
\centering
\def\svgwidth{80truemm} 
\input{slike/07_gamma.pdf_tex}
\caption{Ojačenje v polprevodniku kot funkcija energije vpadnih fotonov. Črna črta
velja pri $T=0$ in rdeča pri $T>0$. Pri frekvencah, večjih od $(F_p-F_v)/\hslash$,
se svetloba absorbira. 
}
\label{s6.11}
\end{figure}

\subsection*{Spoj $\textbf{\textit{p}}$-$\textbf{\textit{n}}$}
\index{Spoj $p$-$n$}
Svetloba se v polprevodniku ojačuje le, če je na istem mestu v prevodnem pasu
dovolj veliko število elektronov in v valenčnem pasu dovolj vrzeli. Za delovanje
polprevodniškega laserja moramo torej s črpanjem vzdrževati neravnovesno stanje, 
podobno kot smo pri navadnih laserjih vzdrževali obrnjeno zasedenost. 
Neravnovesno stanje dosežemo tako, da v degeneriran pol\-pre\-vod\-nik tipa $p$, v
katerem je veliko vrzeli, dovolj hitro dodajamo elektrone v prevodni pas. To lahko storimo s spojem
$p$-$n$, na katerega priključimo napetost v prevodni smeri. \index{Črpanje}

Najpreprostejši primer spoja $p$-$n$ je spoj dveh kosov iste snovi, ki je
na eni strani dopirana z akceptorji ($p$) in na drugi z donorji ($n$). 
Ko staknemo območji $p$ in $n$, elektroni iz območja z višjo koncentracijo 
difundirajo v območje z nižjo koncentracijo in vrzeli ravno obratno. 
Ob spoju nastane v stacionarnem stanju ozek pas, tako imenovani izpraznjeni sloj, 
v katerem ni prostih nosilcev naboja. Na strani $n$ ostanejo pozitivni donorski ioni in 
na strani $p$ negativni akceptorski ioni, ki ustvarjajo električno polje. 
Nastalo polje preprečuje nadaljnjo difuzijo nosilcev naboja. 
V ravnovesju se Fermijeva energija na obeh straneh izenači, 
prevodni in valenčni pas pa se ukrivita (slika~\ref{fig:pnlaser}\,a).
\begin{figure}[ht]
\centering
\def\svgwidth{140truemm} 
\input{slike/07_pnlaser.pdf_tex}
\caption{Energijska pasova v močno dopiranem spoju $p$-$n$ (a) in 
ista pasova ob priključeni napetosti v prevodni smeri (b). $F_p$ in $F_v$
označujeta kvazi-Fermijevi energiji in senčeni del aktivno območje, 
v katerem se elektroni in vrzeli rekombinirajo.
}
\label{fig:pnlaser}
\vglue-3truemm
\end{figure}

Ko na spoj priključimo napetost v prevodni smeri (torej pozitivno napetost na stran
$p$), se potencialni skok zmanjša. Pri tem je potencialna razlika kar sorazmerna priključeni
napetosti $U$. Fermijevi energiji na straneh $p$ in $n$ se razmakneta in v
ozkem območju v bližini spoja pride do hkratne zasedenosti elektronov
v prevodnem pasu in vrzeli v valenčnem pasu (slika~\ref{fig:pnlaser}\,b). 
\index{Spoj $p$-$n$!{aktivno območje}}
To imenujemo aktivno območje, saj v njem prihaja do rekombinacij in do nastanka fotonov. 

Pri nizkih 
priključenih napetostih oziroma nizkih tokovih skozi spoj $p$-$n$ prihaja
do spontane rekombinacije in nizke izsevane moči svetlobe. 
Pri večjih napetostih, ko je $e_0U \approx E_g$, so koncentracije nosilcev naboja velike in
stimulirane rekombinacije omogočajo optično ojačenje. Takrat presežemo prag delovanja
diodnega laserja. \index{Prag delovanja laserja}
Ojačenje v polprevodniških laserjih je precej veliko, lahko več od 
$100~/\si{cm}$, zato je mogoče dobiti delujoč laser že v zelo majhnem aktivnem 
območju, velikem lahko le nekaj mikronov. Navadni polprevodniški laserji so tako 
dolgi $\sim~0,25~\si{mm}$.

\subsection*{Zgradba laserja}
Prvi polprevodniški laser je bil narejen iz galijevega arzenida (GaAs) in je oddajal svetlobo \index{GaAs}
pri $850~\si{nm}$.\footnote{~R. N. Hall 
et al., Phys. Rev. Lett. $\mathbf{9}$, 366 (1962).} 
Shema takega laserja je na sliki~\ref{fig:pnshema}. Svetloba se v njem ojačuje
v tanki aktivni plasti in iz laserja izhaja v vzdolžni smeri. 
Vidimo, da polprevodniški laser nima resonatorja iz dveh odbojnih zrcal,
ampak se svetloba odbija kar na gladko odklanih stranskih ploskvah kristala. Zaradi
velikega lomnega količnika (npr. $n=3,5$ za GaAs) je odbojnost dovolj velika
za učinkovito delovanje.\index{Laser!zgradba}
\begin{figure}[ht]
\centering
\def\svgwidth{140truemm} 
\input{slike/07_pnshema.pdf_tex}
\caption{Shema preprostega polprevodniškega laserja (a) in porazdelitev svetlobne
intenzitete na spoju $p$-$n$. Tipična širina je $100$--$200~\si{\micro\metre}$, 
dolžina $200$--$500~\si{\micro\metre}$ in debelina aktivne plasti  $\sim 1~\si{\micro\metre}$. 
Svetlobni profil v laserju (b). Svetloba 
se ojači v aktivnem območju (vijolična). V območjih $p$ in $n$ se svetloba absorbira.
}
\label{fig:pnshema}
\vglue-3truemm
\end{figure}

Opisani polprevodniški laserji imajo kar nekaj slabosti. So zelo močno dopirani, 
dobro delujejo le močno hlajeni (prvotno pri $77~\si{K}$), poleg tega je snop 
v takih laserjih pogosto širši od debeline aktivnega območja. To znaša
$\sim 1~\si{\micro\meter}$, širina snopa pa nekaj mikronov več, zato
znaten del svetlobe potuje po območju $p$ in $n$, kjer se svetloba absorbira in posledično
snov segreva. Tokovi, potrebni za 
delovanje takega laserja, so visoki ($\sim 100~\si{kA}/\si{cm}^2$ pri sobni temperaturi) in
kakovost snopa razmeroma slaba. 

Bistveno izboljšano delovanje je v tako imenovanih 
heterostrukturah.\footnote{~Za 
iznajdbo heterostruktur sta Zhores I. Alferov in Herbert Kroemer leta 2000 prejela Nobelovo nagrado.}
 To so dvojni spoji $p$-$n$, 
v katerih sta spojeni dve različni snovi, medtem ko se aktivno območje (tipično debelo le okoli $100~\si{nm}$) 
nahaja med polprevodnikoma tipa $p$ in $n$.\footnote{~H. Kroemer, Proc. IEEE $\mathbf{51}$, 1782 (1963).}
Najpomembnejši primer heterostruktur je tanka plast $p$-GaAs, ki se nahaja med plastema  \index{Heterostruktura}
$n$-Al$_x$Ga$_{1-x}$As in $p$-Al$_x$Ga$_{1-x}$As \index{AlGaAs}
(slika~\ref{fig:hetero}\,a). Elektroni v tem primeru tečejo iz tipa $n$ v prevodni pas aktivne plasti
GaAs in vrzeli iz tipa $p$ v valenčni pas. 
Tak laser deluje v območju od $750~\si{nm}$ do $880~\si{nm}$, odvisno od $x$ in koncentracije primesi.
Drug pomemben primer je laser z In$_{x}$Ga$_{1-x}$As$_{1-y}$P$_y$, ki seva svetlobo z valovno 
dolžino med $1,1~\si{\micro\metre}$ in $1,6~\si{\micro\metre}$. To ga naredi še posebej pomembnega za optične
komunikacije, v katerih se najpogosteje uporablja svetlobo pri $1,3~\si{\micro\meter}$ in $1,55~\si{\micro\meter}$. 
Prvi primer si oglejmo podrobneje.\index{Infrardeče valovanje}
\begin{figure}[ht]
\centering
\def\svgwidth{145truemm} 
\input{slike/07_hetero.pdf_tex}
\caption{Shema laserja z dvojno heterostrukturo (a) in oblika energijskih pasov 
na spoju pri napetosti, priključeni v prevodni smeri (b). V aktivnem območju
je število elektronov v prevodnem pasu in vrzeli v valenčnem pasu zelo veliko, zato se 
svetloba močno ojačuje.
}
\label{fig:hetero}
\end{figure}
\vglue-7truemm
\begin{remark}
Pri izdelavi heterostruktur je ključno najti snovi, ki imajo čim bolj podobno strukturo.
Take plastne strukture naredijo z epitaksialno rastjo, zato si morajo biti medatomske
razdalje različnih materialov čim bolj podobne. V nasprotnem primeru se pojavijo 
defekti, zaradi katerih naprava slabše deluje. 
GaAs in AlAs imata enako strukturo in praktično enako medatomsko razdaljo, zato 
lahko brez škode na strukturi atome galija zamenjamo z atomi aluminija. Tipična
vrednost $x$ v Al$_x$Ga$_{1-x}$As je okoli $0,3$. S spreminjanjem koncentracije aluminija $x$ hkrati
zvezno spreminjamo širino energijske reže ($E_g \approx 1,42 + 1,3x~\si{eV}$) in lomni količnik 
zlitine ($n \approx 3,5-0,71x$).
\end{remark}

Heterostruktura ima nekaj poglavitnih prednosti. Zlitina z aluminijem ima večjo 
energijsko režo od čistega GaAs, zato mejni plasti ustvarita potencialno bariero, 
ki preprečuje difuzijo nosilcev naboja iz aktivne plasti (slika~\ref{fig:hetero}\,b). Tako ostane koncentracija 
elektronov v prevodnem pasu in vrzeli v valenčnem pasu že pri razmeroma majhnih tokovih
velika in svetloba se ojačuje. Druga prednost je manjši lomni količnik zlitine, 
zaradi katerega ostane svetloba ujeta v aktivni plasti, podobno kot v valovnem 
vodniku. Ker manjši del svetlobe potuje izven aktivnega območja, je ojačenje večje,
poleg tega se zaradi večje energijske reže občutno zmanjša absorpcija izven aktivnega območja.
Skupaj z zmanjšano debelino aktivnega območja to vodi do znižanja tokov, 
potrebnih za delovanje laserja, za približno dva reda velikosti ($\sim 1~\si{kA}/\si{cm}^2$). 
Z uporabo heterostrukture se tudi izboljša izkoristek in zmanjša gretje.  

Pri strukturi, ki jo kaže slika~\ref{fig:hetero}, je aktivna plast v prečni smeri praktično neomejena, 
zato lahko hkrati sveti mnogo nihanj. Posledično je prečna koherenca snopa razmeroma slaba in 
delovanje laserja nestabilno. To slabost popravijo z jedkanjem plasti ob straneh, da ostane le
kakih $5~\si{\micro\meter}$ širok pas (slika~\ref{fig:heshema}). 
Odjedkani material nadomestijo s čistim Al$_x$Ga$_{1-x}$As, 
tako da je aktivno območje z vseh strani obdano s snovjo z večjim lomnim količnikom in laserski
snop ostane ujet v pravokoten valovni vodnik. Izhodni snop je zaradi oblike aktivne snovi eliptičen s 
premerom okoli $1~\si{\micro\meter}$ v navpični in $5~\si{\micro\meter}$ v prečni smeri. 
V večji oddaljenosti je divergenca snopa lahko tudi $\sim~50\si{\degree}$ v navpični in $\sim~5\si{\degree}$ 
v prečni smeri. Če potrebujemo radialno simetričen snop, ga moramo popraviti z ustreznimi 
lečami.
\begin{figure}[ht]
\centering
\def\svgwidth{95truemm} 
\input{slike/07_heshema.pdf_tex}
\caption{Poenostavljena shema laserja z dvojno heterostrukuro z dodatno prečno omejitvijo
}
\label{fig:heshema}
\end{figure}
\begin{remark}
Omenili smo, da polprevodniški laserji nimajo posebnih zrcal. Zaradi velikega lomnega
količnika polprevodnika se dovolj svetlobe odbije na mejni ploskvi. Vendar naloga resonatorja ni 
samo odboj svetlobe, ampak tudi izbira posameznega nihajnega načina. Ker tega z odbojem 
na ploskvi ne dosežemo, pogosto uporabimo periodične strukture, na katerih se svetloba selektivno
Braggovo odbije. Druga možnost je periodično spreminjajoča se debelina plasti, na kateri pride
do porazdeljene povratne sklopitve ({\it distributed feedback}). \index{Braggov odboj}\index{Porazdeljena
povratna sklopitev}
 
Večja odbojnost mejnih ploskev je nujna v laserjih, pri katerih svetloba ne izhaja iz
tanke aktivne plasti vzporedno z ravnino spoja $p$-$n$, ampak izhaja v smeri pravokotno na aktivno plast.
To so tako imenovani VCSEL ({\it Vertical Cavity Surface-Emitting 
Laser}).\footnote{~F. Koyama et al., Trans. IEICE ${\mathbf E71}$, 1089 (1988).} Ker je sevalna površina večja in 
ojačevalno sredstvo krajše, je ojačenje na prelet manjše in mora biti odbojnost resonatorja toliko večja. 
Kakovost snopa je v takih laserjih občutno boljša kot v navadnih diodnih laserjih, poleg tega ima izhodni 
snop manjšo divergenco.\index{Laser!VCSEL}
\end{remark}
\vglue-4truemm
\begin{remark}
Posebna vrsta polprevodniških laserjev so nizkodimenzionalni laserji. V njih je 
velikost aktivnega območja vsaj v eni smeri primerljiva z medatomsko razdaljo (tipično $\sim 10~\si{nm}$).
Najpreprostejši je laser s potencialno jamo ({\it quantum well laser}), \index{Laser!s potencialno jamo}ki 
je po svoji zgradbi zelo podoben heterostrukturnim (slika~\ref{fig:heshema}), le da so dimenzije precej
manjše. Zaradi majhne dimenzije se v taki snovi bistveno zmanjša gostota stanj. \index{Gostota stanj}
Za dosego zahtevane razlike med kvazi-Fermijevimi nivoji, s čemer dosežemo prag za delovanje laserja,
je zato potrebna precej manjša gostota nosilcev naboja in tok, potreben za delovanje laserja,
se zniža. 

Kvantne pike, ki sicer niso laserji, so kvantno omejene v vseh treh smereh.
V njih so stanja diskretna in pasov ni več. Ker je tehnološko zelo zahtevno izdelati povsem enake
kvantne pike, je spekter izsevane svetlobe iz skupine kvantnih pik zelo nehomogeno razširjen. 
\end{remark}

Diodni laserji so danes eni izmed najpomembnejših laserjev. Uporabljamo jih v vsakodnevnih napravah 
(tiskalnikih, optičnih bralnikih), v medicini, industriji, izrednega
pomena so za tele\-komunikacije, pogosto jih uporabljamo tudi za 
črpanje trdninskih ali vlakenskih laserjev. 
Poleg že naštetih prednosti jih odlikuje tudi zelo dolg življenjski čas ($\sim~50\,\,000$~ur). 

\subsection{Svetleče diode -- LED}
\index{Svetleče diode}\index{LED|see {Svetleče diode}}
Svetleče diode so danes eden izmed najbolj razširjenih izvorov svetlobe in nadomeščajo
navadne žarnice ali svetilke. Oddajajo svetlobo od ultravijolične do infrardeče, 
odlikujeta jih \index{Ultravijolično valovanje}
dober izkoristek (nad $25~\%$) in dolga življenjska doba ($\sim~50\,\,000$~ur).

Svetleče diode delujejo podobno kot polprevodniški laserji. Na spoju $p$-$n$ prihaja do 
spontane emisije fotonov zaradi rekombinacij elektronov in vrzeli. Za razliko od laserjev 
svetleče diode nimajo praga delovanja, ampak začnejo svetiti, čim je na spoj priključena 
napetost v prevodni smeri. Zaradi spontane narave prehodov je svetloba, ki jo oddajajo
svetleče diode, nekoherentna in nepolarizirana. \index{Spoj $p$-$n$}

Prve svetleče diode, ki so oddajale svetlobo v vidnem delu spektra,
so bile izdelane iz GaAsP\index{GaAsP} in so oddajale 
svetlobo pri $710~\si{nm}$.\footnote{~N. Holonyak Jr. in S. F. Bevacqua, Appl. Phys. Lett.
${\mathbf 1}$, 82 (1962).} Pomemben
napredek je bila izdelava svetlečih diod, ki so oddajale svetlobo v modrem delu 
spektra.\footnote{~Za iznajdbo modrih LED so Isamu Akasaki, Hiroshi Amano in 
Shuji Nakamura leta 2014 prejeli Nobelovo nagrado.} Te so sestavljene iz dvojne
heterostrukture, v kateri je plast InGaN, dopiranega s cinkom, 
med plastema $p$-AlGaN in $n$-AlGaN.

Belo svetlobo iz LED svetil lahko dobimo z mešanjem svetlobe iz treh različnih 
diod: rdeče, zelene in modre. Primernejši je pristop z uporabo ene same diode, 
ki oddaja svetlobo v modrem delu spektra. Tako diodo prevlečemo z molekulami fosforja, ki
modro svetlobo absorbira in spontano izseva svetlobo pri nižjih frekvencah 
oziroma daljših valovnih dolžinah. Z ustrezno izbiro parametrov dobimo belo svetlobo
ceneje in preprosteje kot s tremi ločenimi diodami. 

\begin{remark}
Posebna vrsta svetlečih diod so organske svetleče diode (OLED).\footnote{~C. W. Tang in
S. A. VanSlyke, Appl. Phys. Lett. $\mathbf{51}$, 913 (1987).} Te so narejene iz
organskih snovi, ki se obnašajo kot organski polprevodniki. S spreminjanjem kemijske
sestave se spreminja valovno dolžino, ki jo LED oddaja. Njihova poglavitna prednost
je preprosta izdelava, saj je proizvodnja organskih spojin preprostejša in cenejša
od rasti kristalov. \index{Svetleče diode!OLED}
\end{remark}

%Končano

%-------------------------------------------------------------------------------
%	CHAPTER 8
%-------------------------------------------------------------------------------

\chapterimage{slike/Fibri.jpg} % Chapter heading image

\chapter{Optična vlakna}
\label{chap:fibri}
Moderna komunikacijska tehnologija zahteva vedno hitrejši prenos
vedno večje količine podatkov. Navadne kovinske vodnike
so zato v računalniških in telekomunikacijskih povezavah nadomestila optična 
vlakna, ki jih odlikujejo majhne izgube, neobčutljivost na elektromagnetne
in medsebojne motnje ter zmožnost prenosa izjemno velike količine podatkov
na dolge razdalje. V tem poglavju bomo opisali mehanizme prenosa podatkov 
po optičnih vodnikih in spoznali omejitve pri prenosu,
predvsem disperzijo in izgube, ter načine, kako se z njimi spopadamo.

\section{Planparalelni vodnik}
\subsection*{Geometrijski opis}
\index{Optični vodnik}
\index{Optično vlakno}
\index{Optični vodnik!planparalelni}
Klasično pojasnimo delovanje optičnih vlaken s totalnim odbojem~\index{Totalni odboj}
na meji med dvema dielektrikoma. Kadar prehaja svetloba iz snovi 
z večjim lomnim količnikom\index{Lomni količnik} $n_1$ v sredstvo z manjšim lomnim količnikom $n_2$, se pri vpadnih kotih $\vartheta$, ki so večji od mejnega 
kota $\vartheta_T$, totalno odbije. Za mejni kot velja 
(enačba~\ref{eq:totalniodbojkot}): $\sin\vartheta_{T}=n_{2}/n_{1}$.

Najpreprostejši optični vodnik je planparalelna plast 
dielektrika, ki je obdana s snovjo z manjšim lomnim količnikom (slika~\ref{fig:vodnik}). 
Plasti z večjim lomnim količnikom ($n_1$) rečemo sredica\index{Optični vodnik!sredica} in 
okoliški snovi z lomnim količnikom $n_2<n_1$ plašč\index{Optični vodnik!plašč}. 
Žarek potuje po vodniku, če je vpadni kot 
na mejo med sredico in plaščem večji od mejnega kota totalnega odboja $\vartheta_T$.

\begin{figure}[ht]
\centering
\def\svgwidth{100truemm} 
\input{slike/10_Vodnik.pdf_tex}
\caption{Klasična razlaga razširjanja svetlobe po valovnem vodniku. Z modro je označena sredica
vodnika z lomnim količnikom $n_1$ in z zeleno plašč z lomnim količnikom $n_2$. 
Žarek se na meji med sredico in plaščem totalno odbija.}
\label{fig:vodnik}
\vglue-3truemm
\end{figure}
Obstaja največji vpadni kot $\alpha_{\rm max}$, pri katerem se
vpadna svetloba ujame v vlakno.
Z njim povezana je numerična odprtina (apertura) vlakna\index{Numerična odprtina} $NA$, 
ki jo izračunamo kot 
\begin{equation}
NA = \sin \alpha_{\rm max} = n_1 \sin \beta_{\rm max} = 
n_1 \sin(\pi/2-\vartheta_T) =
n_1 \cos\vartheta_T = n_1 \sqrt{1-\sin^2\vartheta_T}.
\label{10NAa}
\end{equation}
Upoštevajoč lomni zakon numerično odprtino zapišemo kot 
\boxeq{10NA}{
NA = \sqrt{n_1^2-n_2^2}.
}
Razlika med lomnima količnikoma sredice in plašča je navadno majhna,
tipično le nekaj stotink, zato je numerična odprtina optičnih 
vodnikov $NA \lesssim 0,1 $. Kot, pod katerim lahko vpada svetloba
v vodnik (ali vlakno), da se vanj ujame, je tako navadno le nekaj stopinj.

\subsection*{Valovni opis}
Za natančen opis širjenja svetlobe po vodnikih ali vlaknih\footnote{~Dogovorimo se, da  
besedo vlakno uporabljamo za cilindrične strukture in besedo vodnik za planparalelne 
in njim podobne strukture.}, ki imajo debelino ali premer
sredice od nekaj mikrometrov do nekaj deset mikrometrov, geometrijska optika ne
zadošča. Rešiti moramo Maxwellove enačbe (enačbe~\ref{eq:Maxwell1}--\ref{eq:Maxwell4}) 
z ustreznimi robnimi pogoji (enačbe~\ref{eq:robni-pogoji} in \ref{eq:robni-pogoji5}),
kar je za cilindrična vlakna dokaj dolg in zapleten račun. Zato določimo najprej 
osnovne značilnosti valovanja, ki se širi po planparalelnem vodniku. Vodnik naj bo omejen 
v smeri $x$, svetloba pa naj potuje v smeri $z$.

Glede na smer polarizacije jakosti električnega polja
ločimo dva primera (slika~\ref{fig:TETM}). Če je smer jakosti električnega polja
vzporedna z mejnima ploskvama (smer $y$), 
govorimo o transverzalnem električnem (TE) valovanju\index{Polarizacija!TE}. 
V nasprotnem primeru, ko je 
z mejnima ploskvama vzporedna jakost magnetnega polja in 
leži jakost električnega polja v ravnini $xz$, 
govorimo o transverzalnem magnetnem (TM) valovanju\index{Polarizacija!TM}.
\begin{figure}[ht]
\centering
\def\svgwidth{140truemm} 
\input{slike/10_TETM.pdf_tex}
\caption{Polarizacija TE valovanja (a) in polarizacija TM valovanja (b) v valovnem vodniku}
\label{fig:TETM}
\end{figure}

Geometrijskemu žarku, ki se odbija na mejah stranice, ustreza v 
valovni sliki val, ki ima prečno komponento valovnega
vektorja\index{Valovni vektor} $k_{x}$ različno od nič. Ker je valovanje v prečni smeri 
omejeno na sredico končne debeline (naj bo debelina sredice enaka $a$), lahko
$k_{x}$ zavzame le diskretne vrednosti. Te so v grobem približno enake 
$N\pi/a$, pri čemer je $N$
celo število. Pravimo, da vsak $N$ določa en rod valovanj v vodniku. Po drugi strani 
obstaja največji $k_x$, za katerega približno velja
\begin{equation}
k_{x \mathrm{max}} \approx k_0 \sin\alpha_{\rm max} = k_0 \sqrt{n_1^2 -n_2^2}.
\end{equation}
Število rešitev za $k_x$ je tako omejeno, točno določeno in odvisno
od razlike lomnih količnikov in od debeline vodnika oziroma polmera vlakna. 
V nadaljevanju bomo spoznali, da v optičnih vlaknih en rod vselej obstaja,
za razliko od kovinskih vodnikov, po katerih se pod določeno frekvenco
valovanje ne more širiti. Enorodovna optična vlakna, torej vlakna, po katerih se širi
en sam rod, imajo še posebej lepe lastnosti za uporabo v komunikacijskih
sistemih\index{Optični vodnik!enorodovni}\index{Optično vlakno!enorodovno}.

Povejmo še nekaj o hitrosti valovanja v vodniku.
Naj bo $\beta$ velikost komponente valovnega vektorja vzdolž smeri $z$ in odvisnost 
polja od koordinate vzdolž vodnika $\exp (i\beta z)$. Velikost
valovnega vektorja v sredici vodnika zapišemo kot
\begin{equation}
k_1 = n_{1}\frac{\omega}{c_0}=\sqrt{\beta^{2}+k_{x}^{2}}
\label{9.0}.
\end{equation}
Za dano vrednost $k_{x}$ zveza med valovnim številom $\beta$
in krožno frekvenco $\omega$ torej 
ni linearna. Fazna hitrost\index{Hitrost valovanja!fazna} 
valovanja $v_{f}=\omega/\beta$ je tako
odvisna od krožne frekvence in nastopi disperzija\index{Disperzija}. Grupna ali skupinska
hitrost\index{Hitrost valovanja!grupna} $v_{g}=d\omega/d\beta$ 
se zaradi nelinearne zveze med $\beta$ in $\omega$ 
razlikuje od fazne hitrosti in njena frekvenčna odvisnost 
ima pomembne posledice za uporabo vlaken pri prenosu podatkov. 

\section{Račun lastnih rodov v planparalelnem vodniku}
\index{Optični vodnik!lastni rodovi}
\index{Optični vodnik!planparalelni}
Poiščimo rešitve valovne enačbe v planparalelnem vodniku. 
To je preprost dvodimenzionalen model optičnega vlakna, ki je sestavljen iz 
plasti prozornega dielektrika z lomnim količnikom $n_1$ in plašča (ali okolice) z lomnim količnikom $n_2$.
Zaradi enostavnosti privzamemo, da je plašč na obeh straneh sredice enak.
Sredica naj bo debela $a$, izhodišče koordinatnega sistema
si izberemo na sredini plasti. Ločimo tri območja, kjer rešujemo valovno enačbo:
območje II označuje sredico, območji I in III naj bosta v plašču nad sredico oziroma pod njo
(slika~\ref{fig:vodnikracun}).

\begin{figure}[ht]
\centering
\def\svgwidth{120truemm} 
\input{slike/10_VodnikRacun.pdf_tex}
\caption{K izračunu lastnih rodov v simetričnem planparalelnem vodniku}
\label{fig:vodnikracun}
\end{figure}

Krajevni del valovne enačbe, ki jo rešujemo, opisuje Helmholtzeva 
enačba\index{Helmholtzeva enačba} (enačba~\ref{eq:Helmholtz})
\begin{equation}
\nabla^{2}\mathbf{E}+n^2(x)\,k_{0}^{2}\,\mathbf{E}=0,
\label{9.1}
\end{equation}
pri čemer je $k_{0}=\omega/c_0$. Lomni količnik $n(x)$  nezvezno spremeni vrednost ob prehodu iz sredice v plašč. 

Nastavek za rešitev naj bo oblike 
\begin{equation}
{\mathbf E}(x,z)=\mathbf{e}\psi\left(x\right)\, e^{i\beta z},
\label{9.2}
\end{equation}
pri čemer $\mathbf{e}$ označuje enotski vektor v smeri polarizacije valovanja.
Omejimo se le na primer TE polarizacije\index{Optični vodnik!TE rodovi} (za izračun lastnih rodov 
TM polariziranega
valovanja glej nalogo~\ref{naloga:TM}). Vstavimo nastavek (enačba~\ref{9.2}) v enačbo
(\ref{9.1}) in dobimo
\begin{equation}
\frac{d^{2}{\bf \psi}}{dx^{2}}+\left(k_{0}^{2}n_1^{2}-\beta^{2}\right){\bf \psi}=0
\qquad \textrm{v sredici (obmo\v cje II)} 
\label{9.3a}
\end{equation}
in 
\begin{equation}
\frac{d^{2}{\bf \psi}}{dx^{2}}+\left(k_{0}^{2}n_2^{2}-\beta^{2}\right){\bf \psi}=0
\qquad \textrm{v plašču (obmo\v cji I in III).} 
\label{9.3b}
\end{equation}
Iz enačbe~(\ref{9.0}) sledi $k_0^2n_1^2-\beta^2=k_x^2$, zato lahko rešitve prve enačbe
zapišemo v obliki
\begin{equation}
\psi_{\mathrm{II}}(x) = C \cos(k_x x)+D \sin(k_x x),
\end{equation}
rešitve v plašču pa so oblike
\begin{equation}
\psi_{\mathrm{I}}(x) = A \exp(-\kappa x)+B \exp(\kappa x)\quad \mathrm{in}\quad
\psi_{\mathrm{III}}(x) = F \exp(-\kappa x)+G \exp(\kappa x),
\end{equation}
pri čemer smo vpeljali $\kappa^2= \beta^2-k_0^2n_2^2$.
Da valovanje ostane ujeto v vlakno, mora biti $\kappa$ realno število.
Le tako namreč dosežemo eksponentno pojemanje jakosti električnega polja 
z oddaljenostjo od sredice,
sicer je valovanje v vseh treh območjih oscilatorno in ni ujeto v vlakno. 

Iz zahteve, da sta $k_x$ in $\kappa$ realna, sledi pogoj za valovno 
število\index{Valovno število} $\beta$
\boxeq{vlaknobeta}{
k_0 n_2 < \beta < k_0 n_1.
}

Poleg tega zahteva po končnosti rešitve da pogoj, da je v območju I 
(pri $x>a/2$) koeficient $B=0$, 
v območju III (pri $x<-a/2$) pa $F=0$. Hitro ugotovimo, da so zaradi 
simetrije vlakna lastne rešitve lahko le sode ali lihe funkcije. 

\subsection*{Sode rešitve}
\index{Optični vodnik!sodi rodovi}
Poglejmo najprej sode rešitve. V sredici je od nič  različen samo $C$, 
v območjih I in III pa bosta amplitudi enaki in $A = G$ (slika~\ref{fig:TESodi}\,a). 
Dobimo 
\begin{align}
\psi_{\mathrm{I}}(x) =&~ A \exp(-\kappa x), \\
\psi_{\mathrm{II}}(x) =&~ C \cos(k_x x) \qquad \mathrm{in}\\
\psi_{\mathrm{III}}(x) =&~ A \exp(\kappa x).
\end{align}
Zvezo med koeficientoma $A$ in $C$ določimo z upoštevanjem robnih pogojev. Na meji
med sredico in plaščem morata biti tangencialni komponenti 
jakosti električnega in magnetnega polja zvezni (enačbi~\ref{eq:robni-pogoji5}). 
Iz tega izpeljemo pogoj, da se za TE valovanje
na meji ohranja amplituda jakosti električnega polja. Pri $x = a/2$ zapišemo
\begin{equation}
A \exp(-\kappa a/2) = C \cos(k_x a/2).
\label{eq:rps1}
\end{equation}
Drugi pogoj izpeljemo iz Maxwellove enačbe~(enačba~\ref{eq:Maxwell2}). Ob upoštevanju
časovne odvisnosti polja dobimo zvezo $\nabla\times{\bf E}=i\omega\mu_{0}{\bf H}$. 
Ker se na meji ohranja
tangencialna komponenta ${\bf H}$, to je v tem primeru $H_z$, se posledično ohranja 
odvod jakosti električnega polja $dE_y/dx$. 
Pri $x = a/2$  velja
\begin{equation}
-A \kappa \exp(-\kappa a/2) = -C k_x \sin(k_x a/2).
\label{eq:rps2}
\end{equation}
Enačbo za $k_x$ izpeljemo iz zahteve, da sta robna 
pogoja  (enačbi~\ref{eq:rps1} in \ref{eq:rps2}) hkrati izpolnjena. Izraza za robna pogoja delimo, tako da
se amplitude pokrajšajo, in dobimo sekularno 
enačbo za sode rešitve \index{Sekularna enačba!sodi rodovi}
\boxeq{sekular1}{
\frac{\kappa}{k_x} = \tan \frac{k_x a}{2}.
}
Rešitve enačbe so diskretne in vsaki vrednosti $k_x$ ustreza 
en sodi rod oziroma sodi lastni način. Pri tem je zveza med $\kappa$ in $k_x$  
\boxeq{kappak}{
\kappa^{2}+ k_x^{2}=k_{0}^{2}\left(n_{1}^{2}-n_{2}^{2}\right)\!.
}

\subsection*{Lihe rešitve}
\index{Optični vodnik!lihi rodovi}
Oglejmo si še lihe rešitve v planparalelnem vodniku. V sredici je od nič različen
le $D$, polji v plašču  sta nasprotno enaki in $A = -G$ (slika~\ref{fig:TESodi}\,b). Sledi
\begin{align}
\psi_{\mathrm{I}}(x) =&~ A \exp(-\kappa x),\\
\psi_{\mathrm{II}}(x) =&~ D \sin(k_x x) \qquad \mathrm{in}\\
\psi_{\mathrm{III}}(x) =&~ -A \exp(\kappa x).
\end{align}
Z upoštevanjem zveznosti jakosti električnega polja in njenega odvoda na 
meji med sredico in plaščem zapišemo robna pogoja pri $x=a/2$
\begin{equation}
A \exp(-\kappa a/2) = D \sin(k_x a/2)
\end{equation}
in 
\begin{equation}
-\kappa A \exp(-\kappa a/2) = D k_x \cos(k_x a/2).
\end{equation}
Ustrezna sekularna enačba za lihe rešitve je\index{Sekularna enačba!lihi rodovi}
 \boxeq{sekular2}{
-\frac{k_x}{\kappa} = \tan \frac{k_x a}{2}.
}
\begin{figure}[ht]
\centering
\def\svgwidth{130truemm} 
\input{slike/10_TESodi.pdf_tex} 
\caption{Prečna odvisnost jakosti električnega polja za sode (a) in lihe (b) rodove v 
simetričnem planparalelnem valovnem vodniku. Modra barva označuje sredico, beli del 
pa plašč vodnika. 
}
\label{fig:TESodi}
\end{figure}
\begin{remark}
Če ne prej, je bralec ob slikah~\ref{fig:TESodi} zagotovo opazil podobnost s kvantnim 
delcem, ujetim v končni enodimenzionalni potencialni jami. Svetloba, ujeta v vodnik ali
vlakno, ustreza vezanim stanjem delca, numerična odprtina pa je tisti parameter, 
ki določa globino potencialne jame. Pri majhnih vrednostih obstaja ena sama rešitev 
za vezano stanje, pri globlji jami je rešitev več. Podobno kot v kvantni mehaniki
tudi v tem primeru ena rešitev za vezano stanje vedno obstaja.\footnote{~Glej npr.
J. Strnad, {\it Fizika, 3. del}, tretja izdaja, DMFA-založništvo (2018).} 
\end{remark}

Sekularnih enačb za lastne rodove (enačbi~\ref{sekular1} in \ref{sekular2}) ne moremo rešiti 
analitično. Rešujemo jih numerično, zelo nazorna je tudi grafična predstavitev
(slika~\ref{fig:TEsec}). Enačbo za sode rodove~(enačba~\ref{sekular1}) pomnožimo s 
$k_xa$ in narišemo funkcijo $k_xa\, \tan (k_xa/2)$ (črne črte). Enačbo za lihe 
rodove~(enačba~\ref{sekular2}) preoblikujemo in narišemo $-k_xa\, \cot (k_xa/2)$ (rdeče črte).
Nato pri danih parametrih $n_1$, $n_2$, $a$ in $k_0$ narišemo krožnico za $\kappa a$, 
ki sledi iz enačbe~(\ref{kappak})
\begin{equation}
 (\kappa a)^2+ (k_x a)^{2}=k_{0}^{2} a^2\left(n_{1}^{2}-n_{2}^{2}\right) = k_0^2\,a^2\,NA^2.
\end{equation}
Število presečišč krožnice s krivuljami določa število lastnih rodov v vodniku in lega presečišča
pripadajočo vrednost $k_x$. Na sliki so narisane tri krožnice za 
tri različne debeline vodnika $a$ (pri istih lomnih količnikih in isti valovni 
dolžini svetlobe). V najtanjšem vodniku (zelena črta) je presečišče
le eno in tak vodnik imenujemo enorodovni vodnik\index{Optični vodnik!enorodovni}. 
Z večanjem debeline število presečišč -- in s tem število lastnih rodov -- narašča in taki vodniki so 
večrodovni\index{Optični vodnik!večrodovni}. Lastni rodovi v večrodovnem vodniku
so izmenično sodi in lihi, pri čemer je osnovni rod vedno sod. 
\begin{figure}[ht]
\centering
\def\svgwidth{60truemm} 
\input{slike/10_TEsekularna_nova.pdf_tex}
\caption{K izračunu $k_x$ v valovnem vodniku
za TE polarizacijo. Število presečišč krožnice s krivuljami določa število lastnih rodov 
v vodniku. Zelena krožnica predstavlja enorodovni vodnik, turkizna dvorodovnega in modra
petrodovnega s tremi sodimi in dvema lihima rešitvama.}
\label{fig:TEsec}
\end{figure}

S slike~\ref{fig:TEsec} razberemo še eno pomembno lastnost vodnikov. 
Ne glede na to, kako majhen je polmer krožnice, krožnica vedno seka črno krivuljo. 
To pomeni, da v še tako tankem vodniku vsaj ena rešitev za $k_x$ vedno obstaja
in ta je vedno soda. 

Ocenimo še število lastnih rodov v vodniku. Iz slike~\ref{fig:TEsec} razberemo, da je 
največja možna vrednost $k_x$ omejena s polmerom krožnice $k_0aNA$, ki ga imenujemo
tudi normirana frekvenca\index{Normirana frekvenca} $V$. Do te vrednosti je 
po ena rešitev na vsak interval dolžine $\pi$, izmenično soda in liha, 
zato je celotno število rodov za eno polarizacijo\index{Optični vodnik!število rodov}
\begin{equation}
N \approx \frac{k_0 a NA}{\pi} = \frac{V}{\pi}.
\end{equation}
Za enorodovni vodnik velja $V < \pi$ oziroma $a< \lambda/2 NA$.
Tipični enorodovni vodnik ima tako debelino $a\lesssim 5~\si{\micro\meter}$, medtem ko je
debelina večrodovnega vodnika z okoli 20 rodovi $a\sim 50~\si{\micro\meter}$.

\begin{definition}
\label{naloga:TM}
Ponovi izračun za TM valovanje\index{Optični vodnik!TM rodovi}\index{Sekularna enačba!TM rodovi}
in pokaži, da sta sekularni enačbi enaki 
\begin{equation}
\frac{\kappa}{k_x} \left(\frac{n_1}{n_2}\right)^2= 
\tan \frac{k_x a}{2} \qquad \mathrm{in} \qquad -\frac{k_x}{\kappa} \left(\frac{n_2}{n_1}\right)^2= 
\tan \frac{k_x a}{2}.
\end{equation}
Namig: zapiši enačbe za jakost magnetnega polja ${\bf H}$ in upoštevaj ustrezne robne pogoje.
\end{definition}

Podoben račun lahko naredimo tudi za TM valovanje (glej nalogo~\ref{naloga:TM}). Zaradi drugačnih
robnih pogojev se sekularni enačbi razlikujeta od tistih za TE polarizacijo. Razlika je v
faktorju $(n_1/n_2)^2$, ki je v tipičnem vodniku zelo blizu ena. Zato se tudi rešitve 
za prečno komponento $k_x$ le malo razlikujejo. Pomembnejša je ugotovitev, da je število
dovoljenih rodov za TM polarizacijo enako številu dovoljenih rodov za TE polarizacijo, 
saj je največji $k_x$ v obeh primerih določen s polmerom krožnice $V$. 
Vse lastne rodove, ki se širijo v danem planparalelnem vodniku, torej zajamemo z opisom sodih
in lihih TE ter sodih in lihih TM rodov.

Ugotovili smo, da je jakost električnega polja tudi izven sredice vodnika različna od nič. 
Poglejmo še, kako je z energijskim tokom. Čeprav se velika večina pretaka po sredici, 
delež, ki se pretaka po plašču, ni vedno zanemarljiv. To posebej velja za višje rodove. 
Delež energijskega toka, ki se pretaka po sredici, izračunamo\index{Gostota energijskega toka}
z integralom
\boxeq{confinement}{
\Gamma = \frac{\int_{-a/2}^{a/2}j\, dS}{\int_{-\infty}^{\infty}j\, dS}.
}
\begin{definition}
Pokaži, da je razmerje med energijskim tokom $P_p$, ki se pretaka po plašču, in energijskim tokom $P_s$, 
ki se pretaka po sredici vodnika, za sode rodove
\begin{equation}
\frac{P_p}{P_s}= \frac{n_2}{n_1}\frac{2 k_x}{\kappa} \frac{\cos^2(k_x a/2)}{k_xa + \sin(k_xa)}
\end{equation}
in za lihe rodove
\begin{equation}
\frac{P_p}{P_s}= \frac{n_2}{n_1}\frac{2 k_x}{\kappa} \frac{\sin^2(k_x a/2)}{k_xa - \sin(k_xa)}.
\end{equation}
\end{definition}

\section{Cilindrično vlakno}
\label{chap:Cilinder}
\index{Optično vlakno}
V praksi svetlobo navadno usmerjamo po optičnih vlaknih, ki imajo cilindrično 
geometrijo.\footnote{~Za doprinos k razvoju in uporabi optičnih vlaken je leta 2009 Charles
K. Kao prejel Nobelovo nagrado.}
Najpreprostejši primer je cilindrično vlakno, v katerem je lomni količnik 
sredice konstanten in nekoliko večji od lomnega količnika plašča. Navadno je 
$n_1 - n_2 \sim 0,001$. Pogosto se uporablja
bolj zapletene konstrukcije, pri katerih se lomni količnik sredice spreminja z
oddaljenostjo od osi. Z zapletenejšo geometrijo namreč zmanjšamo disperzijo v vlaknu in s tem
povečamo zmogljivost prenašanja velike količine podatkov na dolge razdalje -- 
najzmogljivejša vlakna zmorejo prenos več deset terabitov na 
sekundo.\footnote{~D. Hillerkuss et al., Nat. Phot. $\mathbf{5}$, 364 (2011).}

Račun za širjenje svetlobe po cilindričnem vlaknu s homogeno sredico
je podoben kot za planparalelni vodnik, vendar je precej bolj
zapleten. V cilindrični geometriji namreč ni delitve na čiste električne in 
magnetne transverzalne valove, saj so robni pogoji sklopljeni: polje, ki na enem delu
vlakna kaže v radialni smeri, kaže na drugem delu v tangencialni. Na splošno se rešitve izražajo 
v obliki kombinacij Besslovih funkcij. Izkaže se, da je osnovni rod, ki se  širi po
cilindričnem vlaknu, po obliki zelo podoben osnovnemu Gaussovemu snopu, zato je sklopitev
laserskih snopov v optična vlakna zelo učinkovita. Tudi v cilindričnih vlaknih 
obstaja končno število lastnih rodov, ki je odvisno od polmera sredice in
lomnih količnikov sredice in plašča. Če je polmer zadosti majhen (razlika lomnih
količnikov navadno je), obstaja le en lastni rod in optično vlakno je 
enorodovno.\index{Optično vlakno!enorodovno} Sicer je vlakno večrodovno\index{Optično 
vlakno!večrodovno}.

\subsection*{Valovna enačba v cilindričnem vlaknu}
\index{Valovna enačba}
Izračun rodov v cilindričnem vlaknu presega okvir te knjige, zato
si oglejmo le izhodiščne enačbe in rešitve.\footnote{~Za račun glej npr. A. Yariv in 
P. Yeh, {\it Photonics}, šesta izdaja, Oxford
University Press (2007).} Za jakost električnega polja velja 
Helmholtzeva enačba~(enačba~\ref{eq:Helmholtz})\index{Helmholtzeva enačba}
\begin{equation}
\nabla^2 \mathbf{E} + n^2(r)\, k_0^2\, \mathbf{E} = 0,
\end{equation}
pri čemer je $n(r<a)=n_1$ lomni količnik sredice in $n(r>a)=n_2$ 
lomni količnik plašča, ki je dovolj debel, da njegova debelina ne 
vpliva na potovanje svetlobe.
\begin{figure}[ht]
\centering
\def\svgwidth{50truemm} 
\input{slike/10_CilinderRacun.pdf_tex}
\caption{K izračunu lastnih rodov v cilindričnem vodniku}
\label{fig:cilindricnivodnik}
\vglue-5truemm
\end{figure}

$\mathbf{E}$ in $\mathbf{H}$ sta vektorja in imata po 
tri komponente, ki pa so med seboj odvisne. Računamo v cilindričnem koordinatnem sistemu,
v katerem je os $z$ vzdolž osi vlakna. Izračunajmo naprej $E_z$, saj so za to
komponento robni pogoji nesklopljeni. Uporabimo nastavek
\begin{equation}
E_z = R(r)e^{i \nu \varphi}e^{i \beta z},
\end{equation}
pri čemer so $r$ razdalja od osi, $\varphi$ polarni kot in $\nu$ celo število zaradi 
zahteve po enoličnosti rešitve pri spremembi
kota za $2\pi$. Za $R(r)$ v sredici vlakna tako dobimo enačbo
\begin{equation}
r^2 R(r)'' + r R(r)' + (k_s^2r^2 - \nu^2)R(r) = 0,
\label{10BS}
\end{equation}
pri čemer za prečno komponento valovnega vektorja velja
\begin{equation}
k_s^2=k_0^2n_1^2- \beta^2.
\label{eq:ks}
\end{equation}
V plašču velja enačba
\begin{equation}
r^2 R(r)'' + r R(r)' + (-\kappa^2r^2 - \nu^2)R(r) = 0,
\label{10BP}
\end{equation}
pri čemer je 
\begin{equation}
\kappa^2=\beta^2-k_0^2n_2^2.
\end{equation}
V enačbah (\ref{10BS}) in (\ref{10BP}) prepoznamo Besslovo diferencialno enačbo. 
Upoštevajoč le funkcije, ki na izbranem območju ne divergirajo, zapišemo rešitve v sredici kot
\begin{equation}
E_z (r, \varphi, z) = A J_\nu(k_sr)\sin(\nu \varphi)e^{i \beta z} \qquad  \mathrm{in} \qquad 
H_z (r, \varphi, z) = B J_\nu(k_sr)\cos(\nu \varphi)e^{i \beta z}.
\end{equation}
Podobno zapišemo tudi rešitve v plašču
\begin{equation}
E_z (r, \varphi, z)= C K_\nu(\kappa r)\sin(\nu \varphi)e^{i \beta z} \qquad \mathrm{in} \qquad 
H_z (r, \varphi, z)= D K_\nu(\kappa r)\cos(\nu \varphi)e^{i \beta z}.
\end{equation}
Pri tem so $A,B,C$ in $D$ konstante, $J_\nu(x)$ je Besslova funkcija prve vrste reda 
$\nu$ in $K_\nu(x)$ modificirana Besslova funkcija druge vrste reda $\nu$ 
(slika~\ref{fig:J01}). 
\begin{figure}[ht]
\centering
\def\svgwidth{140truemm} 
\input{slike/10_Bessel1.pdf_tex} 
\caption{Besslove funkcije: (a) Besslove funkcije prve vrste 
$J_0(x)$ (črna), $J_1(x)$ (rdeča) in $J_2(x)$ (modra), 
ki predstavljajo oblike rešitev v sredici vlakna, in (b)
modificirane Besslove funkcije druge vrste $K_0(x)$ (črna), $K_1(x)$ (rdeča) in $K_2(x)$ (modra), 
ki prestavljajo rešitev v plašču vlakna.}
\label{fig:J01}
\vglue-3truemm
\end{figure}

Ko enkrat poznamo komponenti $E_z$ in $H_z$, lahko z uporabo Maxwellovih enačb izračunamo še 
preostale komponente $E_r$, $E_\varphi$, $H_r$ in $H_\varphi$. 
Nato z upoštevanjem robnih pogojev zapišemo štiri enačbe za 
pet neznank ($A,B,C,D$ in $\beta$),
tako da ostane ena spremenljivka (amplituda polja) prosta. Tako izračunamo celotni 
jakosti električnega in magnetnega polja v vlaknu in podobno kot pri valovnem vodniku 
 zapišemo sekularno enačbo, ki jo moramo rešiti numerično. Pri vsakem $\nu$ obstaja 
več rešitev, zato lastne rodove označujemo s parom indeksov $\nu$ in $m$, npr. TE$_{01}$.

\subsection*{TE in TM rodovi}
Najprej si oglejmo rešitve, pri katerih je $\nu=0$ in so zato neodvisne od kota $\varphi$. 
V klasični sliki so to žarki, ki potujejo po osi vlakna. Iz robnih pogojev sledi, da 
gre za transverzalne električne rodove, za katere velja $E_z=0$, $E_r=0$ in $E_\varphi \propto J_1(k_sr)$.
Jakost električnega polja za TE je\index{Optično vlakno!TE rodovi}
\begin{equation}
\mathbf{E} \propto \mathbf{e}_\varphi \, J_1(k_s r),
\end{equation}
pri čemer je $\mathbf{e}_\varphi$ enotski vektor.
Podobno lahko prepoznamo tudi TM rodove, pri katerih je $H_z=0$, $H_r=0$ in $H_\varphi \propto J_1(k_sr)$.
Jakost električnega polja za TM rodove je\index{Optično vlakno!TM rodovi}
\begin{equation}
\mathbf{E} \propto \mathbf{e}_r \, J_1(k_s r).
\end{equation}

Amplitudi jakosti električnega polja sta za TE in TM rodove enaki, zato
sta enaki tudi sliki gostote svetlobnega toka (slika~\ref{fig:TE01}). V osi 
vlakna je gostota svetlobnega toka enaka nič, zato sklepamo, da to niso osnovni načini 
širjenja svetlobe. 

\begin{figure}[ht]
\centering
\def\svgwidth{80truemm} 
\input{slike/10_TE01.pdf_tex}
\caption{Intenziteta in smer električnega polja v vlaknu za rodova TE$_{01}$ in TM$_{01}$
}
\label{fig:TE01}
\vglue-5truemm
\end{figure}
Podobno kot smo zapisali sekularno enačbo v valovnem vodniku (enačbi~\ref{sekular1}
in~\ref{sekular2}), tudi tukaj zapišemo enačbo za dovoljene vrednosti $k_s$. 
Ob približku, da se lomna količnika\index{Sekularna enačba!za cilindrično vlakno}
sredice in plašča le malo razlikujeta, je poenostavljena enačba za TE 
valovanje\footnote{~Glej npr. C. R. Pollock, {\it Fundamentals of Optoelectronics}, Irwin (1995).}
\boxeq{sekFiber}{
-k_sa\,\frac{J_0(k_sa)}{J_1(k_sa)}=\kappa a\,\frac{K_0(\kappa a)}{K_1(\kappa a)},
}
pri čemer velja zveza $\kappa^2+k_s^2=(NA)^2k_0^2$. Zaporedne rešitve enačbe ustrezajo rodovom TE$_{0m}$. 
Za izračun valovnih vektorjev za rodove TM$_{0m}$ moramo levo stran enačbe~(\ref{sekFiber}) pomnožiti z $(n_1/n_2)^2$. Ker je ta faktor približno ena, se rešitve enačb med seboj le malo razlikujejo.

\begin{figure}[ht]
\centering
\def\svgwidth{70truemm} 
\input{slike/10_TE_cilinder.pdf_tex}
\caption{K izračunu $k_s$ v 
cilindričnem vlaknu za TE polarizacijo.
Leva stran sekularne enačbe je narisana s črno, desna stran pa za tri različne vrednosti parametra 
$V=k_0aNA$.}
\label{fig:TEsecFib}
\vglue-1truemm
\end{figure}
Zapisano sekularno enačbo (enačba~\ref{sekFiber}) rešujemo ali numerično ali pa se je lotimo grafično in 
na sliki~\ref{fig:TEsecFib} poiščemo presečišča krivulj. Na sliki je leva stran enačbe prikazana
s črno barvo in desna za tri različne vrednosti $V= k_0 a\,NA$. Pri velikem $V$ (modra črta) ima sistem tri 
rešitve, pri srednjem $V$ (turkizna črta) eno rešitev, medtem ko se pri majhnem $V$ (zelena črta) 
krivulji ne sekata. To pomeni, da pri dovolj majhnem polmeru vlakna TE rod ne obstaja. 

Zapišimo to ugotovitev še matematično. Desna stran enačbe da realne rešitve za 
$k_s a \le  V$, leva stran enačbe pa postane pozitivna šele pri 
$J_0 (k_s a)  = 0$, to je pri $k_s a= 2,405$. Sistem ima vsaj eno rešitev, če velja
$V>2,405$. Polmer, pri katerem se valovanje TE$_{01}$ (ali TM$_{01}$) 
z dano valovno dolžino sploh širi po vlaknu, je torej navzdol omejen z 
\boxeq{10_cutoff}{
a \geq \frac{2,405}{k_0 NA}.
}

\subsection*{Hibridni HE in EH rodovi}
\index{Optično vlakno!HE rodovi}
\index{Optično vlakno!EH rodovi}
Poglejmo zdaj rešitve, pri katerih je $\nu \neq 0$. V tem primeru je 
vseh šest komponent električnega in magnetnega polja valovanja različnih od nič in vsi rodovi
imajo tudi komponento polja v smeri $z$. Take rodove imenujemo hibridni rodovi in jih 
označimo s HE, če je $E_z$ razmeroma velik ali vsaj primerljiv z $E_r$ in $E_\varphi$, 
oziroma z EH, če je $H_z$ po velikosti primerljiv s $H_r$ in $H_\varphi$ ali večji od njiju. 

Privzamemo, da se lomna količnika sredice in plašča le malo razlikujeta. Tudi v tem 
približku je sekularna enačba za hibridne rodove precej zapletena in je ne bomo 
zapisali.\footnote{~Glej npr. C. R. Pollock, {\it Fundamentals of Optoelectronics}, Irwin (1995).}
Oglejmo si le njihovo obliko (slika~\ref{fig:HE11}). Najpomembnejši hibridni rod je HE$_{11}$, 
ki je sorazmeren z $J_0(k_sr)$ in zato v središču različen od nič. 
To je osnovni rod, za katerega rešitev sekularne enačbe vedno obstaja in se
zato širi po še tako tankem vlaknu. Zaradi rotacijske simetrije si lahko smer 
polarizacije lastnega roda poljubno izberemo, mi izberimo smeri $x$ in $y$.
\begin{figure}[ht]
\centering
\def\svgwidth{77truemm} 
\input{slike/10_HE11.pdf_tex}\\
\def\svgwidth{77truemm} 
\input{slike/10_HE21.pdf_tex} \\
\def\svgwidth{77truemm} 
\input{slike/10_HE31.pdf_tex} \\
\def\svgwidth{77truemm} 
\input{slike/10_EH11.pdf_tex}
\caption{Intenziteta in smer električnega polja v vlaknu za rodove
HE$_{11}$, HE$_{21}$, HE$_{31}$ in EH$_{11}$}
\label{fig:HE11}
\vglue-4truemm
\end{figure}

Po obliki je osnovni HE$_{11}$ rod zelo podoben Gaussovemu profilu $\exp(-r^2/w^2)$,
zato ga lahko razmeroma dobro opišemo z Gaussovim približkom. 
Pri tem efektivni polmer snopa
$w$ izračunamo po Marcusejevi\index{Marcusejeva formula}
\index{Gaussov snop!efektivni polmer} formuli\footnote{~D. Marcuse, Bell Syst. Tech. J. $\mathbf{56}$, 
703 (1977).}
\begin{equation} 
w = \left(0,65 + \frac{1,619}{V^{3/2}}+\frac{2,879}{V^{6}}\right)\,a,
\label{Marcuse}
\end{equation}
pri čemer je $V = k_0 a\,NA $. Podobnost profila osnovnega
HE$_{11}$ roda z Gaussovo funkcijo omogoča zelo dobro sklopitev Gaussovih
snopov, ki izhajajo iz laserja, v cilindrična vlakna.

Na sliki~\ref{fig:HE11} je poleg osnovnega HE$_{11}$ roda še nekaj primerov višjih rodov. Opazimo, 
da imajo vsi rodovi, razen osnovnega, v osi vlakna minimum. Poleg tega opazimo tudi podobnost med 
oblikami posameznih rodov, ki je posledica majhne razlike med lomnima količnikoma sredice in plašča
($n_1 \approx n_2$). V takem primeru se sekularne enačbe poenostavijo, nekateri rodovi so 
med seboj degenerirani in dajo približno enako rešitev. 
Poleg rodov z enako obliko in različno polarizacijo so tako 
med seboj degenerirani še rodovi HE$_{\nu+1,m}$ in EH$_{\nu-1,m}$. Degenerirane 
rodove lahko združimo v linearne kombinacije in nastanejo pretežno 
linearno polarizirani LP rodovi. 

\subsection*{LP rodovi}
\index{Optično vlakno!LP rodovi}
Za praktično uporabo so najpomembnejši linearno polarizirani (LP) rodovi.\footnote{~Glej 
npr. A. Yariv in P. Yeh, {\it Photonics}, šesta izdaja, Oxford
University Press (2007).}  Taki rodovi niso
točne rešitve valovne enačbe v cilindrični geometriji, ampak jih zapišemo kot linearno 
kombinacijo lastnih rodov, ki so zaradi majhne razlike med lomnima količnikoma sredice
in plašča degenerirani. Tudi te rodove označimo z dvema indeksoma: prvi določa število azimutalnih
vozlov in drugi število radialnih vrhov. Poglejmo nekaj primerov (slika~\ref{fig:LP}).

Osnovni HE$_{11}$ rod je pri majhnih razlikah lomnih količnikov praktično linearno polariziran in ustreza polariziranemu rodu LP$_{01}$. Jakost električnega polja v njem je 
\begin{equation}
\mathbf{E}_\mathrm{LP01} \propto 
\left \{
  \begin{matrix}
  \mathbf{e}_x \\ \mathbf{e}_y 
  \end{matrix}
\right \} \, J_0(k_s r),
\end{equation}
saj ima dve možni smeri polarizacije. Na splošno rodovi HE$_{1m}$ ustrezajo rodovom LP$_{0m}$. 

Višje rodove, na primer LP$_{11}$, sestavimo kot linearno kombinacijo 
TE$_{01}$ ali TM$_{01}$ in HE$_{21}$.
Jakost električnega polja v LP$_{11}$ je tako v obliki štirih možnih kombinacij
\begin{equation}
\mathbf{E}_\mathrm{LP11} \propto \left \{
  \begin{matrix}
  \mathbf{e}_x \\ \mathbf{e}_y 
  \end{matrix}
\right \} \, J_1(k_s r)
\left \{
  \begin{matrix}
  \cos\varphi  \\ \sin\varphi 
  \end{matrix}
\right \}\!,
\end{equation}
kar opisuje štiri različne oblike rodov LP$_{11}$.

Rodove LP$_{21}$, ki nastanejo kot kombinacija rodov HE$_{31}$
in EH$_{11}$, zapišemo kot eno od štirih kombinacij
\begin{equation}
\mathbf{E}_\mathrm{LP21} \propto \left \{
  \begin{matrix}
  \mathbf{e}_x \\ \mathbf{e}_y 
  \end{matrix}
\right \} \, J_2(k_s r)
\left \{
  \begin{matrix}
  \cos 2 \varphi  \\ \sin 2 \varphi 
  \end{matrix}
\right \}\!.
\end{equation}
Linearno polarizirani LP rodovi imajo precejšnjo uporabno vrednost. To so 
namreč rodovi, ki jih v vlaknu vzbudimo, ko nanj posvetimo s polarizirano 
lasersko svetlobo. Zavedati  se moramo, da to niso lastni rodovi vlakna, 
ampak njihove linearne kombinacije, ki po vlaknu potujejo z malenkost različnimi
hitrostmi. Polarizacija svetlobe se zato vzdolž vlakna ne ohranja povsem.
\begin{remark}
Število rodov v večrodovnem cilindričnem vlaknu pri izbrani normirani 
frekvenci $V = k_0a\, NA$ lahko približno ocenimo z uporabo asimptotičnega 
razvoja Besslovih funkcij za velike argumente.\footnote{~Glej npr. C. R. Pollock, {\it Fundamentals 
of Optoelectronics}, Irwin (1995).} Približna ocena vključuje 
vse dovoljene rešitve pri vsaki vrednosti $\nu$ in obe polarizaciji. 
Dobimo \index{Optično vlakno!število rodov}
\begin{equation} 
N = \frac{4 V^2}{\pi^2}.
\end{equation}
V vlaknu s polmerom sredice $20~\si{\micro\meter}$ in numerično aperturo 0,2
se  pri valovni dolžini $1~\si{\micro\meter}$ po tej oceni lahko širi 256 rodov. 
\end{remark}
\begin{figure}[ht!]
\centering
\def\svgwidth{93truemm} 
\input{slike/10_LP01.pdf_tex} \\
\def\svgwidth{93truemm} 
\input{slike/10_LP02.pdf_tex} \\
\def\svgwidth{93truemm} 
\input{slike/10_LP11a.pdf_tex} \\
\def\svgwidth{93truemm} 
\input{slike/10_LP11b.pdf_tex} \\
\def\svgwidth{93truemm} 
\input{slike/10_LP21a.pdf_tex} \\
\def\svgwidth{93truemm} 
\input{slike/10_LP21b.pdf_tex} \\
\caption{Intenziteta in smer električnega polja v vlaknu za približno linearne rodove
LP$_{01}$, LP$_{02}$, LP$_{11}$ in LP$_{21}$}
\label{fig:LP}
\end{figure}
\newpage
\subsection*{Cilindrično vlakno s paraboličnim profilom lomnega količnika}
Čeprav je izračun lastnih načinov v cilindričnem vlaknu zapleten, lahko 
razmeroma enostavno poiščemo rešitve za vlakno, v katerem je dielektrična 
konstanta kvadratna funkcija radialne koordinate $r$\index{Optično vlakno!parabolični profil lomnega količnika}. 
Zapišemo lomni količnik in vpeljemo brezdimenzijski parameter $\Delta$
\begin{equation}
n^2\left(r<a\right)=n_{1}^{2}- \Delta^2 \frac{r^2}{a^2},
\label{9.15}
\end{equation}
pri čemer $a$ označuje polmer sredice.
Enačbo lahko razvijemo za majhno razliko $\Delta$ in za vse smiselne vrednosti $r$
ima tudi lomni količnik parabolični profil. Parabolična
sredica je seveda omejena, okoli nje je plašč s konstantnim
lomnim količnikom $n_2 \approx n_1-\Delta^2/2n_1$ (slika~\ref{fig:GRIN}). 
Tipičen polmer sredice $a$ je nekaj deset mikrometrov in plašča približno petkrat toliko.
\begin{figure}[ht]
\centering
\def\svgwidth{90truemm} 
\input{slike/10_GRIN.pdf_tex} 
\caption{Parabolični profil lomnega količnika sredice zmanjša disperzijo v vlaknu. Plašč
 je praviloma bistveno debelejši od sredice vlakna.}
\label{fig:GRIN}
\end{figure}

Jakost električnega polja za izbrano polarizacijo zapišemo v obliki 
\begin{equation}
E=E_{0}\psi(x,y)\, e^{i\beta z-i\omega t}.
\label{9.16}
\end{equation}
Pri tem smo zanemarili, da zaradi odvisnosti od prečnih koordinat in pogoja $\nabla\cdot{\bf D}=0$
polje ne more imeti povsod iste smeri; za natančnejši račun bi morali zapisati enačbo za
vektorsko polje. Vstavimo približni
nastavek~(enačba~\ref{9.16}) in krajevno odvisnost lomnega količnika~(enačba~\ref{9.15})
v valovno enačbo (enačba~\ref{eq:valovna-skalarna}) 
\begin{equation}
\nabla_{\perp}^{2}\psi+\left(k_{0}^{2}\left(n_{1}^{2}-\Delta^{2}\frac{r^{2}}{a^2}\right)-
\beta^{2}\right)\,\psi=0.
\label{9.17}
\end{equation}
Rešitve lahko zapišemo v obliki
\begin{equation}
\psi(x,y) = X(x)Y(y),
\end{equation}
od koder sledita dve neodvisni enačbi
\begin{equation}
X'' - \frac{k_0^2 \Delta^2}{a^2}\,X\,x^2 - \lambda_1 X = 0 \qquad \mathrm{in} \qquad
Y'' - \frac{k_0^2 \Delta^2}{a^2}\,Y\,y^2 - \lambda_2 Y = 0,
\label{eq:XY}
\end{equation}
pri čemer sta $\lambda_1$ in $\lambda_2$ konstanti. 
Opazimo, da sta enačbi popolnoma enaki enačbama za krajevni del lastnih funkcij 
harmonskega oscilatorja v kvantni mehaniki.
Rešitev posamezne enačbe je tako 
produkt Gaussove in Hermitove funkcije
\begin{equation}
X_n(x) = e^{-\xi^2 x^2/2} H_n(\xi x),
\label{eq:GH}
\end{equation}
pri čemer je $\xi = \sqrt{k_0 \Delta/a}$.
\begin{definition}
Uporabi nastavek (enačba~\ref{eq:GH}) in pokaži, da reši enačbo~(\ref{eq:XY}). Pri tem si pomagaj z 
diferencialno enačbo za Hermitove polinome
\begin{equation}
\left( \frac{d^2}{dx^2}-2x\frac{d}{dx}+2n \right) H_n(x) = 0.
\end{equation}
\end{definition}
Lastne vrednosti enačbe (\ref{9.17}) so oblike
\begin{equation}
\beta_{mn}^{2}=n_{1}^{2}k_{0}^{2}\left(1-\frac{2\Delta}{k_{0}n_{1}^2a}\left(m+n+1\right)\right)\!.
\label{9.19}
\end{equation}
Drugi člen v oklepaju je navadno zelo majhen, zato lahko izraz razvijemo in dobimo
\begin{equation}
\beta_{mn}=n_{1}k_{0}\left(1-\frac{\Delta}{k_{0}n_{1}^2 a}\left(m+n+1\right)\right)
= n_{1}k_{0} - \frac{\Delta \left(m+n+1\right)}{n_{1} a}.
\end{equation}
Ob predpostavki, da je parameter $\Delta$ neodvisen od krožne frekvence, je grupna 
hitrost\index{Hitrost valovanja!grupna}
\begin{equation}
v_{g}=\left(\frac{d\beta_{mn}}{d\omega}\right)^{-1}=\frac{c_{0}}{n_{1}}
\label{9.21}
\end{equation}
in torej enaka za vse rodove. To je pomembna značilnost vlakna s paraboličnim profilom
lomnega količnika. V dejanskem vlaknu je seveda taka odvisnost mogoča
le v omejenem območju sredice, zato je tudi opisani pristop le približen
in velja dobro za tiste rodove, ki se ne raztezajo dosti izven sredice.

Neodvisnost grupne hitrosti od roda je praktično zelo pomembna. 
Grupna hitrost namreč določa čas potovanja svetlobnega sunka, ki
lahko predstavlja en bit informacije. Če se po vlaknu širi več
rodov z različnimi grupnimi hitrostmi, se sunek po prehodu skozi
vlakno podaljša, kar -- kot bomo podrobneje videli v naslednjem razdelku -- omejuje 
uporabno dolžino vlakna. Temu se sicer lahko izognemo z uporabo enorodovnih vlaken,
ki pa so dražja, poleg tega morata divergenca in polmer svetlobnega snopa 
natančno ustrezati značilnostim enorodovnega vlakna, da se izognemo izgubam. 
Zato se za krajše povezave (do nekaj $100~\si{m}$) uporabljajo večrodovna vlakna, ki imajo sredico s 
približno paraboličnim profilom lomnega količnika.

\section{Disperzija}
\label{chap:Disperzija}
Pri prenosu velike količine podatkov na daljavo je zelo pomembno, da
se oblika svetlobnih sunkov, ki prenašajo informacijo, čim bolj ohranja.
Na obliko sunka močno vpliva disperzija, to je odvisnost fazne in grupne hitrosti
valovanja od krožne frekvence. Zaradi disperzije se kratki sunki, ki potujejo po vlaknu, podaljšajo in 
tako omejujejo količino informacije, ki jo lahko prenašamo po vlaknu dane dolžine\index{Disperzija}
(slika~\ref{fig:disp}).
Največja količina vhodnih podatkov na časovno enoto je kar 
obratno sorazmerna z dolžino izhodnih sunkov svetlobe. 
\begin{figure}[ht]
\centering
\def\svgwidth{120truemm} 
\input{slike/10_disperzija.pdf_tex} 
\caption{Zaradi disperzije se sunki svetlobe, ki potujejo skozi vlakno, 
podaljšajo. Na izhodu iz vlakna jih zato ne zaznamo več ločeno.}
\label{fig:disp}
\end{figure}

Pri potovanju svetlobe po optičnih vlaknih poznamo tri 
vrste disperzije: rodovno, materialno in valovodno. V večrodovnih vlaknih
je povsem prevladujoča rodovna disperzija, ki je posledica dejstva, da se 
različni rodovi po vlaknu širijo z različnimi hitrostmi. 
V enorodovnih vlaknih rodovne disperzije ni, zato prideta do izraza 
materialna disperzija, ki se pojavi zaradi odvisnosti lomnega količnika 
vlakna od valovne dolžine svetlobe, in valovodna disperzija, ki se pojavi zaradi 
nelinearne zveze med valovnim številom $\beta$ in krožno frekvenco valovanja. 

\subsection*{Rodovna disperzija}
\index{Disperzija!rodovna}
\index{Optično vlakno!večrodovno}
Na primeru planparalelnega vodnika smo pokazali, da vsaki rešitvi sekularnih 
enačb~(enačbi~\ref{sekular1} in \ref{sekular2}) ustreza en lastni rod v vlaknu.
Ker se vrednosti $k_x$ za različne rodove med seboj razlikujejo in so posledično vrednosti $\beta$ za 
vsak rod drugačne, se posamezni rodovi po vlaknu širijo z različnimi hitrostmi. Kratek
sunek svetlobe, sestavljen iz več različnih rodov, se tako po prehodu skozi vlakno
razdeli na posamezne delne sunke oziroma se efektivno podaljša. 
Izračunajmo razliko med časom, ki ga za širjenje po vlaknu dane dolžine 
potrebuje osnovni rod, in časom, ki ga za isto razdaljo potrebuje zadnji še dovoljeni rod.
 
Osnovnemu rodu ustreza prva rešitev sekularne enačbe (enačba~\ref{sekular1}) in je zato pripadajoča 
vrednost $k_x$ majhna. V prvem približku je $k_x \approx 0$. Ustrezno valovno 
število je po enačbi
\begin{equation}
\beta = \sqrt{\left( \frac{\omega}{c_0}\right)^2n_1^2 - k_x(\omega)^2}
\label{nelinfib}
\end{equation}
kar približno enako $\beta_0 \approx k_0 n_1$. 
Za zadnji še dovoljeni rod velja $k_xa \approx V$ in $\beta_N \approx k_0 n_2$, pri čemer je
$n_2$ lomni količnik plašča. Zapišemo še grupno hitrost, s katero potujejo sunki 
svetlobe po vlaknu
\begin{equation}
v_{g}=\frac{d\omega}{d\beta}=\left(\frac{d\beta}{d\omega}\right)^{-1}\!.
\label{9.51}
\end{equation}
Za prehod vlakna z dolžino $L$ potrebuje osnovni rod čas
\beq
t_0 = \frac{L}{v_{g0}} =  L \,\frac{d\beta_0}{d\omega} = L\,\frac{n_1}{c_0},
\eeq
zadnji rod pa čas
\beq
t_N = \frac{L}{v_{gN}} = L\, \frac{d\beta_N}{d\omega} = L\,\frac{n_2}{c_0}.
\eeq
Podaljšanje sunka zaradi rodovne disperzije je potem 
\boxeq{DispRod}{
\tau\approx 
\frac{L}{c_0} (n_1-n_2).
}
Največja frekvenca modulacije vhodnega signala, pri kateri izhodne sunke še zaznamo ločeno,
je približno obratna vrednost dolžine izhodnih sunkov. Za $1~\si{\kilo\meter}$ dolgo vlakno z 
$\Delta n = 0,05$  je tako modulacijska frekvenca oziroma količina podatkov v časovni
enoti manj od $10~\si{\mega\hertz}$. Čeprav lahko disperzijo zmanjšamo 
s paraboličnim profilom lomnega količnika, so večrodovna vlakna za prenos podatkov na dolge 
razdalje praktično neuporabna.
\begin{remark}
 Med rodovno disperzijo uvrščamo tudi polarizacijsko disperzijo.\index{Disperzija!polarizacijska} 
 V idealnem cilindričnem vlaknu potujeta obe polarizaciji
 z enako hitrostjo. V realnem vlaknu pa imata valovanji z različnima polarizacijama zaradi
 nečistoč in asimetrij v vlaknu različni hitrosti. 
 Ker so nečistoče slučajno in neodvisno razporejene, tako disperzijo zelo težko odpravimo.
\end{remark}

\subsection*{Materialna disperzija}
\index{Disperzija!materialna}
Optična vlakna so navadno iz stekla, katerega lomni količnik je odvisen 
od valovne dolžine svetlobe. Zaradi tako imenovane materialne ali snovne disperzije različne 
spektralne komponente svetlobnega sunka po vlaknu potujejo z različnimi hitrostmi 
in sunek se po prehodu skozi vlakno podaljša. Pri obravnavi\index{Disperzija!snovna|see{Disperzija, materialna}}
se omejimo na enorodovna vlakna, v katerih rodovne disperzije ni. 

Ker je sunek svetlobe končno dolg, je končna tudi njegova spektralna širina 
$\Delta \omega$. 
Dolžino sunka $\tau$ po prehodu skozi vlakno dolžine $L$ približno zapišemo kot
\begin{equation}
\tau = \frac{dt}{d\omega}\Delta \omega = 
\Delta \omega \frac{d}{d\omega}\left(\frac{L}{v_g}\right)
= \Delta \omega \frac{d}{d\omega}\left(L \frac{d\beta}{d\omega}\right) =
\Delta \omega\, L\, \frac{d^2 \beta}{d \omega^2}.
\label{dispračun}
\end{equation}
Za enorodovno vlakno velja $k_x \approx 0$ in $\beta \approx n_1 \omega/c_0$. Dobimo  
\begin{equation}
\tau = \Delta \omega\, \frac{L}{c_0}\,\frac{d^2 (n_1 \omega)}{d \omega^2}.
\end{equation}
Na področju telekomunikacij se podaljšanje sunka navadno zapiše v obliki
\boxeq{eq:dmat1}{
\tau= |D_m| L\, \Delta \lambda.
}
Pri tem smo vpeljali $D_m$ kot koeficient materialne disperzije
\begin{equation}
D_m = - \frac{2\pi}{\lambda^2}\frac{d^2(n_1 \omega)}{d\omega^2},
\end{equation}
ki ga navadno izrazimo v enotah $\si{\pico\second/\nano\meter\, \kilo\meter}$.
Njegova vrednost je lahko pozitivna ali negativna, zato smo v končnem izrazu za dolžino 
sunka dodali absolutno vrednost. V snoveh, ki jih uporabljamo za izdelavo optičnih vlaken, 
je $D_m \sim 10~\si{\pico\second/\nano\meter\, \kilo\meter}$, lahko pa seže
tudi do več $100~\si{\pico\second/\nano\meter\, \kilo\meter}$, odvisno seveda od valovne dolžine
in izbrane snovi.\footnote{~Glej npr. C. R. Pollock, {\it Fundamentals of Optoelectronics}, 
Irwin (1995).}

Materialno disperzijo lahko zmanjšamo na več načinov. Lahko uporabimo čim bolj enobarven
vir svetlobe, da zmanjšamo $\Delta \omega$. Za izbrano snov lahko celo izberemo valovno 
dolžino, pri kateri je materialna disperzija čim manjša in praktično enak nič. Za 
SiO$_2$\index{SiO$_2$} je to pri okoli $1300$--$1500~\si{\nano\meter}$, odvisno 
od primesi. Še najuporabnejša je rešitev, pri kateri 
z materialno disperzijo izničimo vpliv valovodne disperzije in na ta način zmanjšamo skupno 
disperzijo v vlaknu.

\subsection*{Valovodna disperzija}
\index{Disperzija!valovodna}
V optičnem vodniku velja nelinearna zveza med prečno ($k_x$) in vzdolžno ($\beta$)
komponento valovnega vektorja (enačba~\ref{nelinfib}). Vrednost $k_x$ izračunamo numerično, 
pri čemer je rešitev odvisna od frekvence svetlobe. Tudi v cilindričnih vlaknih je valovno število 
$\beta$ nelinearna funkcija $\omega$, zato se pojavi disperzija. Izhajamo iz zveze 
(enačba~\ref{dispračun})
\begin{equation}
\tau = \Delta \omega\, L\, \frac{d^2 \beta}{d \omega^2}.
\end{equation}
Pogosto vpeljemo koeficient valovodne disperzije $D_v$
\begin{equation}
D_v = -\frac{2\pi c_0}{\lambda^2}\frac{d^2 \beta}{d\omega^2}
\end{equation}
in lahko zapišemo
\boxeq{dispVal}{
\tau = |D_v|\,L\, \Delta \lambda.
}
Koeficient valovodne disperzije je praviloma najmanjši, 
$D_v \sim 1$--$10~\si{\pico\second/\nano\meter\,\kilo\meter}$. 
Znaten postane v enorodovnih vlaknih v območju, kjer je materialna disperzija 
zelo majhna ali celo enaka nič. 
V vlaknih s homogeno sredico se valovodni disperziji ne moremo
izogniti, vendar jo lahko pri dani valovni dolžini približno izničimo 
z materialno~(slika~\ref{fig:MatVal}).\footnote{~Slika povzeta po C. R. Pollock, {\it 
Fundamentals of Optoelectronics}, Irwin (1995).}

Ker ima sunek svetlobe vedno neko končno spektralno širino, disperzije 
v optičnem vlaknu nikoli ne moremo povsem odpraviti. Pri celotni disperziji 
$5~\si{\pico\second/\nano\meter\,\kilo\meter}$ in spektralni širini, ki ustreza
$\Delta \lambda = 1~\si{\nano\meter}$, znaša v $100~\si{\kilo\meter}$ dolgem vlaknu najvišja 
frekvenca modulacije vhodnega signala, ki ga na izhodu še lahko razločimo, 
okoli $2~\si{\giga\hertz}$. V nadaljevanju bomo videli, da je pri prenosu podatkov
v vlaknih poglavitni omejujoči faktor ravno disperzija in ne absorpcija. 

\begin{figure}[ht]
\centering
\def\svgwidth{100truemm} 
\input{slike/10_Zero.pdf_tex} 
\caption{Odvisnost koeficientov disperzije od valovne dolžine v vlaknu iz SiO$_2$.\index{SiO$_2$} $D_m$ 
je koeficient materialne disperzije in $D_v$ valovodne, $D$ pa je vsota obeh. Pri $\lambda \approx 1450~\si{\nano\meter}$ se materialna in valovodna disperzija odštejeta in skupna disperzija
je praktično enaka nič.}
\label{fig:MatVal}
\end{figure}
\vglue-4truemm
\begin{remark}
Na valovodno disperzijo je mogoče vplivati tudi s konstrukcijo vlakna. Pokazali smo že, da
v idealnem primeru v vlaknu s paraboličnim profilom lomnega količnika disperzije ni. 
V praksi je sredica sestavljena iz več plasti z različnimi lomnimi količniki in različnimi
debelinami, s čimer se prispevek valovodne disperzije spremeni. Na ta način lahko 
položaj ničle celotne disperzije premaknemo k valovni dolžini izvora oziroma k 
valovni dolžini, pri kateri je v vlaknu najmanj absorpcije in izgub.
\end{remark}

\section{*Potovanje kratkega sunka po enorodovnem vlaknu}
\label{chap:sunvl}
\subsection*{Podaljšanje sunka zaradi disperzije}
\index{Optično vlakno!enorodovno}
Poglejmo si podrobneje, kako po enorodovnem vlaknu ali drugem
sredstvu z disperzijo potuje kratek sunek valovanja z dano začetno obliko.
Sunek zapišemo kot  
\begin{equation}
E\left(x, y, z, t\right)=\psi\left(x,y\right)\, a\left(z,t\right)\!,
\label{9.61}
\end{equation}
pri čemer je $\psi\left(x,y\right)$ lastna rešitev prečnega dela valovne
enačbe, ki določa zvezo $\beta\left(\omega\right)$. 
Funkcija $a\left(z,t\right)$ opisuje obliko in potovanje sunka v smeri $z$. 
Pri $z=0$ jo razvijemo po krožnih frekvencah z ustreznimi amplitudnimi faktorji 
\begin{equation}
a\left(0,t\right)=\int \tilde{A}(\omega)\, e^{- i\omega t}d\omega.
\label{9.62}
\end{equation}
Ko sunek potuje vzdolž osi $z$, vsaki komponenti pripišemo
ustrezen fazni faktor $i \beta (\omega) z$
\begin{equation}
a\left(z,t\right)=\int \tilde{A}(\omega)\, e^{i \beta (\omega) z - i\omega t}d\omega.
\label{9.62f}
\end{equation}
Osnovni sunek naj bo približno monokromatičen s krožno frekvenco $\omega_{0}$. Ta naj bo dovolj 
velika, da lahko privzamemo, da je sunek mnogo daljši od optične periode.

Razvijmo $\beta(\omega)$
okoli $\omega_{0}$, pri čemer vpeljemo razliko krožnih frekvenc $\Omega = \omega - \omega_0$
\begin{equation}
\beta(\omega_0 + \Omega) \approx \beta(\omega_{0})
+\frac{d\beta}{d\omega}\,\Omega+\frac{1}{2}\,\frac{d^{2}\beta}{d\omega^{2}}\,\Omega^{2} = 
\beta(\omega_{0}) +\beta '\,\Omega+\frac{1}{2}\,\beta ''\,\Omega^{2}.
\label{9.62c}
\end{equation}
S črtico smo označili odvod po $\omega$. Enačbo~(\ref{9.62f}) prepišemo v 
\begin{equation}
a\left(z,t\right)=\int \tilde{A}(\Omega)\, e^{i \beta (\omega_0 + \Omega)z - 
i(\omega_0 + \Omega) t}d\Omega  =  e^{i \beta (\omega_0)z - i\omega_0 t} A(z,t).
\label{9.62b}
\end{equation}
Funkcija $A(z, t)$ torej predstavlja prostorsko in časovno odvisnost ovojnice sunka. Z upoštevanjem
razvoja (enačba~\ref{9.62c}) jo zapišemo kot 
\begin{equation}
 A(z,t) = \int \tilde{A}(\Omega)\, \exp \left(i \beta'\, \Omega\,z + 
 \frac{i}{2}\beta''\,\Omega^2\, z - i\, \Omega\, t\right) d\Omega.
 \label{eq:ovojnica967}
\end{equation}
Odvajajmo ovojnico najprej parcialno po $z$
\begin{equation}
 \frac{\partial A(z,t)}{\partial z} = \int \tilde{A}(\Omega)\, 
 \left(i \beta'\, \Omega + \frac{i}{2}\beta''\Omega^2 \right) 
 \exp \left(i \beta'\, \Omega\,z + 
 \frac{i}{2}\beta''\,\Omega^2\, z - i\, \Omega\, t\right) d\Omega,
\label{9.67a}
\end{equation}
nato pa še po $t$
\begin{equation}
 \frac{\partial A(z,t)}{\partial t} = \int \tilde{A}(\Omega)\, 
 \left(-i\Omega\right) 
 \exp \left(i \beta'\, \Omega\,z + 
 \frac{i}{2}\beta''\,\Omega^2\, z - i\, \Omega\, t\right) d\Omega
\label{9.67b}
\end{equation}
Drugi parcialni odvod po $t$ je enak
\begin{equation}
 \frac{\partial^2 A(z,t)}{\partial t^2} = \int \tilde{A}(\Omega)\, 
 \left(-\Omega^2\right) 
 \exp \left(i \beta'\, \Omega\,z + 
 \frac{i}{2}\beta''\,\Omega^2\, z - i\, \Omega\, t\right) d\Omega.
\label{9.67c}
\end{equation}
Primerjamo izračunane odvode in dobimo enačbo
\begin{equation}
 \frac{\partial A(z,t)}{\partial z} = -\beta'\frac{\partial A}{\partial t} - \frac{i}{2} \beta''\frac{\partial^2 A}{\partial t^2}.
 \label{9.68b}
\end{equation}
Enačbo lahko nekoliko poenostavimo z vpeljavo novega para neodvisnih spremenljivk
\begin{equation}
\tau  =  t-\beta'z\nonumber \qquad \mathrm{in} \qquad \zeta = z.
\label{9.70}
\end{equation}
Uporabimo verižno pravilo odvajanja
\begin{equation}
 \frac{\partial}{\partial z}= \frac{\partial}{\partial\tau}\frac{\partial\tau}{\partial z}+ 
 \frac{\partial}{\partial\zeta}\frac{\partial\zeta}{\partial z} = 
 \frac{\partial}{\partial\tau}\left(-\beta'\right)+ \frac{\partial}{\partial \zeta}
\end{equation}
in 
\begin{equation}
 \frac{\partial}{\partial t}= \frac{\partial}{\partial\tau}\frac{\partial\tau}{\partial t}+ 
 \frac{\partial}{\partial\zeta}\frac{\partial\zeta}{\partial t} = 
 \frac{\partial}{\partial\tau}.
\end{equation}
Z novima spremenljivkama se enačba~(\ref{9.68b}) prepiše v 
\begin{equation}
\beta''\frac{\partial^{2}A}{\partial\tau^{2}}-
2\, i\,\frac{\partial A}{\partial\zeta}=0.
\label{9.71}
\end{equation}
Poglejmo enačbo podrobneje. Če ni disperzije in je $\beta''=0$, se $A$
vzdolž koordinate $\zeta$ ne spreminja. To pomeni, da se oblika sunka
ob odsotnosti disperzije ohranja in sunek poljubne začetne oblike nepopačen 
potuje po vlaknu z grupno hitrostjo $1/\beta'$.\index{Hitrost valovanja!grupna}

Če je disperzija različna od nič, ostaneta v enačbi oba člena. Opazimo, da
ima enačba enako obliko kot obosna valovna enačba, ki smo jo v
tretjem poglavju uporabili za obravnavo koherentnih 
snopov (enačba~\ref{eq:obosna-valovna-enacba}). 
Razlika med obosno valovno enačbo in enačbo~(\ref{9.71}) je v tem, da vlogo
prečne koordinate prevzame čas $\tau$. To, kar je bila prej širina snopa, 
je zdaj dolžina sunka. Spomnimo se, da 
obosno valovno enačbo rešijo Gaussovi snopi (enačba~\ref{eq:gaussov-snop}). 
\vglue-4truemm
\begin{remark}
Podobnost med pojavoma seže dlje od formalne oblike. Pri snopih, ki so omejeni 
v prečni smeri, disperzija fazne in grupne hitrosti po prečnih komponentah valovnega
vektorja povzroča spreminjanje prečnega preseka snopa. Pri časovno
omejenih sunkih v sredstvu s frekvenčno disperzijo se namesto preseka sunka
spreminja njegova vzdolžna oblika oziroma njegova dolžina.
\end{remark}
\vglue-4truemm
Celotnega računa
ni treba ponavljati, namesto tega kar v izrazu za Gaussove snope 
(enačba~\ref{eq:gaussov-snop}) napravimo ustrezno zamenjavo spremenljivk. 
Iz enačbe~(\ref{9.71}) razberemo, da valovnemu številu $k$ pri snopih  
ustreza parameter $\mu=(d^{2}\beta/d\omega^{2})^{-1}$. Poleg tega vpeljemo
dolžino sunka $2\sigma$, ki ustreza premeru Gaussovega snopa $2w$, in parameter
$b$, ki ustreza krivinskemu radiju $R$. Oba parametra sta seveda odvisna od $\zeta$, 
tako kot sta parametra $w$ in $R$ odvisna od $z$ (slika~\ref{fig:Gausstau}). 

Na podlagi analogije zapišemo obliko podaljšanega Gaussovega sunka\index{Gaussov sunek}
\begin{equation}
A\left(\zeta,\tau\right)=\frac{A_{0}}{\sqrt{1+\zeta^{2}/
\zeta_{0}^{2}}}\exp\left(-\frac{\tau^{2}}{\sigma^{2}}\right)\exp
\left(-i\frac{\mu\tau^{2}}{2b}\right)e^{i\phi\left(\zeta\right)}.
\label{9.72}
\end{equation}
\vglue-4truemm
Za $\sigma$ velja enaka zveza kot za polmer 
Gaussovega snopa (enačba~\ref{eq:w})
\boxeq{9.73}{
\sigma^{2}=\sigma_{0}^{2}\left(1+\left(\frac{\zeta}{\zeta_{0}}\right)^{2}\right)\!.
}
Pri tem je $2\sigma_{0}$ trajanje sunka pri $\zeta=0$, to je na začetku,
kjer je sunek najkrajši. Krivinskemu radiju valovnih front (enačba~\ref{eq:R})
ustreza količina $b=\zeta\left(1+\zeta_{0}^{2}/\zeta^{2}\right)$.
Po analogiji s snopi sklepamo, da se najmanj 
podaljšuje ravno sunek z Gaussovo časovno odvisnostjo. 
\begin{figure}[ht]
\centering
\def\svgwidth{120truemm} 
\input{slike/10_Gausstau.pdf_tex}
\caption{Primerjava širitve Gaussovega snopa (a) in podaljšanja Gaussovega sunka (b).
Dolžina sunka $2\sigma$ med potovanjem po vlaknu narašča z enako odvisnostjo kot 
narašča premer Gaussovega snopa $2w$ z oddaljenostjo od grla. Znotraj $\zeta_0$ se sunek
še ne podaljša znatno.}
\label{fig:Gausstau}
\vglue-3truemm
\end{figure}

Zanimivo je pogledati odvod faze po $\tau$, ki predstavlja spremembo krožne
frekvence glede na centralno krožno frekvenco sunka $\omega_{0}$:
$\omega-\omega_{0}=\mu\tau/b$.
Za pozitivne vrednosti $\mu$ je krožna frekvenca na začetku sunka,
to je pri $\tau<0$, manjša od $\omega_0$, z naraščajočim časom pa se 
linearno povečuje proti koncu sunka. Obnašanje je zelo podobno čirikanju, 
ki ga bomo spoznali pri obravnavi nelinearnih optičnih pojavov (slika~\ref{fig:chirp}\,a). \index{Čirikanje}
\vglue-1truemm
\begin{remark}
Pri $\zeta=0$ je sunek najkrajši možen pri dani spektralni
širini. Lahko si mislimo, da je sunek najkrajši,
to je omejen s Fourierevo transformacijo spektra, kadar se
vse frekvenčne komponente seštejejo z isto fazo, to je pri $\zeta=0$.
Da nastanejo najkrajši sunki, kadar je faza vseh delnih valov enaka,
smo spoznali že pri fazno uklenjenih sunkih iz večfrekvenčnih laserjev
(razdelek~\ref{chap:Uklepanje}).
Pri potovanju sunka se zaradi disperzije faze frekvenčnih komponent
različno spreminjajo in sunek se podaljša. Pri tem je pomemben  
drugi odvod valovnega števila po krožni frekvenci. Linearno spreminjanje faze 
namreč ne povzroči razširitve, temveč le spremembo v hitrosti.
\end{remark}
\vglue-2truemm
\begin{definition}
\vglue-3truemm
\label{naloga:pulzdisperzija}
Naj bo vpadni sunek svetlobe Gaussove oblike $E(x,y, z=0, t) = 
\psi(x,y) e^{-at^2-i \omega_0 t}$. Pokaži, da je ustrezna funkcija
$\tilde{A}(\Omega)$ oblike
\begin{equation}
\tilde{A}(\Omega) = \frac{1}{\sqrt{4 \pi a}}e^{-\Omega^2/4a}.
\end{equation}
Nato vpelji novi spremenljivki $\tau$ in $\zeta$ ter z neposredno 
integracijo (enačba~\ref{eq:ovojnica967}) pokaži, 
da je podaljšan sunek pri $z\neq 0$ enak ovojnici, 
zapisani z enačbo~(\ref{9.72}), pri čemer je $\zeta_0 = \mu/2 a$.
\end{definition}
\vglue-3truemm
\begin{definition}
\vglue-2truemm
Uporabi enačbo~(\ref{9.73}) in pokaži, da je podaljšanje Gaussovega 
sunka svetlobe oblike $I \propto \exp(-2\tau^2/\sigma^2)$
pri dani dolžini vlakna enako
\begin{equation}
\sigma (L) = \sigma_0\sqrt{1 + \left(\frac{2 L }{\sigma_0^2 \mu}\right)^2}
\end{equation}
in za velike dolžine enako izrazu, ki smo ga izračunali pri 
valovodni disperziji (enačba~\ref{dispVal}).
\end{definition}

\subsection*{Kompenzacija disperzije}
\index{Disperzija!kompenzacija}
\label{kompdisp}
Razširitev sunka zaradi pozitivne disperzije je pri $\mu > 0$ mogoče kompenzirati
s parom vzporednih uklonskih mrežic.\index{Uklonska 
mrežica}\footnote{~E. B. Treacy, IEEE J. Quantum. Electron. $\mathbf{5}$, 454 (1969).}
Prva mrežica različne frekvenčne komponente razkloni in druga ponovno
zbere, vendar so pri tem optične poti za različne komponente različno 
dolge (slika~\ref{fig:comp}).
Vzporednost uklonskih mrežic zagotavlja vzporednost izhodnih žarkov,
vendar so različne komponente vpadne svetlobe med seboj razmaknjene (slika~\ref{fig:comp}\,a).
V praksi zato uporabimo ali dva para uklonskih mrežic ali 
zrcalo, ki svetlobo usmeri po isti poti 
nazaj.
\begin{figure}[h!]
\centering
\def\svgwidth{115truemm} 
\input{slike/10_comp.pdf_tex}
\caption{Kompenzacija disperzije z uklonskima mrežicama (a) in shema z
oznakami za izpeljavo faznega premika (b)}
\label{fig:comp}
\vglue-3truemm
\end{figure}

Naj na par vzporednih uklonskih mrežic vpada ravni val pod kotom $\alpha$, ki se odbija
pod kotom $\beta = \beta(\omega)$ (slika~\ref{fig:comp}\,b). 
Pot, ki jo prepotuje žarek od vpada na mrežico 
do izhoda iz sistema (med točkama $A$ in $B$), je enaka 
\begin{equation}
P = \frac{L}{\cos\beta} \left(1+\cos(\alpha + \beta)\right)\!.
\end{equation}
Zaradi uklona velja zveza $\sin\,\alpha - \sin\,\beta = \lambda/\Lambda$,
pri čemer je $\lambda$ valovna dolžina svetlobe in $\Lambda$ perioda uklonske mrežice. 
Pri fazi moramo upoštevati še fazni premik na drugi mrežici
\begin{equation}
\Phi_m=\frac{2\pi}{\Lambda} \, L \, \tan\,\beta = q L \,\tan\,\beta.
\end{equation}
Celotna sprememba faze je tako $\Phi = \omega P/c + \Phi_m$.

\begin{definition}
\label{nal:dispk}
Pokaži, da je drugi odvod faze po krožni frekvenci enak
\begin{equation}
\frac{d^2 \Phi}{d \omega^2} = - \frac{L\, c\, q^2}
{\left(\omega^2 - (\omega\, \sin\alpha - cq)^2\right)^{3/2}}.
\label{eq:10faza}
\end{equation}
\end{definition}
Račun v nalogi (\ref{nal:dispk})
pokaže, da je disperzija, ki je določena z drugim odvodom faze po krožni frekvenci
(enačba~\ref{eq:10faza}), vedno negativna. Par vzporednih uklonskih mrežic
torej deluje kot sredstvo
z negativno disperzijo in sunek, ki se je razširil zaradi potovanja
po sredstvu s pozitivno disperzijo, lahko ponovno skrajša do meje,
določene s širino spektra. 

\begin{remark}
Postopek kompenzacije disperzije se uporablja za pridobivanje zelo močnih in hkrati zelo
kratkih sunkov svetlobe.\footnote{~D. Strickland in G. Mourou, Opt. Comm. $\mathbf{56}$, 219 (1985).} 
Sunku iz fazno uklenjenega barvilnega\index{Laser!organska barvila} 
ali titan-safirnega\index{Laser!Ti:safir}\index{Uklepanje faz}
laserja se najprej v nelinearnem sredstvu razširi spekter, hkrati
se sunek tudi časovno podaljša. Podaljšan sunek lahko ojačimo, česar 
s prvotnim kratkim in razmeroma močnim sunkom ne bi mogli narediti. Razširjen
in ojačen sunek nato s parom uklonskih mrežic skrajšamo za 
faktor $10$--$100$ glede na prvotno dolžino sunka. Tako nastanejo zelo močni sunki
svetlobe, dolgi le okoli $10~\si{\femto\second}$, kar je le še nekaj optičnih 
period.\footnote{~Za to odkritje sta leta 2018 Donna Strickland in
G\'erard Mourou prejela Nobelovo nagrado.}
\end{remark}

\section{Izgube in ojačenje v optičnih vlaknih}
\index{Izgube v optičnih vlaknih}
Pri prenosu informacij z optičnimi vlakni je poleg disperzije, ki signal popači,
treba upoštevati tudi izgube, ki signal oslabijo. 
Izgube so posledica predvsem absorpcije svetlobe v vlaknu,\index{Absorpcija}
Rayleighovega\index{Rayleighovo sipanje} sipanja na fluktuacijah gostote, 
sipanja na nečistočah in upognjenosti vlakna. Do izgub prihaja tudi na stikih 
med vlakni. Za prenos na dolge
razdalje je tako potreben razmeroma močen signal, a ne premočen,
saj lahko v vlaknu pride do nelinearnih optičnih pojavov. V praksi zato 
optični signal, ki potuje po čezoceanskih vlaknih, ojačujemo in s tem nadomestimo
izgube.

Za merilo izgub v vlaknu vpeljemo atenuacijski 
koeficient\index{Atenuacijski koeficient}, merjen v decibelih ali dB/km
\boxeq{dB}{
A [dB] = -10 \log_{10}\frac{j(z)}{j(0)},
}
pri čemer sta $j(z)$ gostota svetlobnega toka po prepotovani razdalji $z$ in $j(0)$
vpadna gostota svetlobnega toka. Če se po enem prepotovanem
kilometru signal zmanjša na polovico, so izgube $3~\si{\decibel/\kilo\meter}$.

Pri izdelavi optičnih vlaken se najpogosteje uporablja kremenovo steklo, ki 
ima pri valovni dolžini $1,55~\si{\micro\meter}$ izgube okoli 
$0,2~\si{\decibel/\kilo\meter}$.  
Za primerjavo: navadno steklo ima pri vidni svetlobi atenuacijski koeficient okoli 
$1000~\si{\decibel/\kilo\meter}$.

\begin{figure}[ht]
\centering
\def\svgwidth{90truemm} 
\input{slike/10_FibAbs.pdf_tex} 
\caption{Izgube v vlaknu v odvisnosti od valovne dolžine: vijolična črta --
ultravijolična absorpcija, 
črna črta -- infrardeča absorpcija, 
zelena črta -- absorpcija na ionih OH$^{-}$ in modra črta --
izgube zaradi Rayleighovega sipanja. Z rdečo črto so označene skupne izgube.}
\label{FibAbs}
\end{figure}
Slika~\ref{FibAbs} prikazuje odvisnost izgub od valovne dolžine 
za dobro enorodovno vlakno.\footnote{~Slika povzeta po
A. Yariv in P. Yeh, {\it Photonics}, šesta izdaja, Oxford
University Press (2007).}
\index{Optično vlakno!enorodovno}
Celotne izgube (rdeča črta)
so sestavljene iz vrste različnih prispevkov. 
Pri kratkih valovnih dolžinah je absorpcija velika zaradi elektronskih prehodov
v steklu (vijolična črta).\index{Ultravijolično valovanje} 
Širina reže za SiO$_2$\index{SiO$_2$} je namreč okoli $8,9$~eV, 
kar ustreza valovni dolžini\index{Infrardeče valovanje}
okoli $140~\si{\nano\meter}$. Pri velikih valovnih dolžinah je absorpcija posledica
vibracijskih prehodov (črna črta). Čeprav so ti prehodi pri nižjih frekvencah, 
so vrhovi zelo široki in sežejo do okoli $1500~\si{\nano\meter}$. 
Absorpcija na nečistočah lahko ob pazljivi izdelavi postane skoraj v celotnem 
območju praktično zanemarljiva. 
Najbolj problematična nečistoča je voda oziroma ioni OH$^{-}$, ki imajo velik dipolni
moment in izrazito absorpcijo pri $1380~\si{\nano\meter}$ (zelena črta). Zelo pomemben prispevek k 
izgubam, posebej pri krajših valovnih dolžinah, je Rayleighovo sipanje na fluktuacijah 
gostote,\index{Izgube v optičnih vlaknih!Rayleighovo sipanje} 
saj je sorazmerno z $\lambda^{-4}$ (modra črta). 

\begin{remark}
Sipanje na fluktuacijah gostote predstavlja poglavitni del izgub v vlaknu. Na splošno
so gostotne fluktuacije v steklu zaradi amorfne zgradbe neizogibne, vendar so v vlaknih
navadno še precej večje. Med izdelavo steklo namreč močno segrejejo
(na okoli $2000~\si{\celsius}$), da lahko iz njega vlečejo vlakno, in termične 
fluktuacije gostote pri hitrem ohlajanju ostanejo zamrznjene v vlaknu. 
\end{remark}

Iz slike~\ref{FibAbs} je razvidno, da so skupne izgube najmanjše pri 
okoli $1,55~\si{\micro\meter}$, zato se to območje največ uporablja za prenos signalov
na velike razdalje. Izgube so tako majhne, da omogočajo prenos signala 
več sto kilometrov brez vmesnega ojačevanja. Teh izgub se  
ne bo dalo več kaj dosti izboljšati, saj so že zdaj na meji,
določeni s termičnimi fluktuacijami. Pri dolžini optičnih zvez tako glavna omejitev niso izgube,
ampak popačitev signala zaradi disperzije.

\begin{remark}
% Pri prenosu signalov z optičnimi vlakni vpeljemo različne pasove, ki ustrezajo 
% različnim valovnim dolžinam. Pri valovnih dolžinah $1260$--$1360~\si{\nano\meter}$ je tako imenovani
% pas O ({\it original}), ki so ga sprva uporabljali zaradi razpoložljivih virov svetlobe
% in nizke disperzije. Sledita pas E ({\it extended}, $1360$--$1460~\si{\nano\meter}$) in pas S 
% ({\it short}, $1460$--$1530~\si{\nano\meter}$). 
% Najširše uporabljan je pas C ({\it conventional}) pri valovnih dolžinah $1530$--$1565~\si{\nano\meter}$,
% sledita mu še pas L ({\it long}, $1565$--$1625~\si{\nano\meter}$) in pas U 
% ({\it ultralong}, $1625$--$1675~\si{\nano\meter}$).
Po optičnem \index{Razvrščanje po valovni dolžini|see {Multipleksiranje}}
vlaknu lahko prenašamo več signalov hkrati, če za vsakega posebej uporabimo
drugo valovno dolžino. Temu procesu pravimo razvrščanje po valovni dolžini
(WDM -- {\it Wavelength-Division Multiplexing})\index{Multipleksiranje}. 
Z njim dosežemo vzporeden prenos podatkov in hitrosti prenosa do 100~Tb/s.
 
Shematsko je tak način prenosa podatkov prikazan na sliki~\ref{WDM}.
Oddajniki (O) oddajo sunke svetlobe, ki se rahlo razlikujejo v valovni dolžini. 
Z multiplekserjem (M) signale iz različnih kanalov zberemo in jih usmerimo v 
enorodovno vlakno. Vlakno prenaša signal, vmes ga po potrebi ojačimo (A), 
nato pa z demultiplekserjem (DM)  razstavimo na posamezne kanale, ki jih 
zaznamo z ločenimi detektorji (D). Razlika v valovnih dolžinah med posameznimi signali 
je tipično $0,8~\si{nm}$. Zanimivo je tudi, da so (de)multiplekserji pasivni in za 
svoje delovanje ne potrebujejo elektrike.\footnote{~Glej npr. B. E. A. Saleh in M. C. Teich, 
{\it Fundamentals of Photonics}, druga izdaja, John Wiley \& Sons, Inc. (2007).}
\begin{figure}[ht]
\centering
\def\svgwidth{120truemm} 
\input{slike/10_WDM.pdf_tex} 
\caption{Shematski prikaz prenosa več signalov hkrati po enorodovnem vlaknu}
\label{WDM}
\end{figure}
\end{remark}
\subsection*{*Izgube v ukrivljenem vlaknu}
\index{Izgube v optičnih vlaknih!ukrivljeno vlakno}
V vseh primerih do zdaj smo privzeli, da je vlakno povsem ravno in 
da so mejne ploskve valovnega vodnika vzporedne. 
Kadar je vlakno ukrivljeno, del valovanja uhaja v plašč in 
izgube pri prenosu se povečajo. Te izgube postanejo znatne, 
kadar je krivinski radij ukrivljenega vlakna tipično centimeter ali manj. 
Poglejmo si pojav podrobneje na planparalelnem vodniku.
\index{Optični vodnik!planparalelni}

Naj bo vodnik dvodimenzionalna plast debeline $a$ z lomnim količnikom
$n_{1}$, ki je obdana s snovjo z lomnim količnikom $n_{2}$. Vodnik naj
zdaj ne bo raven, temveč ukrivljen s krivinskim radijem $R$, tako da tvori 
del kolobarja z notranjim polmerom $R-a/2$ in zunanjim polmerom $R+a/2$.
Privzamemo, da je $R\gg a$ (slika~\ref{fig:bend}). 

\begin{figure}[ht]
\centering
\def\svgwidth{50truemm} 
\input{slike/10_Krivina.pdf_tex} 
\caption{K izračunu izgub v ukrivljenem vodniku}
\label{fig:bend}
\end{figure}
Zapišemo Helmholtzevo enačbo (enačba~\ref{eq:Helmholtz}) v cilindrični \index{Helmholtzeva enačba}
geometriji
\begin{equation}
\frac{1}{r}\,\frac{\partial}{\partial r}\, r\,\frac{\partial E}{\partial r}
+\frac{1}{r^{2}}\,\frac{\partial^{2}E}{\partial\varphi^{2}}+k_{0}^{2}n^{2}\left(r\right)\, E=0,
\label{9.31}
\end{equation}
pri čemer ima $n\left(r\right)$ vrednost $n_{1}$ v sredici in $n_{2}$ v plašču. 
Pri tem ne pozabimo, da $r$ ni več radialna koordinata vlakna, ampak
označuje oddaljenost od središča krivine. Zanimajo nas rešitve oblike 
\begin{equation}
E(r, \varphi) =\psi\left(r\right)\, e^{im\varphi},
\label{9.32}
\end{equation}
pri čemer bomo privzeli, da je $\psi\left(r\right)$ znatna le v sredici. 

Naj bo $z=R\varphi$ dolžina loka vzdolž sredine sredice. Tedaj je faza nastavka
(enačba~\ref{9.32}) enaka $m\varphi = m z/R$ in valovno število vzdolž 
sredine vlakna $\beta = m/R$.
Ker je valovna dolžina svetlobe dosti manjša od $R$, je $m$ zelo veliko število. 
Funkcija $\psi$ zadošča enačbi 
\begin{equation}
\frac{d^{2}\psi}{dr^{2}}+\frac{1}{r}\,\frac{d\psi}{dr}+\left(k_{0}^{2}\, 
n^{2}\left(r\right)-\frac{m^{2}}{r^{2}}\right)\psi=0.
\label{9.33}
\end{equation}
Rešitve za $\psi$ so kombinacije Besslovih funkcij reda $m$, kar
zaradi velikosti $m$ ni posebno zanimivo. 

Dosti več bomo izvedeli, če se problema lotimo drugače. Namesto $r$
in $\varphi$ vpeljemo koordinati $x=r-R$ in $z=R\varphi$.
S tem preidemo nazaj na koordinate planparalelne plasti in iščemo popravke valovne
enačbe v sredici (\ref{9.3a}), ki so reda $1/R.$ Zapišemo
\begin{equation}
\frac{m^{2}}{r^{2}}=\frac{m^{2}}{\left(R+x\right)^{2}}\approx\frac{m^{2}}
{R^{2}}\,\left(1-2\,\frac{x}{R}\right)=\beta^{2}\left(1-2\,\frac{x}{R}\right)\!.
\label{9.34}
\end{equation}
Enačbo (\ref{9.33}) zdaj nadomestimo s približno enačbo za prečno obliko
polja 
\begin{equation}
\frac{d^{2}\psi}{dx^{2}}+\left(k_{0}^{2}\, n^{2}\left(r\right)-\beta^{2}\right)\,\psi+\frac{1}{R}\,
\left(\frac{d\psi}{dx}+2\,\beta^{2}x\,\psi\right)=0.
\label{9.35}
\end{equation}
Člen, ki vsebuje prvi odvod $d \psi/d x$, lahko odpravimo z nastavkom 
\begin{equation}
\psi(x) = e^{-x/2R} \zeta(x)
\end{equation}
in dobimo enačbo
\begin{equation}
\frac{d^{2}\zeta}{dx^{2}}+\left(k_{0}^{2}\, n^{2}\left(r\right)-\beta^{2}-\frac{1}{4R^2}\right)\,\zeta
+ \frac{2\beta^{2}}{R}\,x\,\zeta=0.
\label{9.35a}
\end{equation}
Enačba je podobna enačbi za izračun lastnih rodov v planparalelnem
vodniku (\ref{9.3a}), pri čemer se $\beta^2$ poveča za $1/4R^2$. Poleg tega je prisoten
člen, ki je linearen v $x$. Če ponovno naredimo primerjavo
med lastnimi načini v valovnem vodniku in stanji delca, ujetega v končno potencialno jamo, 
ta člen ustreza potencialni energiji delca v konstantnem zunanjem električnem 
polju (slika~\ref{fig:tunel}). Podobno kot ujeti delci uhajajo iz potencialne jame
v prisotnosti zunanjega polja (tunelirajo), uhaja tudi svetloba iz ukrivljenega vlakna.
\begin{figure}[ht]
\centering
\def\svgwidth{110truemm} 
\input{slike/10_Tunel.pdf_tex} 
\caption{Lastni načini širjenja svetlobe po ravnem vodniku so analogni stanjem 
delca v končni potencialni jami (a). Načini širjenja po ukrivljenem vodniku
so podobni stanjem delca v konstantnem zunanjem električnem polju (b). Podobno
kot delci zaradi spremenjenega potenciala tunelirajo, uhaja svetloba iz ukrivljenega vlakna.}
\label{fig:tunel}
\vglue-3truemm
\end{figure}

Po analogiji s kvantno mehaniko, kjer ukrivljenost vodnika ustreza jakosti električnega polja,
lahko atenuacijski koeficient v ukrivljenem vlaknu zapišemo približno kot
\begin{equation}
A \propto e^{-CR},
\end{equation}
pri čemer je $C$ konstanta, odvisna od lomnih količnikov sredice in plašča, od polmera vlakna 
ter od valovne dolžine potujoče svetlobe.\footnote{~Glej npr. A. Yariv in 
P. Yeh, {\it Photonics}, šesta izdaja, Oxford
University Press (2007).} 


\subsection*{Ojačevanje v vlaknih}
\index{Optično ojačevanje!v vlaknih}
Zaradi izgub pri prenosu signalov na več tisoč kilometrov dolge razdalje 
je treba signal med prenosom ojačevati. To lahko naredimo elektronsko, kjer optični signal 
pretvorimo v električnega, tega ojačimo in ga nato pretvorimo nazaj v optičnega. 
Precej bolj priročna rešitev je optično ojačevanje v vlaknu 
samem.\footnote{~Glej npr. A. Ghatak in K. Thyagarajan, {\it Introduction to Fiber Optics}, 
Cambridge University Press (1997).} 
\index{Optično vlakno!dopirano z erbijem}
\index{Erbij}

V ta namen se najpogosteje uporablja vlakno, dopirano z erbijevimi 
ioni\footnote{~EDFA -- {\it Erbium-Doped Fiber Amplifier}, 
ojačevalnik na vlakno, dopirano z erbijem.}. 
Na določenih razdaljah (na okoli $100~\si{\kilo\meter}$) svetlobo iz navadnega vlakna 
sklopimo v vlakno, v katerem so erbijevi ioni.
S črpalnim laserjem erbijeve ione vzbudimo, da dosežemo obrnjeno zasedenost. \index{Obrnjena zasedenost}
Ko na dopirani del vlakna vpade svetlobni sunek z valovno dolžino okoli 
$1550~\si{\nano\meter}$, pride do stimulirane emisije in \index{Stimulirano sevanje}
sunek se ojači. To je povsem enak postopek ojačevanja svetlobe, kot ga poznamo iz 
delovanja laserja, le da tukaj svetloba ni ujeta v resonator, ampak se postopoma 
ojačuje vzdolž vlakna. Pri tem se intenziteta črpalnega laserja postopoma zmanjšuje,
kar omejuje dolžino, na kateri se signal ojačuje.
Spektralna širina ojačenja je zaradi sklopitev z ioni v steklu 
razmeroma široka, tudi $40~\si{\nano\meter}$. To pomeni, da se hkrati ojačujejo signali različnih 
valovnih dolžin, kar je še posebej uporabno pri prenosu več signalov naenkrat.

V praksi se uporabljajo vlakna, v katerih je delež erbijevih ionov okoli $\sim 10^{-4}$. 
Črpalni laser je polprevodniški laser, ki 
deluje pri valovni dolžini okoli $900~\si{\nano\meter}$ ali $1480~\si{\nano\meter}$ 
z močjo okoli $20$--$100~\si{\milli\watt}$. Na ta način lahko v $10$--$30~\si{\meter}$
dolgih odsekih vlaken dosežemo več 1000-kratno ojačenje ($30$--$40~\si{\decibel}$), kar 
je dovolj za kompenzacijo izgub.

\section{Sklopitev svetlobe v optične vodnike}
\index{Sklopitev v optično vlakno}
Do zdaj smo govorili o svetlobi, ki potuje po valovnem vodniku ali optičnem vlaknu. Kako pa 
svetlobo sploh sklopimo v vodnik? Poznamo več načinov sklopitve. Obravnavali bomo 
čelno sklopitev, bočno sklopitev s prizmo in bočno sklopitev s periodično strukturo.
Prvi način navadno uporabljamo pri 
cilindričnih vlaknih, medtem ko druga dva načina najpogosteje pri planarnih valovodnih strukturah.

\subsection*{Čelna sklopitev}
\index{Sklopitev v optično vlakno!čelna sklopitev}
Sklopitev svetlobe v večrodovno vlakno lahko obravnavamo geometrijsko, kot smo to naredili
na začetku poglavja (slika~\ref{fig:vodnik}). Izračunali smo, da je največji 
vpadni kot, pod katerim se svetloba še sklopi v vlakno, določen z numerično odprtino 
vlakna $\sin \alpha_{\mathrm{max}}= NA$  (enačba~\ref{10NAa}).

Za natančnejši izračun izkoristka sklopitve svetlobe v optično vlakno vpeljemo 
tako imenovani prekrivalni integral, ki pove, kolikšen delež vpadne svetlobe 
z jakostjo električnega polja $E(r, \varphi)$ se sklopi z izbranim rodom 
vlakna.\index{Prekrivalni integral}
To naredimo tako, da vpadni val razvijemo po lastnih rodovih vlakna, in iz ortogonalnosti
sledi prekrivalni integral, ki ga moramo seveda ustrezno normirati. 

Za sklopitev v rod, 
označen z indeksoma $n$ in $m$, zapišemo prekrivalni integral 
kot\footnote{~Glej npr. C. R. Pollock, {\it Fundamentals of Optoelectronics}, Irwin (1995).}
\boxeq{10:prekint}{
\eta = \frac{|\int E(x,y) E^*_{n,m}(x,y) dx\, dy|^2}
{\left(\int |E(x, y)|^2 dx\,dy \right) \left(\int |E_{n,m}(x, y)|^2 
dx\, dy \right)}.
}
Točen račun prekrivalnega integrala je na splošno precej zapleten, saj vsebuje integrale
Besslovih funkcij. V primeru osnovnega roda račun močno poenostavimo, če 
prečno odvisnost polja nadomestimo z Gaussovim profilom z ustreznim 
efektivnim polmerom (po enačbi~\ref{Marcuse}). 

\subsection*{Bočna sklopitev}
\index{Sklopitev v optično vlakno!bočna sklopitev}
Neposredna sklopitev svetlobe v optični vodnik prek plašča ni mogoča. V vodniku je namreč
lomni količnik sredice vedno večji od lomnega količnika plašča, zato dovolj velikega vstopnega 
kota, pod katerim bi se svetloba ujela v sredico, ni mogoče doseči. Za sklopitev prek stranice 
zato uporabimo drugačen pristop, navadno s prizmo ali s periodično strukturo na 
vlaknu.\footnote{~Glej npr. B. E. A. Saleh in M. C. Teich, 
{\it Fundamentals of Photonics}, druga izdaja, John Wiley \& Sons, Inc. (2007).}

V prvem primeru uporabimo prizmo (slika~\ref{fig:coupler}\,a). Lomni količnik prizme
je pri tem večji od lomnega količnika plašča $n_p > n_2$.
Vhodni žarek vpada na prizmo, se ob prehodu vanjo lomi, nato  se na stranici, ki je vzporedna
z vodnikom, totalno odbije.\index{Totalni odboj} 
V tankem vmesnem območju med prizmo in sredico vodnika se pojavi evanescentni\index{Evanescentno polje}
val s komponento valovnega vektorja $\beta_v  = k_0 n_p \sin \alpha$ v 
smeri vzporedno z vodnikom. 

Pogoj za uspešno 
sklopitev v vlakno je ujemanje vzdolžne komponente valovnega vektorja vpadne svetlobe $\beta_v$ 
z vzdolžno komponento valovnega vektorja $\beta_n$ tistega rodu, ki ga želimo vzbuditi. 
S spreminjanjem vpadnega kota $\alpha$ spreminjamo $\beta_v$ in v vodniku vzbujamo različne rodove. 
Pri tem mora biti razdalja med prizmo in vodnikom dovolj majhna (tipično reda valovne dolžine svetlobe), 
da se v valovod ob izpolnjenem pogoju ujemanja faze sklopi znaten delež vpadne svetlobe.

\begin{figure}[ht]
\centering
\def\svgwidth{140truemm} 
\input{slike/10_coupler.pdf_tex} 
\caption{Bočna sklopitev svetlobe v optični vodnik s prizmo (a) in s periodično strukturo (b)}
\label{fig:coupler}
\end{figure}
Tudi sklopitev s periodično strukturo na valovnem vodniku (slika~\ref{fig:coupler}\,b) deluje na 
ujemanju vzdolžnih komponent valovnega vektorja vpadnega vala in valovnega vektorja ustreznega rodu.
Ko vpade val pod kotom $\alpha$ glede na valovni vodnik, periodična struktura na vodniku 
spremeni njegovo fazo za večkratnik $2 \pi z/\Lambda$, pri čemer je $\Lambda$ perioda strukture.
Če dosežemo, da se komponenta novega valovnega vektorja $\beta = k_0 n_2 \sin \alpha+ 
2 \pi/\Lambda$ izenači s  komponento valovnega vektorja za izbrani rod v vlaknu, 
se vpadna svetloba sklopi v vlakno.

\begin{remark}
 Oba opisana načina za sklopitev svetlobe v vlakno uporabljamo tudi za sklopitev svetlobe 
 iz vlakna, pri čemer mora biti prav tako izpolnjen pogoj ujemanja faz. 
 Sklapljanje svetlobe skozi prizmo je uporabno tudi za raziskave tankih plasti snovi. 
 Iz pogoja za ujemanje faz lahko določimo lastnosti tanke plasti, na primer njen lomni količnik. 
\end{remark}

\section{Sklopitev med optičnimi vodniki}
\index{Sklopitev med valovodi}
\subsection*{Čelna sklopitev dveh vlaken}
\index{Izgube v optičnih vlaknih!spoj dveh vlaken}
Pri telekomunikacijah z optičnimi vodniki so spoji med posameznimi vodniki neizogibni.
V idealnem primeru sta vodnika povsem enaka in se natančno stikata, tako da na spoju
ne prihaja do dodatnih izgub ali popačenja signala. Čim se pojavijo odstopanja 
v velikosti polmera sredice, razlike v vrednostih lomnih količnikov ali nenatančna poravnava 
sredice, na spoju pride do izgub. Tipično znašajo izgube na spoju vlaken do okoli 
$0,2$--$0,5~\si{\decibel}$.

Omejimo se na spoj dveh enorodovnih vlaken, v katerih krajevni del jakosti 
električnega polja osnovnega rodu zapišemo kot
\begin{equation}
E(r, \varphi, z)=\psi(r, \varphi) e^{i\beta z}.
\end{equation} 
Podobno kot smo zapisali prekrivalni integral pri sklopitvi svetlobe v vlakno (enačba~\ref{10:prekint}),
vpeljemo prekrivalni integral za izračun sklopitve med dvema vlaknoma,\index{Prekrivalni integral}
ki pove, kolikšen delež svetlobe moči iz prvega vlakna 
se sklopi v osnovni rod v drugem vlaknu. Sklopitveni faktor je
\boxeq{10:overlap}{
\eta = \frac{|\int \psi_1(r, \varphi) \psi_2^*(r, \varphi) r\, dr\, d\varphi|^2}
{\left(\int |\psi_1|^2 r\, dr\, d\varphi \right) \left(\int |\psi_2|^2 r\, dr\, d\varphi \right)},
}
pri čemer z indeksom $1$ označimo osnovni rod v prvem vlaknu in z indeksom $2$ v drugem. 
Tudi tukaj račun pogosto poenostavimo in namesto Besslovega profila uporabimo 
Gaussov profil z ustreznim efektivnim polmerom snopa (enačba~\ref{Marcuse}).\index{Gaussov 
snop!efektivni polmer}

Izračunajmo za primer sklopitveni faktor in izgube na spoju dveh vlaken z rahlo 
različnima polmeroma. Po Marcusejevi formuli najprej določimo efektivna polmera Gaussovih snopov
v obeh vlaknih $w_1$ in $w_2$. Prečni profil v vlaknih je potem\index{Marcusejeva formula}
\begin{equation}
\label{eq:pripsi1}
\psi_{1} = A_{1} e^{-r^2/w_{1}^2}
\end{equation}
in 
\begin{equation}
\label{eq:pripsi2}
\psi_{2} = A_{2} e^{-r^2/w_{2}^2}.
\end{equation}
Vstavimo nastavka (enačbi~\ref{eq:pripsi1} in \ref{eq:pripsi2}) 
v prekrivalni integral~(enačba~\ref{10:overlap}) in dobimo
\begin{equation}
\eta = \frac{|\int A_1 \, e^{-r^2/w_1^2}\, A_2\, e^{-r^2/w_2^2}\, 2 \pi\, r\, dr|^2}
{\left(\int A_1^2 \,e^{-2r^2/w_1^2} \, 2 \pi \, r\, dr \right) \left(\int A_2^2\, 
e^{-2r^2/w_2^2}\, 2 \pi \, r\, dr \right)},
\label{10:prekintw}
\end{equation}
od koder sledi
\begin{equation}
\eta = \frac{4 w_1^2 w_2^2}{(w_1^2+w_2^2)^2}.
\label{10:w1w2}
\end{equation}
Kadar sta polmera vlaken enaka, je prekrivanje popolno in $\eta = 1$. Z naraščajočo razliko
med polmeroma vrednost $\eta$ pojema. Pri tem
ni pomembno, ali ima večji polmer prvo ali drugo vlakno, v obeh primerih se signal oslabi. 
Intuitivno razumemo, da se signal izgubi pri prehodu iz večjega v manjše vlakno, 
vendar drži tudi obratno, saj se v širšem končnem vlaknu poleg osnovnega vzbudijo
tudi višji rodovi. 

Pri prehodu iz vlakna z $w = 10~\si{\micro\meter}$ v vlakno
s polmerom $w = 8~\si{\micro\meter}$ (ali obratno) je sklopitveni faktor (oziroma
razmerje med prepuščeno in vpadno intenziteto svetlobe) enak $0,95$. Po enačbi~(\ref{dB})
so izgube za izračunano sklopitev enake $0,21~\si{\decibel}$. 

\begin{definition}
Izračunaj prekrivalni integral (enačba~\ref{10:prekintw}) in izpelji enačbo~(\ref{10:w1w2}).
Pokaži tudi, da je sklopitveni faktor za dve enaki vzporedni vlakni, ki sta iz osi izmaknjeni
za $\Delta$, enak
\begin{equation}
\eta = \exp \left( - \frac{\Delta^2}{w^2}\right)\!.
\end{equation}

\end{definition}
\subsection*{Vzdolžna sklopitev}
Ob prenosu signala po optičnem valovodu večina energijskega toka potuje po sredici,
vendar energijski tok seže tudi izven nje, v plašč (enačba~\ref{confinement}). 
Če sta dva vzporedna valovoda dovolj blizu, da\index{Evanescentno polje}
se evanescentni električni polji enega in drugega vodnika v plašču prekrivata, se vodnika 
sklopita in energijski tok se prenese iz enega vodnika v drugega 
(slika~\ref{fig:fcoupler}). 
\begin{figure}[ht]
\centering
\def\svgwidth{95truemm} 
\input{slike/10_fcoupler.pdf_tex} 
\caption{Sklopitev med dvema vzporednima vodnikoma.}
\label{fig:fcoupler}
\vglue-3truemm
\end{figure}

Za podrobnejšo obravnavo bi morali zapisati Maxwellove enačbe z ustreznimi robnimi pogoji 
in jih rešiti za sklopljen primer dveh vzporednih vodnikov. Tak račun je izredno zapleten, 
zato bomo uporabili približek šibke sklopitve in predpostavili, da so rodovi v vodnikih taki, 
kot če bi vodniki ne bili sklopljeni.
Sklopitev torej ne bo spremenila oblike lastnih rodov, bo pa spremenila njihove amplitude, ki 
bodo tako postale odvisne od vzdolžne koordinate $z$.\footnote{~Glej npr. B. E. A. Saleh in M. C. Teich, 
{\it Fundamentals of Photonics}, druga izdaja, John Wiley \& Sons, Inc. (2007).}

Imejmo dva enorodovna vodnika debeline $a$ z lomnima količnikoma sredice $n_1$ in $n_2$, 
med njima in okoli njiju pa naj bo snov z lomnim količnikom $n_0$. Širina 
reže med vodnikoma naj bo $2d$.
Potem zapišemo jakosti električnega polja v prvem in drugem vodniku kot
\begin{equation}
E_1(x,z) = A(z) \psi_1(x) e^{i \beta_1 z}
\end{equation}
in
\begin{equation}
E_2(x,z) = B(z) \psi_2(x) e^{i \beta_2 z},
\end{equation}
pri čemer se $A(z)$ in $B(z)$ le počasi spreminjata s koordinato $z$.

Skupna jakost električnega
polja, ki je v našem približku kar vsota prispevkov $E_1$ in $E_2$, 
mora zadoščati Helmholtzevi enačbi
(enačba~\ref{eq:Helmholtz})
\begin{equation}
\nabla^{2}E(x,z)+k_0^{2}n(x)^2 E(x,z) =0.
\end{equation}
Z $n(x)$ smo označili prečno odvisnost lomnega količnika.
Vstavimo nastavek za jakost električnega polja v Helmholtzevo enačbo in dobimo
\begin{align}
A e^{i \beta_1 z}\left(\frac{\partial^2\psi_1(x)}{\partial x^2} - \beta_1^2\psi_1 + k_0^2
n(x)^2 \psi_1 \right)
&+ 
B e^{i \beta_2 z}\left(\frac{\partial^2\psi_2(x)}{\partial x^2} - \beta_2^2\psi_2 + k_0^2
n(x)^2 \psi_2 \right)+ \nonumber \\ 
2 i \beta_1 \frac{\partial A}{\partial z}\psi_1 e^{i \beta_1 z}
+
2 i \beta_2 \frac{\partial B}{\partial z}\psi_2 e^{i \beta_2 z} &= 0.
\end{align}
Zaradi počasnega spreminjanja  smo člena z drugim odvodom $\partial^2 A/\partial z^2$ in $\partial^2 B/\partial z^2$ zanemarili. 

Zapišemo enačbi za nemoteni funkciji $\psi$
\begin{equation}
\frac{\partial^2\psi_1(x)}{\partial x^2} + \left(k_0^2n_1(x)^2-\beta_1^2\right) \psi_1 =0
\end{equation}
in
\begin{equation}
 \frac{\partial^2\psi_2(x)}{\partial x^2} + \left(k_0^2n_2(x)^2-\beta_2^2\right) \psi_2 =0.
\end{equation}
Lomna količnika $n_1(x)$ in $n_2(x)$ sta tudi tukaj funkciji prečne koordinate. Naj bo $n_1(x)$  
povsod enak $n_0$, razen v sredici prvega vodnika, kjer je $n_1$, in naj bo 
$n_2(x)$ povsod enak $n_0$, razen v sredici drugega vodnika, kjer je enak $n_2$. Sledi
\begin{align}
A e^{i \beta_1 z}k_0^2\left(n(x)^2 -n_1(x)^2\right)\psi_1 
+ 
B e^{i \beta_2 z}k_0^2\left(n(x)^2 -n_2(x)^2\right)\psi_2 &+ \nonumber\\
2 i \beta_1 \frac{\partial A}{\partial z} \psi_1 e^{i \beta_1 z}
+
2 i \beta_2 \frac{\partial B}{\partial z} \psi_2 e^{i \beta_2 z} &= 0.
\end{align}
Enačbo pomnožimo s kompleksno konjugirano vrednostjo $\psi_1^*$ in integriramo po $x$.
Upoštevamo, da se funkciji $\psi_1$ in $\psi_2$ le malo prekrivata, in dobimo
\begin{equation}
\frac{\partial A}{\partial z} = i A K_{11}+i B e^{i(\beta_2-\beta_1)z} K_{12},
\end{equation}
pri čemer sta
\begin{equation}
K_{11}= \frac{k_0^2}{2 \beta_1}\int\psi_1^*\psi_1 (n^2-n_1^2)dx \qquad \mathrm{in} \qquad 
K_{12}= \frac{k_0^2}{2 \beta_1}\int\psi_1^*\psi_2 (n^2-n_2^2)dx.
\end{equation}
Koeficient $K_{11}$ določa spremembo faze v vlaknu zaradi prisotnosti 
drugega vlakna in ga lahko zanemarimo. Tako ostane samo 
\begin{equation}
\frac{\partial A}{\partial z} = i B e^{i(\beta_2-\beta_1)z} K_{12} \qquad \mathrm{in}
\qquad \frac{\partial B}{\partial z} = i A e^{-i(\beta_2-\beta_1)z} K_{21}.
\label{10_B}
\end{equation}
Prvo enačbo odvajamo po $z$, upoštevamo drugo in zapišemo
\begin{equation}
\frac{\partial^2 A}{\partial z^2}-i \Delta \beta \frac{\partial A}{\partial z} + K_{12}K_{21}A = 0,
\label{10_part}
\end{equation}
pri čemer je $\Delta \beta = \beta_2 - \beta_1$. Enačbo~\ref{10_part} rešujemo z nastavkom
\begin{equation}
A = e^{i \Delta \beta z/2}\left( a_1 e^{i \gamma z} + a_2 e^{-i \gamma z}\right)!.
\label{10_nastavek}
\end{equation}
\vglue-3truemm
\begin{definition}
\vglue-3truemm
 Pokaži, da  nastavek (enačba~\ref{10_nastavek}) reši enačbo~(\ref{10_part}), in pokaži,
 da med parametri enačb velja naslednja zveza, pri čemer je $K = \sqrt{K_{12}K_{21}}$
 \begin{equation}
 \gamma^2 = K^2 + \frac{\Delta \beta ^2}{4}.
 \label{10kgamma}
 \end{equation}
 Uporabi enačbi~(\ref{10_B}) in pokaži, da je rešitev za amplitudo $B$
 enaka izrazu v enačbi~(\ref{10_BB}).
\end{definition}
Ko poznamo $A$, lahko z uporabo enačb~(\ref{10_B}) izračunamo še $B$
\begin{equation}
B = \frac{1}{K_{21}}
e^{-i \Delta \beta z/2}\left(\left(\frac{\Delta \beta}{2} +\gamma \right) a_1 e^{i \gamma z} + 
\left(\frac{\Delta \beta}{2} -\gamma \right)a_2 e^{-i \gamma z}\right)\!.
\label{10_BB}
\end{equation}
Naj bo $A(z=0) = A_0$ in $B(z=0)=0$. To pomeni, da potuje svetloba na začetku
le po prvem vlaknu, nato se sklopi v drugega. S tema začetnima pogojema zapišemo izraza 
za $A$ in $B$
\begin{equation}
 A = A_0 e^{i \Delta \beta z/2} \left( \cos(\gamma z) - 
\frac{i \Delta \beta}{2 \gamma}\sin(\gamma z) \right)
\quad \mathrm{in} \quad 
B =  A_0 e^{-i \Delta \beta z/2} \frac{i K_{21}}{\gamma}\sin(\gamma z).
\end{equation}
Moč, ki se pretaka po posameznem vlaknu, je tako z upoštevanjem zveze~(\ref{10kgamma})
\begin{equation}
P_1 = P_0 \left( \cos^2(\gamma z) + \frac{\Delta \beta^2}{4 \gamma^2}\sin^2(\gamma z) \right)
= P_0 \left( 1 - \frac{K^2}{\gamma^2}\sin^2(\gamma z) \right)
\end{equation}
in
\begin{equation}
P_2 =  P_0\, \frac{K^2}{\gamma^2}\sin^2(\gamma z).
\end{equation}
Privzeli smo, da velja $|K_{12}|=|K_{21}|= K$. 
Obe odvisnosti sta oscilirajoči in svetloba se periodično pretaka med vlaknoma
s periodo $\pi/\gamma$ (slika~\ref{fig:foscil}). Amplituda prenosa je 
odvisna od sklopitvenega faktorja $K$ in ujemanja valovnih 
števil v obeh vlaknih. Večji koeficient $K$
in manjše odstopanje $\Delta \beta$ vodita v večji prenos svetlobnega toka v drugo vlakno. 
\begin{figure}[ht]
\centering
\def\svgwidth{140truemm} 
\input{slike/10_Coupler.pdf_tex} 
\caption{Prenos svetlobnega toka med dvema sklopljenima vodnikoma. V prvem primeru (a) sta
vodnika različna, v drugem primeru (b) pa enaka in prenos je popoln.}
\label{fig:foscil}
\end{figure}

Če sta vlakni 
enaki, je $\Delta \beta = 0$ in $\gamma = K$, tako da pride do popolnega prenosa
svetlobnega toka iz enega vlakna v drugo in seveda tudi obratno.
Takrat veljata enačbi
\boxeq{10_couplfib}{
P_1 &= P_0 \cos^2 (\gamma z) \qquad \mathrm{in}\\
P_2 &= P_0 \sin^2 (\gamma z).
}
Na ta način lahko v drugo vlakno sklopimo poljuben delež vpadne svetlobe. 
Prenos celotnega svetlobnega toka iz enega v drugo vlakno nastopi pri dolžini sklopitve $L = \pi/2 \gamma$. 
Pri dolžini $L = \pi/4 \gamma$ sklopimo v drugo vlakno eno polovico gostote vpadnega svetlobnega toka in govorimo o 3-dB sklopitvi. \index{Sklopitev med valovodi!3-dB sklopitev}

\begin{remark}
 Pri izbrani dolžini sklopitve med vlaknoma je intenziteta svetlobe, ki preide v drugo vlakno,
 močno odvisna od parametra $\gamma$, torej od prekrivalnega integrala in od razlike $\Delta \beta$. 
 Z rahlim spreminjanjem parametrov, na primer lomnega 
 količnika enega od vlaken, lahko spreminjamo delež svetlobe v drugem vlaknu. V ta 
 namen pogosto uporabimo elektro-optični pojav in s spreminjanjem priključene napetosti na
 enem od vlaken natančno določimo delež svetlobe, ki preide v drugo vlakno. \index{Elektro-optični pojav}
\end{remark}

\section{*Vpliv spremembe lomnega količnika vlakna na širjenje svetlobe}
Sprememba lomnega količnika sredice ali plašča vlakna povzroči spremembo
valovnega števila $\beta$ za izbran rod. V enorodovnih vlaknih je to
mogoče izkoristiti za izdelavo senzorjev, na primer temperature ali
tlaka. Spremembo valovnega števila, ki je posledica zunanjih vplivov,
izmerimo prek spremembe faze valovanja na izhodu iz vlakna
z ustrezno sestavljenim interferometrom. Ker je dolžina vlakna lahko
velika (v nekaj centimetrov velik tulec lahko brez težav navijemo
kilometre vlakna), je celotna sprememba faze velika že pri majhnih
spremembah merjene količine. Po drugi strani že majhna sprememba 
valovnega števila povzroča neželene spremembe faze in odboje pri prenosu 
informacij.

V tem razdelku zato poglejmo, kako se spremeni valovno število pri
dani spremembi lomnega količnika in koliko svetlobe se v takem primeru odbije.

Obravnavajmo rod z vzdolžno komponento valovnega vektorja $\beta_{lm}$ in obliko prečnega
profila $\psi_{lm}\left(r,\varphi\right)\!.$ Ta mora zadoščati Helmholtzevi enačbi 
(enačba~\ref{eq:Helmholtz})
\begin{equation}
\nabla_{\bot}^{2}\psi_{lm}+\left(\epsilon(r)k_{0}^{2}-\beta_{lm}^{2}\right)\psi_{lm}=0.
\label{9.22}
\end{equation}
Naj se dielektrična konstanta na delu vlakna spremeni za $\delta\epsilon.$
Posledično se na tem mestu spremenita tudi valovno število $\beta=\beta_{lm}+\delta\beta$
in prečna oblika $\psi=\psi_{lm}+\delta\psi.$ 

Tudi popravljena funkcija
$\psi$ mora zadoščati enačbi (\ref{9.22}), zato za perturbacijo velja
\begin{equation}
\nabla_{\bot}^{2}\delta\psi+\left(\epsilon(r)k_{0}^{2}-\beta_{lm}^{2}\right)\delta\psi+
\delta\epsilon\, k_{0}^{2}\psi_{lm}=2\beta_{lm}\delta\beta\,\psi_{lm},
\label{9.23}
\end{equation}
pri čemer smo zanemarili produkte majhnih količin. Množimo obe strani
enačbe s $\psi_{lm}^{*}$, integriramo po preseku vlakna in dobimo
\begin{eqnarray}
 &  & \int\psi_{lm}^{*}\nabla_{\bot}^{2}\delta\psi\,
 dS+\int\left(\epsilon(r)k_{0}^{2}-\beta_{lm}^{2}\right)
 \delta\psi\,\psi_{lm}^{*}dS+k_{0}^{2}\int\delta\epsilon\,\left|\psi_{lm}\right|^{2}dS = \nonumber\\
 & & 2\beta_{lm}\,\delta\beta\int\left|\psi_{lm}\right|^{2}dS.
 \label{9.24}
\end{eqnarray}
Prvi člen na levi preoblikujmo z uporabo zvez 
\beq
\int(u\,\nabla_{\bot}^{2}v-v\nabla_{\bot}^{2}u)\,
dS=\int\nabla_{\bot}\cdot(u\nabla_{\bot}v-v\nabla_{\bot}u)\, 
dS=\oint (u\,\nabla_{\bot}v-v\,\nabla_{\bot}u)\cdot d\mathbf{s}.
\eeq
Funkciji $\psi_{lm}$ in $\delta\psi$ opisujeta vodene valove, zato
morata iti njuni vrednosti za velike $r$ proti nič. Posledično gre
proti nič tudi integral po krivulji $d\mathbf{s}$ in velja 
\beq
\int\psi_{lm}^{*}\nabla_{\bot}^{2}\delta\psi\,
dS=\int\delta\psi\nabla_{\bot}^{2}\psi_{lm}^{*}\, dS.
\eeq
Funkcija $\psi_{lm}^{*}$ zadošča enačbi (\ref{9.22}), zato se v enačbi (\ref{9.24})
prvi in drugi člen odštejeta. Iskan popravek k valovnemu številu je tako 
\begin{equation}
\delta\beta=\frac{k_{0}^{2}\int\delta\epsilon\,\left|\psi_{lm}\right|^{2}dS}{2\,
\beta_{lm}\int\left|\psi_{lm}\right|^{2}dS}.
\label{9.25}
\end{equation}
\begin{remark}
Ta rezultat je seveda analogen kvantnomehanskemu rezultatu, ki sledi iz 
teorije motenj za spremembo energije lastnega stanja delca pri majhni 
spremembi Hamiltonovega operatorja. Rezultat je tudi intuitivno razumljiv: v
najnižjem redu je $\delta\beta$ sorazmerna s uteženim povprečjem
$\delta\epsilon$, pri čemer je utež $\psi_{lm}$.
\end{remark}

Sprememba valovnega števila $\delta \beta$ v delu vlakna ne povzroči le spremembe faze, 
ampak tudi delni odboj.
To je le nekoliko druga oblika odboja na (zvezni ali ostri) meji 
dveh dielektrikov ali, še splošneje, odboja valovanja na območju,
kjer se spremeni fazna hitrost valovanja.
Amplitudo odbitega valovanja, ki se odbije na območju spreminjajočega $\beta$, 
najpreprosteje dobimo z uporabo
enačbe za odboj na meji dveh dielektrikov pri pravokotnem
vpadu. Odbita amplituda je tedaj (enačba~\ref{eq:Fresnel1})
\begin{equation}
E_{r}=\frac{n_{1}-n_{2}}{n_{1}+n_{2}}E_{0},
\label{9.26}
\end{equation}
pri čemer $n_1$ in $n_2$ označujeta nespremenjen in rahlo spremenjen lomni
količnik sredice vlakna. 

Mislimo si, da je sprememba $\beta$ na delu vlakna sestavljena iz
majhnih stopničastih sprememb $\Delta\beta_{i}$ na intervalih $\Delta z$.
Za ravno valovanje je sprememba fazne hitrosti sorazmerna s spremembo
lomnega količnika, zato iz enačbe~(\ref{9.26}) sledi, da je odbito valovanje
na stopničasti spremembi $\Delta\beta_{i}$ enako
\begin{equation}
\Delta E_{i}=\frac{\Delta\beta_{i}}{2\,\beta}\, E_{0}.
\label{9.27}
\end{equation}
Privzeli smo, da je delež odbitega valovanja tako majhen, da ni treba upoštevati 
spremembe amplitude vpadnega vala $E_{0}$. Celotno odbito valovanje je vsota 
prispevkov na posameznih stopnicah $\Delta\beta_{i}$,
pri čemer moramo upoštevati še različne faze delno odbitih valovanj
\begin{equation}
E_{r}=\sum\frac{\Delta\beta_{i}}{2\,\beta}\, e^{2i\beta z_{i}}\, 
E_{0}=\frac{1}{2\,\beta}\sum\frac{d\beta}{dz}\, e^{2i\beta z_{i}}\Delta z\, E_{0}.
\label{9.28}
\end{equation}
Preidemo z vsote na integral in zapišemo amplitudo odbitega valovanja
\begin{equation}
E_{r}=\frac{E_{0}}{2\,\beta}\,\int\frac{d\beta}{dz}\, e^{2i\beta z}dz.
\label{9.29}
\end{equation}
Za primer poglejmo linearno spremembo lomnega količnika in linearno spremembo 
valovnega števila za $\Delta\beta$ na razdalji $L$. Krajši račun pokaže, da je 
delež intenzitete odbitega valovanja 
\begin{equation}
\frac{I_{r}}{I_{0}}=\left( \frac{\Delta\beta}{2 \beta}\,\frac{\sin\beta L}{\beta L}\right)^2\!.
\label{9.30}
\end{equation}
Odbojnost je največja, kadar je $L \ll 1/\beta$,
torej kadar je sprememba $\beta$ ostra stopnica. Čim počasnejša je
sprememba, tem manj je odboja. Kadar je $\sin\beta L=0$, delni 
odboji destruktivno interferirajo in odbojnost je enaka nič.

\begin{definition}
Naj se valovno število ob prehodu spreminja kot funkcija erf
\begin{equation}
\beta (z)= \beta_0 + \frac{2\Delta \beta}{\sqrt{\pi}} \int_0^{z/a} e^{-t^2}dt.
\end{equation}
Pokaži, da je amplituda 
odbitega valovanja za majhne spremembe $\Delta \beta$ enaka
\begin{equation}
\frac{E_r}{E_0} = \frac{\Delta \beta}{\beta_0}e^{-a^2\beta_0^2}.
\end{equation}
Po pričakovanju je odbojnost največja pri $a\to 0$, to je v primeru ostre stopnice.
\end{definition}

%Končano

%-------------------------------------------------------------------------------
%	CHAPTER 9
%-------------------------------------------------------------------------------

\input{Chapter_11_Detektorji}

%Final
%-------------------------------------------------------------------------------
%	CHAPTER 10
%-------------------------------------------------------------------------------

\chapterimage{AOModulator.jpg} % Chapter heading image

\chapter{Modulacija svetlobe}
\label{chap:modulacija}
Spoznali smo, kako svetloba nastane, kako se širi po optičnih vlaknih in kako
jo zaznavamo. Poglejmo zdaj, kako svetlobni signal spreminjamo -- moduliramo, saj je to
nujno potrebno za prenos informacij. 

Pri obravnavi širjenja svetlobe skozi snov je najpomembnejši parameter lomni količnik
oziroma tenzor dielektričnosti. V tem poglavju bomo spoznali, kako z zunanjim poljem vplivamo 
na lomni količnik in širjenje svetlobe skozi snov. Svetlobno 
valovanje lahko moduliramo na več načinov in z ustreznim moduliranjem
lomnega količnika valovanju spreminjamo amplitudo,\index{Elektro-optična modulacija!amplitudna} 
frekvenco ali fazo. Ker imata sprememba frekvence in faze enak učinek na valovanje, ju obravnavamo
skupaj kot frekvenčno oziroma fazno modulacijo\index{Elektro-optična modulacija!frekvenčna}
\index{Elektro-optična modulacija!fazna} (slika~\ref{fig:amfm}). 
\begin{figure}[ht]
\centering
\def\svgwidth{145truemm} 
\input{slike/09_AMFM.pdf_tex}
\caption{Amplitudno (a) in frekvenčno oziroma fazno moduliran signal (b)
}
\label{fig:amfm}
\end{figure}

Delovanje optičnih modulatorjev temelji na različnih pojavih. V tem poglavju bomo 
podrobneje spoznali dva načina, to sta elektro-optični in elasto-optični pojav. 
Pri prvem se lomni količnik spremeni pod vplivom zunanjega električnega polja in
pri drugem zaradi mehanske deformacije. Kadar mehansko deformacijo povzroči zvočno valovanje, 
takim modulatorjem pravimo akusto-optični. Na koncu bomo spoznali še zelo pomemben 
primer elektro-optičnih modulatorjev na osnovi tekočih kristalov.

\section{Elektro-optični pojav}
\label{chap:EO}
Elektro-optični pojav\index{Elektro-optični pojav} opisuje spremembe optičnih lastnosti 
snovi (dielektričnosti in lomnega količnika)\index{Lomni količnik} pod vplivom 
zunanjega električnega polja. Omejimo se na statično zunanje polje oziroma
na polje, ki se le počasi spreminja. Omejitev na nizko 
frekvenco je potrebna zato, da odziv snovi na optično polje še lahko obravnavamo linearno. 
Kako je v nasprotnem primeru, ko je frekvenca polja primerljiva z optično frekvenco, 
bomo na široko obravnavali v poglavju o nelinearni optiki (poglavje~\ref{chap:NLO}).

Vzemimo optično anizotropno snov z nemotenim tenzorjem dielektričnosti  \index{Dielektričnost}
$\underline{\tilde{\epsilon}}$. V snoveh brez absorpcije ali optične aktivnosti
je tenzor $\underline{\tilde{\epsilon}}$ realen in simetričen, zato ga lahko 
diagonaliziramo~(enačba~\ref{eq:gostota-elektricnega-polja-lastni}). Lastne vrednosti 
$\varepsilon_1$, $\varepsilon_2$ in $\varepsilon_3$ ustrezajo kvadratom treh lomnih
količnikov $n_1^2$, $n_2^2$ in $n_3^2$, ki so na splošno različni.
\pagebreak

Kot bomo videli, je namesto dielektričnega tenzorja\index{Dielektričnost!inverzna} 
priročno vpeljati inverzni dielektrični tenzor
\begin{equation}
\underline{b}=\underline{\epsilon}^{-1}.
\end{equation}
V lastnem nemotenem sistemu ga preprosto zapišemo kot
\begin{equation}
\underline{\tilde{b}} = \left[\begin{array}{ccc}
1/\varepsilon_1 & 0& 0\\
0 & 1/\varepsilon_2& 0\\
0 & 0&  1/\varepsilon_3
\end{array}\right] = 
\left[\begin{array}{ccc}
1/n_1^2 & 0& 0\\
0 & 1/n_2^2& 0\\
0 & 0&  1/n_3^2
\end{array}\right]\!.
\end{equation}
Ko priključimo zunanje polje, se tenzor $\underline{\tilde{b}}$
spremeni. Pri elektro-optičnem pojavu so spremembe tenzorja dielektričnosti zaradi vpliva
zunanjega polja navadno majhne in lahko spremembo komponente $\delta b_{ij}$ zapišemo kot 
potenčno vrsto zunanjega polja $\mathbf{E}$, 
pri čemer upoštevamo zgolj prva dva člena v razvoju in uporabimo 
Einsteinov zapis seštevanja
\boxeq{7.1}{
\delta b_{ij}=\sum_{k=1}^3 r_{ijk}E_{k}+ \sum_{k=1}^3 \sum_{l=1}^3 q_{ijkl}E_{k}E_{l} = r_{ijk}E_{k}+q_{ijkl}E_{k}E_{l}.
}
Prvi člen je linearno sorazmeren z zunanjim poljem in opisuje linearni elektro-optični
\index{Elektro-optični pojav!linearni|see {Pockelsov pojav}}
ali Pockelsov\index{Pockelsov pojav} pojav\footnote{~Nemški fizik Friedrich Carl Alwin Pockels, 1865--1913.}. 
Tenzor tretjega ranga $r_{ijk}$ imenujemo elektro-optični 
tenzor\index{Elektro-optični tenzor}
ali tudi Pockelsov tenzor\index{Pockelsov tenzor|see {Elektro-optični tenzor}}. 
Pockelsov tenzor je različen od nič v snoveh brez centra inverzije, značilne vrednosti Pockelsovega
tenzorja so okoli $r \sim 10^{-12}$--$10^{-10}~\si{\m/\V}$.

Kvadratni elektro-optični pojav imenujemo Kerrov\index{Kerrov pojav}
pojav\footnote{~Škotski fizik John Kerr, 1824--1907.} in tenzor $q_{ijkl}$ 
Kerrov tenzor\index{Kerrov tenzor}. \index{Elektro-optični pojav!kvadratni|see {Kerrov pojav}}Kerrov pojav je praviloma precej šibkejši od Pockelsovega, vendar je prisoten 
v vseh snoveh, ne glede na njihove simetrijske lastnosti, torej tudi v tekočinah. 
Značilna vrednost Kerrovega tenzorja je $q \sim 10^{-24}$~m$^2$/V$^2$. 
Navadno ločimo dva primera Kerrovega pojava: Kerrov elektro-optični pojav 
pri zunanjih poljih z nizko frekvenco, in optični Kerrov pojav, ki ga bomo  
podrobneje spoznali v razdelku~\ref{OKP}  pri obravnavi nelinearnih optičnih pojavov.

Pri uporabi trdnih kristalov navadno prevlada linearni člen, zato zapišemo
\boxeq{eq:Pockels}{
\delta b_{ij}=r_{ijk}E_{k}.
}

\subsection*{Elektro-optični ali Pockelsov tenzor}
Simetrija snovi pomembno vpliva na obliko tenzorjev, ki opisujejo njene lastnosti.
Pockelsov tenzor $\underline{r}$ je tenzor tretjega ranga, zato je lahko različen
od nič le v kristalih brez centra inverzije. 
\begin{definition}
Razmisli, zakaj je v kristalih s centrom inverzije tenzor $\underline{r}$
vedno enak nič. 
\end{definition}

Ker je inverzni dielektrični tenzor $\underline{b}$ vedno simetričen, je v 
prvih dveh indeksih simetričen tudi $\underline{r}$
\begin{equation}
r_{ijk} = r_{jik}.
\end{equation}
V najmanj simetričnem primeru triklinskega kristala ima tako namesto 27 zgolj 
18 neodvisnih komponent, v kristalih z višjo simetrijo pa še precej manj. 

Zaradi simetrijskih lastnosti lahko tenzor poenostavljeno zapišemo le z dvema
indeksoma. To naredimo tako, da prva dva indeksa, v katerih je $r_{ijk}$ simetričen, združimo
v enega z vrednostmi od 1 do 6 po dogovoru 
\begin{eqnarray}
 &xx=1 \qquad  &yz=zy = 4 \nonumber \\
 &yy=2 \qquad  &zx=xz = 5 \nonumber \\
 &zz=3 \qquad  &xy=yx = 6.
\end{eqnarray}
S tem $r_{ijk}$ postane matrika velikosti $6\times3$, simetrični tenzor drugega 
ranga $b_{ij}$ pa šestdimenzionalni vektor.
Nekaj primerov Pockelsovih tenzorjev pri različnih kristalnih simetrijah
je podanih v tabeli~\ref{table:Pockels}.\footnote{~A. G. Caba\~nes, J. M. C. Castillo, F. A. Lopez,
v {\it Encyclopedia of Optical Engineering}, ur. R. G. Driggers, CRC Press (2003); A. Yariv in 
P. Yeh, {\it Photonics}, šesta izdaja, Oxford University Press (2007) in
{\it CRC Handbook of Chemistry and Physics}, CRC Press (2002).}

\begin{table}[ht]
 \centering
\begin{tabular}{|c|c|c|c|} \hline  
      Kristal & Grupa & Neničelne komponente tenzorja $r$ & Vrednost ($10^{-12}~\si{\m/\V}$)\\ \hline
      BaTiO$_3$\index{BaTiO$_3$} & 4mm & $r_{xzx} = r_{yzy} = r_{zxx} = r_{zyy} = 
      r_{51} = r_{42}$  &
	    $r_{51} = 1300$ \\
	      & & $r_{xxz} = r_{yyz} = r_{13} = r_{23}$ &  $r_{13} = 8$ \\
	      & & $r_{zzz} = r_{33}$ & $r_{33} = 105$ \\ \hline
      KDP\index{KDP} & 
      $\overline{4}$2m & $r_{yzx} = r_{zyx} = r_{xzy} = r_{zxy} = r_{41} = r_{52}$  &
	    $r_{41} = 8,77$ \\
	    & & $r_{xyz} = r_{yxz} = r_{63}$ &  $r_{63} = 10,3$ \\ \hline
      GaAs\index{GaAs}\index{ZnTe} &  $\overline{4}$3m&
	  $r_{yzx} = r_{zyx} = r_{xzy} = r_{zxy} = r_{xyz} = r_{yxz}$  & 
	   $r_{41} = 1,6$ \\
	ZnTe  & &   $= r_{41} = r_{52}=r_{63}$  & $r_{41} = 4,2$ 
	    \\ \hline
      LiNbO$_3$\index{LiNbO$_3$} & 3m & $r_{xzx} = r_{zxx} = r_{yzy} = r_{zyy} = r_{51} = r_{42}$  &
	    $r_{51} = 32,6$ \\
	     & & $r_{xxz} = r_{yyz} = r_{13} = r_{23}$ &  $r_{13} = 9,6$ \\
	      & & $r_{zzz} = r_{33}$ & $r_{33} = 30,9$ \\
	    & &  $r_{yyy} = - r_{xxy} = -r_{xyx} = -r_{yxx}  = $ & \\
	    & &  $=r_{22} =  -r_{12} =-r_{61} $  &
	    $r_{22}  = 6,8$ \\
\hline 
\end{tabular}
  \caption{Koeficienti Pockelsovega tenzorja za nekaj izbranih snovi. Vrednosti so podane 
  pri valovni dolžini $\sim 600~\si{nm}$, razen za GaAs ($10,6~\si{\micro\metre}$) in 
  ZnTe ($3,4~\si{\micro\metre}$).}
\label{table:Pockels}
\end{table}
\vglue-5truemm
\begin{remark}
Komponente elektro-optičnega tenzorja zaradi nazornosti pogosto ponazarjamo grafično. V matriki $6\times 3$
s piko označimo komponente, ki so enake nič, s polnim krožcem neničelne komponente, povezava med 
komponentami pomeni njihovo enakost, prazen krožec in črtkana črta pa označujeta 
neničelno komponento nasprotnega predznaka. Kot primer sta podana prikaza tenzorjev za 
GaAs\index{GaAs} (levo) in \index{LiNbO$_3$} LiNbO$_3$ (desno).
\begin{figure}[ht]
\centering
\def\svgwidth{20truemm} 
\input{slike/09_tensor.pdf_tex}\qquad \qquad
\def\svgwidth{20truemm} 
\input{slike/09_tensor2.pdf_tex}
\end{figure}
\end{remark}

\section{Longitudinalna modulacija}
Poglejmo podrobneje, kako zunanje električno polje spremeni optične lastnosti 
kristala in kako to vpliva na svetlobo, ki potuje skozi kristal.
Navadno se za elektro-optično modulacijo uporabljajo kristali, ki so dvolomni že brez zunanjega polja. 
Kot primer vzemimo kristal KH$_{2}$PO$_{4}$ (KDP)\index{KDP}, ki ima tetragonalno 
simetrijo ($\bar{4}2$m). Kot razberemo iz tabele~\ref{table:Pockels}, ima 
elektro-optični tenzor KDP dve neodvisni komponenti: $r_{41} = r_{52}=8,77 \times 10^{-12}~\si{\m/\V}$
in $r_{63}= 10,3 \times 10^{-12}~\si{\m/\V}$.

Kristal naj bo odrezan po kristalografskih oseh. Svetloba naj skozi kristal potuje 
v smeri optične osi, to je po dogovoru smer $z$, in v isti smeri na kristal priključimo
polje $E_z$. Ker je smer električnega polja vzporedna s smerjo širjenja svetlobe, taki 
postavitvi pravimo longitudinalna in pojavu longitudinalna 
modulacija (slika~\ref{fig:amshema}).\index{Elektro-optična modulacija!longitudinalna} 
\vglue-5truemm
\begin{figure}[ht]
\centering
\def\svgwidth{80truemm} 
\input{slike/09_AMshema.pdf_tex}
\caption{Shema longitudinalne modulacije signala. Ker je polje priključeno v smeri
potovanja svetlobe, morata biti elektrodi prozorni. Z uporabo polarizatorja in 
analizatorja sestavimo amplitudni modulator (glej razdelek~\ref{chap:ampmod}).}
\label{fig:amshema}
\end{figure}

Inverzni dielektrični tenzor za optično enoosni KDP zapišemo v odsotnosti polja kot
\index{Dvolomnost!enoosne snovi}
\begin{equation}
\underline{\tilde{b}} = 
\left[\begin{array}{ccc}
1/n_o^2 & 0& 0\\
0 & 1/n_o^2& 0\\
0 & 0&  1/n_e^2
\end{array}\right]\!,
\label{7.8}
\end{equation}
pri čemer sta $n_o$ in $n_e$ redni in izredni lomni količnik. Ko priključimo \index{Dielektričnost!inverzna}
polje $E_z$, se tenzor dielektričnosti zaradi Pockelsovega pojava spremeni. Upoštevamo,
da sta $E_x=E_y=0$ in da so od nič različni le $r_{41}$, $r_{52}$ in $r_{63}$. 
Sprememba inverznega tenzorja je potem po enačbi~(\ref{eq:Pockels})
\begin{align}
\delta b_{xx} & =r_{xxx}E_x + r_{xxy}E_y + r_{xxz}E_z = 0,\nonumber \\
\delta b_{xy} & = \delta b_{yx} = r_{xyx}E_x + r_{xyy}E_y + r_{xyz}E_z = r_{63}E_z,\nonumber\\
\delta b_{xz} & = \delta b_{zx} = r_{xzx}E_x + r_{xzy}E_y + r_{xzz}E_z = 0,\nonumber\\
\delta b_{yy} & = r_{yyx}E_x + r_{yyy}E_y + r_{yyz}E_z= 0,\nonumber\\
\delta b_{yz} & = \delta b_{zy} = r_{yzx}E_x + r_{yzy}E_y + r_{yzz}E_z= 0,\nonumber\\
\delta b_{zz} & = r_{zzx}E_x + r_{zzy}E_y + r_{zzz}E_z = 0.
\end{align}
Čeprav je zaradi simetrije večina členov enaka nič, se zaradi zunanjega električnega
polja $E_z$ pojavi izvendiagonalna komponenta $\delta b_{xy}$
\begin{equation}
\underline{b} = 
\left[\begin{array}{ccc}
1/n_o^2 & 0& 0\\
0 & 1/n_o^2 & 0\\
0 & 0& 1/n_e^2
\end{array}\right] + \left[\begin{array}{ccc}
 0& r_{63}E_z& 0\\
r_{63}E_z & 0 & 0\\
0 & 0&  0
\end{array}\right] = \left[\begin{array}{ccc}
1/n_o^2 & r_{63}E_z& 0\\
r_{63}E_z& 1/n_o^2 & 0\\
0 & 0&  1/n_e^2
\end{array}\right]\!.
\label{7.8a}
\end{equation}
Če želimo izračunati, kako se po kristalu pod napetostjo širi vpadni svetlobni
snop, moramo tenzor $\underline{b}$ diagonalizirati. Lastne vrednosti novega tenzorja
in pripadajoče nove lastne osi so
\begin{align}
\lambda_1 &= \frac{1}{n_o^2}+ r_{63}E_z \qquad \mathrm{in} \qquad \mathbf{e}_1 = \frac{1}{\sqrt{2}}(1,1,0),\\
\lambda_2 &= \frac{1}{n_o^2}- r_{63}E_z \qquad \mathrm{in} \qquad \mathbf{e}_2 = \frac{1}{\sqrt{2}}(-1,1,0),\\
\lambda_3 &= \frac{1}{n_e^2} \qquad \mathrm{in} \qquad \mathbf{e}_3 = (0,0,1).
\end{align}
Os $z$ se ohranja, medtem ko sta drugi dve novi lastni osi zasukani za kot $45\si{\degree}$ 
glede na prvotni osi $x$ in $y$ (slika~\ref{fig:amn}).
V novem koordinatnem sistemu je inverzni dielektrični tenzor diagonalen in enak
\begin{equation}
\underline{b} = 
\left[\begin{array}{ccc}
1/n_o^2 + r_{63}E_z& 0& 0\\
0 & 1/n_o^2 - r_{63}E_z& 0\\
0 & 0& 1/n_e^2
\end{array}\right]\!.
\end{equation}
Spomnimo se, da potuje svetloba skozi kristal vzdolž osi $z$. Brez zunanjega električnega
polja je kristal enoosen z optično osjo v smeri $z$\index{Dvolomnost!enoosne snovi}. 
Lomni količnik je neodvisen od
polarizacije vpadnega valovanja in je enak $n_o$. Ko priključimo polje, postane kristal
optično dvoosen, saj so vse tri lastne vrednosti tenzorja dielektričnosti različne. 
\index{Dvolomnost!dvoosne snovi}
\begin{figure}[ht]
\centering
\def\svgwidth{60truemm} 
\input{slike/09_AMindikatrisa.pdf_tex}
\caption{Optično enoosen kristal postane pod napetostjo dvoosen. Indikatrisa, ki je pravokotno
na optično os brez priključenega polja krožnica (polna črta), se pod vplivom 
napetosti spremeni v elipso (črtkana črta). Pri tem sta osi elipse zasukani za kot 
$45\si{\degree}$ glede na prvotni osi $x$ in $y$.}
\label{fig:amn}
\end{figure}

Za svetlobo, 
ki potuje vzdolž osi $z$, torej obstajata dve lastni smeri $x'$ in $y'$ z ustreznima
novima lastnima količnikoma, ki ju izrazimo kot\index{Lomni količnik}
\begin{equation}
\frac{1}{n_{x'}^2} = \frac{1}{n_o^2}+ r_{63}E_z \qquad \mathrm{in} \qquad 
\frac{1}{n_{y'}^2} = \frac{1}{n_o^2}- r_{63}E_z. 
\label{eq:eonxnyr}
\end{equation}
Za vse eksperimentalno dosegljive jakosti električnega polja velja $r_{63}E\ll1/n_o^2$, 
zato lahko enačbi~(\ref{eq:eonxnyr}) lahko razvijemo in dobimo
\begin{equation}
n_{x'} = \sqrt{\frac{n_o^2}{1+ n_o^2 r_{63}E_z}} \approx n_o \sqrt{1- n_o^2 r_{63}E_z} \approx
n_o - \frac{1}{2}n_o^3 r_{63}E_z.
\end{equation}
Nova lomna količnika sta tako 
\boxeq{EOnx}{
n_{x'}&\approx n_o - \frac{1}{2}n_o^3 r_{63}E_z \qquad \mathrm{in}\\
n_{y'}&\approx n_o + \frac{1}{2}n_o^3 r_{63}E_z.
}
Svetloba z različnima lastnima polarizacijama ob priključenem zunanjem polju potuje vzdolž 
osi $z$ z različnima hitrostma. 
Kadar polarizacija vpadnega valovanja ne sovpada z novima lastnima osema $x'$ ali $y'$, je 
svetloba po preletu kristala na splošno eliptično polarizirana. Ko svetloba
prepotuje dolžino kristala $L$, med lastnima polarizacijama nastopi fazna razlika
\begin{equation}
\Delta \phi = k_0 n_{y'} L - k_0 n_{x'} L = \frac{\omega}{c_0}L 
n_o^3 r_{63}E_z = \frac{\omega}{c_0} n_o^3 r_{63}U,
\label{phiAM}
\end{equation} 
pri čemer smo upoštevali zvezo $U = E_zL$. 

Vpeljemo karakteristično napetost $U_\pi$ (imenovano tudi $\pi$-napetost), 
pri kateri je \index{pi@$\pi$-napetost}
fazna razlika enaka $\pi$ in kristal deluje kot ploščica $\lambda/2$\index{Ploščica $\lambda/2$}
\boxeq{UpiL}{
U_\pi = \frac{\pi c_0}{\omega n_o^3 r_{63}} = \frac{\lambda}{2 n_o^3 r_{63}}.
}
Za kristal KDP\index{KDP} je $\pi$-napetost pri valovni 
dolžini $633~\si{\nano\metre}$ okoli  $9000~\si{\volt}$, kar 
je precej visoka napetost. Visoke delovne napetosti
so značilne za kristalne elektro-optične modulatorje in so njihova
glavna pomanjkljivost. 

\section{Transverzalna modulacija}
\index{Elektro-optična modulacija!transverzalna}
Iz praktičnih razlogov je navadno preprosteje priključiti električno polje v smeri, ki 
je pravokotna na smer širjenja svetlobe. Taki postavitvi pravimo transverzalna in pojavu
transverzalna modulacija\index{Elektro-optična modulacija!transverzalna}.
Tudi to postavitev obravnavajmo na primeru. Za zgled vzemimo kristal \index{LiNbO$_3$}LiNbO$_3$, ki 
ima trigonalno simetrijo ($3$m) in po tabeli~\ref{table:Pockels} štiri 
neodvisne komponente: $r_{51}=r_{42}, r_{13}=r_{23}, r_{33}$ in $r_{22}=-r_{12}=-r_{61}$.

Naj se svetloba širi vzdolž osi $z$, ki je hkrati tudi optična os, 
električno polje pa priključimo v smeri $y$ (slika~\ref{fig:tmshema}). 
Krajši račun pokaže, da je inverzni dielektrični tenzor v polju enak
\begin{equation}
\underline{b} = 
 \left[\begin{array}{ccc}
1/n_o^2  -r_{22}E_y& 0& 0\\
0& 1/n_o^2+r_{22}E_y& r_{51}E_y\\
0 & r_{51}E_y&  1/n_e^2
\end{array}\right]\!.
\label{7.8b}
\end{equation}

\begin{figure}[ht]
\centering
\def\svgwidth{80truemm} 
\input{slike/09_TMshema.pdf_tex}
\caption{Shema transverzalne modulacije signala}
\label{fig:tmshema}
\end{figure}

Tudi v tem primeru tenzor diagonaliziramo in poiščemo nove lastne vrednosti.
Če privzamemo, da je sprememba zaradi električnega polja majhna ($rE\ll1$), lahko
zanemarimo člene, v katerih $rE$ nastopa v kvadratni obliki. Nove lastne vrednosti so tako
\begin{align}
\lambda_1 &\approx \frac{1}{n_o^2}-r_{22}E_y, \\
\lambda_2 &\approx \frac{1}{n_o^2}+ r_{22}E_y\qquad \mathrm{in} \\
\lambda_3 &\approx \frac{1}{n_e^2}.
\end{align}
Izračunamo še ustrezne lomne količnike 
\begin{align}
n_{x'} &\approx n_o+\frac{1}{2}n_o^3r_{22}E_y,\\
n_{y'} &\approx n_o-\frac{1}{2}n_o^3r_{22}E_y \qquad \mathrm{in}\\
n_z' &\approx n_e.
\end{align}
Kako pa je z novimi lastnimi osmi?\index{Dvolomnost!dvoosne snovi} 
Hitro ugotovimo (enačba~\ref{7.8b}), da se ob
priključenem polju os $x$ ohranja. Pojavi se torej zasuk okoli osi $x$,
ki ga označimo s kotom $\vartheta$. Račun pokaže, da je za smiselne
vrednosti električnega polja ta kot zelo majhen ($\vartheta~
\approx~r_{51}E_y/(1/n_o^2-1/n_e^2) \sim~1~\si{mrad}$),
tako da lahko v približku rečemo, da se lastne osi ohranjajo. 

Če potuje svetloba vzdolž osi $z$, sta lomna količnika za 
polarizaciji v smeri $x$ in $y$ približno enaka $n_{x'}$ in $n_{y'}$. Fazna razlika med 
polarizacijama po preletu kristala z dolžino $L$ je tako
\begin{equation}
\Delta \phi = k_0 n_{y'} L - k_0 n_{x'} L = \frac{\omega}{c_0}L 
n_o^3 r_{22}E_y = \frac{\omega}{c_0}L n_o^3 r_{22}\frac{U}{d}.
\label{fazaTM}
\end{equation}
Pri tem moramo ločiti med $L$, ki je dolžina kristala v smeri širjenja svetlobe
(smer $z$), in $d$, ki je širina kristala v smeri priključenega električnega polja 
(smer $y$). 
Karakteristična $\pi$-napetost \index{pi@$\pi$-napetost}je 
\boxeq{UpiT}{
U_\pi = \frac{\lambda d}{2 Ln_o^3 r_{22}}.
}
Za izbran kristal ($d=5~\si{mm}$, $L=1~\si{cm}$) je $\pi$-vrednost 
napetosti pri valovni dolžini $633~\si{\nano\metre}$ okoli $2000~\si{\volt}$. S tanjšanjem
kristala lahko to napetost še zmanjšamo.

\begin{remark}
Transverzalno modulacijo lahko dosežemo tudi tako, da se žarek širi vzdolž 
osi $y$, električno polje pa priključimo vzdolž optične osi $z$.
V tem primeru se lastne osi ohranijo in kristal ostane optično enoosen. Vendar 
 ima tudi ta rešitev določene slabosti. Ker je kristal že sam po sebi dvolomen\index{Dvolomnost}, 
povzroči zunanje polje le majhno dodatno fazno razliko, zato je najbolje, če je dolžina 
kristala taka, da velja $k_{0}L(n_{o}-n_{e})=2N\pi$. Pri tem pa nastopi težava. 
Pogoj je lahko zaradi temperaturnega raztezanja in odvisnosti lomnih količnikov od temperature
izpolnjen le pri eni temperaturi, poleg tega se mora svetloba širiti natančno v smeri $y$.
Zato dvolomnost nemotenega kristala navadno kompenziramo, tako da postavimo 
dva enako dolga kristala zapored, pri čemer sta optični
osi med seboj pravokotni, modulacijska napetost na drugem kristalu pa ima
nasproten predznak. Tedaj se fazna razlika med obema polarizacijama zaradi 
naravne dvolomnosti odšteje, zaradi modulacijske napetosti pa sešteje.
\end{remark}

\section{Amplitudna modulacija}
\label{chap:ampmod}
\index{Elektro-optična modulacija!amplitudna}
Poglejmo, kako lahko elektro-optični pojav izkoristimo za modulacijo
amplitude svetlobe. Pod vplivom polja se v kristalu pojavi
fazni zamik med polarizacijama, ki je sorazmeren priključeni napetosti 
(enačbi~\ref{phiAM} in \ref{fazaTM}).
Če za tak kristal postavimo analizator, lahko z napetostjo spreminjamo 
moč prepuščene svetlobe -- amplitudno moduliramo signal.

Vrnimo se k longitudinalni\index{Elektro-optična modulacija!longitudinalna}
modulaciji (slika~\ref{fig:amshema}).
Naj bo vpadno električno polje polarizirano v smeri $y$. 
Ko priključimo napetost, sta novi lastni osi zasukani 
za kot $45\si{\degree}$ glede na prvotni lastni osi (slika~\ref{fig:amn}). Vpadno 
valovanje zato razstavimo na komponenti $x'$ in $y'$
\begin{equation}
\mathbf{E}_0 = E_0\, \mathbf{e}_y = \frac{E_0}{\sqrt{2}}\left(\mathbf{e}_{x'} + \mathbf{e}_{y'}\right)\!.
\end{equation}
Po kristalu se komponenti širita z različnima hitrostma, zato se med njima pojavi 
fazna razlika $\Delta \phi$ 
(enačba~\ref{phiAM}). Ob izstopu iz kristala je 
\begin{equation}
\mathbf{E}_1 = \frac{E_0}{\sqrt{2}}\left(e^{ik_0 n_{x'}L}\mathbf{e}_{x'} + 
e^{ik_0 n_{y'}L}\mathbf{e}_{y'}\right)
= \frac{E_0}{\sqrt{2}}e^{ik_0 n_{x'}L}\left(\mathbf{e}_{x'} + 
e^{i\Delta\phi}\mathbf{e}_{y'}\right)\!.
\end{equation}
Analizator na izhodni strani naj bo obrnjen v smeri $x$, to je pravokotno
na smer vpadne polarizacije. Prepušča le projekcijo obeh lastnih polarizacij
na os $x$
\begin{equation}
\mathbf{E}_2= \left(\mathbf{E}_1 \cdot \mathbf{e}_x\right)\mathbf{e}_x = 
\frac{E_0}{\sqrt{2}}e^{ik_0 n_{x'}L}
\left(\frac{1}{\sqrt{2}} -\frac{1}{\sqrt{2}} e^{i\Delta\phi}\right)\mathbf{e}_x .
\label{7.16}
\end{equation}
Gostota prepuščenega svetlobnega toka ob vpadnem toku $j_0$ je tako 
\begin{equation}
j=\frac{1}{4}j_{0}\left|1-e^{i\Delta\phi}\right|^{2}=\frac{1}{2}j_{0}(1-\cos\Delta\phi).
\label{7.17}
\end{equation}
Preoblikujemo izraz in zapišemo prepustnost  modulatorja $T$ ob upoštevanju 
zveze~(\ref{phiAM})
\boxeq{AMfinal}{
T = \frac{j}{j_0} = \sin\left(\frac{\Delta\phi}{2}\right)^2 = 
\sin\left(\frac{\pi n_o^3 r_{63}U}{\lambda}\right)^2\!.
}
\begin{figure}[ht]
\centering
\def\svgwidth{75truemm} 
\input{slike/09_AMT.pdf_tex}
\caption{Prepustnost amplitudnega modulatorja $T$ v odvisnosti od faznega zamika $\Delta \phi$
(rdeča črta). Če pred vzorec dodamo ploščico $\lambda/4$, 
se faza zamakne za $\pi/2$ in odvisnost prepustnosti od faze (in priključene napetosti)
postane približno linearna (modra črta).}
\label{fig:amt}
\end{figure}

Ko je napetost na kristalu enaka nič, je gostota prepuščenega svetlobnega toka 
$j=0$. To je pričakovano, saj sta analizator in polarizator prekrižana, 
vpadni žarek pa se širi vzdolž lastne osi kristala.
Prepustnost doseže največjo vrednost, ko je $\Delta \phi=\pi$, kar je ravno pri 
$\pi$-napetosti\index{pi@$\pi$-napetost}. Ko torej napetost povečamo z 0 na $U_\pi$, se
prepustnost modulatorja spremeni z 0 na 1 (slika~\ref{fig:amt}).

Pogosto želimo, da je zveza med modulacijsko napetostjo in izhodno
gostoto toka linearna. To lahko dosežemo, če modulator deluje v okolici $\Delta\phi=\pi/2$
(slika~\ref{fig:amt}).
Ena rešitev je dodati stalno visoko napetost in signal modulirati okoli
te vrednosti. Precej bolj praktičen pristop je z uporabo ploščice $\lambda/4$\index{Ploščica $\lambda/4$},
ki jo dodamo med polarizator in kristal, tako da se pojavi stalni
fazni premik $\pi/2$ med redno in izredno polariziranim valom. Potem lahko z razmeroma majhno napetostjo
linearno amplitudno moduliramo svetlobo\index{Elektro-optična modulacija!linearna}.

\section{Fazna in frekvenčna modulacija}
\index{Elektro-optična modulacija!fazna}
\index{Elektro-optična modulacija!frekvenčna}
Svetlobo amplitudno moduliramo tako, da z zunanjim
poljem spremenimo fazi lastnih valov, zaradi česar postane linearno
polarizirano vpadno valovanje po prehodu kristala eliptično polarizirano.
Spremembo polarizacije z analizatorjem prevedemo v spremembo amplitude.

Pogosto želimo modulirati fazo vpadne svetlobe. Tudi to si oglejmo na primeru longitudinalne
 modulacije\index{Elektro-optična modulacija!longitudinalna}. Fazno oziroma 
 frekvenčno modulacijo dosežemo tako,
da vhodno polarizacijo izberemo vzporedno eni od novih lastnih osi kristala, 
na primeri osi $x'$, izhodni polarizator pa odstranimo (slika~\ref{fig:fmshema}). 
\begin{figure}[ht]
\centering
\def\svgwidth{80truemm} 
\input{slike/09_FMshema.pdf_tex}
\caption{Shema fazne modulacije signala. Vpadna polarizacija je vzporedna eni od 
novih lastnih osi kristala, ki se pojavijo pod vplivom zunanjega polja.}
\label{fig:fmshema}
\end{figure}

Celoten fazni zamik po preletu skozi kristal zapišemo kot 
\begin{equation}
\phi =  k_0 n_{x'} L -\omega_0 t= \frac{\omega_0}{c_0}L \left(n_o -
\frac{1}{2}n_o^3 r_{63}\frac{U}{L}\right)-\omega_0 t,
\label{fmphi}
\end{equation}
pri čemer smo lomni količnik zapisali skladno z enačbo~(\ref{EOnx}). Opazimo,
da je fazni zamik odvisen od priključene zunanje napetosti. Pri stalni napetosti je 
ta fazni zamik konstanten, mi pa si oglejmo, kaj se zgodi, če na kristal priključimo
spreminjajočo se napetost.

Obravnavajmo dva primera. V prvem primeru naj bo 
napetost linearna funkcija časa 
\begin{equation}
U= U_0 + \frac{\Delta U}{\Delta t}t.
\end{equation}
Celotna faza prepuščenega valovanja je potem
\begin{equation}
\phi = \frac{\omega_0}{c_0}L n_o - \frac{\omega_0 n_o^3 r_{63}}{2c_0}\left( U_0 + 
\frac{\Delta U}{\Delta t}t\right) - \omega_0 t.
\end{equation}
Krožno frekvenco valovanja v danem trenutku izračunamo kot negativni odvod faze po času
\begin{equation}
\omega = -\frac{d\phi}{dt} = \omega_0 + \frac{\omega_0 n_o^3 r_{63}}{2c_0}\frac{\Delta U}{\Delta t} =
\omega_0 + \Delta \omega.
\end{equation}
Vidimo, da linearno spreminjajoča se modulacijska napetost povzroči spremembo frekvence izhodne svetlobe. 
Dosegljive spremembe frekvence so seveda dokaj majhne,
do nekaj $100~\si{MHz}$, saj so omejene s hitrostjo spreminjanja napetosti.
Napetost seveda tudi ne more neomejeno naraščati. Kadar se napetost
vrača na nič, pride do frekvenčnega premika v nasprotni smeri, ki  ga
lahko zanemarimo, če je čas vračanja bistveno krajši od časa naraščanja.

Poglejmo še primer, ko se priključena napetost periodično spreminja kot
\begin{equation}
U=U_{0}\sin(\omega_{m}t).
\end{equation}
Vstavimo izraz v enačbo~(\ref{fmphi}) in zapišemo fazo izhodnega valovanja 
\begin{equation}
\phi = \frac{\omega_0}{c_0}L n_o - \frac{\omega_0 n_o^3 r_{63}}{2c_0} U_0\sin(\omega_{m}t)
- \omega_0 t.
\end{equation}
V primeru linearno spreminjajoče se napetosti smo na tem mestu fazo odvajali in dobili krožno frekvenco, ki 
je bila konstantna. V tem primeru z odvajanjem dobimo krožno frekvenco, ki se spreminja s časom. Zato
se računa lotimo drugače. Konstantni člen, ki predstavlja le konstantni fazni premik, 
lahko izpustimo in zapišemo jakost električnega polja prepuščenega valovanja  
\begin{equation}
E = E_0 \cos\left( \omega_0 t + \frac{\omega_0 n_o^3 r_{63}}{2c_0} U_0\sin(\omega_{m}t)\right)
= E_0 \cos\left( \omega_0 t + \delta \sin(\omega_{m}t)\right)\!,
\end{equation}
pri čemer je
\begin{equation}
\delta = \frac{\omega_0 n_o^3 r_{63}}{2c_0} U_0.
\end{equation}
Z uporabo Jacobi-Angerjevih identitet 
\begin{equation}
 \begin{split}
  \cos\left(\delta\sin x\right)  &=J_0(\delta)+2J_2(\delta)\cos2x+
2J_4(\delta)\cos4x + \ldots\qquad \mathrm{in} \\
\sin\left(\delta\sin x\right) &=2J_1(\delta)\sin x+2J_3\sin3x+
2J_5\sin5x+\ldots
\label{JA}
\end{split}
\end{equation}
je izhodno polje mogoče zapisati v obliki 
\boxeq{fmJA}{
\begin{split}
\frac{E}{E_0} &= J_0(\delta)\cos(\omega_0 t)+ \\
&+J_1(\delta) \cos\left((\omega_0+\omega_m)t\right)-
J_1 (\delta) \cos\left((\omega_0 -\omega_m)t\right) + \\ 
&+J_2(\delta)\cos\left((\omega_0 +2\omega_m)t\right) + 
J_2(\delta)\cos\left((\omega_0 -2\omega_m)t\right)+ \\
&+ J_3(\delta)\cos\left((\omega_0 +3\omega_m)t\right) - J_3(\delta)\cos\left((\omega_0 -3\omega_m)t\right)+\ldots
\end{split}
}
\begin{definition}
Z uporabo Jacobi-Angerjevih identitet (enačbi~\ref{JA}) pokaži, da električno polje
izhodne svetlobe ob priključeni napetosti $U_0\sin(\omega_{m}t)$ ustreza
polju v enačbi~(\ref{fmJA}).
\end{definition}
Zaradi periodične fazne modulacije priključne napetosti se v spektru izhodne svetlobe pojavijo stranski pasovi, ki se
od osnovne krožne frekvence $\omega_0$ razlikujejo za večkratnike modulacijske krožne frekvence $\omega_m$. 
Njihova amplituda je podana z vrednostjo Besslovih funkcij pri $\delta$.
Ker je vrednost $\delta$ navadno majhna, za opis pogosto zadoščata le člena pri $\omega_0+\omega_m$ in 
$\omega_0-\omega_m$.

\begin{remark}
Elektro-optični pojav izkoriščamo tudi za uklanjanje žarkov.\footnote{~Glej npr. 
B. E. A. Saleh in M. C. Teich, 
{\it Fundamentals of Photonics}, druga izdaja, John Wiley \& Sons, Inc. (2007).}
\index{Elektro-optični deflektor}
Najpreprostejši primer deflektorja je trikotna prizma z elektrodama na osnovnih 
ploskvah. Svetloba se ob prehodu skozi prizmo lomi v odvisnosti od njenega 
lomnega količnika, tega pa lahko spreminjamo 
z napetostjo na elektrodah. Praktično je bolj uporabna dvojna prizma 
(slika~\ref{fig:deflshema}). Sestavljena je iz dveh enakih 
prizem, ki skupaj tvorita kvader, pri čemer osi $z$ zgornje in spodnje prizme
kažeta v nasprotni smeri. S spreminjanjem napetosti, ki jo priključimo prečno na smer
širjenja svetlobe, lahko zelo hitro in zelo natančno spreminjamo smer izhodnega žarka. 
Vendar ta pristop ni splošno uveljavljen, predvsem zaradi velikih napetosti, ki so 
potrebne za znatno uklanjanje. Veliko bolj razširjen je akusto-optični pojav, ki ga 
bomo spoznali v naslednjem razdelku. 
\begin{figure}[ht]
\centering
\def\svgwidth{95truemm} 
\input{slike/09_EOdefl.pdf_tex}
\caption{Shema elektro-optičnega deflektorja}
\vglue-1truecm
\label{fig:deflshema}
\end{figure}
\end{remark}
\begin{remark}
Elektro-optični pojav ima dva prispevka: neposrednega, kjer zunanje polje 
vpliva neposredno na elektronsko polarizirnost, in posredno spremembo
lomnega količnika zaradi piezoelektrične deformacije. Pri nizkih frekvencah sta prispevka 
primerljiva, pri velikih frekvencah pa deformacija kristala ne more slediti
modulacijski napetosti in ostane le neposredni prispevek. Pri akustičnih resonancah, ko 
modulacija v kristalu vzbudi stoječe zvočno valovanje, se piezoelektrični 
prispevek resonančno poveča. Pogoj za akustično resonanco je, da je dimenzija 
kristala mnogokratnik polovice valovne dolžine akustičnega vala v kristalu. 
Pri kristalih velikosti $\sim \si{cm}$ in hitrosti zvočnih valov okoli
$5000~\si{\m/\s}$ so resonance v območju od nekaj sto $\si{kHz}$ do
nekaj deset $\si{MHz}$. 

Pri visokih frekvencah postane pomembna tudi električna vezava modulatorja,
saj kristal predstavlja kapacitivno breme. Njegova impedanca pojema z 
rastočo frekvenco, zato je vedno večji del padca napetosti na notranjem 
uporu izvora napetosti. Pomagamo si z vzporedno vezavo tuljave, 
tako da je resonančna frekvenca $1/(LC)$ nastalega nihajnega kroga 
enaka modulacijski frekvenci, z dodatnim uporom pa spreminjamo širino resonance.
Tipična moč, ki je potrebna za modulacijo, je nekaj $10~\si{W}$, 
kar je za visokonapetosten in hiter izvor že znatna moč. 
\end{remark}

\section{Elasto-optični in akusto-optični pojav}
\index{Elasto-optični pojav}
\index{Akusto-optični pojav}
Pri elasto-optičnem pojavu dielektrične lastnosti snovi in njen lomni količnik
spreminjamo z mehansko deformacijo. Tudi tu opišemo pojav
s spremembo inverznega dielektričnega tenzorja\index{Dielektričnost!inverzna}
\begin{equation}
\underline{b} = \underline{\tilde{b}}+ \delta\underline{b},
\end{equation}
pri čemer sta $\underline{\tilde{b}}$ tenzor v odsotnosti deformacije in
$\delta\underline{b}$ sprememba tenzorja zaradi deformacije snovi. Deformacijo
snovi v linearnem približku opišemo z Greenovim tenzorjem
\begin{equation}
S_{kl}=\frac{1}{2}\left({\frac{\partial u_{k}}{\partial x_{l}}}+
{\frac{\partial u_{l}}{\partial x_{k}}}\right)\!,
\label{7.28}
\end{equation}
pri čemer je $\mathbf{u}$ vektor deformacije.\footnote{~Glej npr. I. Kuščer in A. Kodre,  
{\it Matematika v fiziki in tehniki}, tretji natis, DMFA-založništvo (2016).}
Spremembo tenzorja
$\delta\underline{b}$ v prvem približku zapišemo kot 
\boxeq{7.27}{
 \delta b_{ij}=p_{ijkl}S_{kl}.
}
Vpeljali smo sorazmernostni faktor
$p_{ijkl}$, ki ga imenujemo elasto-optični tenzor\index{Elasto-optični tenzor}. 
Tenzor $\underline{p}$ je različen od nič v vsaki snovi, saj povezuje dva simetrična tenzorja 
drugega ranga. Posledično je simetričen v prvem in drugem paru indeksov
\begin{equation}
p_{ijkl} = p_{jikl} = p_{ijlk} =p_{jilk}.
\end{equation}
V najbolj splošnem primeru triklinske kristalne simetrije 
ima 36 neodvisnih komponent, v bolj simetričnih snoveh se število 
neodvisnih komponent še zmanjša. Če vpeljemo skrajšan zapis
$xx = 1, yy=2, zz = 3, yz = zy = 4, zx = xz = 5, xy = yx = 6$,
zapišemo tenzor za izotropno snov
\begin{equation}
\underline{p}_{\textrm{izo}} = 
\left[\begin{array}{cccccc}
p_{11} & p_{12}& p_{12}&0&0&0\\
p_{12} & p_{11}& p_{12}&0&0&0\\
p_{12} & p_{12}& p_{11}&0&0&0\\
0 & 0& 0&p_{44}&0&0\\
0 & 0& 0&0&p_{44}&0\\
0 & 0& 0&0&0&p_{44}
\end{array}\right]\!,
\label{tenzorp}
\end{equation}
pri čemer je $p_{44}= (p_{11}-p_{12})/2$. Koeficienti tenzorja so 
brezdimenzijski, njihova tipična vrednost je $p\sim0,1$. Za vodo 
velja $p_{11} \simeq p_{12} = 0,31$ in $p_{44} = 0$, za LiNbO$_3$\index{LiNbO$_3$} pa 
$p_{11} = -0,026$, 
$p_{12} = 0,090$, 
$p_{13} = 0,133$, 
$p_{14} = -0,075$, 
$p_{31} = 0,179$, 
$p_{33} = 0,071$,
$p_{41} = -0,151$ in 
$p_{44} = 0,146$.\footnote{~A. Yariv in 
P. Yeh, {\it Photonics}, šesta izdaja, Oxford University Press (2007)}

Iz zveze $\underline{b} = \underline{\varepsilon}^{-1}$ in enačbe~(\ref{7.27}) izrazimo
spremembo dielektričnega tenzorja 
\begin{equation}
\delta\epsilon_{ij}=-\tilde{\epsilon}_{ii}\tilde{\epsilon}_{jj}p_{ijkl}S_{kl},
\label{7.29}
\end{equation}
pri čemer smo privzeli, da je nemoteni $\tilde{\epsilon}$ diagonalen. Ob mehanski deformaciji
torej v splošnem tudi izotropna snov postane dvolomna.
Dvojni lom\index{Dvolomnost}, ki se pojavi v deformirani snovi, izkoriščamo za proučevanje
mehanskih napetosti v snoveh, na primer v prozornih plastičnih modelih. 
Nas bo v nadaljevanju zanimal uklon \index{Uklon} svetlobe na periodični
modulaciji lomnega količnika, ki nastane zaradi zvočnega valovanja v snovi. Takemu pojavu
pravimo akusto-optični pojav.

\begin{definition}
\label{nalogaAO}
\vglue-3truemm
Po izotropni snovi se širi longitudinalno valovanje vzdolž smeri $z$, tako da 
deformacijo v snovi zapišemo kot
\begin{equation}
\mathbf{u} = A \cos(q z - \Omega t)\mathbf{e}_z.
\end{equation}
Pokaži, da je taka snov dvolomna z optično osjo vzdolž osi $z$, lastni 
lomni količniki pa so 
\begin{align}
n_{x'} &\approx \tilde{n}+\frac{1}{2}\tilde{n}^3p_{12}q A \sin (q z - \Omega t),\\
n_{y'} &\approx \tilde{n}+\frac{1}{2}\tilde{n}^3p_{12}q A \sin (q z - \Omega t) \qquad \mathrm{in}\\
n_z' &\approx \tilde{n}+\frac{1}{2}\tilde{n}^3p_{11}q A \sin (q z - \Omega t),
\end{align}
pri čemer je $\tilde{n}$ lomni količnik v odsotnosti motnje.\index{Lomni količnik}
\end{definition}

\section{Uklon svetlobe na zvočnem valovanju}
\index{Akusto-optični pojav}
Vzbudimo v plasti prozorne izotropne snovi zvočno valovanje z valovno dolžino $\Lambda$, 
ki potuje v smeri $x$. To naredimo tako, da na eno stran snovi priključimo piezoelektrik, 
ki se pod izmenično napetostjo periodično krči in razteza s krožno frekvenco $\Omega$.
Na drugo stran damo akustični absorber ali reflektor, tako da lahko 
v snovi vzbudimo stoječe valovanje. 
Zaradi zvočnega valovanja se v snovi periodično spreminjata gostota in 
z njo lomni količnik\index{Lomni količnik} (glej nalogo~\ref{nalogaAO})
\begin{equation}
n = \tilde{n} + \delta n \,\sin \left(2\pi x/\Lambda- \Omega t\right)\!.
\end{equation}
V zgoščini je lomni količnik nekoliko večji kot v razredčini, zato je optična pot na takem mestu
skozi plast daljša. Ravno svetlobno valovanje, ki vpada na plast pravokotno glede
na smer širjenja zvoka, po izstopu zato nima povsod enake faze in valovno čelo 
je periodično modulirano s periodo 
valovne dolžine zvočnega valovanja. Zvočno valovanje v snovi torej deluje kot 
optična fazna mrežica in vpadni žarek se na splošno uklanja (slika~\ref{fig:ao}). 
Tipična frekvenca, s katero vzbujamo elastično
deformacijo, je okoli $\Omega/2\pi \sim 50~\si{MHz}$ in ustrezna valovna dolžina okoli 
$\Lambda \sim 100~\si{\micro\metre}$. Frekvence, ki so v uporabi, navadno sežejo od 
nekaj $\si{MHz}$ prek $10~\si{GHz}$. Čeprav so vse te frekvence
daleč nad slišnimi, taka valovanja imenujemo zvočna oziroma akustična. 
\index{Uklon}\index{Uklonska mrežica}
\begin{figure}[ht]
\centering
\def\svgwidth{50truemm} 
\input{slike/09_AOshema.pdf_tex}
\caption{Zvočno valovanje v snovi deluje kot optična fazna mrežica in vpadna svetloba 
se na njem uklanja. Debelina plasti je $d$, valovna dolžina zvočnega valovanja $\Lambda$
in vpadni kot $\vartheta$.}
\label{fig:ao}
\end{figure}
\vglue-3truemm
Oglejmo si dva limitna primera. V prvem primeru je debelina plasti, 
v kateri vzbujamo zvočno valovanje, zelo majhna 
($d \ll \Lambda^2/\lambda$) in modulator deluje kot tanka uklonska mrežica.
Pojavi se veliko 
uklonskih vrhov, vendar je intenziteta posameznega žarka razmeroma majhna  (slika~\ref{fig:ao_bragg}\,a). 
Kote $\beta$, pod katerimi se pojavijo ojačitve, izračunamo po preprosti enačbi
\boxeq{R-N}{
\Lambda (\sin \vartheta - \sin \beta ) = N \lambda,
}
pri čemer so $\vartheta$ vpadni kot, $\lambda$ valovna dolžina svetlobe v snovi, 
$\Lambda$ valovna dolžina zvočnega valovanja in 
$N$ celo število. Takemu pojavu 
pravimo Raman-Nathov uklon\footnote{~Indijski fizik in nobelovec Sir Chandrasekhara 
Venkata Raman, 1888--1970, 
in indijski fizik N. S. Nagendra Nath.}\index{Raman-Nathov uklon}. 
Opazimo ga pri razmeroma nizkih zvočnih frekvencah 
(pod $\sim10~\si{MHz}$) in majhnih debelinah (pod $\sim 1~\si{cm}$) pri poljubnem vpadnem 
kotu $\vartheta$.
\begin{figure}[ht]
\centering
\def\svgwidth{110truemm} 
\input{slike/09_AO_1.pdf_tex}
\caption{Ob vpadu svetlobe na tanko plast, v kateri je vzbujeno zvočno valovanje, se 
pojavi veliko uklonskih vrhov (a). Na debeli plasti je opazen zgolj en uklonjen vrh in  
še ta le ob izpolnjenem Braggovem pogoju (b).}
\label{fig:ao_bragg}
\end{figure}
V nasprotnem limitnem primeru modulator deluje 
kot debela uklonska mrežica  (slika~\ref{fig:ao_bragg}\,b). 
Na splošno je delež uklonjene svetlobe na taki mrežici neuporabno majhen. 
Znaten postane le tedaj, kadar je izpolnjen Braggov\index{Braggov uklon}
pogoj\footnote{~Angleška znanstvenika in nobelovca, oče in sin, Sir William Henry Bragg, 1862--1942,
in Sir William Lawrence Bragg, 1890--1971.}
\boxeq{7.29a}{
2 \Lambda\sin\vartheta= N\lambda
}
in je vpadni kot enak uklonjenemu.
Ker je valovna dolžina svetlobe precej manjša od valovne dolžine zvočnega valovanja, je $\vartheta$
praviloma zelo majhen $\vartheta \sim 10^{-2}$. 

Poglejmo natančneje, kako pridemo do Braggovega pogoja (enačba~\ref{7.29a}). Pogoj
za ohranitev gibalne količine fotona pri sipanju na zvočnem valu je
\begin{equation}
\mathbf{k}_{0}\pm\mathbf{q}=\mathbf{k}_{1},
\label{7.30}
\end{equation}
pri čemer so $\mathbf{k}_{0}$ valovni vektor vpadne svetlobe, $\mathbf{k}_{1}$
valovni vektor uklonjene svetlobe in $\mathbf{q}$ valovni
vektor zvočnega vala. Znak plus velja, kadar potuje zvok proti projekciji
$\mathbf{k}_{0}$ na $\mathbf{q}$, negativen predznak pa ob potovanju zvoka v nasprotno smer. 
Frekvenca zvočnega vala je dosti nižja od frekvence svetlobe, zato se frekvenca svetlobe 
pri sipanju le malo spremeni in $|\mathbf{k}_{0}| \approx |\mathbf{k}_{1}|$.
Tedaj je $|\mathbf{q}| = q=2k_{0}\sin\vartheta$ 
(slika~\ref{fig:ao_bragg3}), od koder sledi Braggov pogoj 
(enačba~\ref{7.29a}). Obenem je vpadni kot enak izhodnemu, kar pomeni, da se na
zvočnem valu Braggovo sipana svetloba zrcalno odbije.\index{Braggov odboj} Razmere so torej
povsem analogne Braggovemu sipanju rentgenske svetlobe na kristalnih
ravninah. Ob izpolnjenem Braggovem pogoju je mogoče doseči, 
da se vsa vpadna svetloba uklanja.
\begin{figure}[ht]
\centering
\def\svgwidth{40truemm} 
\input{slike/09_AO_3.pdf_tex}
\caption{K izpeljavi Braggovega pogoja. Vektorja $\mathbf{k}_0$ in $\mathbf{k}_1$ označujeta
valovna vektorja vpadne in sipane svetlobe, $\mathbf{q}$ pa je valovni vektor zvočnega vala.}
\label{fig:ao_bragg3}
\end{figure}
\begin{remark}
Naredimo še oceno, kdaj je v veljavi Raman-Nathov in kdaj Braggov režim. 
Izhajamo iz pogoja, da je laserski snop na poti skozi plast zvočnega 
valovanja tako ozek, da ostane v celoti znotraj
ene razredčine ali zgoščine. Premer laserskega snopa $2w$ naj bo tako približno enak širini
razredčin in zgoščin, ki jo ocenimo na $\Lambda/2$. Dolžina, znotraj katere se snop še ne razširi znatno,
je območje bližnjega polja $2z_0 = 2 \pi w^2/\lambda$ (enačba~\ref{eq:z0}). Največja dolžina poti, pri 
kateri je pogoj izpolnjen in snop ostane znotraj ene razredčine ali zgoščine, je tako
\begin{equation}
d_c \sim 2z_0 \sim \frac{2 \pi w^2}{\lambda} \sim \frac{\pi \Lambda^2}{8 \lambda}.
\end{equation}
Pri debelinah $d \ll d_c$ velja Raman-Nathov približek in pri debelinah $d \gg d_c$ Braggov uklon.
Poglejmo primer. Če svetloba z $\lambda = 1~\si{\micro\metre}$ vpade na kristal, 
v katerem je vzbujeno zvočno valovanje s frekvenco $\Omega/2\pi = 50~\si{\mega\hertz}$ in
pripadajočo valovno dolžino $\Lambda = 0,1~\si{\milli\metre}$, je mejna debelina $d_c \sim 6~\si{mm}$.
\end{remark}

Če je zvočno valovanje potujoče, kar smo privzeli že 
s tem, da smo mu pripisali natanko določen valovni vektor $\mathbf{q}$,
se spremeni tudi frekvenca sipanega vala zaradi Dopplerjevega premika
pri odboju na zvočnem valovanju, ki potuje s hitrostjo $v_{z}$. Upoštevati
moramo le projekcijo na smer vpadne in odbite svetlobe, zato je 
\begin{equation}
\frac{\Delta\omega}{\omega}=\pm\frac{2v_{z}\sin\vartheta}{c}=
\pm\frac{2\Omega\Lambda\sin\vartheta}{2 \pi c}=\pm\frac{\Omega}{\omega},
\label{7.32}
\end{equation}
pri čemer smo uporabili Braggov pogoj (enačba~\ref{7.29a}). Sprememba frekvence
sipane svetlobe je torej kar enaka frekvenci zvočnega valovanja. To je seveda v skladu 
z zahtevo, da se pri uklonu na zvočnem valovanju ohranja skupna energija
vpadnega fotona in kvanta zvočnega valovanja (fonona), ki se pri sipanju 
absorbira ali pri njem nastane.

Malenkost drugačno je obnašanje, ko v snovi vzbudimo stoječe zvočno valovanje. 
Takrat lahko sipanje obravnavamo kot vsoto sipanja na dveh valovanjih z valovnima 
vektorjema $\mathbf{q}$ in $-\mathbf{q}$. Smer Braggovo sipanega vala je obakrat enaka, 
krožna frekvenca pa se enkrat poveča, drugič zmanjša za $\Omega$. Zato se pojavi utripanje
sipanega vala s krožno frekvenco $2\Omega$.

\subsection*{Uporaba akusto-optičnih modulatorjev}
\index{Akusto-optični modulator}
Spoznali smo, da z zvočnim valovanjem spreminjamo smer vpadne svetlobe.
Bistvena razlika od navadnih uklonskih mrežic\index{Uklonska 
mrežica} je dinamičnost akusto-optičnih modulatorjev, 
saj lahko uklonski kot svetlobe hitro spreminjamo. Poglavitna omejitev je,
da mora biti vsaj približno izpolnjen Braggov pogoj. S kombinacijo dveh med seboj 
pravokotnih akusto-optičnih modulatorjev snop 
premikamo po ravnini, kar s pridom uporabljamo v različnih optičnih napravah, 
na primer v optičnih pincetah, optičnih bralnikih ali 
optičnih litografskih zapisovalnikih.

Z vklapljanjem in izklapljanjem zvočnega valovanja, ki ga vzbujamo s piezoelektričnim 
elementom, na katerega pritisnemo izmenično napetost, moduliramo intenziteto
svetlobnega snopa. To potrebujemo na primer pri preklapljanju
dobrote laserskega resonatorja (razdelek~\ref{qswitch}).\index{Sunkovni laser!{preklop dobrote}}
Akusto-optične modulatorje uporabljamo tudi za uklepanje faz\index{Uklepanje faz}
v laserskem resonatorju (razdelek~\ref{chap:Uklepanje}). Če v Braggovem 
elementu vzbudimo stoječe zvočno valovanje, se amplituda direktnega snopa 
periodično spreminja. Kadar je frekvenca spreminjanja ravno enaka frekvenci
obhoda snopa, se faze vzbujenih nihanj uklenejo in nastanejo kratki
sunki izhodne svetlobe.

Naslednji primer uporabe je spreminjanje frekvence svetlobe. Možne so spremembe
do nekaj $100~\si{\mega\hertz}$, kar je ravno primerno za uporabo v laserskih merilnikih
hitrosti, kjer merimo frekvenco utripanja med referenčno svetlobo in svetlobo, odbito od
merjenega predmeta. Če ima referenčna svetloba
isto frekvenco kot merilni snop, ni mogoče določiti predznaka hitrosti
predmeta, če pa referenčni svetlobi nekoliko spremenimo frekvenco,
se pojavi utripanje tudi tedaj, ko predmet miruje. Frekvenca utripanja
se poveča ali zmanjša glede na predznak hitrosti predmeta.

Zanimiva je tudi uporaba Braggovega elementa za izdelavo
hitrega frekvenčnega analizatorja električnih signalov.  
Piezoelektrični element vzbujamo z električnim signalom,
ki ima neznan spekter. Enak spekter imajo tudi vzbujeni zvočni valovi, 
pri čemer vsakemu valu določene frekvence ustreza določen kot odklona svetlobnega
snopa. Za Braggovim elementom postavimo lečo. Vsak uklonjeni
snop da v goriščni ravnini svetlo točko, katere položaj je odvisen
od kota odklona in torej od frekvence zvočnega vala. Spekter zaznamo
z vrstičnim detektorjem. Akusto-optični element oziroma Braggova celica 
tako frekvenčni spekter zvočnih valov prevede v prostorski
spekter prepuščene svetlobe. Prostorski spekter svetlobe
analiziramo z lečo, ki v goriščni ravnini da prostorsko
Fourierevo transformiranko svetlobnega snopa pred lečo.

\section{*Račun akusto-optičnega pojava}
\index{Akusto-optični pojav}
Izračunajmo gostoto energijskega toka svetlobe, ki se uklanja na zvočnem valovanju. Izhajamo 
iz valovne enačbe v nehomogenem sredstvu, kar je dokaj težaven problem, zato
se moramo zateči k približkom. Uporabili bomo metodo sklopljenih 
valov.\index{Metoda sklopljenih valov}\footnote{~Glej npr. A. Yariv in 
P. Yeh, {\it Photonics}, šesta izdaja, Oxford University Press (2007).}

Naj vzporeden snop zvočnega valovanja s širino $d$ in valovnim vektorjem $\mathbf{q}$ 
potuje v smeri $x$.
Nanj pod kotom $\vartheta$ glede na os $z$ (normalo na kristal) 
vpada ravno svetlobno valovanje z valovnim vektorjem 
$\mathbf{k}=(k_{x},0,k_{z})= k(\sin\vartheta,0,\cos\vartheta)$.
Vsa valovanja (vpadno na levi od zvočnega snopa in izhodna valovanja na njegovi desni)
obravnavajmo znotraj snovi, da ni treba upoštevati še loma, ki
le zaplete izraze. 

Privzamemo, da se v snovi zaradi zvočnih valov spremeni le velikost
dielektrične konstante. Zaradi enostavnosti računajmo z efektivno elasto-optično
konstanto $p$ in efektivno deformacijo $S_0$. Ob upoštevanju zveze med
spremembo dielektričnosti in deformacijo
v zvočnem valu~(enačba~\ref{7.29}) spremembo dielektričnosti
zapišemo kot  \index{Dielektričnost}
\begin{equation}
\epsilon=\tilde{\epsilon}+\delta\epsilon = 
\tilde{\varepsilon} -\tilde{\epsilon}^{2}pS_{0}\sin(qx-\Omega t).
\label{7.33}
\end{equation}
Zaradi spremembe dielektričnosti se pojavi dodatna električna polarizacija 
$\delta P$\index{Električna polarizacija}
\begin{equation}
\delta P = \varepsilon_0 \delta \varepsilon E = - \varepsilon_0 
\tilde{\varepsilon}^2 p S_0 \sin(qx-\Omega t)E.
\end{equation}
Dodatna polarizacija v valovno enačbo (enačba~\ref{eq:valovna-skalarna}) 
doprinese nehomogen člen. Dobimo\index{Valovna enačba}
\begin{equation}
\nabla^{2}E-\frac{\tilde{\epsilon}}{c^{2}}{\frac{\partial E^{2}}
{\partial t^{2}}}=\mu_{0}{\frac{\partial^2 \delta P}{\partial t^{2}}},
\label{7.33a}
\end{equation}
pri čemer smo privzeli, da je $\nabla\cdot\mathbf{E}\approx 0$, čeprav je
$\epsilon$ funkcija kraja. 

Enačbo~(\ref{7.33a}) brez dodane polarizacije $\delta P$ rešijo ravni valovi 
z valovnim vektorjem $\mathbf{k}$ in krožno frekvenco $\omega$. Tej rešitvi se 
primešajo valovi z valovnim vektorjem $\mathbf{k}\pm n\mathbf{q}$
in krožno frekvenco $\omega\pm n\Omega$. Zato iščemo rešitve v obliki vsote
ravnih valov, torej Fouriereve vrste
\begin{equation}
E=\sum_{n}A_{n}(z)e^{in(qx-\Omega t)}e^{i(k_{x}x+k_{z}z-\omega t)}.
\label{7.34}
\end{equation}
Zaradi sklopitve prek $\delta P$ smo dopustili, da so amplitude
$A_{n}$ funkcije $z$. Če je $\delta\epsilon$ dovolj majhen, se $A_{n}(z)$
le počasi spreminjajo.

Izračunajmo 
\begin{equation}
\nabla^{2}E=\sum_{n}\left( -(k_{z}^{2}+(k_{x}+nq)^{2})A_{n}(z)+2ik_{z}A_{n}'(z)\right) \, 
e^{i((k_x+nq)x+k_{z}z-(\omega+n\Omega)t)}.
\label{7.35}
\end{equation}
Člene z $A_{n}''$ lahko izpustimo, če je le $k_{z}A_{n}'\gg A_{n}''$ oziroma 
kadar se $A_{n}$ spreminjajo počasi v primerjavi z $\exp(ik_{z}z)$. Drugi odvod 
polarizacije po času da
\begin{equation}
\begin{split}
\frac{\partial^2 \delta P}{\partial t^2} =&-\frac{\varepsilon_0 \tilde{\varepsilon}^2pS_0}{2i} 
\sum_{n} A_n(z) \exp\left(i((k_x+nq)x+k_{z}z-(\omega+n\Omega)t)\right) \times \\ &\left(
-(n\Omega+\omega+\Omega)^2e^{i(qx-\Omega t)} + (n\Omega+\omega-\Omega)^2e^{i(-qx+\Omega t)} \right)\!.
\label{7.35a}
\end{split}
\end{equation}
Izračunamo še drugi odvod jakosti elektičnega polja po času 
\begin{equation}
\frac{\partial E^{2}}{\partial t^{2}} = - \sum_{n}(n\Omega + \omega)^2 
A_{n}(z)e^{in(qx-\Omega t)}e^{i(k_{x}x+k_{z}z-\omega t)}.
\label{7.35b}
\end{equation}
Vstavimo izraze (\ref{7.35}), (\ref{7.35a}) in (\ref{7.35b}) v valovno enačbo (\ref{7.33a})
in izenačimo člene z isto časovno in prostorsko odvisnostjo, na primer
s $k_z z+(k_x+mq)x-(\omega+m\Omega)t$. Sledi
\begin{equation}
\begin{split}
-\left(k_{z}^{2}+(k_{x}+mq)^{2}\right)A_{m}+2ik_{z}A_{m}' + \frac{\tilde{\varepsilon}}{c^2}(m\Omega+\omega)^2A_m
=\\ =\frac{\mu_0\varepsilon_0\tilde{\varepsilon}^2pS_0}{2i}(\omega+m\Omega)^{2}(A_{m-1}-A_{m+1}).
\end{split}
\end{equation}
Upoštevamo, da je 
\begin{equation} 
k_{x}^{2}+k_{z}^{2}=k^{2}=\frac{\tilde{\epsilon}\omega^2}{c^{2}},
\end{equation}
in naredimo približek $(\omega +m\Omega)^2 \approx \omega^2$.
Tako dobimo sistem enačb
\begin{equation}
A_{m}'+i\beta_{m}A_{m}+\xi(A_{m+1}-A_{m-1})=0,
\label{7.37}
\end{equation}
pri čemer sta
\begin{equation}
\beta_{m}=\frac{mq}{k_{z}}(k_{x}+\frac{1}{2}mq) \qquad \mathrm{in} \qquad \xi=-\frac{\tilde{\epsilon} pS_0k^2}{4k_z}.
\label{7.38}
\end{equation}
Reševanje sistema enačb~(\ref{7.37}) je težavno, zato poiščimo rešitve le v treh
pomembnih limitnih primerih. Amplituda vpadnega vala naj bo $A_{0}(0)=A_{0}$, 
za ostale naj velja $A_{n}(0)=0$.

\subsection*{Braggov uklon ob šibki pretvorbi}
Najprej privzamemo, da je $\xi d\ll 1$, da je torej velikost $\delta \epsilon$
majhna in debelina zvočnega snopa ne prevelika. Tedaj pri vseh
$z$ in za pozitivne $m$ velja $A_{m+1}\ll A_{m}$ in lahko člen $A_{m+1}$
v enačbi~(\ref{7.37}) izpustimo. S tem zapišemo preprost sistem enačb
\begin{equation}
A_{m}'+i\beta_{m}A_{m}=\xi A_{m-1},
\label{7.40}
\end{equation}
 ki jih lahko zapored integriramo
\begin{equation}
A_{m}(z)=\xi e^{-i\beta_{m}z}\int_{0}^{z}A_{m-1}(z')
e^{i\beta_{m}z'}dz'.
\label{7.41}
\end{equation}
Podobne izraze izpeljemo tudi za negativne $m$.

Poglejmo posebej prvi uklonjeni val z amplitudo $A_{1}$. Po predpostavki,
da je $A_{\pm1}\ll A_{0}$, se le malo energije uklanja iz osnovnega
vala in lahko privzamemo, da je $A_{0}(z)$ skoraj konstanta. 
Potem lahko integral v enačbi~(\ref{7.41})
izračunamo in dobimo
\begin{equation}
A_{1}(d)=A_{0}\xi d\,\frac{\sin\beta_{1}d/2}{\beta_{1}d/2}\, e^{-i\beta_{1}d/2},
\label{7.41a}
\end{equation}
pri čemer je $d$ debelina plasti zvočnega valovanja, parametra $\xi$ in $\beta_1$ pa 
sta določena z enačbama~(\ref{7.38}).
Funkcija $A_{1}(d)$ ima vrh pri $\beta_{1}=0$, to je po enačbi~(\ref{7.38}) pri 
\begin{equation}
k_x+ \frac{q}{2} = k \sin\vartheta + \frac{q}{2} = 0
\qquad \mathrm{ali} \qquad 
2\Lambda\sin\vartheta=-\lambda.
\label{7.43}
\end{equation}
Vidimo, da predstavlja $\beta_{1}=0$ ravno pogoj za Braggovo sipanje vpadnega
vala.\index{Braggov uklon}

Delež moči uklonjenega vala je 
\boxeq{7.45}{
\frac{I_{1}}{I_{0}}=\left|\frac{A_{1}}{A_{0}}\right|^{2}=(\xi d)^{2}
\left(\frac{\sin\beta_{1}d/2}{\beta_{1}d/2}\right)^{2}\!.
}
Če je Braggov pogoj izpolnjen, je $I_{1}/I_{0}=(\xi d)^{2}$ in naraščanje parabolično. To seveda
lahko velja le, dokler je $\xi d\ll1$. Kadar intenziteta uklonjenega žarka
tako naraste, da ta pogoj ni več izpolnjen, je treba v računu upoštevati tudi 
zmanjšanje moči vpadnega snopa.

\subsection*{Braggov uklon ob znatni pretvorbi}
Drugi primer naj bo približek, da sta le $A_{0}$ in $A_{1}$ različna od nič, 
vendar opustimo omejitev $\xi d\ll 1$. Ta približek je smiseln, saj je 
Braggov pogoj izpolnjen le za en uklonjeni val naenkrat, na
primer $m=1$. Tedaj so vse ostale amplitude $A_{m\ne0,1}$
majhne in ne vplivajo na $A_{1}$. Zaradi velike pretvorbe 
$A_{0}(z)$ ne smemo več obravnavati kot konstante. Upoštevamo 
izpolnjen Braggov pogoj (enačba~\ref{7.43}) in iz sistema enačb~(\ref{7.37})
sledi\index{Braggov uklon}
\begin{equation}
A_{0}'+\xi A_{1}  =  0 \qquad \mathrm{in} \qquad A_{1}'-\xi A_{0} =  0.
\end{equation}
Ob začetnih pogojih $A_{0}(0)=A_{0}$ in $A_{1}(0)=0$ sta rešitvi
\begin{equation}
A_{0}(d) = A_{0}\cos (\xi d) \qquad \mathrm{in} \qquad A_{1}(d) = A_{0}\sin (\xi d).
\end{equation}
Če je izpolnjen Braggov pogoj, se moč vpadnega vala na razdalji $\pi/(2\xi)$
skoraj vsa pretoči v uklonjeni snop in z naraščajočo debelino ponovno vsa v
prepuščeni val (slika \ref{s7.10}).
Za učinkovito delovanje akusto-optičnega modulatorja seveda
želimo doseči ravno take pogoje.
\vglue-1truemm
\begin{figure}[ht]
\centering
\def\svgwidth{80truemm} 
\input{slike/09_AOcalc.pdf_tex}
\caption{Delež prepuščenega (črna črta) in uklonjenega (rdeča črta) 
valovanja na zvočnem valovanju v odvisnosti od debeline plasti zvočnega valovanja, če
je izpolnjen Braggov pogoj.}
\label{s7.10}
\end{figure}

Vstavimo še parameter $\xi$, ki je podan z enačbo~(\ref{7.38}).
Razmerje med intenziteto uklonjenega in vpadnega snopa je tako
\boxeq{7.48}{
\frac{I_{1}}{I_{0}}=\sin^{2}\left(\frac{\pi \tilde{n}^{3}pS_{0}d}{2\lambda\cos\vartheta}\right)\!,
}
pri čemer je $\tilde{n}=\sqrt{\tilde{\varepsilon}}$.
Izračunamo še amplitudo deformacije $S_0$. Za longitudinalne (zvočne) valove v snovi oblike 
$u = u_0 \cos(qx - \Omega t)$ je deformacija enaka
\begin{equation}
S_{xx} = \frac{1}{2}\left(\frac{\partial u}{\partial x}+ 
\frac{\partial u}{\partial x}\right) = \frac{\partial u}{\partial x} = - u_0 q\, \sin(qx - \Omega t).
\end{equation}
Povprečna gostota energijskega toka zvočnega valovanja v času $T$ je\footnote{~Glej 
npr. J. Strnad, {\it Fizika 1. del}, druga izdaja, DMFA-založništvo (2016).}
\index{Gostota energijskega toka}
\begin{equation}
j_{z}= \frac{1}{T}\int_0^T 
v_z \left(\frac{1}{2} \varrho \left(\frac{\partial u}{\partial t}\right)^2
+ \frac{1}{2} \varrho v_z^2\left(\frac{\partial u}{\partial x}\right)^2
\right) dt = \frac{1}{2} \varrho v_z^3 S_0^2,
\label{7.49}
\end{equation}
pri čemer sta $v_z$ hitrost zvoka v snovi in $\varrho$ gostota snovi. 

Sledi 
\begin{equation}
S_{0}=\sqrt{\frac{2j_{z}}{\varrho v_{z}^{3}}}.
\label{7.50}
\end{equation}
Praktično je vpeljati merilo uporabnosti neke snovi za akusto-optični modulator. To je 
koeficient \index{Koeficient $M$}
\begin{equation}
M=\frac{\tilde{n}^{6}p^{2}}{\varrho v_{z}^{3}}.
\label{7.51}
\end{equation}
Večja kot je vrednost koeficienta $M$, izrazitejši je akusto-optični pojav v dani snovi. 
Izkoristek pretvorbe s koeficientom $M$ zapišemo kot
\begin{equation}
 \frac{I_{1}}{I_{0}}=\sin^{2}\left(\frac{\pi d}{\lambda\cos\vartheta}\sqrt{\frac{j_z M}{2}}\right)\!.
\end{equation}
Poglejmo primer. V kremenu z gostoto $\varrho=2200~\si{\kilo\gram/\metre^3}$ 
je hitrost zvoka $v_{z}=6000~\si{\metre/\second}$,
$\tilde{n}=1,46$ in $p=0,2$. To da $M=8\times 10^{-16}~\si{\metre^2/\watt}$.
Pri gostoti zvočnega toka $10~\si{\watt/\centi\metre^2}$ in valovni dolžini svetlobe 
$633~\si{\nano\metre}$ nastopi popoln prenos moči v uklonjeni snop pri 
debelini $d \sim 3~\si{\centi\metre}$. Upoštevana gostota zvočnega toka je kar velika
in je ni prav lahko doseči, zato so uklonski izkoristki navadno nekaj manjši od 1.

\begin{remark}
Opisani račun izkoristka uklona na zvočnih valovih je uporaben tudi
pri računu izkoristka holograma.\index{Hologram} V primeru faznega holograma je račun
povsem enak in jasno kaže razliko med tankim in debelim hologramom.
Poleg faznih poznamo tudi amplitudne holograme, pri katerih je 
modulirana absorpcija v snovi.\footnote{~H.
Kogelnik, Bell Syst. Tech. J. $\mathbf{48}$, 2909 (1969).}
\end{remark}

\subsection*{Raman-Nathov uklon}
Oglejmo si še tretji primer. Izhajamo iz sistema enačb~(\ref{7.37}), ki smo ga 
zaenkrat rešili za primer Braggovega uklona oziroma v njegovi bližini. 
Enačbe je preprosto rešiti še v primeru tanke mrežice oziroma tako imenovanega Raman-Nathovega približka. 
\index{Raman-Nathov uklon}Vpeljimo novo neodvisno spremenljivko $\zeta=2\xi z$. 
Zveza~(\ref{7.37})
preide v 
\begin{equation}
2\frac{dA_{m}(\zeta)}{d\zeta}+A_{m+1}(\zeta)-A_{m-1}(\zeta)=\frac{\beta_{m}}{i\xi}A_{m}.
\label{7.53}
\end{equation}
Upoštevamo izraza za $\beta_m$ in $\xi$~(enačbi~\ref{7.38}). Člen na desni lahko izpustimo, če je 
\begin{equation}
\frac{\beta_{m}}{\xi}=\left| \frac{4mq}{\tilde{\varepsilon}pS_0k}\left(\sin\vartheta+\frac{mq}{2k}\right)\right| 
\ll 1
\label{7.54}
\end{equation}
oziroma če je valovna dolžina zvoka dovolj velika v primerjavi z valovno dolžino svetlobe. Potem 
v enačbi~(\ref{7.53}) prepoznamo rekurzijsko zvezo za Besslove funkcije 
\begin{equation}
2J_{n}'+J_{n+1}-J_{n-1}=0
\label{7.55}
\end{equation}
z rešitvijo $A_{m}(z)=A_{0}J_{m}(2\xi z)$. Kadar je $2\xi d$ ničla funkcije
$J_{0}$, prvič je to pri $2\xi d\approx 2,4$, se vsa energija ukloni iz
vpadnega snopa, vendar se v tem primeru 
razporedi v mnogo uklonjenih snopov.
\begin{remark}
Do istega rezultata lahko pridemo tudi preprosteje, z uporabo Fraunhoferjevega uklona
(enačba~\ref{eq:FraunhoferApprox}). Upoštevamo, da zvočni val ustvari periodično fazno 
mrežico in izračunamo njeno Fourierevo transformiranko, pri čemer uporabimo 
Jacobi-Angerjev razvoj (enačba~\ref{JA}). 
\end{remark}

\section{Modulacija s tekočimi kristali}

\subsection*{Nematični tekoči kristali}
\index{Tekoči kristali}
\index{Tekoči kristali!nematik}
Za konec opišimo še modulacijo svetlobe s tekočimi kristali.\footnote{~Glej npr.
Marija Vilfan in Igor Muševič, {\it Tekoči kristali}, DMFA-založništvo (2002).} 
Tekoči kristali so anizotropne kapljevine. To pomeni, da so tekoči kot 
kapljevine, imajo pa določene anizotropne lastnosti kot trdni kristali. 
Tekoče kristale tvorijo podolgovate ali ploščate molekule, 
ki odražajo različne stopnje urejenosti. 

Omejimo se na najosnovnejši
primer, to so podolgovate organske molekule v nematični fazi tekočega kristala. 
Navadno so to molekule z razmeroma togim jedrom iz
dveh ali treh benzenovih obročev, ki imajo na koncih krajše ali daljše
alifatske verige (slika~\ref{fig:5CB}). Značilnost nematične faze je, da
so v njej težišča molekul neurejena, enako kot v navadni tekočini, medtem ko so 
osi molekul v povprečju urejene v določeno smer. Pravimo, da imajo molekule
v nematiku orientacijsko ureditev dolgega dosega. Če nematik segrejemo,
preide v izotropno tekočo fazo, če pa ga ohladimo, neposredno ali prek drugih
tekočekristalnih faz preide v trdno kristalno obliko. 

Smer povprečne urejenosti podolgovatih molekul opišemo z enotskim vektorjem 
$\mathbf{n}$, ki ga imenujemo direktor.\index{Tekoči kristali!direktor} 
Smeri $\mathbf{n}$ in $-\mathbf{n}$ sta 
enakovredni, saj molekule z enako verjetnostjo kažejo tako v smer $+\mathbf{n}$ kot 
v $-\mathbf{n}$. Stopnja urejenosti v mikroskopski sliki ni prav velika, povprečen
odklon molekul od $\mathbf{n}$ je nekaj deset stopinj, odvisno seveda od temperature.
\index{Tekoči kristali!5CB}
\begin{figure}[ht]
\centering
\def\svgwidth{100truemm}
\input{slike/09_5CB.pdf_tex}
\caption{Molekula enega izmed najbolj razširjenih tekočih kristalov, 4-ciano-4'pentil-bifenila 
ali 5CB (a) in shematski prikaz podolgovatih molekul v nematični fazi z označenim direktorjem (b)}
\label{fig:5CB}
\end{figure}

Molekule so v nematični fazi v povprečju orientacijsko urejene, zato se nematik
obnaša kot enoosni dvolomni kristal\index{Dvolomnost!enoosne snovi}. Njegova optična os je vzporedna 
z $\mathbf{n}$, lastni vrednosti dielektričnega tenzorja sta $\varepsilon_\bot$ in
$\varepsilon_{\parallel}$, ki ustrezata rednemu ($n_o$) in izrednemu ($n_e$) 
lomnemu količniku.\index{Dielektričnost}\index{Lomni količnik}
Ker je optična polarizirnost benzenovih obročev vzdolž osi molekul precej večja kot
v prečni smeri, je razlika med rednim in izrednim lomnim količnikom v nematiku razmeroma 
velika, navadno med $0,1$ in $0,2$, seveda spet odvisno od temperature.

V povprečju so molekule urejene v smeri direktorja. Če se smer direktorja lokalno
spremeni, je energija takega deformiranega 
stanja nekoliko večja od energije homogenega urejenega stanja. Tekoči kristal na
drugače orientiran delček snovi deluje z navorom v smeri zmanjševanja 
nehomogenosti $\mathbf{n}$. To lastnost, ki je značilna za tekoče kristale,
imenujemo orientacijska elastičnost. Vendar so v makroskopskem vzorcu
nematičnega tekočega kristala elastični navori prešibki,
da bi uredili celoten vzorec, zato se na splošno smer direktorja $\mathbf{n}$ 
po vzorcu neurejeno spreminja. Da dobimo urejene vzorce, ki jih potrebujemo za izdelavo
optičnih naprav, moramo ureditev vzorca 
vsiliti. To naredimo z zunanjim električnim ali magnetnim poljem
ali pa vzorce pripravimo dovolj tanke, da ureditev vsilijo mejne površine. 

Poglejmo, kako nastane urejen vzorec v tankih plasteh. Če površino,
ki je v stiku s tekočim kristalom, ustrezno pripravimo (prevlečemo s posebnimi 
plastmi ali mehansko obdelamo), se molekule tekočega kristala tik ob površini uredijo
v dani smeri. Tako na primer podrgnjena tanka plast najlona uredi
$\mathbf{n}$ ob površini v smeri drgnjenja vzporedno s površino (slika~\ref{s7.20a}\,a). 
Po drugi strani tanka plast lecitina ali silana uredi direktor 
pravokotno na površino (slika~\ref{s7.20a}\,b). Ti dve snovi imata namreč
polarno glavo, ki se adsorbira na stekleno površino, in alifatsko verigo, 
ki stoji približno pravokotno na površino. Zato se tudi alifatski repi molekul
tekočega kristala uredijo pravokotno na steklo. V obeh primerih, 
vzporedni (planarni) ali pravokotni (homeotropni) ureditvi ob steni, 
se urejenost zaradi orientacijske elastičnosti
ohranja tudi stran od stene, tako da lahko brez težav naredimo urejene
vzorce debeline do kakih $200~\si{\micro\metre}$. Pri večjih debelinah so elastični
navori prešibki in v vzorcu nastanejo defekti.
\begin{figure}[ht]
\centering 
\def\svgwidth{120truemm} 
\input{slike/09_planarno.pdf_tex}
\caption{Ureditev tekočega kristala navadno vsilimo z urejevalno površino. Dva primera
sta planarna ureditev (a), kjer je direktor vzporeden z urejevalno površino, in 
homeotropna ureditev (b), kjer je direktor pravokoten na mejno ploskev.}
\label{s7.20a}
\end{figure}
 
Na ureditev molekul tekočega kristala vpliva zunanje električno ali magnetno polje.
Zaradi urejenosti molekul električna (ali magnetna) susceptibilnost nematičnega tekočega
kristala ni skalar, temveč ima dve različni lastni vrednosti,
eno za smer vzporedno z $\mathbf{n}$, drugo za pravokotno nanj. Zato je
elektrostatična energija odvisna od kota med zunanjim poljem $\mathbf{E}$
in direktorjem $\mathbf{n}$. Pri konstantnem zunanjem električnem polju 
gostoto električne energije\index{Gostota energije} zapišemo kot 
\begin{equation}
w_{el}=-\frac{1}{2}\mathbf{E}\cdot \mathbf{D}.
\label{lcwe}
\end{equation}
Električno polje lahko razstavimo na del, ki je vzporeden z $\mathbf{n}$, in del, ki je
pravokoten nanj
\begin{equation}
\mathbf{E} = (\mathbf{E} \cdot \mathbf{n}) \mathbf{n} + \left( \mathbf{E} - 
(\mathbf{E} \cdot \mathbf{n}) \mathbf{n} \right)\!.
\end{equation}
Potem je 
\begin{equation}
\mathbf{D} = \varepsilon_0 \varepsilon_\bot \mathbf{E} + \varepsilon_0 \varepsilon_a
(\mathbf{E}\cdot\mathbf{n})\mathbf{n},
\label{7.56a}
\end{equation}
pri čemer je $\varepsilon_a = \varepsilon_\parallel - \varepsilon_\bot$ anizotropni
del dielektrične konstante. Anizotropni del gostote energije je tako do konstante
\boxeq{7.56}{
w_a = -\frac{1}{2}\epsilon_{0}\epsilon_{a}(\mathbf{E}\cdot\mathbf{n})^{2}.
}
Če je $\epsilon_{a}>0$, se molekule tekočega kristala uredijo v smeri 
zunanjega polja, sicer se uredijo pravokotno nanj.

Urejenost tekočekristalnega vzorca je tako odvisna od orientacijske
elastičnosti, robnih pogojev, ki jih določimo z obdelavo mejne
površine, in od jakosti ter smeri zunanjega električnega ali magnetnega polja.

\subsection*{Tekočekristalni prikazovalnik}
\index{Tekočekristalni prikazovalnik}
Vzemimo tanko plast tekočega kristala med dvema površinama, ki vsiljujeta
vzporedno planarno ureditev. Vzorec je urejen in homogen, optična os leži v ravnini 
plasti. Če dodamo na površini še prozorni elektrodi, lahko
z zunanjo napetostjo spreminjamo orientacijo molekul v plasti in s tem tudi 
smer optične osi. Dovolj velika napetost zasuče $\mathbf{n}$ in optična os
se postavi pravokotno na stene, razen tik ob površini. Tipično so take napetosti okoli 
nekaj voltov.

Ta pojav izkoristimo za izdelavo preprostega optičnega preklopnika. 
Naj debelina plasti $d$ ustreza debelini ploščice $\lambda/2$\index{Ploščica $\lambda/2$} 
za izbrano valovno dolžino svetlobe 
\begin{equation}
d(n_{e}-n_{o})=(2N+1)\frac{\lambda}{2},
\label{7.57}
\end{equation}
pri čemer so $N$ celo število, $n_e$ izredni in $n_o$ redni 
lomni količnik. Ker je v nematikih $n_{e}-n_{o}\sim0,1$, 
je ustrezna debelina $d$ nekaj $\si{\micro\metre}$. Tak vzorec damo med dva prekrižana 
polarizatorja s prepustno smerjo pod kotom $45\si{\degree}$ glede na $\mathbf{n}$
oziroma optično os. Vzorec, ki deluje kot ploščica $\lambda/2$, 
polarizacijo svetlobe z izbrano valovno dolžino zasuče za $90\si{\degree}$ in 
svetloba prehaja skozi analizator. Ko
priključimo napetost, se optična os obrne v smeri polja. Polarizacija vpadne 
svetlobe se pri prehodu skozi plast ohrani in 
analizator je ne prepusti. Z električnim poljem smo torej preklopili iz
stanja, ki prepušča svetlobo, v stanje, ki svetlobe ne prepusti.
Vendar ima tak preklopnik nekaj slabosti. Prepustnost je odvisna
od valovne dolžine svetlobe in od temperature, poleg tega mora biti debelina 
plasti povsod povsem enaka. Bolj praktični so zasukani nematiki. 
\index{Tekoči kristali!zasukan nematik}

Zasukan  nematik nastane tako, da površini, ki vsiljujeta planarno ureditev,
zasučemo za kot $90\si{\degree}$ eno glede na drugo (slika~\ref{LCD1}\,a), zato se $\mathbf{n}$ 
v plasti zvezno zavrti.\footnote{~M. 
Schadt in W. Helfrich, Phys. Rev. Lett. $\mathbf{27}$, 561 (1971).}
Pokazali bomo, da polarizacija svetlobe, ki je ob vstopu 
v plast polarizirana v smeri urejanja, pri prehodu skozi plast približno
sledi $\mathbf{n}$ in je ob izstopu iz plasti pravokotna
na vpadno polarizacijo. Ko priključimo električno polje, se direktor 
obrne v smer pravokotno na plast tekočega kristala (slika~\ref{LCD1}\,b). V tem primeru 
se polarizacija ne zasuče in analizator svetlobe ne prepusti. Plast med prekrižanima
polarizatorjema brez polja torej prepušča svetlobo, s poljem pa ne. Pri tem
delovanje prikazovalnika ni dosti odvisno niti od debeline plasti, niti
od valovne dolžine svetlobe. 
\begin{figure}[ht]
\centering
\def\svgwidth{90truemm} 
\input{slike/09_LCD1.pdf_tex}
\caption{V zasukani nematični celici polarizacija $P$ sledi smeri zasukanega 
direktorja in analizator (A) prepušča svetlobo (a). V električnem polju 
($\mathbf{E}$) se tekočekristalne molekule zasučejo v smer polja. 
Polarizacija svetlobe se ohranja in analizator svetlobe ne prepušča (b).}
\label{LCD1}
\end{figure}

\begin{figure}[ht]
\centering
\includegraphics[width=80truemm]{slike/09_LCD.jpg}
\caption{Na močno povečani fotografiji računalniškega tekočekristalnega zaslona se 
jasno vidi, da je vsak piksel sestavljen iz treh delov: rdečega, modrega in zelenega.}
\label{fig:LCD}
\end{figure}
\begin{remark}
Tekočekristalni zasloni, ki jih uporabljamo danes, so precej bolj zapleteni.
Najpreprostejši so črno-beli prikazovalniki, ki delujejo z odbito svetlobo (npr. 
v žepnih računalih), zato imajo za analizatorjem odbojno površino. Večina 
sodobnih prikazovalnikov (npr. računalniški ali telefonski zasloni) ima svoj izvor svetlobe, 
praviloma so to LED ali fluorescenčna svetila.
Barve dosežemo z barvnimi filtri (rdečim, modrim in zelenim, slika~\ref{fig:LCD}) na vsakem 
slikovnem elementu (pikslu) posebej, krmiljenje pikslov pa s tankoplastnimi 
tranzistorji (TFT -- {\it Thin Film Transistors}). Večina
sodobnejših zaslonov ima tekoče kristale urejene planarno in v njih poteka
preklapljanje v ravnini (IPS -- {\it In-Plane Switching}), lahko so molekule
urejene vertikalno in jih s poljem nagibamo (VA -- {\it Vertical Alignment}), tekočekristalne 
zaslone lahko naredimo tudi občutljive na dotik.
\end{remark}
\vglue-3truecm
\section{*Račun prehoda svetlobe skozi zasukan nematik}
\index{Tekoči kristali!zasukan nematik}
Pokazati moramo še, da polarizacija svetlobe pri prehodu skozi zasukan nematik 
približno sledi zasuku optične osi.\footnote{~Glej npr. P. G. de Gennes in J. Prost, 
{\it The Physics of Liquid Crystals}, Oxford University Press (1995).}\index{Dvolomnost!enoosne snovi}
Vzemimo vzorec, kakršen je na sliki~\ref{LCD1}\,a, in ga obravnavajmo
kot lokalno optično enoosno snov. Pri $z=0$ naj bo optična
os v smeri $x$, ko se premikamo vzdolž osi $z$, naj se optična os suče
v ravnini $xy$. Kot med optično osjo in osjo $x$ naj bo 
\begin{equation}
\varphi=qz.
\label{7.58}
\end{equation}
Zanimajmo se le za širjenje svetlobe v smeri $z$. 
Tedaj potrebujemo le del dielektričnega
tenzorja v ravnini $xy$.\footnote{~Podobno opišemo holesterični tekoči kristal, v
katerem so molekule kiralne in se $\mathbf{n}$ spontano suče okoli
smeri, pravokotne na $\mathbf{n}$.} 

\begin{definition}
Pokaži, da se dielektrični tenzor v zasukani nematični plasti zapiše kot\index{Dielektričnost}
\begin{equation}
\varepsilon (z)=\left[\begin{array}{cc}
\bar{\varepsilon}+\frac{1}{2}\varepsilon_{a}\cos(2qz) & \frac{1}{2}\varepsilon_{a}\sin(2qz)\\
\frac{1}{2}\varepsilon_{a}\sin(2qz) & \bar{\varepsilon}-\frac{1}{2}\varepsilon_{a}\cos(2qz)
\end{array}\right]\!,
\label{7.59}
\end{equation}
pri čemer je $z$ razdalja od plasti, v kateri je direktor
obrnjen v smeri $x$. Vpeljali smo še povprečno vrednost $\bar{\varepsilon}$ 
\begin{equation}
\bar{\varepsilon}=(\varepsilon_{\parallel}+\varepsilon_{\perp})/2.
\label{7.60}
\end{equation}
Namig: uporabi rotacijsko matriko $A(\varphi)$ in izračunaj  $\varepsilon (z) = A(\varphi) \cdot 
\tilde{\varepsilon} \cdot A(\varphi)^\mathrm{T}$.
\end{definition}

Iz Maxwellovih enačb~(enačbe~\ref{eq:Maxwell1}--\ref{eq:Maxwell4}) 
hitro uvidimo, da je valovna enačba\index{Valovna enačba}
za valovanje s krožno frekvenco $\omega$ v našem primeru oblike
\begin{equation}
\frac{d^{2}\mathbf{E}}{dz^{2}}+\frac{\omega^{2}}{c^{2}} \epsilon
(z)\mathbf{E}=0.
\label{7.61}
\end{equation}
Zapišimo jo po komponentah, upoštevajoč tenzor dielektričnosti~(enačba~\ref{7.59})
\begin{equation}
\frac{d^{2}E_{x}}{dz^{2}} + 
(\beta^{2}+\alpha^{2}\cos(2qz))E_{x}+\alpha^{2}E_{y}\sin(2qz) = 0
\label{7.62a}
\end{equation}
in 
\begin{equation}
\frac{d^{2}E_{y}}{dz^{2}} +
\alpha^{2}E_{x}\sin(2qz)+(\beta^{2}-\alpha^{2}\cos(2qz))E_{y} = 0,
\label{7.62b}
\end{equation}
pri čemer sta $\alpha^{2}=\epsilon_{a}\omega^{2}/(2c^{2})$ in 
$\beta^{2}=\bar{\epsilon}\omega^{2}/c^{2}$. S tem smo dobili sistem
dveh sklopljenih diferencialnih enačb, ki ga lahko rešimo.
\begin{remark}
Opis potovanja svetlobe skozi zasukan nematik je lepa ilustracija uporabe 
Blochovega teorema, ki ga sicer poznamo iz fizike trdne snovi za 
reševanje Schr\"odingerjeve enačbe za delec v periodičnem potencialu. Teorem pravi, 
da lastne rešitve zapišemo kot produkt ravnega vala in funkcije, ki ima enako 
periodo kot potencial.\index{Blochov teorem}\footnote{~Glej npr. N. W. Ashcroft in 
N. D. Mermin, {\it Solid State Physics}, Harcourt College
Publishers (1976).}
\end{remark}

Za reševanje je ugodno vpeljati krožni polarizaciji 
$E_{+}=E_{x}+iE_{y}$ in $E_{-}=E_{x}-iE_{y}$. Potem lahko
enačbi~(\ref{7.62a}) in (\ref{7.62b}) prepišemo v
\begin{equation}
-\frac{d^{2}E_{+}}{dz^{2}}=\beta^{2}E_{+}+\alpha^{2}E_{-}e^{2iqz}
\label{lcm1}
\end{equation}
in 
\begin{equation}
-\frac{d^{2}E_{-}}{dz^{2}}=\alpha^{2}E_{+}e^{-2iqz}+\beta^{2}E_{-}.
\label{lcm2}
\end{equation}

Lastne rešitve poiščemo v obliki 
\begin{equation}
E_{+}  =  Ae^{i(k+q)z} 
\qquad
\mathrm{in}
\qquad
E_{-}  =  Be^{i(k-q)z}.
\label{7.65}
\end{equation}
Nastavek reši sistem enačb~(\ref{lcm1}) in (\ref{lcm2}) 
natanko takrat, kadar $A$ in $B$ rešita sistem homogenih linearnih enačb 
\begin{equation}
\left((k+q)^{2}-\beta^{2}\right)A-\alpha^{2}B  =  0 \qquad \mathrm{in} \qquad 
-\alpha^{2}A+\left((k-q)^{2}-\beta^{2}\right)B  =  0.
\label{7.66d}
\end{equation}
 Sistem je netrivialno rešljiv, če je determinanta koeficientov enaka
nič
\begin{equation}
\left(k^{2}+q^{2}-\beta^{2}\right)^{2}\!-4k^{2}q^{2}-\alpha^{4}=0.
\label{7.66}
\end{equation}
Spomnimo se, da sta $\beta$ in $\alpha$ sorazmerna z $\omega$,
zato dobljena enačba predstavlja disperzijsko relacijo -- zvezo med 
$\omega$ in $k$ -- za svetlobo v zasukanem sredstvu
\begin{equation}
\left(k^{2}+q^{2}-\frac{\bar{\epsilon}\omega^{2}}{c^{2}}\right)^{2}\!-
4k^{2}q^{2}- \frac{\epsilon_{a}^2\omega^{4}}{4c^{4}}
=0.
\label{7.66a}
\end{equation}

Da dobimo disperzijsko odvisnost, moramo rešiti kvadratno enačbo (enačba~\ref{7.66a}). 
Vendar za razlago delovanja zasukane nematične celice zadošča približek 
$q\ll\alpha, \beta$, saj je perioda sukanja optične osi
velika v primerjavi z valovno dolžino svetlobe. Tedaj lahko $q$ v disperzijski
zvezi (enačba~\ref{7.66}) zanemarimo in velja
\begin{equation}
k^{2}=
\begin{cases}
\beta^{2}+\alpha^{2}=\frac{\omega^{2}}{c^{2}}\epsilon_{\parallel}\\
\beta^{2}-\alpha^{2}=\frac{\omega^{2}}{c^{2}}\epsilon_{\bot}.
\end{cases}
\label{7.67}
\end{equation}
Ti dve vrednosti ustrezata velikosti valovnega vektorja za izredni
in redni val v navadnem enoosnem kristalu. Vstavimo ju v enačbi~(\ref{7.65})
in za polarizaciji lastnih valov dobimo $B=\pm A$.
Izračunajmo še obe kartezični komponenti električnega polja za prvo rešitev ($A=B$)
\begin{align}
E_{x} &=  \frac{1}{2}(E_{+}+E_{-})  =  \frac{1}{2}Ae^{ikz}(e^{iqz}+e^{-iqz})  =  Ae^{ikz}\cos qz\\
E_{y} & = \frac{1}{2i}(E_{+}-E_{-})  =  \frac{1}{2i}Ae^{ikz}(e^{iqz}-e^{-iqz})  =  Ae^{ikz}\sin qz.
\label{7.68}
\end{align}
Zasuk polarizacije torej sledi zasuku optične osi. Druga rešitev ($A=-B$) da val,
ki je polariziran pravokotno na lokalno optično os in se prav tako
suče z njo. Pri tem se prvi val širi kot izredni val s fazno hitrostjo $c/n_{e}$ in  
drugi kot redni val s hitrostjo $c/n_{o}$. Če na zasukano
nematično celico vpada svetloba, ki je polarizirana ali vzporedno z 
optično osjo ob meji ali pravokotno nanjo, izstopa iz celice svetloba, katere 
polarizacija je zasukana za enak kot, kot je zasukana optična os. 
V primeru, da vpadna polarizacija ne sovpada z eno od
lastnih osi, jo razstavimo na lastni in po prehodu skozi
vzorec zopet sestavimo, s čemer seveda na splošno nastane eliptična
polarizacija.
\vglue-4truemm
\begin{remark}
Disperzijsko zvezo (enačba~\ref{7.66a}) 
lahko rešimo (slika~\ref{gap}). Pri izbrani vrednosti $\alpha$ 
obstajajo pri vseh frekvencah, razen v ozkem območju -- rečemo mu frekvenčna reža --
štiri realne rešitve za $k$, po dve za valovanji v pozitivni in v negativni smeri.
V območju reže je en par rešitev imaginaren. Vsaki vrednosti $k$
pripada neko razmerje amplitud $A$ in $B$, ki ga izračunamo
iz enačb~(\ref{7.66d}) in ki določa polarizacijo lastnega vala. Polarizacije
lastnih valov so na splošno eliptične in pri dani frekvenci med
seboj niso pravokotne, saj zapisani sistem enačb ne predstavlja čisto 
navadnega problema lastnih vektorjev simetrične matrike. 
V območju frekvenčne reže le en par rešitev predstavlja
potujoč val, drug pa polje, ki eksponentno pojema v sredstvo. Zato
se svetloba s frekvenco v reži in z ustrezno polarizacijo totalno odbije. Pojav v zasukanih
nematskih celicah ni opazen, saj je tam $q \ll \alpha$. Če pa je perioda vijačnice
primerljiva z valovno dolžino svetlobe, kot na primer v holesteričnih
tekočih kristalih, pride do značilnega obarvanega videza. Pojav
je povsem analogen Braggovemu odboju na kristalih.\index{Tekoči kristali!holesterik}\index{Braggov odboj}
\begin{figure}[ht]
\centering
\def\svgwidth{80truemm} 
\input{slike/09_gap.pdf_tex}
\caption{Rešitve disperzijske zveze (enačba~\ref{7.66}) v zasukanem nematiku
pri izbranem $\alpha$. Razen v frekvenčni reži (modra pasova) obstajajo pri vsaki frekvenci
štiri rešitve za $k$.}
\label{gap}
\end{figure}
\end{remark}

\section{Račun preklopa v tekočem kristalu -- Frederiksov prehod}
Ob opisu tekočekristalnih prikazovalnikov smo omenili, da lahko z dovolj velikim 
zunanjim električnim poljem molekule tekočega kristala, razen tik ob urejevalni površini,
obrnemo v smeri polja. Izračunajmo jakost polja, ki je potrebna za ta zasuk. 
\index{Frederiksov prehod}

Energija nematičnega tekočega kristala je najnižja, kadar je direktor $\mathbf{n}$
povsod obrnjen v isto smer. Povečanje energije zaradi krajevne odvisnosti $\mathbf{n}$
zapišemo z orientacijsko elastično energijo oziroma Frankovo prosto 
energijo\footnote{~Angleški fizik Sir Frederick Charles Frank, 1911--1998.}
\index{Frankova prosta energija}
\boxeq{7.70}{
F_{e}=\frac{1}{2}\int\left\{ K_{1}(\nabla\cdot\mathbf{n})^{2}+K_{2}
\left(\mathbf{n}\cdot(\nabla\times\mathbf{n})\right)^{2}+K_{3}
\left(\mathbf{n}\times(\nabla\times\mathbf{n})\right)^{2}\right\} dV.
}
Pri tem so $K_{1}$, $K_{2}$ in $K_{3}$ tri Frankove elastične
konstante, ki so odvisne od snovi in tudi od temperature. 
Prvi člen predstavlja povečanje energije zaradi deformacije v obliki 
pahljače, drugi zaradi zasuka in tretji zaradi upogiba (slika~\ref{s7.20}).\footnote{~F. C.
Frank, Discuss. Faraday Soc. $\mathbf{25}$, 19 (1958).}
\begin{figure}[ht]
\centering
\def\svgwidth{140truemm} 
\input{slike/09_KKK.pdf_tex}
\caption{Trije načini deformacije ureditve molekul tekočega kristala so pahljačasta deformacija,
zasuk in upogib.}
\label{s7.20}
\end{figure}

V zunanjem električnem polju se energija tekočega kristala dodatno spremeni. 
Navadno je neodvisna količina jakost električnega polja, saj je polje posledica
zunanje napetosti na elektrodah. Ustrezen člen v prosti energiji je tedaj 
(enačbi~\ref{lcwe} in~\ref{7.56a})
\begin{equation}
F_{el} = -\int \frac{1}{2} \mathbf{D}\cdot\mathbf{E}\, dV= -\frac{1}{2} \int
\left( \varepsilon_0 \varepsilon_\bot \mathbf{E}\cdot\mathbf{E} + 
\varepsilon_{0}\varepsilon_{a}(\mathbf{E}\cdot\mathbf{n})^{2}\right)dV = F_0 + F_{el,a}.
\end{equation}
Pri tem $F_{0}$ predstavlja del proste energije, ki je neodvisen od $\mathbf{n}$ in 
ni pomemben pri izračunu preklopa. Prosta energija
nematičnega tekočega kristala v električnem polju je tako 
\begin{equation}
F=F_e + F_{el} = F_0 + F_e + F_{el,a} = F_{0}+F_{e}-\int \frac{1}{2}\varepsilon_{0}\varepsilon_{a}
(\mathbf{E}\cdot \mathbf{n})^{2}dV.
\label{7.72}
\end{equation}
Tekoči kristal je v ravnovesju, ko je prosta energija najmanjša. Kadar je
$\epsilon_{a}>0$, se zato skuša $\mathbf{n}$ postaviti vzporedno s
poljem, vendar popoln zasuk onemogoča mejna urejevalna plast. 
Da lahko z minimizacijo $F$ izrazimo $\mathbf{n}(\mathbf{r})$, moramo
torej poznati še robne pogoje.

Naj bo nematični tekoči kristal med dvema vzporednima
steklenima ploščama, med katerima je razmik $d$. Na obeh ploščah naj bo $\mathbf{n}$ vzporeden
s površino in obrnjen v isto smer, tako da je brez zunanjega 
polja $\mathbf{n}$ povsod enako usmerjen. Naj bo to smer $x$.
Na stekleni plošči dodamo elektrodi, ki ustvarjata polje pravokotno na 
prvotno smer direktorja, naj bo to smer $z$.
Ko priključimo polje, je energijsko ugodnejše, da
se molekule vsaj delno zasučejo v smer polja. Ta zasuk opišemo s
komponento vektorja $\mathbf{n}$ v smeri $z$
\begin{equation}
\mathbf{n}(z)=(n_{x}(z),0,n_{z}(z)).
\label{7.73}
\end{equation}
Robni pogoj, kateremu mora direktor zadostiti,
je $n_{z}(0)=n_{z}(d)=0$. Približno rešitev zato iščemo z nastavkom 
\begin{equation}
n_{z}(z)=a\sin (qz), \qquad q=\frac{\pi}{d},
\label{7.74}
\end{equation}
ki ni nič drugega kot prvi člen razvoja prave rešitve v Fourierevo vrsto.
Ker je direktor enotski vektor, velja
\begin{equation}
n_x = \sqrt{1-a^2\sin^2(qz)} \approx 1 - \frac{a^2}{2}\sin^2(qz).
\end{equation}
Vzdolž smeri $x$ in $y$ se direktor ne spreminja, zato velja
\begin{equation}
\nabla\times\mathbf{n}=\left(0,\,\frac{dn_{x}}{dz},\,0\right)
\label{7.75}
\end{equation}
 in 
\begin{equation}
\mathbf{n}\times(\nabla\times\mathbf{n})=\left(-n_{z}\frac{dn_{x}}{dz},\,0,\,
n_{x}\frac{dn_{x}}{dz}\right)\!.
\label{7.76}
\end{equation}
Prosta energija na enoto površine je tako do konstante
\begin{align}
F_S & =  \frac{1}{2}\int\left(K_{1}\left(\frac{dn_{z}}{dz}\right)^{2}+
K_{3}\left(n_x^2+n_{z}^{2}\right)\left(\frac{dn_{x}}{dz}\right)^{2}-
\epsilon_{0}\epsilon_{a}(n_{z}E)^{2}\right)dz=\nonumber \\
 & =  \frac{1}{2}\int_{0}^{d}
 \left(K_{1}q^{2}a^{2}\cos^{2}(qz)+K_{3}q^{2}a^{4}\sin^{2}(qz)\cos^2(qz)-
 \epsilon_{0}\epsilon_{a}E^2a^{2}\sin^{2}(qz)\right)dz=\nonumber \\
 & =  \frac{d}{4}\left( K_{1}q^{2}a^2+\frac{1}{4}K_{3}q^{2}a^4-\epsilon_{0}\epsilon_{a}E^2a^2\right)\!.
\end{align}
V našem primeru smo integral lahko izračunali, saj smo uporabili nastavek (enačba~\ref{7.74}).
Sicer bi morali uporabiti Euler-Lagrangeevo metodo za minimizacijo proste energije, ki 
jo poznamo iz variacijskega računa.

Zdaj lahko poiščemo amplitudo deformacije $a$, pri kateri je prosta energija
najmanjša in odvod $dF_S/da=0$. Tedaj mora biti $a$ rešitev enačbe 
\begin{equation}
2(K_{1}q^{2}-\epsilon_{0}\epsilon_{a}E^2)a+K_{3}q^{2}a^{3}=0.
\label{7.78}
\end{equation}
 Rešitvi sta 
\begin{equation}
a=0
\end{equation}
in
\begin{equation}
a^{2}=2\frac{\epsilon_{0}\epsilon_{a}E^2-K_{1}q^{2}}{K_{3}q^{2}}.
\label{7.79}
\end{equation}
 Pri majhnih poljih, ko je $\epsilon_{0}\epsilon_{a}E^2<K_{1}q^{2}$,
je fizikalno smiselna le prva rešitev, ki predstavlja vzorec brez deformacije. Pri 
velikih poljih postane stabilna druga rešitev. Takrat deformacija
v sredini plasti hitro naraste, tako da se $\mathbf{n}$ postavi skoraj
popolnoma v smer zunanjega polja. Tedaj naša rešitev seveda ni dobra,
saj smo pri računu privzeli, da je $n_{z}\ll1$. Prehodu iz nedeformiranega
stanja v deformirano stanje pravimo  Frederiksov prehod\footnote{~Ruski fizik
Vsevolod Konstantinovič Frederiks, tudi Fr\'{e}edericksz, 1885--1944.}. Na njem
temelji preklapljanje optičnih prikazovalnikov na nematične tekoče 
kristale.\footnote{~V. Frederiks in V. Zolina, Trans. Faraday Soc. $\mathbf{29}$, 919 (1933).} 

Izračunajmo še kritično jakost električnega polja, pri kateri preide tekoči kristal v deformirano fazo.
To se zgodi pri 
\begin{equation}
\epsilon_{0}\epsilon_{a}E_c^2-K_{1}q^{2} = 0
\end{equation}
oziroma
\boxeq{FreeE}{
E_c = \frac{\pi}{d}\sqrt{\frac{K_1}{\varepsilon_0\varepsilon_a}}.
}
V tipičnem tekočekristalnem prikazovalniku je napetost, potrebna za prehod, $U = E_cd \sim 3~\si{\volt}$. 
\index{Tekočekristalni prikazovalnik}

Poglejmo še, kako narašča amplituda deformacije v bližini Frederiksovega prehoda. To lahko 
izračunamo iz enačbe~(\ref{7.79}) in dobimo 
\begin{equation}
a = \sqrt{\frac{2 \varepsilon_0 \varepsilon_a}{K_3 q^2 }(E^2-E_c^2)}.
\end{equation}
Pogosto naredimo približek enakih konstant, kjer privzamemo, da so vse Frankove 
elastične konstante enake vrednosti. V tem približku je 
\begin{equation}
a \approx \sqrt{\frac{2(E^2-E_c^2)}{E_c^2}}
\end{equation}
in amplituda korensko narašča z naraščajočim poljem (slika~\ref{Fred}). Tak prehod je
fazni prehod drugega reda, saj količina, ki opisuje prehod (amplituda deformacije $a$),
zvezno preide iz vrednosti $a=0$ v končno vrednost. 
\begin{figure}[ht]
\centering
\def\svgwidth{70truemm} 
\input{slike/09_Fred.pdf_tex}
\caption{Kvalitativno obnašanje amplitude deformacije $a$ ob Frederiksovem prehodu. Pri poljih, 
manjših od kritičnega, so molekule tekočega kristala urejene, nad kritičnim poljem
pa se ureditev deformira.}
\label{Fred}
\end{figure}

\begin{definition}
Izračunaj Frederiksov prehod v zasukani nematični celici (kot zasuka med zgornjo in spodnjo 
mejno ploskvijo naj bo $\pi/2$) in pokaži, da je kritično polje za prehod enako
\begin{equation}
E_c =  \frac{\pi}{d}\sqrt{\frac{K_1}{\varepsilon_0\varepsilon_a}}
\sqrt{1 + \frac{K_3-2K_2}{4K_1}}.
\end{equation}
Namig: uporabi nastavek $\varphi = z \pi/2d$ in $\vartheta = a \sin(\pi z/d)$. 
\end{definition}

%Final
% -------------------------------------------------------------------------------
% 	CHAPTER 11
% -------------------------------------------------------------------------------

\chapterimage{slike/NLO.jpg} 

\chapter{Nelinearna optika}
\label{chap:NLO}
Pri obravnavi svetlobnega valovanja v snovi smo doslej vedno privzeli linearno 
zvezo med polarizacijo in jakostjo električnega polja. To 
je seveda približek, ki je dovolj dober le pri razmeroma majhnih jakostih
polja. Kadar doseže jakost električnega polja velike vrednosti -- in v laserskih snopih
jih nedvomno lahko doseže -- je treba upoštevati tudi višje člene v razvoju. Takrat
govorimo o nelinearni optiki\index{Nelinearna optika}, saj zveza med polarizacijo
in električnim poljem ni linearna. V tem poglavju bomo spoznali zanimive pojave, ki jih 
povzroči nelinearni del polarizacije, med drugim optično 
frekvenčno podvajanje, optično parametrično ojačevanje, optično usmerjanje, 
samozbiranje laserskega snopa, optične solitone in optično fazno konjugacijo. 

\section{Nelinearna susceptibilnost}
\label{Chap:Chi}
V linearnem približku odziva snovi velja, da je polarizacija snovi\index{Električna polarizacija} 
$\mathbf{P}$ linearna funkcija jakosti električnega polja 
$\mathbf{E}$\index{Električno polje!jakost}. Takrat zapišemo (enačba~\ref{eq:PM})
\begin{equation}
\mathbf{P} = \mathbf{D} - \varepsilon_0 \mathbf{E} = 
\varepsilon_0 \underline{\epsilon} \cdot\mathbf{E} - \varepsilon_0 \mathbf{E} = 
\varepsilon_0 (\underline{\epsilon} - 1)\cdot\mathbf{E}. 
\end{equation}
Pri tem smo uporabili splošen zapis dielektričnosti v obliki tenzorja. Če vpeljemo še tenzor linearne susceptibilnosti\index{Susceptibilnost!linearna}
\begin{equation}
\chi^{(1)} = \underline{\epsilon} - 1,
\end{equation}
linearni odziv snovi zapišemo strnjeno kot
\begin{equation}
\mathbf{P}_{\mathrm{L}} =  \varepsilon_0 \chi^{(1)} \cdot \mathbf{E}.
\end{equation}
Ta približek je dober za majhne jakosti električnega polja. Pri večjih jakostih polja
postanejo pomembni tudi členi višjega reda v razvoju polarizacije
po $\mathbf{E}$
\boxeq{8.1}{
\mathbf{P}=\mathbf{P}_{\mathrm{L}}+ \mathbf{P}_{\mathrm{NL}}=
\epsilon_{0} \chi^{(1)}\cdot \mathbf{E}+
\epsilon_{0}\chi^{(2)}:\mathbf{E}\, \mathbf{E}+
\epsilon_{0}\chi^{(3)}\vdots \mathbin \mathbf{E}\mathbin \mathbf{E}\mathbin\mathbf{E} + \dots
}
Vpeljali smo nelinearni susceptibilnosti\index{Susceptibilnost!nelinearna} 
$\chi^{(2)}$ in $\chi^{(3)}$, ki sta tenzorja tretjega in četrtega ranga. 
Za nazornješo predstavo izpišimo notranja produkta tenzorjev z vektorji 
po komponentah
\begin{equation}
\left(\mathbf{P}_{\mathrm{NL,2}}\right)_i= 
\epsilon_{0}\sum_{j,k}\chi^{(2)}_{ijk} \,E_j \,E_k =
\epsilon_{0}\chi^{(2)}_{ijk} \,E_j \,E_k 
\label{eq:nlin2}
\end{equation}
in 
\begin{equation}
\left(\mathbf{P}_{\mathrm{NL,3}}\right)_i= 
\epsilon_{0}\sum_{j,k,l}\chi^{(3)}_{ijkl} \,E_j \,E_k\, E_l=
\epsilon_{0}\chi^{(3)}_{ijkl} \,E_j \,E_k\, E_l,
\label{eq:nlin3}
\end{equation}
pri čemer smo uporabili Einsteinov zapis seštevanja po indeksih. Značilne vrednosti
susceptibilnosti v trdnih snoveh so $\chi^{(1)} \sim 1$, 
$\chi^{(2)} \sim 10^{-11}~\si{\metre/\volt}$ 
in $\chi^{(3)} \sim 10^{-22}~\si{\metre^2/\volt^2}$. Obravnavali bomo samo snovi, v katerih
ni izgub in so susceptibilnosti realne.

\begin{definition}
Oceni gostoto svetlobnega toka, pri kateri postane nelinearni 
prispevek k polarizaciji znaten in velja 
 $$P_{NL}/P_L \sim 10^{-5}.$$
Tipične vrednosti so $\sim~1~\si{\giga\watt/\metre^2}$. Ker je to z navadnimi
svetili povsem nedosegljivo, je bilo mogoče nelinearne
optične pojave opazovati šele po iznajdbi laserjev.
\end{definition}
 
Tenzor $\chi^{(2)}$ je od nič različen le v snoveh, ki nimajo centra inverzije. 
Ker lahko v produktu (enačba~\ref{eq:nlin2}) vrstni red $E_j E_k$ zamenjamo, mora biti
tenzor invarianten na zamenjavo
\begin{equation}
\chi_{ijk} = \chi_{ikj}.
\label{eq:chijk}
\end{equation}
Zato lahko vpeljemo poenostavljen zapis, pri katerem prvi indeks 
prepišemo ($x=1$, $y=2$, $z=3$) in zadnja dva indeksa združimo, 
podobno kot pri elektro-optičnem pojavu (razdelek~\ref{chap:EO}).
Dogovorjene oznake so enake: $xx=1$, $yy=2$, $zz=3$, $yz=zy=4$, 
$xz=zx=5$, $xy=yx=6$. Namesto
splošnega tenzorja tretjega ranga smo uvedli matriko velikosti $3\times6$,
v kateri je zaradi simetrije navadno le nekaj komponent 
različnih od nič. 

Kadar je absorpcija v snovi dovolj majhna, lahko matriko poenostavimo
z dodatnim približkom, tako imenovano  
\index{Kleinmanova domneva} Kleinmanovo domnevo\footnote{~D. A. Kleinman, Phys. Rev. $\mathbf{126}$, 1977 (1962).}.
Ta pravi, da je 
\begin{equation}
\chi_{ijk} = \chi_{ikj} = \chi_{kij} = \chi_{kji} = \chi_{jik} = \chi_{jki}.
\label{Klein}
\end{equation}
\begin{table}[ht]
 \centering
\begin{tabular}{|c|c|c|c|} \hline  
      Kristal & Grupa & Neničelne komponente tenzorja $\chi$ & Vrednosti ($10^{-12}~\si{\metre/\volt}$)\\ \hline
      BaTiO\index{BaTiO$_3$}$_3$ & 4mm & $\chi_{xxz} = \chi_{yyz} = \chi_{xzx} = \chi_{yzy} = 
      \chi_{15} = \chi_{24}$  &
	    $\chi_{15} = 42,6$ \\
	      & & $\chi_{zxx} = \chi_{zyy} = \chi_{31} = \chi_{32}$ &  $\chi_{31} = 45,2$ \\
	      & & $\chi_{zzz} = \chi_{33}$ & $\chi_{33} = 16,0$ \\ \hline
      KDP\index{KDP} & 
      $\overline{4}$2m & $\chi_{xyz} = \chi_{yxz} = \chi_{xzy} = \chi_{yzx} = \chi_{14} = \chi_{25}$  &
	    $\chi_{14} = 0,88$ \\
	    & & $\chi_{zxy} = \chi_{zyx} = \chi_{36}$ &  $\chi_{36} =1,12$ \\ \hline
      Telur\index{Telur} & 32 & $\chi_{xxx} = -\chi_{xyy} = -\chi_{yyx} = -\chi_{yxy} =$  & \\
      & &  = $\chi_{11} = -\chi_{12}=-\chi_{26}$  &
	    $\chi_{11} = 1300$ \\
	    & & $\chi_{xyz} = \chi_{xzy} = -\chi_{yxz}= - \chi_{yzx}= \chi_{14} = 
	    -\chi_{25}$ &  $\chi_{14} \approx 0$ 
	    \\ \hline
      LiNbO$_3$\index{LiNbO$_3$} & 3m & $\chi_{xxz} = \chi_{yyz} = \chi_{xzx} = \chi_{yzy} = \chi_{15} = \chi_{24}$  &
	     \\
	     & & $\chi_{zxx} = \chi_{zyy} = \chi_{31} = \chi_{32}$ &  $\chi_{31} = -11,9$ \\
	      & & $\chi_{zzz} = \chi_{33}$ & $\chi_{33} = 68,8$ \\
	    & &  $-\chi_{xxy} = - \chi_{xyx} = \chi_{yyy} = -\chi_{yxx}  = $ & \\
	    & & $=-\chi_{16} = \chi_{22}$ = $-\chi_{21}$  &
	    $\chi_{22}  = 5,52$ \\
\hline 
\end{tabular}
  \caption{Koeficienti nelinearne susceptibilnosti za nekaj izbranih 
  snovi}
  \index{Susceptibilnost!nelinearna} 
\label{table:chi}
\end{table}

Poglejmo primer. 
Vzemimo barijev titanat (BaTiO$_3$)\index{BaTiO$_3$} s točkovno grupo 4mm. To pomeni, da
ima 4-števno os simetrije in dve zrcalni ravnini, od katerih ena preslika $x \to -x$ ali $y \to -y$, 
druga pa $x\to y$ in $y\to x$. Od nič različni elementi susceptibilnosti so tako samo
\begin{equation}
\chi_{xxz} = \chi_{xzx} =   \chi_{yyz} = \chi_{yzy}, \quad  \chi_{zzz} \qquad \mathrm{in} 
\qquad \chi_{zxx} = \chi_{zyy}.   
\end{equation}
Z upoštevanjem Kleinmanove domneve se število različnih členov še zmanjša in ostaneta le dva
\begin{equation}
\chi_{xxz} = \chi_{xzx} = \chi_{yyz} = \chi_{yzy} =\chi_{zxx} = \chi_{zyy} \qquad \mathrm{in} \qquad \chi_{zzz}.   
\end{equation}
Primerjajmo rezultat s tabelo~\ref{table:chi}\footnote{~A. Yariv in 
P. Yeh, {\it Photonics}, šesta izdaja, Oxford University Press (2007) in
{\it CRC Handbook of Chemistry and Physics}, CRC Press (2002).}, v kateri
so navedene izmerjene nelinearne susceptibilnosti\footnote{~Izmerjene vrednosti, 
ki jih najdemo v literaturi, se od vira do vira pogosto znatno razlikujejo.}. Vidimo, da Kleinmanova
domneva ni povsem točna, je pa razmeroma dober približek. 

\section{Nelinearni optični pojavi drugega reda}
\index{Nelinearna optika!drugega reda}
Vzemimo optično nelinearni kristal s $\chi^{(2)} \neq 0$. V smeri pravokotno 
glede na njegovo mejno ploskev naj vpadata dve valovanji s frekvencama\footnote{~Tudi 
v tem poglavju bomo $\omega$ namesto krožna frekvenca pogosto imenovali zgolj frekvenca.}
$\omega_{1}$ in $\omega_{2}$. Zaradi nelinearne sklopitve nastajajo v snovi nova 
valovanja z različnimi kombinacijami frekvenc (slika~\ref{fig:nl2}).
Tako poleg valovanj z osnovnima frekvencama izhajajo iz kristala tudi 
valovanja pri dvakratnikih obeh vstopnih frekvenc, pri njuni vsoti, 
razliki in celo pri frekvenci nič. Oglejmo si te pojave podrobneje.
\begin{figure}[ht]
\centering
\def\svgwidth{140truemm} 
\input{slike/08_nl3.pdf_tex}
\caption{Shematski prikaz nastanka valovanj pri nelinearnih optičnih pojavih drugega reda (a)
in spekter izhodne svetlobe (b). Intenzitete izhodnih žarkov niso v merilu.}
\label{fig:nl2}
\end{figure}

\begin{remark}
Nastanku valovanja pri podvojeni frekvenci oziroma optičnemu frekvenčnemu podvajanju pravimo tudi
SHG\index{SHG|see {Frekvenčno podvajanje}} ({\it Second Harmonic 
Generation})\index{Frekvenčno podvajanje}, 
nastanku valovanja pri vsoti frekvenc SFG\index{SFG|see {Generacija vsote frekvenc}}
({\it Sum Frequency Generation})\index{Generacija vsote frekvenc}, 
nastanku valovanja pri razliki frekvenc DFG\index{DFG|see {Generacija razlike frekvenc}} 
({\it Difference Frequency Generation})\index{Generacija razlike frekvenc} in pojavu 
statičnega polja pri $\omega = 0$ optično usmerjanje\index{Optično usmerjanje}
({\it optical rectification}).  
\end{remark}

Navadna valovna enačba za opis svetlobe v snovi  
ne velja za opis pojavov pri velikih 
intenzitetah vpadnih valovanj, saj se pojavi nelinearna polarizacija. Valovanje 
v takem primeru opišemo z nelinearno valovno 
enačbo\index{Valovna enačba!nelinearna}
\boxeq{8.3}{
\nabla^{2}\mathbf{E}-\frac{\epsilon}{c_0^{2}}{\frac{\partial^2\mathbf{E}}{\partial t^2}}=
\mu_{0}{\frac{\partial^2\mathbf{P}_{\textrm{NL}}}{\partial t^2}}.
}

\begin{definition}
Iz Maxwellovih enačb (enačbe~\ref{eq:Maxwell1}--\ref{eq:Maxwell4}) izpelji 
nelinearno valovno enačbo (enačba~\ref{8.3}), pri čemer upoštevaj enačbo~(\ref{8.1}). 
Pomagaj si z identiteto
\begin{equation}
\nabla \times (\nabla \times \mathbf{A}) = \nabla (\nabla \cdot \mathbf{A}) 
- \nabla^2 \mathbf{A}.
\end{equation}
\end{definition} 

Nelinearne valovne enačbe na splošno ne znamo rešiti, zato se zatečemo k približkom.
Prva poenostavitev je omejitev na vzporedna vpadna žarka,
ki se širita v smeri osi $z$. Poleg tega se omejimo na izračun samo enega
nastalega valovanja in privzamemo, da je neodvisno od drugih nastalih valovanj.
Ta omejitev ni huda. Dokler so namreč amplitude nastalih valovanj majhne, 
jih lahko obravnavamo ločeno. Ni sicer nujno,
da so amplitude nastalih valov vedno majhne, vendar je lahko, kot bomo videli 
pozneje, le eno nastalo valovanje naenkrat po jakosti primerljivo z vpadnima. 

V snovi so tako prisotna tri valovanja:
dve vpadni in eno novonastalo. Zapišemo jih z 
\begin{align}
\mathbf{E}_{1} & =  \frac{\mathbf{e}_{1}}{2}\left(A_{1}(z)\, 
e^{i(k_{1}z-\omega_{1}t)}+A_{1}^{*}(z)\, e^{-i(k_{1}z-\omega_{1}t)}\right)\!,\nonumber \\
\mathbf{E}_{2} & =  \frac{\mathbf{e}_{2}}{2}\left(A_{2}(z)\, 
e^{i(k_{2}z-\omega_{2}t)}+A_{2}^{*}(z)\, e^{-i(k_{2}z-\omega_{2}t)}\right) \qquad \mathrm{in} \nonumber \\
\mathbf{E}_{3} & =  \frac{\mathbf{e}_{3}}{2}\left(A_{3}(z)\, 
e^{i(k_{3}z-\omega_{3}t)}+A_{3}^{*}(z)\, e^{-i(k_{3}z-\omega_{3}t)}\right)\!.
\end{align}
Ker valovna enačba (enačba~\ref{8.3}) ni linearna, smo polja 
zapisali v realni obliki s kompleksno konjugiranimi
deli. Upoštevali smo tudi,
da so zaradi nelinearnih pojavov amplitude $A$ funkcije kraja, za
katere privzamemo, da se le počasi spreminjajo. Njihova kompleksna vrednost
dopušča pojav dodatnega faznega zamika. Za valovna
števila velja $k_{n}^{2}=\epsilon_{n}\omega_n^{2}/c_0^{2}$,
pri čemer je $\epsilon_{n}$ dielektrična konstanta pri frekvenci
$\omega_{n}$ in polarizaciji $\mathbf{e}_{n}$, indeks $n = 1...3$ pa označuje
valovanje. S tem nastavkom vsako od treh valovanj
pri konstantni amplitudi reši linearni del valovne enačbe. 

Naša naloga
je ugotoviti, kako se zaradi nelinearnih pojavov spreminjajo amplitude posameznih valovanj.
Nastavek za polje, ki bo približno rešil nelinearno valovno enačbo, je tako
\begin{equation}
\mathbf{E}(z,t) = \sum_{n=1}^3 \frac{\mathbf{e}_{n}}{2}\left(A_{n}(z)\, 
e^{i(k_{n}z-\omega_{n}t)}+A_{n}^{*}(z)\, e^{-i(k_{n}z-\omega_{n}t)}\right)\!.
\label{eq:nlnastavek}
\end{equation}
Izračunajmo najprej 
\begin{equation}
\nabla^{2}\mathbf{E}=-\sum_{n=1}^3 \frac{\mathbf{e}_{n}}{2}\left(k_{n}^{2}A_{n}(z)-2ik_{n}
\frac{dA_{n}(z)}{dz}\right)\, e^{i(k_{n}z-\omega_{n}t)}+\mathrm{k.~k.}
\label{8.5}
\end{equation}
S k.\,k. smo označili kompleksno konjugirani del. Upoštevali smo,
da se amplituda $A_{n}(z)$ le počasi spreminja s krajem, zato smo njen
drugi odvod po kraju zanemarili.
Izračunamo še drugi odvod po času 
\begin{equation}
\frac{\partial^2\mathbf{E}}{\partial t^2}=\sum_{n=1}^3 \frac{\mathbf{e}_{n}}{2}
\left(-\omega_n^2\right) \left(A_{n}(z)\, e^{i(k_{n}z-\omega_{n}t)}+\mathrm{k.~k.}\right)\!.
\label{8.5a}
\end{equation}
Nelinearna polarizacija vsebuje produkte polj, ki nihajo z
vsotami in razlikami parov frekvenc $\omega_{1}$, $\omega_{2}$ in
$\omega_{3}$.

Vstavimo nastavek za polje (enačba~\ref{eq:nlnastavek}) 
in dobimo\footnote{~Spomnimo, da je $\chi^{(2)}:\mathbf{e}_{n}\,\mathbf{e}_{m}$ 
notranji produkt tenzorja z enotskima vektorjema polarizacije, 
katerega $i$-ta komponenta se izračuna kot $\sum_{jk}\chi^{(2)}_{ijk}\mathbf{e}_{nj}\,\mathbf{e}_{mk}$.
Indeksa $n$ in $m$ označujeta valovanje, $i, j$ in $k$ pa kartezične koordinate.}
\begin{equation}
\begin{split}
\mathbf{P}_{\mathrm{NL}}= \epsilon_{0}\chi^{(2)}:\mathbf{E}\, \mathbf{E} =
\varepsilon_0 \sum_{n=1}^3 \sum_{m=1}^3 
 \left( \frac{1}{4} \chi^{(2)}:\mathbf{e}_{n}\,\mathbf{e}_{m}\right) 
 A_{n}(z)\,A_{m}(z) e^{i(k_{n}+k_{m})z-i(\omega_{n}+\omega_{m})t}+  \\
\left( \frac{1}{4} \chi^{(2)}:\mathbf{e}_{n}\,\mathbf{e}_{m}\right)
A_{n}(z)\,A_{m}^*(z) e^{i(k_{n}-k_{m})z-i(\omega_{n}-\omega_{m})t}+ \mathrm{k.~k.}
\label{8.5b}
\end{split}
\end{equation}
Valovna enačba (enačba~\ref{8.3}) je izpolnjena ob vsakem času $t$, če se izrazi 
pri istih časovnih odvisnostih, to je pri istih frekvencah, ujemajo. Če
zberemo člene pri $\omega_3 = \omega_1 + \omega_2$, dobimo 
\begin{equation}
ik_{3}\mathbf{e}_{3}\frac{dA_{3}}{dz}e^{ik_{3}z}=-\frac{\mu_{0} 
\varepsilon_0 \omega_{3}^{2}}{4}\chi^{(2)}:\mathbf{e}_{1}\mathbf{e}_{2}\,A_{1}\,A_{2}e^{i(k_{1}+k_{2})z}.
\label{8.7}
\end{equation}
Množimo obe strani skalarno z $\mathbf{e}_{3}$, upoštevamo zvezo med $k_{3}$ in $\omega_{3}$
ter ravnamo podobno še za drugi dve valovanji. Dobimo sistem sklopljenih
enačb za amplitude valovanj v optično nelinearnem sredstvu
\boxeq{eq:nlAz}{
\frac{dA_{3}}{dz} &= \frac{i\omega_{3}\chi_{ef}}{4c_0 n_3} A_{1}\, A_{2}\, e^{-i\Delta kz}\\
\frac{dA_{2}}{dz} &= \frac{i\omega_{2}\chi_{ef}}{4c_0 n_2} A_{1}^*\, A_{3}\, e^{i\Delta kz}\\
\frac{dA_{1}}{dz} &= \frac{i\omega_{1}\chi_{ef}}{4c_0 n_1} A_{2}^*\, A_{3}\, e^{i\Delta kz}\label{eq:nlA3}.
}
Pri tem je $\Delta k$ razlika valovnih vektorjev
\begin{equation}
\Delta k = k_{3}-k_{1}-k_{2}.
\end{equation}
Čeprav je $\omega_{3}-\omega_{2}-\omega_{1}=0$, je $\Delta k$ navadno različen od nič zaradi 
disperzije lomnega količnika. Videli bomo, da je to ključnega pomena 
pri vrsti nelinearnih optičnih pojavov. S $\chi_{ef}$ pa smo označili efektivno 
susceptibilnost.\footnote{~Glej npr. F. Zernike in J. E. Midwinter, 
{\it Applied Nonlinear Optics}, John Wiley \& Sons, Inc. (1973).}
Izračunamo jo kot \index{Susceptibilnost!efektivna}
\begin{equation}
\chi_{ef}=\mathbf{e}_{3}\cdot\chi:\,\mathbf{e}_{1}\,\mathbf{e}_{2} = 
\sum_{ijk} \chi_{ijk}^{(2)} e_{3i} e_{1j} e_{2k}.
\label{eq:chicomp}
\end{equation}
Ker polarizacijski vektorji niso nujno vzporedni s koordinatnimi osmi, $\chi_{ef}$ 
niso čiste kartezične komponente tenzorja nelinearne susceptibilnosti.
\begin{definition}
\vglue-3truemm
Pokaži, da iz Kleinmanove domneve (enačba~\ref{Klein}) sledi, da so \index{Kleinmanova domneva}
efektivne susceptibilnosti $\chi_{ef}$ v vseh treh enačbah~(\ref{eq:nlAz}--\ref{eq:nlA3}) enake.
\end{definition}
\begin{definition}
\vglue-3truemm
Pokaži, da nastavek za polje v nelinearni snovi (enačba~\ref{eq:nlnastavek}) reši nelinearno
valovno enačbo (enačba~\ref{8.3}), in pokaži, da spreminjanje amplitude posameznih valovanj 
ustreza enačbam~(\ref{eq:nlAz}--\ref{eq:nlA3}).
\end{definition}
Čeprav je $\omega_{3}-\omega_{2}-\omega_{1}=0$, je $\Delta k$ navadno različen od nič zaradi 
frekvenčne disperzije lomnega količnika. Videli bomo, da je to ključnega pomena 
pri vrsti nelinearnih optičnih pojavov. 

Zapisani sistem diferencialnih enačb (enačbe~\ref{eq:nlAz}--\ref{eq:nlA3}) opisuje več pojavov, 
odvisno od začetnih pogojev in relativnih intenzitet. Opisali bomo nekaj
najpomembnejših primerov.

\section{Optično frekvenčno podvajanje}
\label{chap:SHG}
Obravnavajmo optično nelinearno sredstvo, na katerega vpadata valovanji ${\mathbf E}_1$ in
$\mathbf{E}_2$ z enakima frekvencama $\omega_{1}=\omega_{2}=\omega$. Vpadni valovanji
razlikujemo zaradi možnosti dveh različnih polarizacij. Izhodna svetloba vsebuje valovanje
s frekvenco $\omega_{3}=2\omega$, zato govorimo o frekvenčnem 
podvajanju\index{Frekvenčno podvajanje}. To je
najpreprostejši in tudi najpomembnejši nelinearni optični pojav.\footnote{~P. A. Franken et al, Phys. Rev. Lett.
$\mathbf{7}$, 118 (1961).} 
Pogosto ga uporabljamo za pridobivanje laserske svetlobe pri krajših valovnih dolžinah, na primer
pri Nd:YAG laserju\index{Laser!Nd:YAG}, ko infrardeče izhodno valovanje ($1064~\si{\nano\metre}$) 
pretvorimo v vidno svetlobo zelene barve ($532~\si{\nano\metre}$).\index{Infrardeče valovanje} 

Zanima nas, kako se $A_{3}(z) = A_{2\omega}(z)$ spreminja vzdolž nelinearnega kristala
pri začetnem pogoju $A_{2\omega}(0)=0$.
Privzamemo, da se pretvori le manjši del vpadnega energijskega toka in da ostaneta 
amplitudi $A_{1}=A_{2}=A_0$ približno konstantni. Tedaj lahko
enačbo za $A_{3}(z)$ (enačba~\ref{eq:nlAz}) brez težav integriramo do dolžine kristala $L$ in 
dobimo
\begin{equation}
A_{2\omega}(L)=\frac{i\omega \chi_{ef} A_0^2}{2c_0 n_{2\omega}}
\,e^{-i\Delta kL/2}\, \frac{\sin\left(\frac{\Delta k L}{2}\right)}{\frac{\Delta kL}{2}}L,
\label{8.9}
\end{equation}
pri čemer smo z $n_{2\omega}$ označili lomni količnik pri dvojni frekvenci.
Iz tega izraza izračunamo izhodno gostoto svetlobnega toka pri dvojni
frekvenci. Upoštevamo enačbo~(\ref{eq:jcw}) in dobimo
\begin{equation}
j_{2\omega}(L) =\frac{1}{2}\epsilon_{0}n_{2\omega}c_0|A_3|^2 = 
\frac{\omega^2 \chi_{ef}^2}{2 n_{2\omega} n_\omega^2c_0^3\varepsilon_0} j_\omega^2 L^2
\left(\frac{\sin\left(\frac{\Delta k L}{2}\right)}{\frac{\Delta kL}{2}}\right)^2.
\label{8.10}
\end{equation}
\vglue-6truemm
\begin{remark}
Pri izpeljavi frekvenčnega podvajanja iz enačb za nelinearne pojave drugega
reda (enačbe~\ref{eq:nlAz}--\ref{eq:nlA3}) moramo biti pazljivi. 
Uporabili smo splošne enačbe in predpostavili, da je 
vpadno valovanje se\-stav\-lje\-no iz dveh ločenih valovanj s frekvenco $\omega$
in gostoto svetlobnega toka $j_\omega$. Lahko pa frekvenčno podvajanje obravnavamo
z enim vpadnim valovanjem s frekvenco $\omega$ in gostoto svetlobnega toka $2j_\omega$, 
ki nelinearno interagira samo s sabo. Takrat je zapis enačb za predfaktor
drugačen, končni rezultat pa seveda enak. 
\end{remark}
\vglue-2truemm
Gostota energijskega toka frekvenčno podvojene svetlobe torej narašča s kvadratom
intenzitete vpadne svetlobe. Naj bo $S$ presek snopa. Potem je razmerje med 
energijskim tokom pri podvojeni in osnovni frekvenci (izkoristek pretvorbe) enako
\boxeq{8.11}{
\frac{P_{2\omega}}{P_{\omega}}=
\frac{\omega^2 \chi_{ef}^2}{2 S n_{2\omega} n_\omega^2c_0^3\varepsilon_0} P_\omega L^2
\left(\frac{\sin\left(\frac{\Delta k L}{2}\right)}{\frac{\Delta kL}{2}}\right)^2\!.
}
Izkoristek pretvorbe v frekvenčno podvojeno valovanje narašča z močjo vpadnega valovanja.
Vendar je pomembnejši faktor, ki v enačbi~(\ref{8.11}) nastopa v oklepaju.
Odvisnost faktorja v oklepaju od $\Delta kL/2$ je prikazana 
na sliki~\ref{fig:shg2}. Faktor je največji pri $\Delta kL = 0$, potem pa zelo hitro
pade na zelo majhne vrednosti. 
\begin{figure}[ht]
\centering
\def\svgwidth{72truemm} 
\input{slike/08_shg2.pdf_tex}
\caption{Izkoristek pretvorbe v frekvenčno podvojeno valovanje $j_{2\omega}/j_\omega$ je 
sorazmeren s funkcijo $(\sin(x)/x)^2$,
pri čemer je $x = \Delta k L/2$ (enačba~\ref{8.11}). 
Pri tem $\Delta k$ označuje razliko valovnih vektorjev in
$L$ prepotovano pot v kristalu.}
\label{fig:shg2}
\end{figure}
Izkoristek pretvorbe v frekvenčno podvojeno valovanje je tako velik le pri  majhnih vred\-no\-stih 
$\Delta kL$. To lahko dosežemo z majhno debelino kristala, veliko bolj smiselno
pa je poiskati pogoje, pri katerih je $\Delta k = 0$. Če uspemo izpolniti pogoj, da
se faze valovanj ujemajo, je vrednost faktorja 
$\sin(\Delta kL/2)/(\Delta kL/2)=1$ in neodvisna od dolžine poti $L$.

V tem primeru izkoristek pretvorbe narašča sorazmerno s kvadratom poti
\begin{equation}
\frac{P_{2\omega}}{P_{\omega}}=
\frac{\omega^2 \chi_{ef}^2}{2 S n_{2\omega} n_\omega^2c_0^3\varepsilon_0} P_\omega L^2.
\label{eq:shgl2}
\end{equation}
Za čim večjo pretvorbo v frekvenčno podvojeno valovanje je torej treba 
izpolniti pogoj $\Delta k = 0$\index{Ujemanje faz}. Kako to naredimo,
bomo spoznali v nadaljevanju.

\begin{definition}
\label{deplet}
Pokazali smo, da izkoristek pretvorbe v frekvenčno podvojeno valovanje 
pri $\Delta k = 0$  narašča sorazmerno s kvadratom 
dolžine kristala (enačba~\ref{eq:shgl2}). Takšna odvisnost velja, 
dokler je $j_{2\omega}$ bistveno manjša od $j_{\omega}$ oziroma $A_3 \ll A_1, A_2$ (slika~\ref{fig:shg2dep}). 
Pokaži, da v nasprotnem primeru gostota svetlobnega toka frekvenčno
podvojenega valovanja $j_{2\omega}(L)$ narašča kot
\begin{equation}
j_{2\omega} (L) = j_\omega \tanh^2 \left(\chi_{ef}\omega L \sqrt{\frac{j_\omega}
{2 n_{2\omega} n_\omega^2 c_0^3 \varepsilon_0}} \right) = j_\omega \tanh^2(\kappa L).
\label{eq:nondepl}
\end{equation}
Namig: upoštevaj, da se celotna energija ohranja.
\end{definition}

\begin{figure}[ht]
\centering
\def\svgwidth{72truemm} 
\input{slike/08_shg_depletion.pdf_tex}
\caption{Izkoristek pretvorbe v frekvenčno podvojeno valovanje $j_{2\omega}/j_\omega$ 
pri $\Delta k = 0$.
Če privzamemo, da se gostota svetlobnega toka osnovnega žarka ne zmanjšuje, 
je odvisnost parabolična (rdeča krivulja), kar 
je dober približek le za majhne gostote toka. Natančnejši izračun pokaže, da je izkoristek 
pretvorbe sorazmeren s $\tanh^2(\kappa L)$ (enačba~\ref{eq:nondepl}, črna krivulja).}
\label{fig:shg2dep}
\end{figure}

Kaj pa se zgodi, kadar pogoj ujemanja faz ni izpolnjen in 
 $\Delta k \neq 0$? Takrat lahko  $L^2$ v enačbi~(\ref{8.11})
okrajšamo in izkoristek pretvorbe z naraščajočim
$L$ sinusno niha med nič in neko največjo vrednostjo. Ta pojav lahko opazimo, če
uporabimo klinast vzorec, ki se mu spreminja debelina, ali če vzorec sučemo 
in tako spreminjamo razliko faz. Pojav, imenujemo ga Makerjeve 
oscilacije\footnote{~P. D. Maker et al., Phys. Rev. Lett. $\mathbf{8}$, 21 (1962).}, 
uporabljamo za merjenje nelinearne susceptibilnosti kristalov.\index{Makerjeve oscilacije}

\subsection*{Ujemanje faz}
\index{Ujemanje faz}
Poglejmo, kako lahko dosežemo ujemanje faz, ki je nujno za učinkovito optično
frekvenčno podvajanje. Spomnimo se, da je pogoj za ujemanje faz 
\begin{equation}
\Delta k = k_3 - k_1 -k_2 = k_3(2\omega) - k_1(\omega) -k_2(\omega) = 
\frac{2\omega}{c_0} n_3(2\omega) - \frac{\omega}{c_0} n_1(\omega)- \frac{\omega}{c_0} n_2(\omega) =0.
\end{equation}
Valovna števila izrazimo s krožnimi frekvencami, pri čemer 
lomnim količnikom pripišemo ustrezno frekvenco svetlobe in jo navedemo v oklepajih. Dobimo
pogoj za ujemanje faz 
\boxeq{eq:dk0}{
n_1(\omega) + n_2(\omega) = 2n_3(2\omega).
}
Da lahko zadostimo temu pogoju, izkoristimo dvojni lom\index{Dvolomnost} v 
anizotropnih kristalih. Omejimo se le na optično 
enoosne kristale\index{Dvolomnost!enoosne snovi} brez absorpcije in z 
normalno disperzijo, pri katerih oba lomna količnika naraščata s frekvenco.  

Za razumevanje je najnazornješi grafični prikaz (slika~\ref{fig:dk}), pri katerem
rišemo presek ploskve valovnega vektorja z ravnino, določeno z optično osjo in valovnim
vektorjem (glej razdelek~\ref{chap:anizotropni}). 
Za vsako smer valovnega vektorja obstajata dve rešitvi:
rednemu žarku, katerega polarizacija je pravokotna na omenjeno ravnino,
ustreza krožnica s polmerom $n_o$, izrednemu, katerega polarizacija leži v ravnini, 
pa elipsa s polosema $n_o$ in $n_e$.\index{Lomni količnik} 
Rdeča barva nakazuje presek ploskve pri vpadni frekvenci, modra pa pri podvojeni. 
Ekscentričnost elipse in frekvenčna disperzija sta zaradi večje nazornosti na sliki 
močno pretirani. 

Podrobneje poglejmo primer s slike~\ref{fig:dk}\,a, za katerega velja $n_e>n_o$. 
Opazimo, da se v neki točki rdeča elipsa, ki ustreza vpadnemu valovanju s frekvenco $\omega$, 
seka z modro krožnico, ki ustreza valovanju s podvojeno frekvenco $2\omega$. Pri kotu 
$\vartheta_m$ je torej redni lomni 
količnik pri dvojni frekvenci enak izrednemu pri osnovni
frekvenci. Če izberemo izredno polarizacijo vpadnega valovanja, je za podvojeno 
valovanje z redno polarizacijo pri kotu $\vartheta_m$ izpolnjen pogoj ujemanja 
faz~(enačba~\ref{eq:dk0}). Takrat leži polarizacija vpadnega valovanja v ravnini,
ki jo določata smer optične osi in smer valovnega vektorja, polarizacija 
izhodnega frekvenčno podvojenega žarka pa 
je pravokotna na optično os. Zapišimo ta razmislek še z enačbo.

\begin{figure}[ht]
\centering
\def\svgwidth{145truemm} 
\input{slike/08_phasematch.pdf_tex}
\caption{Štirje primeri, pri katerih je izpolnjen pogoj za ujemanje faz pri kotu $\vartheta_m$. 
Ujemanje faz prve vrste za pozitivno anizotropno snov (a), 
ujemanje faz prve vrste za negativno anizotropno snov (b) ter 
ujemanje faz druge vrste za pozitivno (c) in negativno (d) anizotropno snov.}
\label{fig:dk}
\end{figure}

V obravnavanem primeru mora biti lomni količnik za redno polarizirano valovanje pri 
podvojeni frekvenci $n_o(2\omega)$ enak lomnemu količniku za izredno 
polarizirano valovanje pri osnovni frekvenci $n(\omega)$. Lomni količnik
za izredno valovanje je seveda odvisen od kota (enačba~\ref{eq:izreden})
\begin{equation}
\frac{1}{n^2(\omega,\vartheta)}=
\frac{\cos^{2}\vartheta}{n_{o}^2(\omega)}+\frac{\sin^{2}\vartheta}{n_{e}^2(\omega)}.
\label{8.12}
\end{equation}
Ko izenačimo $n_o(2\omega) = n(\omega,\vartheta_m)$, dobimo
\begin{equation}
\cos^{2}\vartheta_m=\frac{(n_o(2\omega))^{-2}-(n_{e}(\omega))^{-2}}
{(n_{o}(\omega))^{-2}-(n_{e}(\omega))^{-2}},
\label{8.13}
\end{equation}
iz česar lahko izračunamo kot $\vartheta_m$, pri katerem pride do ujemanja faz.
\begin{definition}
\vglue-3truemm
Pokaži, da v primeru negativne anizotropije (slika~\ref{fig:dk}\,b) 
pogoj za ujemanje faz zapišemo kot
\begin{equation}
\cos^{2}\vartheta_m=\frac{(n_o(\omega))^{-2}-(n_{e}(2\omega))^{-2}}
{(n_{o}(2\omega))^{-2}-(n_{e}(2\omega))^{-2}}.
\label{8.13a}
\end{equation}
\end{definition}

S slik~\ref{fig:dk}\,c in d razberemo, da obstaja še en primer, pri 
katerem je izpolnjen pogoj za ujemanje faz. Kot zgled tako imenovanega ujemanja faz druge vrste
obravnavajmo primer na sliki~\ref{fig:dk}\,c.
Vzemimo različno polarizirani vhodni valovanji z ustreznima različnima lomnima
količnikoma $n_o(\omega)$ in $n(\omega,\vartheta)$, ki ju na sliki predstavljata
rdeča krog in elipsa.
Enačba~(\ref{eq:dk0}) je izpolnjena, kadar je povprečje lomnih količnikov
vhodnih valovanj  enako lomnemu količniku frekvenčno podvojenega žarka $n_o(2\omega)$. 
To se zgodi pri tistem kotu $\vartheta_m$, pri katerem je modra krožnica ravno na 
sredini med rdečo krožnico in rdečo elipso. Za praktično uporabo je ta izbira, kadar obstaja,
celo ugodnejša, saj je pri njej kot ujemanja faz bliže $\pi/2$. 
Ujemanje faz je zato manj občutljivo na majhna odstopanja v kotu ali na temperaturne
spremembe lomnih količnikov. 

\subsection*{Efektivna susceptibilnost}
\index{Susceptibilnost!efektivna}
Na izhodno moč frekvenčno podvojenega snopa (enačba~\ref{8.11}) 
poleg faznega faktorja bistveno vpliva tudi efektivna 
susceptibilnost $\chi_{ef}$ (enačba~\ref{eq:chicomp}). Ta je odvisna 
od polarizacij vhodnega in izhodnega žarka ter seveda od simetrije kristala. 

Ugotovili smo, da je v optično enoosnih kristalih pogoj ujemanja faz 
določen s smerjo valovnega vektorja glede na smer optične osi 
$\vartheta_m$ (enačbi~\ref{8.13} in \ref{8.13a}). Zaradi simetrije
to pomeni, da obstaja cel stožec smeri, v katerih se faze ujemajo.
Drugi kot, ki določa smer širjenja valovanja v ravnini, pravokotni na optično os, 
izberemo tako, da izkoristimo največje komponente nelinearne 
susceptibilnosti.\footnote{~Glej npr. C. C. Davis, {\it Lasers and Electro-Optics}, 
Cambridge University Press (2006).}

Poglejmo primer KDP (KH$_{2}$PO$_{4}$)\index{KDP}, ki je negativno anizotropen 
z vrednostmi $n_o(\omega) = 1,4942$, 
$n_e(\omega) = 1,4603$, $n_o(2\omega) = 1,5129$ in $n_e(2\omega) = 1,4709$.\footnote{~A. Yariv in 
P. Yeh, {\it Photonics}, šesta izdaja, Oxford University Press (2007).}
Podatki so za valovno dolžino osnovnega snopa 1064~nm. 
Zaradi negativne anizotropije za izračun kota ujemanja faz 
uporabimo enačbo~(\ref{8.13a}) in dobimo $\vartheta_m = 41,25^\circ$. 
Poleg tega iz tabele~\ref{table:chi} razberemo, da ima nelinearna susceptibilnost v tetragonalni
simetriji $\bar{4}2$m od nič različne komponente $\chi_{xyz}$, $\chi_{xzy}$, $\chi_{yxz}$,
$\chi_{yzx}$, $\chi_{zxy}$ in $\chi_{zyx}$.
Zaradi poenostavitve privzamemo, da so njihove vrednosti enake. 

Označimo enotski vektor v smeri valovnih vektorjev osnovnega in frekvenčno 
podvojenega valovanja s $\mathbf{s}$. Pomagamo si s sliko~\ref{fig:chi} in zapišemo 
vektor $\mathbf{s}$, pri čemer $\varphi$ označuje kot med osjo $x$ in projekcijo 
$\mathbf{s}$ na ravnino $xy$
\begin{equation}
\mathbf{s}=(\cos\varphi\sin\vartheta_m,\sin\varphi\sin\vartheta_m,\cos\vartheta_m).
\label{8.14}
\end{equation}

\begin{figure}[ht]
\centering
\def\svgwidth{80truemm} 
\input{slike/08_chi.pdf_tex}
\caption{K izračunu efektivne susceptibilnosti. Črtkan krog opisuje osnovno ploskev
stožca, ki je določen s $\vartheta_m$, pikčast krog pa njegovo projekcijo
na ravnino $xy$. Rdeč vektor označuje polarizacijo redno polariziranega valovanja in
moder polarizacijo izredno polariziranega valovanja.}
\label{fig:chi}
\end{figure}

Naša naloga je poiskati kot $\varphi$, pri katerem je 
$\chi_{ef}$ največji in s tem največja tudi moč frekvenčno podvojenega valovanja.
Iz pogoja za ujemanje faz smo določili, da mora biti vpadna svetloba redno polarizirana in 
izhodna frekvenčno podvojena izredno polarizirana. Pri tem je redna polarizacija pravokotna na 
ravnino, ki jo tvorita os $z$ in vektor $\mathbf{s}$. Zapišemo jo kot
\begin{equation}
\mathbf{e}_o=(e_{ox}, e_{oy}, e_{oz}) = (\sin\varphi,-\cos\varphi,0).
\label{8.15}
\end{equation}
To najlažje preverimo tako, da postavimo vektor $\mathbf{s}$ enkrat v ravnino $xz$ in
drugič v ravnino $yz$. Izredna polarizacija leži v ravnini, ki jo tvori 
vektor $\mathbf{s}$ z osjo $z$, hkrati pa je približno pravokotna na vektor $\mathbf{s}$. 
Majhno odstopanje zaradi anizotropije kristala tukaj zanemarimo. Izredno polarizacijo 
tako zapišemo kot  
\begin{equation}
\mathbf{e}_e=(e_{ex}, e_{ey}, e_{ez}) 
=(-\cos \varphi \cos \vartheta_m,-\sin \varphi \cos \vartheta_m ,\sin \vartheta_m).
\label{8.15a}
\end{equation}
Zdaj lahko izračunamo efektivno susceptibilnost (enačba~\ref{eq:chicomp}), 
pri čemer upoštevamo, da sta žarka 1 in 2 pri osnovni frekvenci redno polarizirana, medtem ko
žarek z oznako 3 opisuje izredno polariziran žarek pri podvojeni frekvenci
\begin{equation}
\chi_{ef} = \sum_{ijk} \chi_{ijk} e_{3i} e_{1j} e_{2k} = \sum_{ijk} \chi_{ijk} e_{ei} e_{oj} e_{ok}.
\end{equation}
Krajši račun pokaže, da je zaradi oblike tenzorja nelinearne susceptibilnosti v izbranem 
primeru od nič različna le ena komponenta nelinearne polarizacije, komponenta $z$. Dobimo
\begin{equation}
\chi_{ef} = \chi_{zxy} e_{ez} e_{ox} e_{oy} + \chi_{zyx} e_{ez} e_{oy} e_{ox}.
\end{equation}
Vstavimo komponente vektorjev $\mathbf{e}_o$ in $\mathbf{e}_e$ (enačbi~\ref{8.15} in \ref{8.15a}) in dobimo
\begin{align}
P_{z}^{2\omega}=- 2\varepsilon_0\, \chi_{zxy}E_{0}^2\cos\varphi\sin\varphi
\sin\vartheta_m = - \varepsilon_0\, \chi_{zxy}E_{0}^2\sin(2\varphi) \sin\vartheta_m.
\label{8.151}
\end{align}
Nelinearna polarizacija je največja pri $\varphi=\pi/4$, največji $\chi_{ef}$  pa je 
\begin{equation}
\chi_{ef}= 
\sin\vartheta_m \chi_{zxy} \approx 0,66\, \chi_{zxy} \approx 
0,74~\si{\pico\metre/\volt}.
\label{8.16}
\end{equation}

\begin{definition}
Izračunaj največjo efektivno nelinearno susceptibilnost za
frekvenčno podvajanje svetlobe z valovno
dolžino $10~\si{\micro\metre}$ v kristalu telurja s simetrijsko grupo 32 (glej tabelo~\ref{table:chi}). 
Lomni količniki: $n_o(\omega) = 4,7969$, 
$n_e(\omega) = 6,2455$, $n_o(2\omega) = 4,8657$ in $n_e(2\omega) = 6,3152$.\index{Telur}
\end{definition}

\begin{remark}
Namesto zvezne svetlobe za optično podvajanje frekvenc pogosto uporabimo kratke laserske sunke, saj je 
vršna moč v njih zelo velika in je zato velika tudi pretvorba v frekvenčno podvojeni signal.
Vendar je treba biti pazljiv, saj lahko zaradi disperzije grupne hitrosti osnovni in 
podvojeni signal ne potujeta z enakima hitrostma. Navadno podvojeni signal potuje 
počasneje in zaostaja za osnovnim, zato lahko iz kristala izhaja razmeroma sploščen in 
precej razvlečen frekvenčno podvojeni sunek svetlobe. Pojav je izrazit predvsem pri podvajanju v
ultravijolični del spektra.\index{Ultravijolično valovanje}
\end{remark}

\section{Frekvenčno podvajanje Gaussovih snopov}
\index{Frekvenčno podvajanje!Gaussovih snopov}
Doslej smo vpadno in frekvenčno podvojeno svetlobo obravnavali kot ravni valovanji,
ki sta bili razsežni v prečni smeri. Izračunali smo, da v primeru \index{Ujemanje faz}
ujemanja faz ($\Delta k=0$)
moč frekvenčno podvojene svetlobe narašča s kvadratom dolžine poti po nelinearnem
sredstvu. Pretvorba v frekvenčno podvojeno svetlobo je po enačbi~(\ref{8.11}) sorazmerna
z gostoto svetlobnega toka pri osnovni frekvenci.
Zato v praksi vpadno svetlobo vselej zberemo in tako povečamo gostoto toka. 
Pri tem moramo paziti, da je nelinearni kristal odporen proti poškodbam
zaradi velike gostote svetlobnega toka. Odpornost in možnost izpolnitve kriterija ujemanja 
faz sta poglavitna kriterija pri izbiri snovi za frekvenčno podvajanje. 

Poglejmo, kako se enačbe spremenijo, če je vpadni snop pri osnovni 
frekvenci Gaussove oblike\index{Gaussov snop!frekvenčno podvajanje}. 
Rezultat lahko ocenimo, če vzamemo, da je
efektivna dolžina za pretvorbo $L$ kar enaka dolžini območja bližnjega polja; izven tega območja je 
gostota toka znatno manjša, s tem pa tudi izkoristek pretvorbe v 
frekvenčno podvojeni snop.
Celotna  dolžina $L$ je (enačba~\ref{eq:z0})
\begin{equation}
L=2z_{0}=2\frac{\pi w_{0}^{2}}{\lambda/n} = \frac{n w_0^2 \omega}{c_0}  \qquad \mathrm{in~~zato} \qquad 
w_{0}^{2} = \frac{c_0 L}{n \omega}.
\label{SHGG}
\end{equation}
Pri zapisu preseka vpadnega snopa upoštevajmo še faktor ena polovica, ki ga dobimo ob integraciji
 gostote svetlobnega toka
snopa po celotni površini (glej nalogo~\ref{naloga-širina-snopa}). Sledi
\begin{equation}
S=\frac{1}{2}\pi w_{0}^{2} = \frac{\pi c_0 L}{2 n \omega}.
\end{equation}
Večja ko je dolžina $L$, na kateri se svetloba frekvenčno podvaja, 
večji je presek snopa $S$ in zato intenziteta svetlobe manjša, kar zmanjša
učinek pretvorbe.
V enačbi~(\ref{8.10}) upoštevamo ujemanje faz in $S$ pri podvojeni frekvenci. Dobimo
\begin{equation}
\frac{P_{2\omega}}{P_{\omega}}=
\frac{\omega^3 \chi_{ef}^2}{2\pi n_{2\omega} n_\omega c_0^4\varepsilon_0} P_\omega\, L.
\label{8.17}
\end{equation}
Ob optimalnem zbiranju je izkoristek pretvorbe torej sorazmeren z dolžino kristala.
\begin{definition}
Naj na $1~\si{\centi\metre}$ dolg kristal KH$_{2}$PO$_{4}$\index{KDP} vpada svetloba
z valovno dolžino $1,06~\si{\micro\metre}$ in vhodno močjo $P_\omega = 5~\si{\kilo\watt}$.
Efektivna nelinearna susceptibilnost je $\chi_{ef}=7\times 10^{-13}~\si{\metre/\volt}$, 
$\Delta k=0$ in $n=1,5$. Izračunaj
faktor pretvorbe v frekvenčno podvojeno svetlobo.
Da je Rayleighova dolžina $2z_{0}=1~\si{\centi\metre}$, mora biti polmer
grla okoli $40~\si{\micro\metre}$. Gostota svetlobnega toka v kristalu je pri
tem $2\times 10^{8}~\si{\watt/\centi\metre^{2}}$, kar je že blizu praga za poškodbe,
predvsem na vstopni ali izstopni ploskvi. 
\end{definition}

\section{{*}Račun podvajanja Gaussovih snopov}
\index{Gaussov snop!frekvenčno podvajanje}
V prejšnjem razdelku smo grobo ocenili vpliv oblike Gaussovih snopov
na frekvenčno podvajanje. Naredimo zdaj natančnejši izračun. Vrnimo se k valovni
enačbi (enačba~\ref{8.3}), vpadna snopa naj bosta pri krožnih frekvencah
$\omega_{1}$ in $\omega_{2}$ in nastajajoči snop pri frekvenci
$\omega_{3}=\omega_{1}+\omega_{2}$.
Vsako od polj ($i=1,2,3$) naj ima obliko 
\begin{equation}
\mathbf{E}_{i}  = \frac{\mathbf{e}_{i}}{2}\left(\tilde{A}_{i}(r,z)\, 
e^{i(k_{i}z-\omega_{i}t)}+\tilde{A}_{i}^{*}(r,z)\, e^{-i(k_{i}z-\omega_{i}t)}\right)\!,
\end{equation}
pri čemer je $\tilde{A}(r,z)$ funkcija tako vzdolžne kot prečne koordinate. Privzamemo, 
da se vzdolž smeri  $z$ le počasi spreminja.
Zaradi poenostavljenega zapisa vpeljemo nove spremenljivke 
\begin{equation}
\psi_i = \sqrt{\frac{n_i}{\omega_i}}\tilde{A}_i.
\end{equation}
Tako je nastavek za jakost električnega polja
\begin{equation}
\mathbf{E}_{i}=\frac{\mathbf{e}_{i}}{2}\sqrt{\frac{\omega_{i}}{n_{i}}}\psi_{i}(r,z)
e^{i(k_{i}z-\omega_{i}t)}+\mathrm{k.~k.}
\label{8.18}
\end{equation}
Vstavimo nastavek (enačbe~\ref{8.18}) v valovno
enačbo (enačba~\ref{8.3}) in ločimo na levi in desni strani člene z enako časovno odvisnostjo.
Zaradi počasnega spreminjanja vzdolž smeri $z$ smo zanemarili druge odvode 
$\psi$ po $z$. Od tod sledi sklopljen sistem obosnih enačb 
\begin{align}
\nabla_{\perp}^{2}\psi_{1}+2ik_{1}\psi_{1}^{\prime} & =  -
\frac{k_{1}}{2}\kappa\psi_{2}^{\ast}\psi_{3}e^{i\Delta kz},\label{SHGGAuss-1}\\
\nabla_{\perp}^{2}\psi_{2}+2ik_{2}\psi_{2}^{\prime} & =  -
\frac{k_{2}}{2}\kappa\psi_{1}^{\ast}\psi_{3}e^{i\Delta kz},\\
\nabla_{\perp}^{2}\psi_{3}+2ik_{3}\psi_{3}^{\prime} & =
 - \frac{k_{3}}{2}\kappa\psi_{1}\psi_{2}e^{-i\Delta kz}
\label{SHGGauss_3}
\end{align}
in pripadajoč sistem kompleksno konjugiranih enačb. S črtico smo označili odvajanje po $z$. 
Vpeljali smo še parameter $\kappa$, za katerega velja
\begin{equation}
\kappa=\frac{\chi_{ef}}{c_0} \sqrt{\frac{\omega_{1}\omega_{2}\omega_{3}}{n_{1}n_{2}n_{3}}}.
\label{8.20}
\end{equation}
Sistem enačb~(\ref{SHGGAuss-1}--\ref{SHGGauss_3}) je očitno
posplošitev sistema enačb~(\ref{eq:nlAz}--\ref{eq:nlA3}) za primer, ko je valovanje odvisno
tudi od prečne koordinate. Reševanje tega nelinearnega sistema parcialnih
diferencialnih enačb je na splošno zelo zapleteno.

Poglejmo najenostavnejši primer frekvenčnega podvajanja, ko je 
$\omega_{3}=2\omega_{1}=2\omega$.
Vpadna snopa naj bosta enaka in Gaussove oblike~(enačba~\ref{eq:gaussov-snop}), 
njuna amplituda  naj bo enaka $A_1$
\begin{equation}
\psi_{1} = \psi_2 = A_{1}\frac{1}{1+iz/z_1}
\exp\left(-\frac{r^{2}}{w_1^{2}(z)}+\frac{ik_1r^{2}}{2R_1(z)}\right)\!.
\label{8.21}
\end{equation}
Privzamemo, da je izpolnjen pogoj za ujemanje faz\index{Ujemanje faz} 
($\Delta k=0$) in da je pretvorba dovolj majhna, da zmanjševanja $\psi_{1}$
ni treba upoštevati. Tudi za podvojeni snop privzamemo Gaussovo 
obliko, vendar naj njegova amplituda $A_3$ le počasi narašča. Zapišemo ga kot
\begin{equation}
\psi_{3}=A_{3}(z)\psi_{3H}(z,r)=A_{3}(z)\frac{1}{1+iz/z_{3}}
\exp\left(-\frac{r^{2}}{w_{3}^{2}(z)}+\frac{ik_{3}r^{2}}{2R_{3}(z)}\right)\!,
\label{8.22}
\end{equation}
pri čemer $\psi_{3H}$ reši homogeno obosno valovno 
enačbo (enačba~\ref{eq:obosna-valovna-enacba}). Ko izraza za $\psi_{1}$
in $\psi_{3}$ vstavimo v enačbo (\ref{SHGGauss_3}),
ostane na levi le člen oblike $2ik_{3}A_{3}^{\prime}(z)\psi_{3H}$. Tako dobimo pogoj
\begin{equation}
\begin{split}
A_{3}^{\prime}(z)\frac{1}{1+iz/z_3}\exp\left(-\frac{r^{2}}{w_{3}^{2}(z)}+\frac{ik_{3}r^{2}}
{2R_{3}(z)}\right)=\\
\frac{i\kappa}{4}A_{1}^{2}\frac{1}{(1+iz/z_{1})^{2}}\exp\left(-\frac{2r^{2}}
{w_{1}^{2}(z)}+\frac{ik_{1}r^{2}}{R_{1}(z)}\right)\!.
\label{8.23}
\end{split}
\end{equation}
Poiščimo rešitev te enačbe v obliki, za katero velja $w_{30}^{2}=w_{10}^{2}/2$. Tedaj je 
\begin{equation}
z_{3}=\frac{k_{3}w_{30}^{2}}{2}=\frac{2k_{1}w_{10}^{2}}{4}=z_{1}
\end{equation}
in je tudi $w_{3}^{2}(z)=w_{1}^{2}(z)/2$. Poleg tega je $R_{3}(z)=R_{1}(z)$
in lahko na obeh straneh krajšamo eksponentna faktorja. Ostane 
\begin{equation}
A_{3}^{\prime}(z)=\frac{i\kappa}{4}A_{1}^{2}\frac{1}{1+iz/z_1}.
\label{8.24}
\end{equation}
Enačbo seveda brez težav integriramo. Naj bo grlo vpadnega
snopa ravno na sredini nelinearnega sredstva, tako da integriramo
od $-L/2$ do $L/2$
\begin{align}
A_{3}(L) & =  \frac{i\kappa}{4}A_{1}^{2}\int_{-L/2}^{L/2}\frac{dz}{1+iz/z_1} 
  = \frac{\kappa}{4}A_{1}^{2}z_{1}\ln\frac{1+i\frac{L}{2z_{1}}}{1-i\frac{L}{2z_{1}}}\nonumber \\
 & =  \frac{\kappa}{2}A_{1}^{2}z_{1}\arctan\frac{L}{2z_{1}}\;.
\end{align}
Moč Gaussovega snopa je
\begin{equation}
P_{i}=\frac{1}{2}\pi w_{i0}^{2} \frac{1}{2}c_0 n_i \epsilon_{0}E_{i0}^{2}=
\frac{\pi}{4}w_{i0}^{2}\varepsilon_0 c_0 \omega_{i} A_{i}^{2},
\label{8.26}
\end{equation}
tako da je izkoristek pri frekvenčnem podvajanju Gaussovega snopa 
\begin{align}
\frac{P_{2\omega}}{P_{\omega}}=\frac{A_3^2}{A_1^2}  = &
\frac{\chi_{ef}^2 \omega^3 P_\omega z_1}{\pi c_0^4 \varepsilon_0 n_\omega n_{2\omega}} 
\arctan^2 \left( \frac{L}{2z_1}\right) \nonumber\\
 = &\frac{\chi_{ef}^2 \omega^3 P_\omega}{\pi c_0^4 \varepsilon_0 n_\omega n_{2\omega}} \frac{L}{2}
\left(\frac {\arctan^2 \left( L/2z_1\right)}{L/2z_1}\right)\!.
\label{8.27}
\end{align}
Funkcija $(\arctan^{2}x)/x$ zavzame največjo vrednost 0,64 pri $x =L/2z_1=1,39$.
Pri dani dolžini nelinearnega sredstva $L$ je torej 
izkoristek največji, kadar je $z_{1}=0,36\,L$, kar je malo manj kot pri
preprosti oceni $z_{1}=0,5\,L$ (enačba~\ref{SHGG}). Največji izkoristek
frekvenčnega podvajanja Gaussovih snopov je tako
\begin{equation}
\frac{P_{2\omega}}{P_{\omega}}
= 0,64 \frac{\omega^3 \chi_{ef}^2}{2\pi n_{2\omega} n_{\omega} c_0^4 \varepsilon_0 } P_\omega L.
\label{8.28}
\end{equation}
S preprosto oceno, ki smo jo naredili v prejšnjem razdelku (enačba~\ref{8.17}), smo tako 
rezultat le malo zgrešili, v obeh primerih pa izkoristek narašča linearno z dolžino kristala.

\section{Optično parametrično ojačevanje}
\index{Optično parametrično ojačevanje}
\index{Parametrično ojačevanje|see{Optično parametrično ojačevanje}}

Oglejmo si še en zelo uporaben primer mešanja treh valovanj, 
ki ga opisujejo enačbe (\ref{eq:nlAz}--\ref{eq:nlA3}). To je
optično parametrično ojačevanje, pri katerem nelinearne optične pojave
izkoristimo za ojačenje optičnih signalov.\footnote{~N. M. Kroll, Phys. Rev. $\mathbf{127}$, 1207 (1962).}
Imejmo razmeroma šibek vhodni
signal pri frekvenci $\omega_{1}$, ki ga želimo ojačiti, in močno črpalno valovanje
pri frekvenci $\omega_{3}>\omega_{1}$. Zaradi nelinearnosti v snovi  
intenziteta valovanja pri $\omega_{1}$ narašča, 
intenziteta valovanja pri $\omega_{3}$ se zmanjšuje, hkrati pa zaradi
ohranitve energije nastaja dodatno valovanje pri razliki frekvenc
$\omega_{2}=\omega_{3}-\omega_{1}$ (slika~\ref{fig:opa2}). Proces parametričnega ojačevanja 
si torej lahko predstavljamo kot pretvorbo enega fotona pri frekvenci 
$\omega_{3}$ v dva fotona pri $\omega_{1}$ in $\omega_{2}$.
Parametrično ojačevanje pogosto uporabljamo za ojačenje šibkih signalov 
v infrardečem delu spektra.\index{Infrardeče valovanje}
\begin{figure}[ht]
\centering
\def\svgwidth{70truemm} 
\input{slike/08_opa.pdf_tex}
\caption{Shematski prikaz valovanj pri optičnem parametričnem ojačevanju. Valovanje
pri krožni frekvenci $\omega_1$ se ojačuje na račun črpalnega valovanja s krožno frekvenco
$\omega_3$. Pri tem nastane nedejaven žarek s krožno frekvenco $\omega_2 =  \omega_3-\omega_1$.}
\label{fig:opa2}
\end{figure}

Izhajamo iz splošnih enačb za nelinearne optične pojave drugega reda 
(enačbe~\ref{eq:nlAz}--\ref{eq:nlA3})
\begin{align}
\frac{dA_{3}}{dz} &=\frac{i\omega_{3}\chi_{ef}}{4c_0 n_3} A_{1}\, A_{2}\, e^{-i\Delta kz}, \\
\frac{dA_{2}}{dz} &=\frac{i\omega_{2}\chi_{ef}}{4c_0 n_2} A_{1}^*\, A_{3}\, e^{i\Delta kz}\qquad \textrm{in}\\
\frac{dA_{1}}{dz} &=\frac{i\omega_{1}\chi_{ef}}{4c_0 n_1} A_{2}^*\, A_{3}\, e^{i\Delta kz}.
\label{eq:opaA}
\end{align}
Privzamemo, da je črpalno valovanje vselej dosti močnejše od drugih dveh
($A_{3}\gg A_{1}$, $A_{2}$) in da je njegova jakost približno konstantna $A_3 = A_{30}$.
Poskrbimo še, da je izpolnjen pogoj za ujemanje faz $\Delta k=0$ in
začetna pogoja zapišemo kot $A_{1}(z=0)=A_{10}$ in $A_{2}(z=0)=0$. Ko vse to upoštevamo,
dobimo sklopljeni enačbi
\begin{align}
\frac{dA_{1}}{dz} &= \frac{i\omega_{1}\chi_{ef}}{4c_0 n_1} A_{2}^*\, A_{30}\label{eq:opaA1} 
\qquad \mathrm{in} \\
\frac{dA_{2}^*}{dz} &= -\frac{i\omega_{2}\chi_{ef}}{4c_0 n_2} A_{1}\, A_{30}^*.
\label{eq:opaA2}
\end{align}
Enačbi lahko rešimo tako, da prvo odvajamo po $z$ in vanjo vstavimo drugo enačbo.
Dobimo
\begin{equation}
\frac{d^2 A_1}{d z^2} = \frac{\omega_1 \omega_2 \chi_{ef}^2|A_{30}|^2}
{16 c_0^2 n_1 n_2} A_1 = \kappa^2 A_1
\end{equation}
in podobno za $A_2$
\begin{equation}
\frac{d^2 A_2}{d z^2} = \frac{\omega_1 \omega_2 \chi_{ef}^2|A_{30}|^2}
{16 c_0^2 n_1 n_2} A_2 = \kappa^2 A_2.
\end{equation}
Ob upoštevanju začetnih pogojev izračunamo rešitev za naraščanje amplitude signalnega žarka
z začetno amplitudo $A_{10}$
\boxeq{eq:opa}{
A_1 = A_{10} \cosh (\kappa L),
}
pri čemer je $L$ dolžina nelinearnega sredstva in 
\begin{equation}
\kappa^2 = \frac{\omega_1 \omega_2 \chi_{ef}^2|A_{30}|^2}
{16 c_0^2 n_1 n_2}.
\label{opakapa}
\end{equation}
Hkrati s signalnim žarkom narašča tudi amplituda dodatnega nedejavnega 
({\it idle}) žarka, ki nastane med procesom ojačenja\index{Nedejavni žarek}
\boxeq{eq:opan}{
A_2 = A_{20} \sinh (\kappa L).
}
Vrednost $A_{20}$ lahko izrazimo z začetno amplitudo signalnega žarka $A_{10}$
\begin{equation}
A_{20} = i \sqrt{\frac{\omega_2 n_1}{\omega_1 n_2}} A_{10}.
\label{opakapaA}
\end{equation}
Na začetku intenziteti obeh valovanj naraščata približno eksponentno na račun črpalnega
valovanja (slika~\ref{fig:opagraf}). Ko postane njuna intenziteta znatna in se 
začne $A_3$ zmanjševati, je treba to seveda
upoštevati v računu. Rešiti je treba zahtevnejši sistem treh 
sklopljenih enačb, podobno~\textendash~a še bolj zapleteno~\textendash~kot v nalogi~(\ref{deplet}).

\begin{figure}[ht]
\centering
\def\svgwidth{90truemm} 
\input{slike/08_opagraf.pdf_tex}
\caption{Normirani intenziteti ojačenega žarka ($|A_1/A_{10}|^2$) in dodatnega 
nedejavnega žarka ($|A_2/A_{20}|^2$), ki nastane zaradi zahteve po ohranitvi energije. 
Naraščajoči funkciji sta seveda samo približek, ki velja, dokler je ojačenje majhno in 
se intenziteta črpalnega žarka ne zmanjšuje znatno.}
\label{fig:opagraf}
\end{figure}

\begin{definition}
Pokaži, da sta izraza za amplitudi polji $A_1$ in $A_2$ (enačbi~\ref{eq:opa} in~\ref{eq:opan})
rešitvi sklopljenih enačb~(\ref{eq:opaA1} in \ref{eq:opaA2}).
\end{definition}

Do zdaj smo privzeli, da je izpolnjen pogoj ujemanja faz \index{Ujemanje faz}
in $\Delta k=k_{3}-k_{1}-k_{2}=0$. 
Ta pogoj lahko izpolnimo na enak način kot pri podvajanju frekvence: v dvolomnem kristalu 
izberemo ustrezne polarizacije in smer širjenja svetlobe glede na optično os, 
tako da velja $\omega_{3}n_{3}=\omega_{1}n_{1}+\omega_{2}n_{2}$.

Če na primer vzamemo izredno polarizacijo za črpalno valovanje
in redni polarizaciji za obe ojačevani valovanji, mora biti izpolnjen naslednji pogoj 
\begin{equation}
\left(\left(\frac{\cos\vartheta_{m}}{n_{o}(\omega_3)}\right)^{2}
+\left(\frac{\sin\vartheta_{m}}{n_{e}(\omega_3)}\right)^{2}\right)^{-1/2}=
\frac{\omega_{1}}{\omega_{3}}n_{o}(\omega_1)+\frac{\omega_{2}}{\omega_{3}}n_{o}(\omega_2).
\label{8.34}
\end{equation}

\begin{definition}
Pokaži, da v primeru neujemanja faz $\Delta k \neq 0$ amplitudi ojačevanega in dodatnega 
nedejavnega žarka naraščata kot \index{Neujemanje faz}
\begin{equation}
A_1 = A_{10} \left( \cosh(\kappa z) - \frac{i \Delta kz}{2 \kappa} \sinh (\kappa z) 
\right) e^{\frac{i \Delta kz}{2}}
\label{phmisa1}
\end{equation}
in
\begin{equation}
A_2 = A_{20} \sinh(\kappa z) e^{\frac{i \Delta k}{2}},
\end{equation}
pri čemer sta
\begin{equation}
\kappa^2 = \frac{\omega_1 \omega_2 \chi_{ef}^2|A_{30}|^2}
{16 c_0^2 n_1 n_2} - \frac{\Delta k^2}{4}
\end{equation}
in
\begin{equation}
A_{20} = i \sqrt{\frac{\omega_2 n_1}{\omega_1 n_2}} \sqrt{1 + \frac{\Delta k^2}{4 \kappa^2}}
~A_{10}.
\label{phmisa20}
\end{equation}
Pokaži še, da so enačbe~(\ref{phmisa1}--\ref{phmisa20}) v limitnem primeru, 
ko se faze ujamejo in je $\Delta k = 0$,
enake enačbam~(\ref{eq:opa}--\ref{opakapaA}).
\end{definition}

Za konec ocenimo koeficient ojačenja v kristalu 
LiNbO$_{3}$\index{LiNbO$_3$}, v katerem želimo
ojačiti svetlobo z valovno dolžino $\lambda = 1~\si{\micro\metre}$. Črpamo z laserjem z 
valovno dolžino okoli $500~\si{\nano\metre}$ in gostoto svetlobnega 
toka $5~\si{\mega\watt}/\si{\centi\metre}^{2}$. Lomni količnik snovi je 
$n = 2,2$ in efektivna nelinearna susceptibilnost  $\chi_{ef} = 5~\si{\pico\metre}/\si{\volt}$. 
Vstavimo podatke v enačbo~(\ref{opakapa}) in izračunamo vrednost 
$\kappa \sim 0,15~/\si{\centi\metre}$. To pomeni, da je porast intenzitete vpadne svetlobe v $1~\si{cm}$ 
dolgem kristalu le približno $2~\%$. 

\subsection*{Optični parametrični oscilator (OPO)}
\index{Optični parametrični oscilator}

Navedeni primer kaže, da optično parametrično ojačevanje svetlobe pri prehodu skozi kristal ni prav veliko
kljub dokaj močnemu črpalnemu žarku. Zato je smiselno, da svetloba večkrat preleti
ojačevalno sredstvo in se postopoma ojačuje. To naredimo tako, 
da optično ojačevalno sredstvo zapremo v optični 
resonator\index{Resonator!parametrični oscilator}
in signal se ob vsakem obhodu ojači. Sestavili smo tako imenovani optični parametrični oscilator
(slika~\ref{fig:opo}). 
\begin{figure}[ht]
\centering
\def\svgwidth{80truemm} 
\input{slike/08_opo.pdf_tex}
\caption{Shematski prikaz tipičnega optičnega parametričnega oscilatorja. Ojačevalno sredstvo
zapremo med resonatorja, da se signalni žarek ($\omega_1$) ob vsakem obhodu ojači.}
\label{fig:opo}
\end{figure}

V optičnem resonatorju je odbojnost zrcal za črpalni žarek ($\omega_3$) zelo majhna, 
odbojnost za ojačeni žarek pa blizu ena. Valovanje pri $\omega_1$,
ki se v parametričnem oscilatorju ojačuje, nastane spontano, prav tako valovanje pri 
$\omega_2 = \omega_3 -\omega_1$. Njuni frekvenci sta dodatno določeni s pogojem za 
ujemanje faz $ k_3 - k_1 - k_2 = 0$, 
hkrati mora ojačevano nihanje sovpadati z lastnim nihanjem resonatorja. 
S sukanjem ojačevalnega kristala lahko na ta način spreminjamo
ojačeno frekvenco in naredili smo nastavljiv izvor svetlobe, navadno v 
infrardečem delu spektra.\index{Infrardeče valovanje}
\pagebreak

Za delovanje oscilatorja mora biti jakost črpalnega žarka tako velika, da je 
ojačenje signala na obhod večje od izgub. Signal z močjo 
$P_0$ se ob prehodu skozi ojačevalno sredstvo ojači (enačba~\ref{eq:opa})
\begin{equation}
P_1 = P_0 \cosh^2 (\kappa L),
\end{equation}
hkrati pa se zaradi izhodnega zrcala z odbojnostjo $\mathcal{R}<1$ in notranjih izgub $\Lambda_0$ 
intenziteta žarka zmanjšuje. Ker je pogoj ujemanja faz izpolnjen le v eni smeri, se svetloba
ojačuje le enkrat na celoten obhod. Ob preletu v drugo smer je namreč $\Delta k \neq 0$ in 
žarek se ne ojačuje. V stacionarnem stanju je ojačenje ravno enako izgubam, moč 
signalnega žarka po obhodu $P_2$ pa je enaka začetni moči $P_0$. Velja
\begin{equation}
P_2 = P_1\,(1-\Lambda_0)\mathcal{R} = P_0 \,(1-\Lambda_0) \mathcal{R} \,\cosh^2 (\kappa L) = P_0
\end{equation}
oziroma
\begin{equation}
\cosh^2 (\kappa L) =\frac{1}{(1-\Lambda_0)\, \mathcal{R}}.
\end{equation}
Iz tega pogoja izračunamo parameter $\kappa$, po enačbi~(\ref{opakapa}) pa mejno 
amplitudo oziroma intenziteto črpalnega žarka. Nadaljujmo še prejšnji primer ojačenja 
svetlobe v $1~\si{cm}$ dolgem kristalu LiNbO$_{3}$.\index{LiNbO$_3$}
Če je odbojnost izhodnega zrcala $\mathcal{R}=0,85$, notranje izgube $\Lambda_0 = 0,05$ in prečni presek 
žarka $10~\si{\micro\metre^2}$, je moč praga $P_{\omega_3} = 5~\si{\watt}$.
\vglue-4truemm
\begin{remark}
Optični parametrični oscilator oddaja svetlobo, podobno kot laser. Tudi sicer
sta si do neke mere podobna: oba sistema potrebujeta močen črpalni mehanizem, oba 
sta sestavljena iz resonatorja, v katerem se žarek velikokrat odbije in postopoma ojačuje,
in oba oddajata koherentno svetlobo pri točno določeni valovni dolžini. Vendar
je med parametričnim oscilatorjem in laserjem velika razlika. Pri laserju se svetloba
ojači zaradi obrnjene zasedenosti stanj, pri oscilatorju pa 
zaradi nelinearnega optičnega pojava. Pri oscilatorju energija torej ni shranjena v
snovi, ampak se signal ojačuje sproti. Velika prednost oscilatorjev pred laserji 
je zvezno nastavljiva frekvenca delovanja v zelo širokem frekvenčnem območju, saj ni določena
s prehodom med nivoji, ampak z izpolnjevanjem pogoja za ujemanje faz.
\end{remark}

\section{Optično usmerjanje in teraherčno valovanje}
\index{Optično usmerjanje}
\index{Teraherčno valovanje}
Ko smo obravnavali nelinearne optične pojave drugega reda, smo zapisali
različne frekvence, ki so vsebovane v izhodnem signalu (slika~\ref{fig:nl2}). Eno izmed
izhodnih valovanj ima tudi frekvenco enako nič, kar pomeni, da je to statično električno polje. Iz analogije
z elektronskimi vezji, v katerih izmenično napetost z usmernikom spremenimo v enosmerno napetost, 
pojav imenujemo optično usmerjanje, saj iz svetlobnega valovanja nastane statično polje. Tako statično 
polje navadno ni veliko, saj sunek svetlobe z vršno močjo nekaj $\si{\mega\watt}$ tipično povzroči 
nekaj $\si{\milli\volt}$ napetosti v smeri prečno na smer potovanja svetlobe.\footnote{~M. Bass
et al., Phys. Rev. Lett. $\mathbf{9}$, 446 (1962).}

\begin{definition}
Pokaži, da je napetost, ki se pojavi pri optičnem usmerjanju, približno enaka
\begin{equation}
U = \frac{\chi P_0}{n^3 \varepsilon_0 c_0 a},
\end{equation}
pri čemer so $P_0$ moč vpadne svetlobe, $n$ lomni količnik snovi in $a$ širina kristala.
Namig: nelinearni kristal obravnavaj kot ploščati kondenzator in zapiši polarizacijo.

Oceni še napetost, če je
$\chi = 3~\si{\pico\meter/\volt}$, $P_0 = 1~\si{\mega\watt}$, $n = 2,2$ in $a = 5~\si{\milli\metre}$. 
\end{definition}

Precej bolj uporaben je pojav, ko na nelinearni kristal posvetimo z ultrakratkimi 
sunki svetlobe, tipično dolgimi okoli $\si{ps}$ ali še krajšimi. Spomnimo se, da je povsem 
monokromatsko valovanje lahko samo tako, ki je časovno neomejeno in ima neskončen koherenčni čas
(enačba~\ref{eq:spektralna-sirina-zveza}). 
Čim je valovanje časovno omejeno, na primer v obliki kratkega sunka, ima njegov spekter 
končno širino, pri čemer imajo krajši sunki svetlobe širši spekter valovanja. Označimo 
osrednjo krožno frekvenco z $\omega$ in širino spektra z $\Delta \omega$. 

Ko z ultrakratkim sunkom osvetlimo optično 
nelinearni kristal, vanj vstopajo vse frekvence z danega intervala $\omega \pm \Delta \omega/2$.
Optično usmerjanje ni več popolno, saj se frekvence ne odštejejo povsem, ampak se 
namesto statičnega polja pojavi sunek svetlobe s širokim spektrom, ki sega od ničelne
frekvence do neke največje vrednosti. Celotna spektralna širina tega signala je 
približno enaka spektralni širini vstopnega sunka, ta pa je obratno sorazmerna z njegovo dolžino
(slika~\ref{fig:THz}).
Ocenimo te vrednosti še numerično. 

\begin{figure}[ht]
\centering
\def\svgwidth{100truemm} 
\input{slike/08_THz.pdf_tex}
\caption{Shematski prikaz nastanka teraherčnega valovanja v optično nelinearnem sredstvu}
\label{fig:THz}
\end{figure}

Vzemimo kratek sunek svetlobe dolžine $\tau$ s spektralno širino
$\Delta \nu = \Delta \omega/2 \pi = 1/\pi \tau$.
Če je sunek svetlobe dolg $1~\si{\pico\second}$, je razlika v frekvencah 
spektra 
\begin{equation}
\Delta \nu = \frac{1}{3 \times 10^{-12}~\si{s}} = 0,3~\si{\tera\hertz}.
\end{equation}
Valovanje, ki nastane pri takem optičnem kvazi-usmerjanju, ima torej frekvence v teraherčnem
področju in naredili smo izvor teraherčnega valovanja. 

Teraherčno valovanje, to je 
elektromagnetno valovanje s frekvencami v območju od 0,3 do $3~\si{\tera\hertz}$
oziroma z valovnimi dolžinami med 0,1 in $1~\si{\milli\metre}$, 
se uporablja za neinvazivno slikanje in preiskave tkiv in materialov. Kristali, ki 
se najpogosteje uporabljajo za nastanek teraherčnega valovanja, so ZnTe\index{ZnTe}, 
GaP, GaSe in GaAs. Slabost optičnega usmerjanja je razmeroma nizek izkoristek zaradi majhne frekvence
 nastalega valovanja, poleg tega je treba paziti, da absorpcija teraherčnega valovanja
 v snovi ni prevelika.\index{GaP} \index{GaSe} \index{GaAs}

\section{Nelinearni pojavi tretjega reda}
\index{Nelinearna optika!tretjega reda}
Doslej smo obravnavali najnižji red nelinearnosti, katerega glavni
učinek je mešanje treh frekvenc, na primer optično frekvenčno podvajanje ali
optično parametrično ojačevanje. Ti pojavi so mogoči le v kristalih brez centra
inverzije. Nelinearna polarizacija in z njo povezani nelinearni pojavi tretjega reda pa so
možni v vsaki snovi. 
V nelinearni polarizaciji tretjega reda nastopa jakost električnega polja v tretji potenci
\boxeq{eq:nl3P}{
\mathbf{P}_{\mathrm{NL,3}}= \epsilon_{0}\chi^{(3)}\vdots \mathbin 
\mathbf{E}\mathbin \mathbf{E}\mathbin\mathbf{E}
}
oziroma izpisano po komponentah
\begin{equation}
\left(\mathbf{P}_{\mathrm{NL,3}}\right)_i= \epsilon_{0}\chi^{(3)}_{ijkl} \,E_j \,E_k\, E_l.
\end{equation}
Pri tem je $\chi^{(3)}$ tenzor četrtega ranga, njegova tipična velikost je okoli 
$10^{-22}~\si{\metre^2/\volt^2}$. Na splošno ima 81 različnih neodvisnih komponent, vendar se to
število lahko zelo zmanjša zaradi simetrije snovi. V izotropni snovi je tako
21 neničelnih elementov, od katerih so le trije neodvisni. 

Če vsebuje vpadno polje le eno frekvenco, se zaradi nelinearnosti tretjega
reda pojavi polarizacija pri 3$\omega$ in $\omega$. Pri dveh vpadnih
frekvencah $\omega_{1}$ in $\omega_{2}$ so mogoče kombinacije $2\omega_{1}\pm\omega_{2}$
in $\omega_{1}\pm2\omega_{2}$, pri treh vpadnih frekvencah pa vse
mogoče vsote in razlike frekvenc, to so $\omega_1$, $\omega_2$, $\omega_3$, 
$3\omega_1$, $3 \omega_2$, $3\omega_3$, 
$\omega_1 + \omega_2 + \omega_3$, $\omega_1 + \omega_2 - \omega_3$, 
$\omega_1 - \omega_2 + \omega_3$, $- \omega_1 + \omega_2 + \omega_3$, 
$2 \omega_1\pm\omega_2$, $2 \omega_1\pm\omega_3$, $2 \omega_2\pm\omega_1$,
$2 \omega_2\pm\omega_3$, $2 \omega_3\pm\omega_1$, $2 \omega_3\pm\omega_2$.
Možnosti je torej precej več kot pri nelinearnosti drugega reda in računi so zato 
precej bolj zapleteni.

Obravnava nastanka valovanja pri
kombinaciji frekvenc je zelo podobna obravnavi frekvenčnega podvajanja ali  parametričnega
ojačevanja. V enačbah za nastanek novega valovanja ali ojačevanje
katerega od vpadnih snopov spet nastopi fazni faktor, ki vsebuje razliko
vseh valovnih vektorjev $\Delta{\bf k}$. Da je intenziteta novega
valovanja znatna, mora biti $\Delta kL\simeq0$, spet mora biti torej
izpolnjen pogoj ujemanja faz. Ker v tem primeru nastopajo na splošno štirje
valovni vektorji, je seveda tudi pri izbiri geometrije in polarizacij
za dosego ujemanja faz precej več možnosti.\index{Ujemanje faz}

Omejimo se na najpreprostejši primer, pri katerem ima vpadno valovanje le eno 
frekvenco. Takrat se pojavi valovanje pri potrojeni frekvenci, pa tudi
pri frekvenci, ki je enaka vpadni. Pojavi se torej polarizacija pri 
vpadni frekvenci, ki spremeni obnašanje osnovnega valovanja, in valovanje vpliva samo nase.
Ti pojavi, ki jih poimenujemo s predpono {\it samo-}, kot na primer samozbiranje, so
značilni za nelinearne pojave tretjega reda\index{Samozbiranje}.\footnote{~Glej 
npr. R. W. Boyd, {\it Nonlinear Optics}, tretja izdaja, Academic Press (2008).}

\section{Optični Kerrov pojav}
\index{Optični Kerrov pojav|see Kerrov pojav!optični}
\index{Kerrov pojav!optični}
\label{OKP}
Naj valovanje vpada na nelinearno snov, za katero velja $\chi^{(2)} = 0$.
Polarizacija je potem enaka vsoti linearnega in nelinearnega dela tretjega reda 
(enačba~\ref{8.1})\index{Električna polarizacija}
\begin{equation}
\mathbf{P}=
\epsilon_{0} \chi^{(1)}\cdot \mathbf{E}+
\epsilon_{0}\chi^{(3)}\vdots \mathbin \mathbf{E}\mathbin \mathbf{E}\mathbin\mathbf{E}.
\end{equation}
Ker obravnavamo nelinearne pojave, moramo tudi v tem primeru zapisati realna
električna polja. To naredimo z vsoto dveh kompleksno konjugiranih členov
\begin{equation}
\mathbf{E}=\frac{\mathbf{e}}{2}\left(Ae^{i(kz-\omega t)}+A^{*}e^{-i(kz-\omega t)}\right)\!.
\label{8.71}
\end{equation}
Podobno zapišemo polarizacijo, pri čemer nas zanimajo samo členi pri krožni frekvenci $\omega$
\begin{equation}
\mathbf{P}=\frac{\mathbf{e}}{2}\left(P_\omega e^{i(kz-\omega t)}+P_\omega^{*}e^{-i(kz-\omega t)}\right)\!.
\label{8.71a}
\end{equation}
Nelinearna polarizacija tretjega reda ima frekvenco $\omega$, kadar v produktu 
$\mathbf{E}\mathbin \mathbf{E}\mathbin\mathbf{E}$ 
dvakrat nastopa nekonjugirani del, enkrat pa konjugirani. To se lahko zgodi na tri
načine, zato dobimo
\begin{equation}
\frac{\mathbf{e}}{2}P_{\omega,\mathrm{NL}} = 3 \frac{1}{8} A A^* \left( 
\varepsilon_0 \chi^{(3)}\vdots \mathbf{e}\, \mathbf{e} \, \mathbf{e} \right) A.
\label{pomega}
\end{equation}
Celotna polarizacija je
\begin{equation}
\mathbf{P}=
\epsilon_{0} \chi^{(1)}\cdot \mathbf{E}+\frac{3}{4} |A|^2 \left( 
\varepsilon_0 \chi^{(3)}\vdots \mathbin \mathbf{e}\mathbin \mathbf{e} \right) \mathbf{E}.
\label{eq:ptnl}
\end{equation}
Upoštevamo enačbo za povprečno gostoto energijskega toka (enačba~\ref{eq:jcw}) in
zapišemo
\begin{equation}
\mathbf{P}=
\epsilon_{0} \left( \chi^{(1)} +\frac{3}{4} \frac{2  j }
{\varepsilon_0 \tilde{n} c_0} \chi^{(3)}\vdots \mathbin \mathbf{e}\mathbin 
\mathbf{e} \right) \mathbf{E}.
\label{eq:pppeee}
\end{equation}
Z $\tilde{n}$ smo označili lomni količnik pri krožni frekvenci $\omega$. 
Faktor v oklepaju ni nič drugega kot efektivna susceptibilnost, ki je neposredno povezana
z lomnim količnikom snovi $\chi_{ef} = \varepsilon -1 =n^2 -1$. Enačba~(\ref{eq:pppeee}) 
torej opisuje pojav, 
pri katerem vpadna svetloba vpliva na lomni količnik snovi, po kateri 
potuje.\index{Susceptibilnost!efektivna}
To je podoben učinek kot pri navadnem Kerrovem pojavu, pri katerem se lomni količnik 
spremeni pod vplivom zunanjega električnega polja (enačba~\ref{7.1}).
Opisani optični ekvivalent zato imenujemo optični 
Kerrov pojav\footnote{~Škotski fizik John Kerr, 1824\textendash1907.}.

Poglejmo pojav na primeru izotropne snovi. Na snov naj vpada valovanje, ki je polarizirano
v smeri $x$, tako da ima nelinearna polarizacija le komponento 
\begin{equation}
P_{\mathrm{NL},x}=
\epsilon_{0} \left(\chi_{xx} +\frac{3}{4} \chi_{xxxx}\frac{2 j }
{\varepsilon_0 \tilde{n} c_0}\right)E = \varepsilon_0 \chi_{ef}E = \varepsilon_0 (n^2-1) E.
\end{equation}
Izrazimo še efektivni lomni količnik
\begin{equation}
n \approx \tilde{n} + \frac{3 \chi_{xxxx}}{4 \varepsilon_0 c_0 \tilde{n}^2} j,
\end{equation}
ki ga lahko zapišemo v obliki 
\boxeq{eq:nnl}{
n= \tilde{n} + n_2 j,}\index{Lomni količnik!efektivni}
pri čemer smo vpeljali nelinearni lomni količnik\index{Lomni količnik!nelinearni}
\boxeq{eq:n2}{
n_2 = \frac{3 \chi_{xxxx}}{4 \varepsilon_0 c_0 \tilde{n}^2}.
}
Efektivni lomni količnik snovi je torej odvisen od gostote svetlobnega toka, ki vpada nanjo. 
Tipične vrednosti nelinearnega lomnega količnika za vidno svetlobo so $10^{-20}~\si{\metre^2/\watt}$.
V tekočini CS$_2$ je $n_2 = 3,2 \times 10^{-18}~\si{\metre^2/\watt}$, v nekaterih \index{CS$_2$}
drugih snoveh (npr. polprevodnikih) je lahko vrednost $n_2$ večja še za nekaj 
velikostnih redov, $n_2$ pa je lahko tudi negativen (tabela~\ref{table:chi3}).\footnote{~R.
W. Boyd, {\it Nonlinear Optics}, tretja izdaja, Academic Press (2008).}

\begin{table}[ht]
 \centering
\begin{tabular}{|c|c|c|} \hline  
      Snov & $\chi^{(3)}~(\si{\metre^2/\volt^2})$ & $n_2~(\si{\metre^2/\watt})$\\ \hline
     steklo BK7 & 2,8 $\times 10^{-22}$ & $3,4 \times 10^{-20}$ \\ \hline
     voda & $2,5 \times 10^{-22}$ & $4,1 \times 10^{-20}$ \\ \hline
     GaAs & $1,4 \times 10^{-18}$ & $3,3 \times 10^{-17}$ \\ \hline\index{GaAs}
     ZnSe & $6,2 \times 10^{-20}$ & $3,0 \times 10^{-18}$ \\ \hline\index{ZnSe}
     CS$_2$ & $3,1 \times 10^{-20}$ & $3,2 \times 10^{-18}$ \\ \hline \index{CS$_2$}
     polimer 4BCMU  & $-1,3 \times 10^{-19}$ & $-1,5 \times 10^{-17}$ \\ \hline      
\end{tabular}
  \caption{Nelinearna susceptibilnost tretjega reda in nelinearni lomni količnik za nekaj snovi}
\label{table:chi3}
\end{table}

Zanimivi posledici lomnega količnika, odvisnega od gostote svetlobnega toka vpadne svetlobe, 
sta samozbiranje svetlobnega snopa in širjenje solitonov po optičnih vodnikih, 
kar si bomo pogledali v naslednjih razdelkih.

\begin{remark}
Ničesar nismo povedali o ujemanju faz, ki je sicer nujno potrebno za učinkovite nelinearne 
optične pojave. V tem primeru vpada na snov en sam laserski žarek in pogoj ujemanja faz
je vedno izpolnjen. 
\end{remark}

\section{Samozbiranje in krajevni solitoni}
\index{Samozbiranje}
\index{Soliton!krajevni}
Za začetek si oglejmo pojav samozbiranja svetlobe.\footnote{~P. L. Kelley, Phys. Rev. Lett. $\mathbf{15}$,
1005 (1965).}
Osnovni Gaussov snop 
(enačba~\ref{eq:gaussov-snop}) naj vpada na sredstvo, v katerem je lomni 
količnik odvisen od gostote energijskega toka vpadne svetlobe po enačbi~(\ref{eq:nnl}).
Naj bo $n_{2}>0$, tako da je lomni količnik v sredini snopa večji 
od nemotenega lomnega količnika na robu. V osi snopa se optična pot 
zaradi optično gostejšega sredstva podaljša in valovna fronta 
v osi zaostaja glede na fronte na robu snopa. Če je zaostajanje dovolj veliko,
lahko krivinski radij valovne fronte postane negativen in snop se
ne širi, temveč oža (slika~\ref{fig:sf1}). Temu pojavu pravimo 
samozbiranje. Samozbiranje je pri dovolj
veliki moči snopa lahko tako veliko, da pride do katastrofične zožitve snopa
in s tem do tolikšnega povečanja gostote svetlobnega toka, da nastanejo
poškodbe v snovi.
\begin{figure}[ht]
\centering
\def\svgwidth{95truemm} 
\input{slike/08_sf1.pdf_tex}
\caption{V Gaussovem snopu je intenziteta valovanja $j$ odvisna od prečne koordinate $r$, 
zato je tudi lomni količnik nelinearnega sredstva  $n$ odvisen od nje. To vodi do 
 samozbiranja svetlobe. Na sliki so fronte vpadnega Gaussovega snopa narisane kot ravni valovi.}
\label{fig:sf1}
\end{figure}
\begin{definition}
Gaussov snop svetlobe z močjo $P$ in polmerom $w$ naj vpada pravokotno na ploščico
kristala debeline $d$. Pokaži, da ploščica deluje na snop kot leča, pri čemer je njena
goriščna razdalja enaka  
\begin{equation}
f = \frac{\pi w^4}{8 n_2 d P}.
\end{equation}
Z $n_2$ smo označili nelinearni lomni količnik ploščice.
\end{definition}

Zaradi uklona se Gaussov snop širi, medtem ko ima pojav samozbiranja ravno nasprotni
učinek. Snop samemu sebi ustvarja valovni vodnik, v katerem je v sredini lomni količnik 
večji kot na robu. Pri določeni moči snopa se oba pojava po
velikosti ravno izenačita. Snop, ki potuje po snovi, ima tako konstanten polmer, 
valovne fronte pa so ravne -- nastane tako imenovani krajevni soliton.\footnote{~Glej npr. G. New, {\it Introduction
to Nonlinear Optics}, Cambridge University Press (2011).}
\index{Soliton!krajevni}

Ocenimo, pri kolikšni moči vpadne svetlobe se pojavijo krajevni solitoni. 
Vzemimo, da je na izbranem mestu valovna fronta ravna. Lahko si mislimo,
da je tam ravno grlo Gaussovega snopa. Brez samozbiranja bi bil na razdalji
dolžine bližnjega polja $z_{0}$ krivinski radij valovne fronte (enačba~\ref{eq:R})
\begin{equation}
R(z_{0})=z_{0}\left( 1+\left(\frac{z_{0}}{z_{0}}\right)^{2}\right)=2z_{0}.
\label{8.75}
\end{equation}
Po enačbi~(\ref{eq:nnl}) je odvisnost lomnega količnika približno
\begin{equation}
n(r)=\tilde{n}+n_2 j_0 e^{-2r^2/w_0^2}.
\label{8.76}
\end{equation}
Razlika med lomnim količnikom na osi (pri $r=0$) in pri $r = w_{0}$ 
od osi je kar približno $\Delta n= j_{0} n_{2}$.
Zaradi tega je na poti od grla do $z_0$ razlika optičnih poti med žarkoma na osi $(r=0)$ in 
pri $r= w_{0}$ enaka $\Delta nz_{0} = n_2 j_0 z_0$ in valovna fronta se 
ukrivi na nek krivinski radij $-R$. Iz preproste geometrije velja zveza 
\begin{equation}
\Delta nz_{0}=R-R\sqrt{1-\frac{w_{0}^{2}}{R^{2}}}\approx \frac{w_{0}^{2}}{2R}.
\label{8.77}
\end{equation}
Da valovna fronta ostane ravna, se morata krivinska radija zaradi uklona 
(enačba~\ref{8.75}) in samozbiranja (enačba~\ref{8.77}) po velikosti izenačiti. 
Od tod sledi 
\begin{equation}
\Delta n=\frac{w_{0}^{2}}{4z_{0}^{2}}.
\label{8.78}
\end{equation}
Moč snopa, pri katerem se polmer ne spreminja, je potem 
\begin{equation}
P_{s}= \frac{1}{2}\pi w_0^2 \,j_0 = \frac{1}{2}\pi w_0^2 \, \frac{\Delta n}{n_2} = 
\frac{1}{2}\pi w_0^2 \,\frac{w_{0}^{2}}{4z_{0}^{2}}\,\frac{1}{n_2} = \frac{\lambda^2}{8\pi n_2},
\label{8.79}
\end{equation}
pri čemer smo upoštevali zvezo med $z_0$ in $w_0$ (enačba~\ref{eq:z0}).

Pri moči, ki je manjša od kritične moči, se vpadli Gaussov snop širi, 
čeprav nekoliko počasneje kot v sredstvu s konstantnim lomnim količnikom. 
Če pa je moč znatno večja od kritične moči, lahko
nastopita katastrofično samozbiranje in porušitev snovi.
Zanimivo je, da kritična moč, pri kateri se pojavijo solitoni, 
ni odvisna od začetnega polmera snopa.

\begin{definition}
Nariši skico k enačbi~(\ref{8.77}) in izpelji izraz za moč, pri kateri se pojavijo
solitoni (enačba~\ref{8.79}). 

Izračunaj še kritično moč za pojav solitonov v CS$_{2}$,\index{CS$_2$}
če je valovna dolžina vpadnega valovanja $1~\si{\micro\metre}$, 
nelinearni lomni količnik te tekočine pa 
 $n_{2}=3,2 \times 10^{-18}~\si{\metre^2/\watt}$. 
\end{definition}

\begin{remark}
Eksperimentalna metoda, \index{Metoda vzdolžnega premika}
\index{Z-scan|see {Metoda vzdolžnega premika}}s katero merimo nelinearni 
lomni količnik, je tako imenovana
metoda vzdolžnega premika ({\it Z-scan}).\footnote{~M. Sheik-bahae, A. A. Said in E. W. Van Stryland, 
Opt. Lett. $\mathbf{14}$, 955 (1989).} 
Optično nelinearno sredstvo (naj ima $n_2>0$)
postavimo v zbran laserski snop (slika~\ref{fig:zscan}). 
Zaradi samozbiranja deluje vzorec kot leča, njena goriščna razdalja
pa je odvisna od intenzitete snopa in nelinearnega lomnega količnika. Ko vzorec 
premikamo vzdolž snopa, se skupna efektivna goriščna razdalja leče in nelinearne snovi 
spreminja in snop na detektorju je enkrat bolj zbran, drugič manj. 
Za lege vzorca desno od prvotnega gorišča ($z>0$) je skupna goriščna
razdalja daljša od goriščne razdalje leče, snop je bolj zbran (pikčasta črta) in signal 
na detektorju (D) naraste. Za lege vzorca levo
od prvotnega gorišča ($z<0$) je ravno obratno, snop se razširi (črtkana črta) in 
signal na detektorju se zmanjša. Za snovi z negativnim nelinearnim lomnim količnikom
je odziv ravno nasprotnega predznaka. Pri določanju nelinearnega lomnega količnika je
ključno uporabiti zaslonko (Z), s katero omejimo premer vpadnega snopa pred detektorjem. 
Če zaslonko odstranimo in merimo 
odvisnost celotne vpadne intenzitete od lege vzorca, nelinearnega lomnega količnika 
ne moremo meriti, lahko pa določimo nelinearni absorpcijski koeficient. 

\begin{figure}[ht]
\centering 
\def\svgwidth{110truemm} 
\input{slike/08_zscan.pdf_tex}
\caption{Shema metode vzdolžnega premika (a): Z -- zaslonka, D -- detektor, $n_2$ - optično
nelinearno sredstvo in $f$ -- leča. S premikanjem nelinearnega sredstva se spreminja intenziteta svetlobe, 
ki vpade na detektor (b).}
\label{fig:zscan}
\end{figure}
\end{remark}

\section{*Izpeljava krajevnih solitonov}
\index{Soliton!krajevni}
\label{chap:ks}
V prejšnjem razdelku smo opisali nastanek solitonov v optično nelinarnih snoveh. 
Za podrobnejšo obravnavo krajevnih solitonov moramo rešiti valovno 
enačbo. Reševali jo bomo v obosnem približku in začeli s skalarno obliko Helmholtzeve enačbe (enačba~\ref{eq:Helmholtz})
\begin{equation}
\nabla^{2}E+n^{2}\frac{\omega^{2}}{c_0^{2}}E=0.
\label{8.80}
\end{equation}
Polje zapišemo v obliki počasi spreminjajoče se amplitude $\psi$ in faznega faktorja, podobno kot 
smo to naredili pri izpeljavi Gaussovega snopa (enačba~\ref{eq:ravni-val-nastavek})
\begin{equation}
E=\psi(\mathbf{r},z)e^{ik_{0}z},
\label{8.81}
\end{equation}
pri čemer sta $\mathbf{r}$ krajevni vektor v ravnini $xy$ in 
\begin{equation}
k_{0}=\frac{\tilde{n}\omega}{c_0} 
\end{equation}
valovno število brez upoštevanja nelinearnosti.
Funkcija $\psi(\mathbf{r},z)$ naj se v smeri osi $z$ le počasi spreminja, tako da lahko
drugi odvod po $z$ zanemarimo in zapišemo \index{Obosna valovna enačba}
\begin{equation}
\nabla_{\bot}^{2}\psi+\frac{\omega^{2}}{c_0^{2}}(n^{2}-\tilde{n}^{2})\psi+2ik_{0}
\frac{\partial\psi}{\partial z}=0.
\label{8.82}
\end{equation}
Upoštevamo odvisnost lomnega količnika od intenzitete, pri čemer
zanemarimo člen z $n_{2}^{2}$, ker je gotovo majhen. Zapišemo
\begin{equation}
\nabla_{\bot}^{2}\psi+2k_{0}^{2}\frac{n_{2}}{\tilde{n}}j\psi+2ik_{0}\frac{\partial\psi}{\partial z}=0.
\label{8.83}
\end{equation}
Izrazimo še gostoto svetlobnega toka z amplitudo jakosti električnega polja in dobimo
\begin{equation}
\nabla_{\bot}^{2}\psi+
k_{0}^{2} n_2 \varepsilon_0 c_0 |\psi|^2 \psi+
2ik_{0}\frac{\partial\psi}{\partial z}=0.
\label{8.83a}
\end{equation}
Preden se lotimo reševanja enačbe, vpeljemo še
\begin{equation}
\kappa=k_{0}^{2} n_2 \varepsilon_0 c_0
\end{equation}
 in novo spremenljivko vzdolž osi $z$
\begin{equation}
\zeta=\frac{z}{2k_{0}}.
\end{equation}
 S tem preide enačba (\ref{8.83a}) v standardno obliko nelinearne Schr\"odingerjeve
enačbe, le da namesto odvoda po času tukaj nastopa odvod po koordinati $\zeta$. Sledi
\index{Schr\"odingerjeva enačba!nelinearna}
\boxeq{8.84}{
i\frac{\partial\psi}{\partial\zeta}+\nabla_{\bot}^{2}\psi+\kappa\left|\psi\right|^{2}\psi=0.
}
V treh dimenzijah je reševanje enačbe (\ref{8.84}) težavno in analitične
rešitve niso znane. V dveh dimenzijah pa stacionarno rešitev znamo
poiskati. Stacionarni rešitvi se vzdolž $\zeta$ lahko spreminja le faza, zato
rešitev iščemo v obliki 
\begin{equation}
\psi=e^{i\eta^{2}\zeta}\, u(x),
\label{8.87}
\end{equation}
pri čemer je $\eta$ konstanta, katere pomen bomo videli v nadaljevanju, 
 funkcija $u(x)$ pa naj bo realna. 
Uporabimo nastavek (enačba~\ref{8.87}) v enačbi (\ref{8.84}) in dobimo
\begin{equation}
\frac{d^{2}u}{dx^{2}}=\eta^{2}u-\kappa u^{3}.
\end{equation}
 Z množenjem obeh strani z $u^{\prime}$ lahko enačbo enkrat integriramo
\begin{equation}
\left(\frac{du}{dx}\right)^{2}=\eta^{2}u^{2}-\frac{1}{2}\kappa u^{4}.
\end{equation}
Ločimo spremenljivki in zapišemo 
\begin{equation}
\int_{\eta\sqrt{2/\kappa}}^{u}\frac{du}{\sqrt{\eta^{2}u^2-\frac{1}{2}\kappa u^{4}}}=x-x_{0},
\label{8.85}
\end{equation}
pri čemer smo uvedli integracijsko konstanto $x_{0}$ in integracijsko mejo postavili 
tako, da so vrednosti pod korenom pozitivne.
Integral brez težav izračunamo
\begin{equation}
\frac{1}{\eta}\ln\left(\sqrt{\frac{\kappa}{2}}\frac{u}{\eta+
\sqrt{\eta^{2}-\kappa u^{2}/2}}\right)=x-x_{0}
\end{equation}
in izrazimo iskano funkcijo $u(x)$
\begin{equation}
u=\sqrt{\frac{2}{\kappa}}\frac{2 \eta }{e^{\eta(x-x_{0})}+e^{-\eta(x-x_{0})}}=
\sqrt{\frac{2}{\kappa}}\frac{\eta}{\cosh \left(\eta(x-x_{0})\right)}.
\label{8.86}
\end{equation}
Po enačbi (\ref{8.87}) je rešitev
\begin{equation}
\psi(x,z)=\sqrt{\frac{2}{\kappa}}\,\eta\,\frac{e^{i\eta^{2}\zeta}}{\cosh \left(\eta(x-x_{0})\right)}.
\label{8.88}
\end{equation}
Vidimo, da predstavlja spremenljivka $1/\eta$  karakteristično širino snopa, $x_{0}$ pa
je le njegov prečni premik, ki ga lahko brez škode postavimo na $x_0=0$. Tako
zapišemo celotno polje stacionarnega snopa 
\begin{equation}
E_{s}(x,z)=\sqrt{\frac{2}{\kappa}}\,\frac{\eta}{\cosh(\eta x)}\,\exp\left(ik_{0}z\left(1+
\frac{\eta^{2}}{2k_{0}^{2}}\right)\right)\!.
\label{8.89}
\end{equation}
Gostoto svetlobnega toka,\index{Gostota energijskega toka} ki je sorazmerna 
kvadratu amplitude polja, zapišemo kot
\boxeq{8.89a}{
j_{s}(x,z)= j_0 \frac{1}{\cosh^2(\eta x)}.
}
Iz zgornje enačbe sledi, da je gostota svetlobnega toka neodvisna od $z$, kar pomeni, 
da sunek  ohranja svojo obliko, ko potuje vzdolž osi $z$ (slika~\ref{fig:soliton}).

\begin{figure}[ht]
\centering
\def\svgwidth{100truemm} 
\input{slike/08_soliton.pdf_tex}
\caption{Prečna odvisnost relativne intenzitete krajevnega solitona $j/j_0$ v dveh dimenzijah. 
Vzdolž koordinate $z$ se profil ohranja.}
\label{fig:soliton}
\end{figure}

Če se vrnemo k izrazu za jakost električnega polja (enačba~\ref{8.89}), vidimo, da
parameter $\eta$ nastopa tudi v faznem faktorju. To pomeni, da je od njega odvisna 
tudi fazna hitrost
\index{Hitrost valovanja!solitonov}
\begin{equation}
v_{f}= \frac{c_0}{\tilde{n}\left(1+\frac{\eta^{2}}{2k_{0}^{2}}\right)}.
\end{equation}
Fazna hitrost omejenih snopov oziroma solitonov je torej vedno manjša od fazne hitrosti ravnih valov. 
Bolj ko je snop omejen, manjša je fazna hitrost, za velike polmere snopa pa doseže 
limitno vrednost $c_0/\tilde{n}$.

Celotna moč svetlobe v dvodimenzionalnem snopu je enaka integralu
gostote svetlobnega toka (enačba~\ref{8.89a}) po $x$. Integriramo in zapišemo 
\begin{equation}
P_s = \int j_s\,dx \propto \int |E_s|^2 dx  = 
\frac{2}{\kappa}\,\eta^{2}\int_{-\infty}^{\infty}\frac{dx}
{\cosh^{2}\eta x}=\frac{4\eta}{\kappa}.
\label{eq:solj}
\end{equation}
Moč stacionarnega snopa (solitona) je torej obratno sorazmerna 
z njegovo širino $1/\eta$. Manjši kot je polmer snopa, močnejši je uklon
in večja moč je potrebna, da nelinearni pojavi izničijo vpliv uklona. 
V dveh dimenzijah obstaja stacionarna rešitev, v treh dimenzijah pa 
se snop z nadkritično močjo skrči v singularnost.

\section{Optični solitoni}
\index{Soliton!optični}
\label{chap:soliton}
V prejšnjem razdelku smo ugotovili, da pojav samozbiranja lahko izniči širjenje 
svetlobnega snopa zaradi uklona, tako da ima pri
ustrezni moči snop vzdolž smeri širjenja konstantno širino in obliko. Takim snopom 
smo rekli krajevni solitoni. Povsem podoben pojav poznamo tudi v časovni 
domeni, kjer se pojavijo časovni ali optični solitoni.\footnote{~Glej npr. R. W. 
Boyd, {\it Nonlinear Optics}, tretja izdaja, Academic Press (2008).}

Sunek svetlobe  naj se širi po valovnem vodniku. Zaradi disperzije je lomni količnik\index{Disperzija}
odvisen od frekvence valovanja in sunek svetlobe se med potovanjem po vodniku podaljšuje. 
Več o tem smo spoznali pri 
obravnavi disperzije v optičnih vlaknih (razdelek~\ref{chap:Disperzija}). 
Ob primernih pogojih lahko nelinearna odvisnost lomnega količnika $n(j)$ 
ravno izniči disperzijo $n(\omega)$ in sunek
ohranja obliko. Sunkom svetlobe, ki potujejo po sredstvu brez spremembe
oblike, pravimo optični solitoni. Posebej so pomembni v optičnih vlaknih, 
kjer želimo  vpliv disperzije zaradi učinkovitosti prenosa
informacije kar se da zmanjšati. 

Pojava optičnih solitonov ni težko pojasniti. Naj na optično nelinearno sredstvo
vpade sunek svetlobe, ki je Gaussove oblike v času
\begin{equation}
j(t) = j_0 e^{-2t^2/\tau^2}.
\label{08_pulz}
\end{equation}
Faza takega sunka je ob upoštevanju enačbe~(\ref{eq:nnl}) enaka
\begin{equation}
\phi (t) = k_0 n z - \omega_0 t = k_0 (\tilde{n} + n_2 j)z - \omega_0 t = 
\phi_0 + k_0 n_2 z j - \omega_0 t,
\end{equation}
krožna frekvenca pa 
\begin{equation}
\omega = -\frac{d\phi}{dt} = \omega_0 - k_0 n_2 z \frac{dj}{dt}.
\end{equation}
Če vstavimo časovno obliko sunka svetlobe (enačba~\ref{08_pulz}), vidimo, da se 
frekvenca takega sunka spreminja s časom
\begin{equation}
\omega = \omega_0 + \frac{4k_0 n_2 z j_0}{\tau^2} \, t \, e^{-2t^2/\tau^2}.
\label{eq:chirpi}
\end{equation}
Začetnemu delu sunka (pri $t<0$) se krožna frekvenca zmanjša, medtem ko se zadnjemu delu sunka
(pri $t>0$) poveča (slika~\ref{fig:optsoliton}). 
Ta pojav spreminjanja frekvence znotraj kratkega sunka imenujemo čirikanje
({\it chirping}), \index{Čirikanje} po podobnosti z oglašanjem čričkov.

\begin{figure}[ht]
\centering
\def\svgwidth{70truemm} 
\input{slike/08_OpticniSoliton.pdf_tex}
\caption{Kratkemu sunku svetlobe, ki se širi po valovnem vodniku, se zaradi nelinearnega 
lomnega količnika snovi v začetnem delu frekvenca zmanjša, v končnem delu pa poveča.
Spreminjanje frekvence znotraj kratkega sunka imenujemo čirikanje.}
\label{fig:optsoliton}
\end{figure}
Pri prehodu optičnega sunka z osnovno krožno frekvenco $\omega_0$ se različnim delom sunka
frekvenca različno spremeni (slika~\ref{fig:chirp}\,a), začetnemu delu se zmanjša in 
končnemu poveča. Po drugi strani v snoveh poznamo barvno disperzijo, 
kar pomeni, da se valovanja z različnimi frekvencami širijo z različnimi hitrostmi.
\index{Disperzija} Pojav disperzije je še bolj zapleten pri potovanju sunkov svetlobe po vlaknih
(glej razdelka~\ref{chap:Disperzija} in \ref{chap:sunvl}). Ključni parameter je disperzija 
grupne hitrosti oziroma drugi odvod valovnega vektorja po krožni frekvenci (enačba~\ref{9.71}).

Pod določenimi pogoji (izbrana snov in določeno frekvenčno območje) 
lahko dosežemo, da potuje del valovanja z daljšo valovno dolžino počasneje kot del valovanja
s krajšo valovno dolžino (slika~\ref{fig:chirp}\,b). V tem primeru končni del sunka 
dohiteva sprednjega in učinek disperzije ravno izniči učinek nelinearnosti. 
Nastane signal, ki ohranja svojo obliko~\textendash~soliton. 
\begin{figure}[ht]
\centering
\def\svgwidth{145truemm} 
\input{slike/08_Chirp.pdf_tex}
\caption{Čirikanje sunkov svetlobe zaradi nelinearnega pojava (a). Z ustrezno disperzijo lahko
čirikanje izničimo (b) in nastane soliton.}
\label{fig:chirp}
\end{figure}

\section{*Izpeljava optičnih solitonov}
\index{Soliton!optični}
Za matematični opis optičnih solitonov izhajamo iz nelinearne \index{Valovna enačba!nelinearna}
valovne enačbe (enačba~\ref{8.3}), ki jo zapišemo v skalarni obliki
\begin{equation}
\nabla^{2}E-\frac{n^2}{c_0^{2}}{\frac{\partial^2 E}{\partial t^2}}=
\mu_{0}{\frac{\partial^2P_{\textrm{NL}}}{\partial t^2}},
\end{equation}
pri čemer je 
$P_\textrm{NL}$ nelinearna polarizacija tretjega reda (enačba~\ref{eq:nlin3}).
Namesto v časovni domeni je enačbo prikladnejše reševati v frekvenčni domeni, zato
namesto $E$ in $P_{\mathrm{NL}}$ vpeljemo Fourierevi transformiranki $\tilde{E}$ in $\tilde{P}$.

Dobimo
\begin{equation}
\nabla^{2}\tilde{E}+\frac{n^2}{c_0^{2}}\omega^2 \tilde{E}=
- \mu_{0}\omega^2 \tilde{P}.
\label{eq:soleq1}
\end{equation}
Enačbo rešujemo z nastavkoma
\begin{equation}
\tilde{E} = \tilde{A} (z,\omega - \omega_0) e^{ik_0z}
\end{equation}
in 
\begin{equation}
 \tilde{P} = \tilde{B} (z,\omega - \omega_0) e^{ik_0z},
\end{equation}
pri čemer je $\omega_0$ osrednja krožna frekvenca svetlobnega sunka in $k_0 = \omega_0 \tilde{n}/c_0$. 
Vpeljemo še 
$\Omega =\omega - \omega_0$ in valovna enačba~(\ref{eq:soleq1}) se prepiše v 
\begin{equation}
\left(\frac{\partial^2}{\partial z^2}+k^2\right)\tilde{A}(z,\Omega) e^{ik_0z} =
- \mu_{0}\omega^2 \tilde{B} (z,\Omega) e^{ik_0z}.
\end{equation}
Da lahko rešimo to enačbo, naredimo nekaj približkov. Ker je $\omega \approx \omega_0$, na desni strani
enačbe nadomestimo frekvenco z osrednjo frekvenco. Poleg tega upoštevamo, da se amplituda 
glede na valovno dolžino le počasi spreminja, zato drugi odvod zanemarimo in zapišemo
\begin{equation}
2 i k_0 \frac{\partial \tilde{A}}{\partial z} + (k^2-k_0^2) \tilde{A} = - \mu_{0}\omega_0^2 \tilde{B}.
\end{equation}
Člen $k^2 - k_0^2$ razstavimo kot razliko kvadratov, ob šibki disperziji pa $k(\omega_0 + \Omega)$
razvijemo v Taylorjevo vrsto za majhne $\Omega$ 
okoli osrednje krožne frekvence $\omega_0$ do tretjega člena. Dobimo
\begin{equation}
k^2 - k_0^2 \approx 2k_0 (k-k_0) \approx 2k_0 (k'\Omega + \frac{1}{2}k''\Omega^2),
\end{equation}
pri čemer $'$ označuje odvod po krožni frekvenci. Enačbo prepišemo v 
\begin{equation}
2 i k_0 \frac{\partial \tilde{A}}{\partial z} + 2k_0(k'\Omega + \frac{1}{2}k''\Omega^2) \tilde{A} 
= - \mu_{0}\omega_0^2 \tilde{B}.
\end{equation}
Vrnimo se v časovno domeno, tako da naredimo inverzno Fourierevo transformacijo. Naj bo 
$A(z,t)$ kompleksna amplituda jakosti električnega polja in inverzna transformiranka 
funkcije $\tilde{A}(z,\Omega)$, funkcija $B(z,t)$ pa naj bo 
amplituda polarizacije in inverzna transformiranka 
funkcije $\tilde{B}(z,\Omega)$.
Sledi
\begin{equation}
i \left(\frac{\partial}{\partial z}+\frac{1}{v_{g}}\frac{\partial}{\partial t}\right)A-
\frac{1}{2}\frac{d^{2}k}{d\omega^{2}}\,\frac{\partial^{2}A}{\partial t^{2}}=
-\frac{\mu_0\omega_0^2}{2 k_0}B,
\label{8.93}
\end{equation}
pri čemer smo z 
\begin{equation}
 v_g = \frac{d\omega}{dk} = \frac{1}{k'}
\end{equation}
označili grupno hitrost.\index{Hitrost valovanja!grupna}
Vpeljemo novo spremenljivko 
\begin{equation}
\tau=t-\frac{z}{v_{g}},
\label{nelinver}
\end{equation}
s katero opišemo obliko sunka $A_S(z,\tau)$, kot ga vidi opazovalec, ki se giblje
z grupno hitrostjo skupaj s sunkom. Uporabimo pravilo verižnega odvajanja 
\begin{equation}
\frac{\partial A}{\partial z} = \frac{\partial A_S}{\partial z} + \frac{\partial A_S}{\partial \tau}
\frac{\partial \tau}{\partial z}
= \frac{\partial A_S}{\partial z} -\frac{1}{v_g} \frac{\partial A_S}{\partial \tau}.
\end{equation}
Podobno naredimo še za odvod po času $\tau$, ki pa se ne razlikuje od odvoda po času $t$
\begin{equation}
\frac{\partial A}{\partial t} = \frac{\partial A_S}{\partial z}\frac{\partial z}{\partial \tau}+
\frac{\partial A_S}{\partial \tau}\frac{\partial \tau}{\partial t} 
= \frac{\partial A_S}{\partial \tau} 
\end{equation}
in
\begin{equation}
\frac{\partial^2 A}{\partial t^2} = \frac{\partial^2 A_S}{\partial\tau^2}.
\end{equation}
Vstavimo še amplitudo nelinearne polarizacije (enačba~\ref{eq:ptnl}), pri čemer izraz popravimo
za faktor $4$, ker smo drugače vpeljali parameter $A$. Dobimo
\begin{equation}
B = 3\varepsilon_0\chi^{(3)} |A|^2 A
\end{equation}
in enačbo~(\ref{8.93}) zapišemo kot 
\begin{equation}
i\,\frac{\partial A_S}{\partial z}-\frac{1}{2}\frac{d^{2}k}{d\omega^{2}}\,
\frac{\partial^{2}A_S}{\partial\tau^{2}}+\kappa\left|A_S\right|^{2}A_S=0.
\label{8.95}
\end{equation}
Pri tem je parameter
\begin{equation}
\kappa = \frac{3\omega_0\chi^{(3)}}{2c_0 \tilde{n}} = 2 \omega_0 \varepsilon_0 n_2 \tilde{n}
\end{equation}
sorazmeren nelinearnemu lomnemu količniku $n_2$ 
(enačba~\ref{eq:n2}). Enačba~(\ref{8.95}) ni nič drugega kot nelinearna Schr\"odingerjeva 
enačba\index{Schr\"odingerjeva enačba!nelinearna}, ki smo jo 
zapisali že pri izpeljavi krajevnih solitonov~(enačba~\ref{8.84}). Enačbi se razlikujeta po tem, da
ima vlogo prečne koordinate $x$ tukaj čas $\tau$ in rešitve nimajo več konstantnega premera,
ampak imajo konstantno dolžino sunka. Stacionarne rešitve obstajajo le v primeru, kadar je  $d^{2}k/d\omega^{2}<0$ oziroma kadar ima drugi odvod nasprotni predznak od nelinearnega lomnega količnika $n_2$. Kot pri krajevnih solitonih tudi tukaj vpeljemo parameter $\eta$, ki je sorazmeren 
z energijo solitona (enačba~\ref{eq:solj}). Sledi 
\begin{equation}
A_S\left(z,\tau\right)=\sqrt{\frac{2}{\kappa}}\eta\frac{e^{i\eta^{2}z}}{{\cosh}\left(\eta \tau 
\sqrt{2\left|\frac{d^{2}k}{d\omega^{2}}\right|^{-1}}\right)}
\end{equation}
oziroma
\begin{equation}
A\left(z,t\right)=\sqrt{\frac{2}{\kappa}}\eta\frac{e^{i\eta^{2}z}}{{\cosh}\left(\eta (t-\frac{z}{v_g}) 
\sqrt{2\left|\frac{d^{2}k}{d\omega^{2}}\right|^{-1}}\right)}.
\label{8.96}
\end{equation}
Zapisana je oblika solitona, ki potuje z grupno hitrostjo $v_g$ in pri tem ohranja obliko. Zaradi tega
so solitoni izredno zanimivi za prenos velike gostote informacij na velike razdalje, saj se izognemo
omejitvam disperzije. 

\begin{remark}
Ena izmed snovi, ki izpolnjuje pogoj, da je $k''$ nasprotnega predznaka kot $n_2$, so kremenova 
optična vlakna. Pri valovnih dolžinah vidne svetlobe to sicer ne velja, velja pa za 
$\lambda \gtrsim 1,3~\si{\micro\metre}$.\index{SiO$_2$}
Pogoj je torej izpolnjen pri valovnih dolžinah okoli $1,5~\si{\micro\metre}$, ki se navadno uporabljajo 
pri prenosu signalov po optičnih vlaknih, in signal lahko potuje brez podaljševanja. 
\end{remark}

\section{Optična fazna konjugacija}
\index{Optična fazna konjugacija}
Optična fazna konjugacija je zanimiv in danes tudi praktično pomemben
pojav, pri katerem nastane iz danega valovanja novo valovanje z enakimi valovnimi
frontami, vendar potuje novo valovanje v nasprotni smeri od prvotnega.\footnote{~Glej npr. R.
W. Boyd, {\it Nonlinear Optics}, tretja izdaja, Academic Press (2008).} Novo valovanje je tako,
kot bi začetnemu valovanju obrnili predznak časa in ga ``zavrteli nazaj''.\footnote{~R. W. Hellwarth, 
J. Opt. Soc. Am $\mathbf{67}$, 1 (1977).}

Vzemimo optično nelinearno snov, na katero posvetimo z dvema močnima 
snopoma, ki potujeta v nasprotnih smereh. To sta črpalna snopa z valovnima vektorjema
${\bf k}_{1}$ in ${\bf k}_2 = -{\bf k}_{1}$, ki naj bosta kar se da podobna ravnemu valu. 
Poleg njiju naj na snov vpada
še tretji, signalni snop, ki ni nujno ravni val (slika~\ref{08_OPC1}). 
Frekvence vseh snopov naj bodo enake.
Signalni snop interferira s prvim črpalnim valom in zaradi nelinearnosti 
tretjega reda povzroči modulacijo lomnega količnika, na kateri se
drugo črpalno valovanje uklanja. Uklonjeno valovanje je enake oblike
kot signalno, le potuje v nasprotni smeri, saj ima drugo črpalno valovanje
nasprotno smer od prvega. Črpalni valovanji sta seveda enakovredni in ni
mogoče ločiti med valovanjem, s katerim signalno valovanje interferira, in valovanjem, 
ki se uklanja.

\begin{figure}[ht]
\centering
\def\svgwidth{60truemm} 
\input{slike/08_opc1.pdf_tex}
\caption{Optična fazna konjugacija. Dva močna črpalna žarka (modra) z valovnima
vektorjema ${\bf k}_{1}$ in ${\bf k}_2$ vpadata na optično nelinearno snov v 
nasprotnih smereh, vpadni signal (rdeč) pa se odbije v 
smer, iz katere vpada.}
\label{08_OPC1}
\end{figure}
\vglue-3truemm
\begin{remark}Optična fazna konjugacija je zelo podobna holografiji, 
le da pri holografiji najprej zapišemo predmetni snop, ki ga kasneje reproduciramo, 
medtem ko pri fazni konjugaciji zapis valovanja in njegova reprodukcija \index{Hologram}
potekata sočasno. 
\end{remark}

Koordinatni sistem izberemo tako, da se signalno valovanje s frekvenco $\omega$ in amplitudo $A_3$
širi v smeri $z$. Zapišemo ga kot
\begin{equation}
E_{3}=\mathfrak{\Re}\left(A_3\left(z\right)\, e^{i\left(kz-\omega t\right)}\right)\!,
\label{8.97}
\end{equation}
pri čemer $\mathfrak{\Re}$ označuje realni del. 
V naslednjem razdelku bomo z računom pokazali, da je novonastalo valovanje sorazmerno
\begin{equation}
E_{4} \propto \mathfrak{\Re}\left(A_3^{*}\left(z\right)\, e^{i\left(-kz-\omega t\right)}\right)\!.
\label{8.98}
\end{equation}
Zaradi nasprotnega predznaka $k$ potuje nastalo valovanje v obratni smeri od signalnega
valovanja, poleg tega je kompleksno konjugirana tudi njegova amplituda. To seveda
ne vpliva na obliko valovnih front, saj so te popolnoma enake kot pri signalnem
valovanju. Ker novo valovanje iz signalnega nastane tako,
da krajevni del faze kompleksno konjugiramo, nastalemu valovanju pravimo fazno
konjugirano valovanje.
\begin{figure}[ht]
\centering
\def\svgwidth{110truemm} 
\input{slike/08_opc2.pdf_tex}
\caption{Primerjava odbojev na navadnem zrcalu (zgoraj) in faznem konjugatorju (spodaj): odboj ravnega
vala (a), odboj krogelnega valovanja (b) in odboj popačenega vala (c). Valovne fronte 
vpadnega vala so označene s polno črto in odbitega s črtkano.}
\label{08_OPC2}
\vglue-2truemm
\end{figure}

Uporabna posledica fazne konjugacije je prikazana na sliki~\ref{08_OPC2}.
Najpreprostejši primer je vpad ravnega vala (a), ki se ne odbije po 
odbojnem zakonu (slika zgoraj), ampak se odbije v smer, iz katere 
je vpadel na snov (slika spodaj). Drugi primer je krogelni val 
ali v približku tudi Gaussov snop (b). Po odboju od navadnega zrcala (zgoraj), se 
valovanje še naprej razširja. Na fazno konjugiranem zrcalu se odbiti krogelni val spet
zbere v izvoru (spodaj). 

V tretjem primeru vpada svetloba skozi sredstvo, ki valovanju doda naključno
fazo, zato po prehodu valovne fronte niso več gladke (c). Od navadnega zrcala
se popačen snop odbije, pri ponovnem prehodu skozi sredstvo pa se popačenje
še poveča. Povsem drugačno je obnašanje pri odboju na faznem konjugatorju. 
Ko popačen snop vpade na fazni konjugator, v njem ustvari fazno konjugirani snop, 
ki potuje v nasprotni smeri in ima enako nepravilne valovne fronte kot vpadni val. Po prehodu
skozi nepravilno sredstvo se neravnosti valovne fronte izničijo
in nastanejo enake gladke valovne fronte ravnega vala kot na začetku. 
To lastnost popravljanja valovne fronte je mogoče 
koristno uporabiti, na primer namesto enega zrcala v laserskem resonatorju.
\vglue-3truemm
\begin{remark}
Omenili smo, da se fazno konjugirana zrcala uporabljajo v laserjih za izničenje
popačenj Gaussovega snopa. Drugi primer uporabe je v optični astronomiji
ali optičnih komunikacijah skozi atmosfero. Naključne spremembe gostote v atmosferi
signalu dodajo naključni fazni premik, ki signal popači. Če se signal odbije od zrcala nazaj
proti izvoru, je torej dvakratno popačen. Če pa se odbije od fazno konjugiranega zrcala, 
se vpliv nehomogenosti atmosfere ravno izniči in na prenos signala ne vpliva, poleg
tega je šibek vpadni signal lahko še dodatno ojačen. 
\end{remark}

\section{*Izpeljava optične fazne konjugacije}
\index{Optična fazna konjugacija}
Poglejmo podrobneje, kako v nelinearnem sredstvu nastane fazno konjugirani
val. Kot je razvidno iz slike~\ref{08_OPC1}, je celotno polje v nelinearnem
sredstvu vsota štirih valovanj, dveh močnih črpalnih (oznaki 1 in 2), signalnega 
(oznaka 3) in novonastalega (oznaka 4)
\begin{equation}
E=\frac{1}{2}A_{1}e^{i{\bf k}_{1}\cdot{\bf r}-i\omega t}+\frac{1}{2}A_{2}e^{-i{\bf k}_{1}\cdot{\bf r}-
i\omega t}+\frac{1}{2}A_{3}\left(z\right)e^{ikz-i\omega t}+\frac{1}{2}A_{4}
\left(z\right)e^{-ikz-i\omega t}+\mathrm{k.~k.}
\label{8.99}
\end{equation}
S k.~k. smo spet označili kompleksno konjugirane člene.  Vsa valovanja naj imajo
enako frekvenco, zaradi enostavnosti privzamemo, da so enake tudi vse polarizacije.
Račun poenostavimo še s predpostavko, da sta črpalna vala dosti močnejša 
od signalnega in novonastalega, tako da sta njuni
amplitudi $E_{1}$ in $E_{2}$ konstantni, $E_{3}\left(z\right)$ in $E_{4}\left(z\right)$
pa se le počasi spreminjata.

Vstavimo nastavek za $E$ (enačba~\ref{8.99}) v valovno enačbo (enačba~\ref{8.3})
\index{Valovna enačba!nelinearna}
\begin{equation}
\nabla^{2}E+n^2\frac{\omega^{2}}{c_0^{2}}\, 
E=\mu_{0}\frac{\partial^2 P_{\mathrm{NL}}}{\partial t^2},
\label{8.100}
\end{equation}
pri čemer je $n\,\omega/c_0=k$. $P_{\textrm{NL}}$ je po enačbi~(\ref{eq:nl3P})
enak $P_\mathrm{NL}= \epsilon_{0}\chi^{(3)}E^3$. S $\chi^{(3)} = \chi$ smo označili
efektivno nelinearno susceptibilnost
za izbrano polarizacijo. 

Ker je $E$ zapisan kot vsota osmih členov
(enačba~\ref{8.99}), vsebuje produkt $E^3$ kar 512 členov. Vendar se njihovo število znatno zmanjša, 
če upoštevamo le tiste z enako časovno odvisnostjo oziroma enako frekvenco.
Poleg tega nas ne zanimajo različne kombinacije valovnih vektorjev, ampak k enačbi za $E_{3}$ 
prispevajo le tisti členi s krajevnim faznim faktorjem $\exp(ikz)$, 
k enačbi za $E_4$ pa tisti z $\exp(-ikz)$. Sledi
\begin{equation}
\begin{split}
P_{\mathrm{NL}\,3,4} = \frac{\varepsilon_0\chi}{8} \big(
\left(6 A_1 A_2 A_4^*+ 6A_1 A_1^*A_3 + 6A_2A_2^*A_3 + 3 A_3A_3^*A_3 + 6 A_4 A_4^* A_3\right)
e^{i k z - i\omega t} \\
+
\left(6 A_1 A_2 A_3^*+6 A_1 A_1^*A_4 + 6A_2A_2^*A_4 + 6 A_3A_3^*A_4 + 3 A_4 A_4^* A_4\right)
e^{-i k z - i\omega t} \big).
\end{split}
\end{equation}
Če zanemarimo še člene, v katerih nastopata $A_3$ in $A_4$ v višjih potencah, dobimo
\begin{equation}
\begin{split}
P_{\mathrm{NL}\,3,4} = \frac{3\varepsilon_0\chi}{4} \big(
\left( A_1 A_2 A_4^*+ |A_1|^2 A_3 + |A_2|^2 A_3 \right)
e^{i k z - i\omega t} \\
+ 
\left( A_1 A_2 A_3^*+|A_1|^2 A_4 + |A_2|^2A_4 \right)
e^{-i k z - i\omega t} \big).
\end{split}
\end{equation}
Vstavimo izračunani izraz v valovno enačbo~(enačba~\ref{8.100}) in upoštevamo, 
da se $A_i(z)$ le počasi 
spreminja (kar pomeni, da zanemarimo drugi odvod po $z$). Za $A_3$ so pomembni 
samo členi s potenco $ikz-i\omega t$, za $A_4$ pa členi s potenco $-ikz-i\omega t$. Dobimo 
\begin{equation}
i k \frac{dA_3}{dz} = - \frac{3}{4} \mu_0\varepsilon_0 \chi \omega^2 
\left( A_1 A_2 A_4^*+ (|A_1|^2 + |A_2|^2) A_3 \right)
\label{eq:opc1}
\end{equation}
in 
\begin{equation}
-i k \frac{dA_4}{dz} = - \frac{3}{4} \mu_0\varepsilon_0 \chi \omega^2 
\left( A_1 A_2 A_3^*+ (|A_1|^2 + |A_2|^2) A_4 \right)\!.
\label{eq:opc2}
\end{equation}
Drugi člen na desni že poznamo: opisuje odvisnost lomnega količnika
od intenzitete črpalnih valov, torej optični Kerrov\index{Kerrov pojav!optični}
pojav, in je zato le dodaten prispevek
k fazi. Vpeljemo novi amplitudi, ki se od prejšnjih razlikujeta zgolj v faznem faktorju.
\begin{equation}
\tilde{A}_3 = A_3 \exp\left(-i\frac{ 3 \chi \omega}{4 c_0 n}(|A_1|^2 + |A_2|^2) z\right)
\end{equation}
in 
\begin{equation}
\tilde{A}_4 = A_4 \exp\left(i\frac{ 3 \chi \omega}{4 c_0 n}(|A_1|^2 + |A_2|^2)z\right)\!.
\end{equation}
Ko novi amplitudi vstavimo v diferencialni enačbi~(enačbi~\ref{eq:opc1} in 
\ref{eq:opc2}), se Kerrov prispevek k fazi odšteje
in enačbi se prepišeta v 
\begin{equation}
\frac{d\tilde{A}_{3}}{dz}=i\frac{ 3 \chi \omega}{4 c_0 n}\,
A_{1}A_{2}\tilde{A}_{4}^{*} \qquad \textrm{in} \qquad 
\frac{d\tilde{A}_{4}}{dz}=-i\frac{ 3 \chi \omega}{4 c_0 n}\,
A_{1}A_{2}\tilde{A}_{3}^*.
\label{8.105}
\end{equation}
Vpeljemo še sklopitveno konstanto
\begin{equation}
\kappa=\frac{ 3 \chi \omega}{4 c_0 n}A_1 A_2.
\label{8.106}
\end{equation}
Enačbi se poenostavita v 
\begin{equation}
\frac{d\tilde{A}_{3}}{dz}=i\kappa \tilde{A}_{4}^{*} \qquad
\textrm{oziroma} \qquad \frac{d\tilde{A}^*_{3}}{dz}=-i\kappa^* \tilde{A}_{4} 
\label{8.107}
\end{equation}
in
\begin{equation}
\frac{d\tilde{A}_{4}}{dz}=-i\kappa \tilde{A}_{3}^*.
\label{8.107a}
\end{equation}
Zelo težaven problem nelinearne valovne enačbe smo prevedli na linearni
sistem dveh preprostih sklopljenih enačb za amplitudi signalnega in
odbitega vala. Rešitvi sistema sta
sta 
\begin{align}
\tilde{A}_3^* \left(z\right) & =  C_{1}\cos(\left|\kappa\right|z)+
C_{2}\sin(\left|\kappa\right|z)
\label{8.108} \qquad \mathrm{in}\\
\tilde{A}_4 \left(z\right) & =  D_{1}\cos(\left|\kappa\right|z)+
D_{2}\sin(\left|\kappa\right|z).
\label{8.108a}
\end{align}
Z upoštevanjem zveze, ki izhaja  iz prve diferencialne enačbe 
(enačba~\ref{8.107}), zapišemo
\begin{equation}
C_1 = \frac{i \kappa^*}{|\kappa|}D_2 \qquad
\textrm{in} \qquad 
C_2 = -\frac{i \kappa^*}{|\kappa|}D_1. 
\end{equation}
Potrebujemo še robne pogoje za obe valovanji. Z leve, pri $z=0$,
poznamo $\tilde{A}_{3}^{*}\left(0\right)$, pri $z=L$ pa ne more biti odbitega
vala in je zato $\tilde{A}_{4}\left(L\right)=0$. S tem določimo konstanti $D_{1}$
in $D_{2}$
\begin{equation}
D_2 = -\frac{i|\kappa|}{\kappa^*} \tilde{A}_3^*(0) \qquad
\textrm{in} \qquad 
D_1 = -D_2 \tan(|\kappa|L). 
\end{equation}
Zdaj lahko zapišemo amplitudi znotraj nelinearne snovi
\begin{align}
\tilde{A}_{3}\left(z\right) & =  \tilde{A}_3(0)
\frac{\cos\left(|\kappa|(L-z)\right)}{\cos\left(|\kappa|L\right)}
\qquad \mathrm{in} \qquad
\tilde{A}_{4}\left(z\right) & =  \tilde{A}_3^*(0)\frac{i \kappa}{|\kappa|}
\frac{\sin\left(|\kappa|(L-z)\right)}{\cos\left(|\kappa|L\right)}.
\label{8.109}
\end{align}
Izračunajmo še amplitudi odbitega in prepuščenega vala. Amplituda odbitega vala 
pri $z=0$ je 
\boxeq{8.110}{
\tilde{A}_{4}(0)  =  \tilde{A}_3^*(0)\frac{i \kappa}{|\kappa|}
\tan \left(|\kappa|L\right)\!,
}
amplituda prepuščenega pri $z = L$ pa
\boxeq{8.110a}{
\tilde{A}_{3}(L)  =  \frac{\tilde{A}_3^*(0)}{
\cos \left(|\kappa|L\right)}.
}
Oglejmo si rezultat podrobneje. Vidimo, da je odbiti val sorazmeren 
kompleksno konjugirani amplitudi vpadnega vala, kar je poglavitna značilnost
fazne konjugacije. 
Poleg konjugirane amplitude ima tudi natanko nasprotni valovni vektor. 
Zanimiva je tudi njegova velikost. Ker 
je lahko $\tan\left(|\kappa|L\right)>1$, je odbiti val lahko močnejši od vpadnega.
Ojačenje odbitega vala gre seveda na račun moči črpalnih
valov. V našem računu bi lahko amplituda odbite svetlobe narasla proti neskončnosti, 
vendar zapisane enačbe takrat niso več veljavne, saj smo privzeli, 
da sta signalni in odbiti žarek precej šibkejša od črpalnih.

Poglejmo še prepuščeni žarek. Ker je $\cos(x)\leq1$, je amplituda prepuščenega
žarka vedno večja od amplitude vpadnega. To pomeni, da smo na račun črpalnih žarkov
dobili prepustnost, ki je vedno večja od $100~\%$, in odbojnost, ki je lahko 
večja od $100~\%$.

Pri računu smo predpostavili, da je vpadni signal ravni val. Če je njegova
amplituda odvisna še od prečne koordinate, ga lahko razvijemo po ravnih
valovih in zgoraj izpeljana enačba~(\ref{8.110}) velja za vsako komponento posebej. 
Odbite komponente so sorazmerne s konjugiranimi komponentami signalnega valovanja
z nasprotnim valovnim vektorjem in se seštejejo v valovno fronto enake
oblike kot pri signalnem valovanju, le da potujejo v nasprotni smeri.

\section{Stimulirano Ramanovo in stimulirano Brillouinovo sipanje}
\label{chap:SRS}
\index{Ramanovo sipanje!stimulirano}
\index{Brillouinovo sipanje!stimulirano}
Ko svetloba vpade na snov, se je del siplje. Poleg elastičnega Rayleighovega sipanja,
pri katerem se energija vpadnih fotonov (in z njo frekvenca svetlobe) ohranja, opazimo 
tudi sipanje, pri katerem se energija izhodnih fotonov razlikuje od energije 
vpadnih. 

Če se energija fotonov spremeni zaradi prehajanja molekul snovi med različnimi
vibracijskimi ali rotacijskimi stanji, govorimo o Ramanovem\index{Ramanovo sipanje}
sipanju\footnote{~Indijski fizik in nobelovec Sir Chandrasekhara Venkata Raman, 1888--1970.}. 
Ta pojav je mogoč tako v plinih in tekočinah kot v trdnih snoveh. Navadno 
ločimo dva primera: Stokesovo sipanje\footnote{~Irski fizik in matematik Sir George Gabriel
Stokes, 1819--1903.}, pri katerem foton prek virtualnega vmesnega stanja 
odda energijo molekuli (slika~\ref{08_Raman}\,a),
in anti-Stokesovo sipanje,
\index{Ramanovo sipanje!Stokesovo}\index{Ramanovo sipanje!anti-Stokesovo}
pri katerem foton prejme energijo od vzbujene molekule (slika~\ref{08_Raman}\,b).
V prvem primeru je frekvenca
sipane svetlobe $\nu_s=\nu_0-\nu_v$, pri čemer $\nu_0$ označuje frekvenco vpadne
svetlobe in $\nu_v$ vibracijsko frekvenco, ter v drugem primeru pa $\nu_{as}=\nu_0+\nu_v$.
Ker je v termičnem ravnovesju razmeroma malo molekul v vzbujenem stanju, so ti 
procesi redki in intenziteta anti-Stokesovega sipanja je zato 
še znatno šibkejša od že tako šibkega Stokesovega sipanja. Tipični Ramanov premik 
$\nu_0-\nu_s$ znaša okoli $10^{12}$--$10^{13}~\si{\hertz}$.
Drugi zanimiv primer je, kadar se energija fotonov spremeni zaradi 
vzbujanja akustičnih valov (fononov). Takrat govorimo o Brillouinovem 
sipanju (slika~\ref{08_Raman}\,c).
Tipični Brillouinov premik je $\sim 10^{10}~\si{\hertz}$. \index{Brillouinovo sipanje}

\begin{figure}[ht]
\centering
\def\svgwidth{140truemm} 
\input{slike/08_Raman.pdf_tex}
\caption{Prehodi med energijskimi nivoji za Ramanovo sipanje in shema Brillouinovega sipanja. Vpadna 
svetloba ima frekvenco $\nu_0$, sipana pa $\nu_s$ oziroma $\nu_{as}$. Pri Stokesovem sipanju 
snov energijo prevzame (a), pri anti-Stokesovem sipanju energijo odda (b).
Pri Brillouinovem sipanju svetloba vzbuja akustične valove s frekvenco $\nu_f$ in se na njih odbije (c).}
\label{08_Raman}
\end{figure}

\subsection*{Stimulirano Ramanovo sipanje}
Pri spontanem Ramanovem sipanju se svetloba siplje na termično vzbujenih fluktuacijah
v snovi. Lahko pa se svetloba siplje na fluktuacijah, ki jih povzroči vpadno
svetlobno polje.\footnote{~G. Eckhardt et al., Phys. Rev. Lett.
$\mathbf{9}$, 455 (1962).} To se zgodi, kadar na snov svetimo z dvema žarkoma hkrati: 
s črpalnim žarkom s frekvenco $\nu_0$ in s Stokesovim žarkom s frekvenco $\nu_s$. 
Fotoni črpalnega žarka se absorbirajo in molekule vzbudijo v virtualno stanje, nato pa 
Stokesov žarek stimulira točno določen prehod. Če razlika frekvenc vpadnih 
žarkov ustreza frekvenci določenega vzbujenega stanja, se  molekule vračajo
v izbrano vzbujeno stanje in pri tem izsevajo dodatne fotone s frekvenco $\nu_s$. Zaradi
stimuliranega sevanja\index{Stimulirano sevanje} je izsevana svetloba po fazi in smeri enaka vpadni in Stokesov
žarek se ojačuje. Tako resonančno ojačenje svetlobe na račun črpalne svetlobe imenujemo stimulirano  
Ramanovo sipanje. Moč svetlobe pri $\nu_s$  narašča eksponentno -- do določene meje, seveda.
Stimulirano Ramanovo sipanje je navadno zelo močno, saj se lahko v Stokesov žarek pretvori 
več kot $\sim 10~\%$ energije črpalnega žarka. Za primerjavo, tipični delež sipane svetlobe v primeru 
spontanega Ramanovega sipanja je $10^{-6}/\si{\centi\meter}$.\footnote{~Glej npr. 
G. New, {\it Introduction to Nonlinear Optics}, Cambridge University Press (2011).}

Obravnavajmo stimulirano Ramanovo sipanje v klasičnem približku, v katerem snov opišemo 
z $N$ neodvisnimi enodimenzionalnimi harmonskimi oscilatorji na enoto prostornine.\footnote{~Glej npr. 
R. W. Boyd, {\it Nonlinear Optics}, tretja izdaja, Academic Press (2008).} Nihanje
posameznega oscilatorja zadošča enačbi\index{Harmonski oscilator}
\begin{equation}
\frac{d^2X(z,t)}{dt^2}+ \gamma \frac{dX}{dt}+\omega_v^2X = \frac{F(z,t)}{m}.
\label{srs:X}
\end{equation}
Pri tem so $X$ odmik od ravnovesne lege, $\gamma$ koeficient dušenja, $m$ masa in $F$ zunanja sila.
Glavna predpostavka modela je, da polarizirnost molekul ni konstantna, ampak odvisna od 
``raztega'' nihajoče molekule oziroma oscilatorja. Polarizirnost $\alpha$, ki jo vpeljemo
kot kvocient med induciranim dipolnim momentom in jakostjo električnega polja $\mathbf{p} = \alpha \mathbf{E}$,
razvijemo v Taylorjevo vrsto okoli ravnovesne polarizirnosti  $\alpha_0$ do prvega člena in zapišemo dielektričnost
\begin{equation}
\varepsilon = 1+N\alpha/\varepsilon_0 = 1+\frac{N}{\varepsilon_0}\left(\alpha_0 + \left(\frac{d\alpha}{dX}\right) _0 X\right)\!.\label{srs:a}
\end{equation}
Silo na en oscilator izračunamo kot odvod energije po odmiku
\begin{equation}
F = \frac{dW}{dX}= \frac{1}{2}\left(\frac{d\alpha}{dX}\right)_0\,\overline{E^2}.
\label{srs:F}
\end{equation}
Jakost električnega polja smo zapisali kot povprečje, saj so optične frekvence 
tako hitre, da jim molekule ne morejo slediti. 
Celotno električno polje zapišemo kot vsoto črpalnega in Stokesovega polja
\begin{equation}
E(z,t)= \frac{1}{2}\left( E_0(z)e^{-i\omega_0t}+ E_s(z)e^{-i\omega_st} + \mathrm{k.\,k.}\right)
\label{eq:srsE}
\end{equation}
in časovno odvisno povprečje kvadrata
\begin{equation}
\overline{E^2} = \frac{1}{4}\left(2E_0E_s^* e^{-i(\omega_0-\omega_s)t}+\mathrm{k.\,k.}\right)\!.
\end{equation}
Vstavimo zapisano jakost električnega polja v izraz za silo (enačba~\ref{srs:F}),
to pa v enačbo oscilatorja (enačba~\ref{srs:X}). Dobimo
\begin{equation}
\frac{1}{2}\left(\omega_v^2-\omega^2-i\omega\gamma\right)\tilde{X} = 
\frac{1}{4m}\left(\frac{d\alpha}{dX}\right)_0 E_0 E_s^*,
\end{equation}
pri čemer smo vpeljali $\omega = \omega_0-\omega_s$ in $\tilde{X}$, tako da velja
\begin{equation}
X(z,t) = \frac{1}{2}\left(\tilde{X}e^{-i\omega t}+ \mathrm{k.~k.}\right)\!.
\label{eq:srsX}
\end{equation}
Oscilatorji torej nihajo s kompleksno amplitudo
\begin{equation}
\tilde{X} = \frac{\left(\frac{d\alpha}{dX}\right)_0 E_0 E_s^*}{2m\left(
\omega_v^2-(\omega_0-\omega_s)^2-i(\omega_0-\omega_s)\gamma\right)}.
\end{equation}
Zdaj lahko zapišemo polarizacijo $P = N\alpha E$. Nelinearni del
polarizacije dobimo z uporabo razvoja v enačbi~(\ref{srs:a}) \index{Električna polarizacija}
in z upoštevanjem enačb~(\ref{eq:srsE}) in (\ref{eq:srsX}) 
\begin{equation}
P_{\mathrm{NL}} = \frac{1}{4} N \left(\frac{d\alpha}{dX}\right)_0 \, \left(\tilde{X}
e^{-i(\omega_0-\omega_s)t}
+ \mathrm{k.~k.}\right) \cdot \left( E_0(z)e^{-i\omega_0t}+ E_s(z)e^{-i\omega_st} + \mathrm{k.~k.}\right)\!.
\end{equation}
Omejimo se le na polarizacijo pri krožni frekvenci $\omega_s$ in zapišemo
\begin{equation}
P_{\mathrm{NL},\omega_s} = \varepsilon_0 \chi_s E_s = 
\frac{N }{8m}\left(\frac{d\alpha}{dX}\right)_0^2 \, 
\frac{|E_0|^2}{\omega_v^2-(\omega_0-\omega_s)^2+i(\omega_0-\omega_s)\gamma}\,E_s.
\label{srs:chi}
\end{equation}
Efektivna susceptibilnost za svetlobo pri krožni frekvenci $\omega_s$ 
je \index{Susceptibilnost!efektivna} torej kompleksna in sorazmerna intenziteti 
vpadne laserske svetlobe pri krožni frekvenci $\omega_0$. 
V resonanci, ko je $\omega_0-\omega_s = \omega_v$, je efektivna susceptibilnost
imaginarna in negativnega predznaka, kar ima, kot bomo videli, zelo pomembne 
fizikalne posledice. Iz zapisanega izraza je tudi razvidno, zakaj gre pri 
stimuliranem Ramanovem sipanju za nelinearni optični pojav tretjega reda. 

Kompleksna susceptibilnost pomeni kompleksni lomni količnik. Če upoštevamo
le prve člene v razvoju, se vpadna svetloba po snovi širi kot\index{Susceptibilnost!kompleksna}
\begin{equation}
E_s(z) = E_s(0)\exp\left(i k z + ikz\frac{1}{2}\Re(\chi_s)- k z \frac{1}{2}\Im(\chi_s)\right)\!,
\label{srs:Es}
\end{equation}
pri čemer $\Re$ označuje realni in $\Im$ imaginarni del.
\begin{definition}
\label{nalogasrs}
 Izpelji enačbo~(\ref{srs:Es}), tako da iz efektivne susceptibilnosti izračunaš lomni količnik,
in pokaži, da je imaginarni del susceptibilnosti $\Im(\chi_s)$ vedno negativen. 
\end{definition}
Ker je imaginarni del efektivne susceptibilnosti vedno negativen (naloga~\ref{nalogasrs}), 
jakost e\-lek\-trič\-nega polja eksponentno narašča na račun črpalnega laserskega snopa. 
Največjo ojačenje je seveda v primeru, ko je sistem v resonanci
in razlika frekvenc vpadne svetlobe enaka vibracijski frekvenci.
Če zanemarimo izgube, lahko zapišemo
\begin{equation}
|E_s(L)|^2 = |E_s(0)|^2 e^{G_RjL}.
\end{equation}
Vrednosti $G_R$ za najmočnejša nihanja so $0,024~\si{\cm/\mega\watt}$ za CS$_2$, \index{CS$_2$}
$0,029~\si{\cm/\mega\watt}$ za LiNbO$_3$ \index{LiNbO$_3$}
in $0,0008~\si{\cm/\mega\watt}$ za SiO$_2$.\index{SiO$_2$}\footnote{~A. Yariv in 
P. Yeh, {\it Photonics}, šesta izdaja, Oxford University Press (2007).}
V $1~\si{m}$ dolgem odseku optičnega vlakna je tako pri gostoti svetlobnega toka 
$10^{10}~\si{\watt/\meter^2}$  faktor ojačenja $1,08$, na vlaknu dolžine $20~\si{\metre}$
pa $5$.

\begin{remark}
Pojav stimuliranega Ramanovega sipanja se uporablja za izdelavo optičnih 
ojačevalnikov. Posebej pomemben je ta pojav za ojačenje signala v 
optičnih vlaknih, predvsem zaradi velike intenzitete na velikih dolžinah.
\index{Optično ojačevanje!v vlaknih}
\end{remark}

\subsection*{Stimulirano Brillouinovo sipanje}
\index{Brillouinovo sipanje!stimulirano}
Stimulirano Brillouinovo sipanje\footnote{~Francoski fizik L\'eon Nicolas Brillouin, 1889--1969.} je 
pojav, pri katerem vpadno svetlobno valovanje
vzbudi akustični val (fonon), nato pa se na njem siplje. Poleg vpadne svetlobe pri $\nu_0$
se tako pojavi še Stokesova svetloba pri frekvenci $\nu_s = \nu_0-\nu_f$, pri čemer 
$\nu_f$ označuje frekvenco akustičnega vala  (slika~\ref{08_Raman}\,c). Interferenca
vpadnega in Stokesovega valovanja, ki ima komponento ravno pri $\nu_f$, povratno
povečuje intenziteto vzbujenega zvočnega valovanja, njegovo povečanje pa vodi do
večje intenzitete Stokesovega valovanja. Zaradi pozitivne povratne zanke se 
svetloba eksponentno ojačuje. Najmočnejši pojav je 
ravno v nasproti smeri od vpadne svetlobe, v smeri naprej pa Brillouinovega sipanja ni.
Če bi namreč vstopna in izstopna svetloba bili vzporedni, bi bila razlika njunih
valovnih vektorjev enaka nič. 

Računa za Brillouinovo sipanje ne bomo naredili, ga pa lahko bralec poišče
v literaturi.\footnote{~Glej npr. R. W. Boyd, {\it Nonlinear Optics}, tretja izdaja, Academic Press (2008).}
Poglavitno 
je, da se tudi pri stimuliranem Brillouinovem sipanju signal pri zmanjšani frekvenci
eksponentno ojačuje
\begin{equation}
|E_s(z)|^2 = |E_s(L)|^2 e^{G_Bj(L-z)}.
\end{equation}
Pri zapisu smo upoštevali, da se val širi in ojačuje v nasprotni smeri od vpadnega.
Vrednosti parametra $G_B$ so na primer $0,13~\si{\cm/\mega\watt}$ za CS$_2$ \index{CS$_2$}
in $0,0045~\si{\cm/\mega\watt}$ za SiO$_2$.\index{SiO$_2$}\footnote{~A. Yariv in 
P. Yeh, {\it Photonics}, šesta izdaja, Oxford University Press (2007).}
Ker je koeficient $G_B$ odvisen od zunanjih parametrov, na primer od 
temperature ali pritiska, lahko stimulirano Brillouinovo sipanje uporabimo tudi za
izdelavo natančnih senzorjev. 

\section{Nelinearni pojavi v optičnih vlaknih}
\label{NLOFIB}
V osmem poglavju smo podrobno obravnavali optična vlakna. Omenili smo, da pri 
velikih intenzitetah lahko nastopijo nelinearni pojavi. Oglejmo si nekaj najpomembnejših.\footnote{~Glej npr.
G. P. Agrawal, {\it Applications of Nonlinear Fiber Optics}, Academic Press (2001).}

\subsection*{Pojavi drugega reda}\index{Nelinearna optika!v vlaknih}
Optična vlakna so praviloma narejena iz SiO$_2$, \index{SiO$_2$}
za katerega je zaradi simetrije molekul $\chi^{(2)}=0$. 
Nelinearnih pojavov drugega reda zato ne opazimo, razen izjemoma na morebitnih 
nepravilnostih v steklu.\index{Nelinearna optika!drugega reda}
Da bi izkoristili nelinearne optične pojave
drugega reda in v vodnikih dosegli na primer optično frekvenčno podvajanje,
\index{Frekvenčno podvajanje}
morajo biti vodniki zgrajeni iz snovi, ki imajo nelinearno susceptibilnost 
različno od nič. Prednost frekvenčnega podvajanja v vodnikih pred navadnimi 
kristali je v tem, 
da svetloba znotraj sredice vodnika potuje brez uklona. Posledično je
pretvorba iz osnovne v frekvenčno podvojeno svetlobo, ki je sorazmerna s kvadratom 
dolžine poti (enačba~\ref{eq:shgl2}), zelo učinkovita. Seveda 
mora biti izpolnjen tudi pogoj za ujemanje faz.\index{Ujemanje faz} To 
omogoča rodovna disperzija, zaradi \index{Disperzija!rodovna}
katere lahko rodova pri osnovni in podvojeni frekvenci potujeta po vodniku
z enako komponento valovnega vektorja $\beta$.

\subsection*{Pojavi tretjega reda}
V navadnih optičnih vlaknih  
prevladujejo nelinearni pojavi tretjega reda,\index{Nelinearna 
optika!tretjega reda} ki jih v grobem delimo v dve skupini. Prva vključuje
neelastično sipanje (Ramanovo in Brillouinovo), druga pa pojave, ki 
temeljijo na optičnem Kerrovem pojavu.\index{Kerrov pojav!optični}

Obravnavajmo najprej stimulirano Ramanovo sipanje 
(SRS)\index{Ramanovo sipanje!stimulirano}, pri katerem se intenziteta 
svetlobe pri vpadni frekvenci zmanjšuje, na njen račun pa se eksponentno povečuje 
intenziteta valovanja z malenkost nižjo frekvenco. Razlika frekvenc ustreza
vibracijskemu prehodu molekul v snovi. Zaradi SRS se signal v telekomunikacijskih vlaknih 
popači in poveča se njegova spektralna širina. Po drugi strani stimulirano Ramanovo sipanje 
izkoriščamo za ojačenje signala v vodnikih. Če v vlakno posvetimo z močno črpalno
svetlobo, katere frekvenca se od signalnega žarka razlikuje za vibracijsko frekvenco
($\sim 13~\si{THz}$), se signalni žarek ojači. Ker je spekter SiO$_2$ razmeroma 
širok ($\sim 5~\si{THz}$), tega pogoja ni težko izpolniti. 

Pri stimuliranem Brillouinovem sipanju (SBS) se vpadna svetloba
\index{Brillouinovo sipanje!stimulirano} odbije na 
optično vzbujenem akustičnem valu v snovi. Signal
v smeri naprej oslabi, pojavi pa se odbiti val, katerega intenziteta
narašča eksponentno z intenziteto vpadne svetlobe. 
Neželenemu pojavu se lahko izognemo z zmanjšanjem vpadne
moči (pod $\sim 100~\si{mW}$ na kanal) 
ali povečanjem spektralne širine vpadne svetlobe.

Druga skupina nelinearnih pojavov temelji na spreminjanju 
lomnega količnika z intenziteto vpadne svetlobe (optični Kerrov pojav).
Opišimo tri pojave:
\begin{enumerate}
\item 
Samomodulacija faze ({\it SPM -- Self-Phase Modulation}). Različni deli
sunka zaradi različne intenzitete občutijo različen lomni količnik in pride
do tako imenovanega čirikanja~(enačba~\ref{eq:chirpi} 
in slika~\ref{fig:optsoliton}). \index{Samomodulacija faze}
Pojav vodi do spektralne razširitve in zaradi\index{Čirikanje}
disperzije tudi do časovnega podaljšanja sunka. Z ustrezno disperzijo dosežemo 
krajšanje sunkov ali pojav optičnih solitonov (razdelek~\ref{chap:soliton}).
\index{Soliton!optični}

\item
Navzkrižna modulacija faze ({\it CPM -- Cross-Phase Modulation}).\index{Navzkrižna
modulacija faze} Ko po vlaknu
potuje več svetlobnih sunkov hkrati, prvi sunek povzroči spremembo lomnega količnika, 
drugi sunki pa to spremembo občutijo. Medsebojna motnja med sunki povzroči 
spektralno razširitev. Pojav lahko izkoristimo za krajšanje sunkov ali
za izdelavo optičnih stikal, 
saj lahko z zunanjim kontrolnim žarkom spreminjamo fazo izbranega sunka.

\item
Sklopitev štirih valov ({\it FWM -- Four-wave mixing}). Če po vlaknu potuje več
valovanj z različnimi frekvencami,\index{Sklopitev štirih valov} 
se zaradi nelinearne sklopitve pojavijo
valovanja pri dodatnih frekvencah (vsotah in razlikah frekvenc obstoječih valovanj). To velja
predvsem v vlaknih z zelo majhno disperzijo, v katerih se faze lahko ujemajo.  
\end{enumerate}

Čeprav so nelinearni pojavi v vlaknih pogosto nezaželeni, jih lahko s pridom
izkoriščamo pri izdelavi na primer novih vlakenskih laserjev, ojačevalnikov, preklopnikov ali 
optičnih logičnih vezij. 

%Končano

\newpage
% \chapterimage{Lit.jpg} % Chapter heading image
\thispagestyle{plain}
\vglue1truemm
\section*{\LARGE{Recenziji}}
\vskip1truecm
V monografiji avtorja obravnavata izbrane vsebine s področja fotonike. 
Delo nagovarja tako raziskovalce področja kot tiste, ki v svet fotonike 
šele vstopajo. Monografija na strnjen, a kljub temu poglobljen način 
obravnava ključne fizikalne vsebine, ki so potrebne za temeljno razumevanje 
področja, hkrati pa omogočajo nadaljnje delo na številnih področjih fotonike,
ki jih pričujoče delo ne vsebuje.

Na temo fotonike lahko v tujih jezikih najdemo številne monografije, ki se lotevajo različnih vsebin, saj je področje izjemno široko. Značilnost pričujoče monografije je, da z osnovnimi vsebinami, kot so razširjanje svetlobe skozi snov, polarizacija elektromagnetnega valovanja, lom, odboj in uklon, opravi že v prvem poglavju, ki je namenjeno osvežitvi osnovnih pojmov. Knjigo naj zato v roke vzamejo tisti, ki so z osnovami teorije elektromagnetnega polja in valovne optike že dobro podkovani. Že v naslednjem poglavju namreč avtorja obravnavata koherenco in nato koherentne snope svetlobe, predvsem osnovni Gaussov snop, ošvrkneta pa tudi snope višjih redov. Pri obravnavi optičnih resonatorjev se lotita tudi zahtevnih vsebin, na primer iskanja električnega polja v nestabilnih resonatorjih z uklonskim integralom in sklopitve resonatorja z okolico ali drugim resonatorjem. Nato se posvetita interakciji svetlobe s snovjo in podrobno obravnavata optično črpanje trinivojskega sistema, ponudita tudi kvantnomehansko izpeljavo verjetnosti za prehod in obravnavata Rabijeve oscilacije v dvonivojskih sistemih. Osnova fotonike so seveda laserji. Monografija podrobno obravnava ojačanje svetlobe v laserjih, spektralno širino izsevane svetlobe, stabilizacijo frekvence laserja, stacionarno in nestacionarno delovanje laserja, doseganje kratkih laserskih sunkov s preklopom dobrote ter aktivnim in pasivnim uklepanjem faz, v posebnem poglavju pa avtorja obravnavata tudi nekaj najpomembnejših vrst laserjev.

Najbolj znana in razširjena uporaba laserjev je nedvomno za prenos informacije po optičnih vlaknih. Reševanje Maxwellovih enačb v omejeni geometriji je vedno izziv, a avtorja suvereno peljeta bralca skozi izpeljave rodov v vlaknih s stopničastim in paraboličnim profilom lomnega količnika. Obravnavata tudi izgube v vlaknih, podaljšanje curka zaradi disperzije in možnosti za njeno kompenzacijo z uklonskima mrežicama in s fazno samomodulacijo. Ne izogneta se niti kvantni obravnavi sklopitve svetlobe v optične vodnike ter čelne in vzdolžne sklopitve dveh vlaken. Za prenos informacij je treba svetlobo modulirati, zato avtorja podrobneje obravnavata modulacijo svetlobe z elektro-optičnim in elasto-optičnim pojavom, na koncu pa tudi detektirati, kjer največ pozornosti namenita polprevodniškim kvantnim detektorjem.

Velike jakosti električnega polja v laserski svetlobi so omogočile opazovanje nelinearnega odziva snovi. Področju nelinearne optike je posvečeno obširno poglavje, ki je dober preplet nizanja ključnih nelinearnih pojavov in poglobljene obravnave nekaterih od njih, kot na primer optično frekvenčno podvajanje tako za ravne kot Gaussove snope, optični Kerrov pojav, samozbiranje, krajevni in optični solitoni, optična fazna konjugacija in stimulirano Ramanovo sipanje.

Knjigo odlikujejo dopadljive naslovne fotografije poglavij in slike, ki so odlična podpora tekstu, saj zanje res velja pogosto izrečeno, da ena dobra slika nadomesti tisoč besed. Tako raziskovalci kot študenti bodo nedvomno hvaležni tudi za številne izpeljave, prek katerih si bodo pridobili podrobno razumevanje in s tem zmožnost aplikacije na raziskovalne in strokovne probleme, s katerimi se srečujejo pri delu. Za izpeljave sta avtorja sicer izbrala dokaj nekonvencionalno pot pisanja več enačb v eni vrstici in nizanja izrazov z več enačaji, kar je na začetku morda moteče, a se bralec hitro privadi in je hvaležen za podajo korakov. Monografija vsebuje tudi številne naloge, v okviru katerih lahko bralec preveri svoje razumevanje ali dobi namige za izpeljave, ki so v besedilu izpuščene. Pravi užitek pa je tudi prebiranje ``zvezdic'', kjer avtorja obravnavano temo povežeta z analognimi temami ali izpeljavami na drugih področjih fizike, z drugimi temami v tej monografiji, ponudita pa tudi primere praktične uporabe. Monografijo odlikujeta lepo tekoča misel in izboren jezik, njen velik pomen pa je nedvomno tudi v utemeljitvi slovenskega izrazoslovja na področju fotonike.

Maribor, 22. 5. 2020\hfill Prof. dr. Nataša Vaupotič  

\vglue0.5truecm
Knjiga Fotonika, napisana v slovenskem jeziku, avtorjev prof. dr.
Martina Čopiča in doc. dr. Mojce Vilfan, predstavlja zanimivo znanstveno branje s področja fotonike s poglobljenim fizikalnim opisom. Številna poglavja fotonike so ustrezno obdelana, s čimer nudijo bralcu zelo dober vpogled v tematiko.

Fotonika je znanstveno področje, ki se ukvarja z generacijo, detekcijo in spreminjanjem svetlobe, pri čemer se naslanja na številne linearne ali nelinarne fizikalne pojave -- od ojačevanja svetlobe do pretvorbe valovne dolžine. Knjiga se začne s poglavji, ki opisujejo elektromagnetno valovanje, koherenco, koherentne snope in optičnine resonatorje, v okviru katerih se osredotoči predvsem na Gasussove snope. Nadaljuje se z obravnavanjem interakcije svetlobe s snovjo in laserji. Po pregledu osnovnih tipov laserjev pridejo na vrsto optična vlakna, detekcija in modulacija svetlobe. Knjiga se zaključi s poglavjem o nelinarnih optičnih pojavih.

Na področju fotonike delujejo številni slovenski raziskovalci, tako v večjih javnih institucijah, kot so Univerza v Ljubljani, Institut Jožef Stefan, Univerza v Mariboru in Univerza na Primorskem, kot tudi v nekaterih podjetjih, kot so Fotona, LPKF, Optotek, Lumentum, če naštejem samo največja. Ker knjiga med drugim predstavi praktično vsa pomembnejša področja fotonike, bo pomembna tako za raziskovalce v osnovnih in aplikativnih raziskavah kot za inženirje, ki jim bo lahko služila kot neke vrste poglobljen priročnik.

Prav tako ne smemo spregledati pomena takšne publikacije s stališča jezika, saj uvaja številne marsikdaj težko prevedljive izraze s področja fotonike ter s tem bogati slovensko znanstveno in strokovno izrazoslovje ter s tem jezik kot tak.

Ljubljana, 25. 5. 2020\hfill Izr. prof. dr. Rok Petkovšek 

\newpage
% \chapterimage{Lit.jpg} % Chapter heading image
\thispagestyle{plain}
\vglue1truemm
\section*{\LARGE{Dodatna literatura}}
\vskip1truecm
\begin{itemize}
\setlength\itemsep{0.5em}
 \item A. Yariv in P. Yeh, {\it Photonics}, šesta izdaja, Oxford University Press (2007).
 \item B. E. A. Saleh in M. C. Teich, {\it Fundamentals of Photonics}, druga izdaja, John Wiley \& Sons, Inc. (2007). 
 \item G. A. Reider, {\it Photonics, An Introduction}, Springer (2016).
 \item V. Degiorgio in I. Cristiani, {\it Photonics, A Short Course}, druga izdaja, Springer (2016).
 \item W. T. Silfvast, {\it Laser Fundamentals}, druga izdaja, Cambridge University Press (2004). 
 \item C. C. Davis, {\it Lasers and Electro-Optics}, Cambridge University Press (2006).
 \item O. Svelto in D. C. Hanna, {\it Principles of Lasers}, peta izdaja, Springer (2010).
 \item K. F. Renk, {\it Basics of Laser Physics}, druga izdaja, Springer (2017).
 \item J. W. Goodman, {\it Statistical Optics}, druga izdaja, John Wiley \& Sons, Inc. (2015).
 \item W. Demtr\"oder, {\it Laser Spectroscopy}, peta izdaja, Springer (2014).
 \item A. Ghatak in K. Thyagarajan, {\it Introduction to Fiber Optics}, Cambridge University Press (1997).
 \item G. H. Rieke, {\it Detection of Light}, druga izdaja, Cambridge University Press (2003).
 \item G. New, {\it Introduction to Nonlinear Optics}, Cambridge University Press (2011).
 \item R. W. Boyd, {\it Nonlinear Optics}, tretja izdaja, Academic Press (2008).
\end{itemize}

\cleardoublepage
\thispagestyle{plain}
\mbox{}
\cleardoublepage
\chapterimage{Lit.jpg}
\printindex

\end{document}
